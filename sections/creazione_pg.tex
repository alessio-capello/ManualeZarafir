\documentclass[../manuale_main.tex]{subfiles}



\begin{document}

Prima di immergerci nella creazione di un nuovo personaggio, è fondamentale comprendere la \textbf{“Regola Zero”} di \textbf{Zarafir}, una regola comune a tutti i giochi di ruolo: \textbf{il Master ha la facoltà di modificare o ignorare le regole del manuale per qualsiasi motivo, purché ciò migliori l’esperienza di gioco}. Questo potere deve essere utilizzato con buon senso: le regole servono come guida, ma la narrazione e il divertimento dei giocatori sono sempre al primo posto.

\vspace{0.3cm}

\section{Un Sistema Libero e Flessibile}
Zarafir non utilizza un sistema di classi predefinite. I personaggi sono definiti dalle loro \textbf{Caratteristiche} e \textbf{Abilità}, che possono essere combinate liberamente per creare personaggi unici. 

\begin{itemize}
    \item \textbf{Libertà Creativa:} Ogni giocatore può creare un personaggio che rispecchia la propria visione, senza essere vincolato a schemi rigidi.
    \item \textbf{Ruolo del Master:} Il Master ha il compito di valutare le richieste dei giocatori durante la creazione dei personaggi, garantendo che le caratteristiche e le abilità scelte siano coerenti con l’ambientazione e il tipo di personaggio che i giocatori desiderano interpretare.
    \item \textbf{Adattamento e Coerenza:} Eventuali modifiche possono essere discusse e concordate insieme, per garantire che il personaggio sia equilibrato e in linea con il mondo di gioco.
\end{itemize}

\vspace{0.5cm}
\noindent
\begin{center}
\rule{\textwidth}{0.4pt} 
\end{center}
\vspace{0.5cm}

\section{La Scheda del Personaggio}
Creare un nuovo personaggio è un processo semplice. Potete utilizzare la \textbf{Scheda del Personaggio} fornita in appendice oppure annotare le seguenti informazioni su un foglio di carta:

\begin{itemize}
    \item \textbf{Nome e Cognome:} (se l'ambientazione lo prevede)
    \item \textbf{Razza:} La specie o etnia del personaggio.
    \item \textbf{Sesso:} Identità di genere del personaggio.
    \item \textbf{Età:} L’età apparente o reale.
    \item \textbf{Altezza e Peso:} Aspetto fisico del personaggio.
    \item \textbf{Allineamento:} Orientamento morale o etico (facoltativo).
    \item \textbf{Punti Vita e Stress Mentale:} I valori iniziali di salute e resistenza mentale.
    \item \textbf{Punti Esperienza:} I punti accumulati durante l'avventura.
    \item \textbf{Caratteristiche:} Forza, Agilità, Costituzione, Intelligenza, Sensibilità, Reattività e Carisma.
    \item \textbf{Abilità e Talenti:} Le competenze del personaggio.
    \item \textbf{Inventario ed Equipaggiamento:} Da completare successivamente.
\end{itemize}

\textbf{Suggerimento:} Si consiglia di aggiungere una sezione nella scheda per annotare i valori usati frequentemente, come \textbf{i Modificatori del personaggio, il Potenziale Offensivo, Il valore dell'armatura, ecc}, per semplificare il gioco ed evitare di ricalcolarli ogni volta.

\vspace{0.5cm}\noindent\centering\rule{\textwidth}{0.4pt}\vspace{0.5cm}

\section{Le Caratteristiche del Personaggio}
Il personaggio è definito da sette caratteristiche principali, che rappresentano le sue capacità fisiche, mentali e sociali:

\begin{itemize}
    \item \textbf{Forza (Fo):} Capacità fisica e controllo della stessa. Influenza il combattimento corpo a corpo e la capacità di trasportare oggetti pesanti.
    \item \textbf{Agilità (Ag):} Rapidità e coordinazione dei movimenti. Condiziona il combattimento a distanza.
    \item \textbf{Costituzione (Co):} Resistenza fisica e capacità di sopportare sforzi, ferite e malattie.
    \item \textbf{Intelligenza (In):} Capacità di comprendere, apprendere e utilizzare la magia.
    \item \textbf{Sensibilità (Se):} Saggezza, empatia e percezione delle sfumature del mondo.
    \item \textbf{Reattività (Re):} Rapidità di pensiero e azione. Influenza la capacità di reagire prontamente.
    \item \textbf{Carisma (Ca):} Capacità di relazionarsi con gli altri. Determina l’efficacia nelle interazioni sociali.
\end{itemize}

\subsection*{Assegnazione delle Caratteristiche}
Alla creazione del personaggio, ogni giocatore assegna un valore a ciascuna delle sette caratteristiche, scegliendo tra i seguenti numeri:

\[
\textbf{6, 7, 8, 9, 10, 11, 12}
\]

Questi valori rappresentano la base del personaggio e possono essere modificati successivamente grazie ai bonus e malus dati dalla razza scelta e dalle competenze.

\vspace{0.3cm}

\subsection*{Esempi di Assegnazione delle Caratteristiche}
\textbf{Esempio: Personaggio Combattente}
\begin{itemize}
    \item Forza (Fo): 12
    \item Agilità (Ag): 6
    \item Costituzione (Co): 11
    \item Intelligenza (In): 7
    \item Sensibilità (Se): 8
    \item Reattività (Re): 10
    \item Carisma (Ca): 9
\end{itemize}

\textbf{Esempio: Personaggio Mago}
\begin{itemize}
    \item Forza (Fo): 6
    \item Agilità (Ag): 7
    \item Costituzione (Co): 8
    \item Intelligenza (In): 12
    \item Sensibilità (Se): 11
    \item Reattività (Re): 10
    \item Carisma (Ca): 9
\end{itemize}

\vspace{0.5cm}
\noindent
\begin{center}
\rule{\textwidth}{0.4pt} 
\end{center}
\vspace{0.5cm}


\section{Razze e Origini}

\noindent
Questo manuale non è vincolato a un’ambientazione fissa: le razze presentate in questa sezione sono \textbf{archetipi generici}, adattabili a qualunque mondo immaginabile. Ogni razza rappresenta un insieme di caratteristiche fisiche o culturali che possono essere interpretate come vere e proprie specie diverse, ceppi genetici, stirpi mutanti, o semplicemente popoli di una stessa civiltà.

\medskip
\noindent
I nomi sono indicativi. Il Master può rinominarli per adattarli all’ambientazione: per esempio, la “Razza Forte e Goffa” potrebbe essere un orco, un costrutto, un gigante o un cyborg; la “Razza Alta e Agile” potrebbe rappresentare elfi, androidi o felinoidi. La “Razza Umana” è una categoria neutra, adatta a personaggi versatili.

\begin{itemize}
  \item \textbf{Razze Forti e Goffe}: +1 Forza, +1 Costituzione, -1 Reattività. Tipiche di creature fisicamente potenti ma poco agili.
  \item \textbf{Razze Alte e Agili}: +1 Destrezza, +1 Reattività, -1 Costituzione. Adatte a esseri slanciati, scattanti e fragili.
  \item \textbf{Razze Intuitive}: +1 Sensibilità, +1 Carisma, -1 Costituzione. Specie sensibili e percettive, ma fisicamente deboli.
  \item \textbf{Razze Figlie dell’Ombra}: +1 Sensibilità, +1 Reattività, -1 Carisma. Adatte a chi è nato per vedere dove altri temono guardare.
  \item \textbf{Razze Mistiche}: +1 Intelligenza, +1 Sensibilità, -1 Reattività. Indicano una particolare affinità con le forze invisibili del mondo.
  \item \textbf{Razze Predatrici}: +1 Forza, +1 Reattività, -1 Costituzione. Specie selvatiche, istintive, e potenzialmente aggressive.
  \item \textbf{Razze Civili}: +1 Carisma, +1 Intelligenza, -1 Costituzione. Popoli istruiti, abituati alla diplomazia e alla vita urbana.
  \item \textbf{Razze Umane}: A discrezione del master. L’archetipo più flessibile e adattabile.
\end{itemize}

\medskip
Tutti i modificatori sono pensati per offrire un tocco iniziale di differenziazione. Non rappresentano obblighi narrativi o limiti comportamentali: ogni personaggio è libero di definirsi al di là della propria biologia.

\medskip
Il Master può introdurre nuove razze, sostituire i bonus numerici con tratti narrativi, o ignorare del tutto questa sezione se l’ambientazione prevede un solo tipo di essere senziente.

\vspace{0.3cm}

\section{Competenze}
Oltre ai modificatori della razza, ogni personaggio può avere una \textbf{Competenza}, che rappresenta una predisposizione o un talento naturale in una caratteristica particolare. 

\begin{itemize}
    \item \textbf{+1 a una caratteristica e -1 a un’altra.}
    \item La competenza non può portare una caratteristica oltre il valore di 13 o sotto il valore di 5.
\end{itemize}


\end{document}