\documentclass[../manuale_main.tex]{subfiles}



\begin{document}

Questo manuale contiene una piccola lista di creature che potrete utilizzare durante le vostre avventure. Non è e non potrebbe essere una lista esaustiva: non conoscendo l'ambientazione che andrete ad utilizzare, sarebbe impossibile definire i tipi e le capacità di ogni creatura del vostro mondo. Abbiamo deciso di integrare questa sezione principalmente per fornire un riferimento al master nella costruzione delle creature in modo che siano coerenti con la \emph{sua} ambientazione.\\
Le creature che verranno mostrate appartengono a un'ambientazione fantasy classica.
Nella descrizione delle statistiche o abilità non sono presenti talenti o specializzazioni, per mantenere i parametri il più generico possibile. Sarà compito del master deciderli, per fornire loro una caratterizzazione migliore e rendere le sfide sempre diverse.

\subsection{Animali}
Fanno parte di questa categoria tutte le creature appartenenti all'ambientazione non contaminate dalla magia in alcun modo (che potrebbero quindi appartenere a un mondo realistico e non "Fantasy"). Gli animali sono considerati tutti come "Non senzienti".

\paragraph{Animali di Piccole dimensioni o inoffensivi}\mbox{}\\ 
Questa categoria racchiude tutti gli animali non pericolosi per un avventuriero che possono essere incontrati. Fanno parte di questa categoria conigli, cani, gatti, cavalli, ecc. Queste creature vengono considerate innocue, pertanto provare a domarle sarà abbastanza semplice e gli attacchi contro di loro verranno gestiti direttamente dal master.  Nota bene: possano esistere versioni degli animali sopracitati particolarmente aggressive o addestrate per esserlo (vedi cani da guerra); in questo caso si consiglia al Master di elaborare le statistiche relative da una creatura che possa essere coerente con la situazione che si desidera rappresentare.
\paragraph{Animale di Medie dimensioni}\mbox{}\\ 
In questa sottocategoria ricadono gli animali di medie dimensioni che potrebbero aggredire gli avventurieri (per fame, se minacciati, ecc). Fanno parte di questa categoria orsi, tigri, lupi, ecc. Queste creature, sebbene più piccole in dimensioni, si possono rivelare più feroci e pericolose rispetto ad animali di grandi dimensioni. Sono tipicamente predatori carnivori, in grado di ferire a morte animali più grandi di loro.\\
Anche in questo caso le linee guida generali per strutturare le caratteristiche queste creature è di capire concettualmente cosa devono rappresentare: un grosso predatore potrebbe avere un buon quantitativo di punti vita e danni base, ma essere meno capace di evitare gli attacchi; un predatore più piccolo, ma molto feroce potrebbe avere alti danni, buona abilità di combattimento ma pochi punti vita.\\
Seguendo questi due archetipi vengono proposti due esempi di animali di medie dimensioni.\\

\begin{center}
\textbf{Orso bruno}\\
\renewcommand{\arraystretch}{1.2}
\begin{tabularx}{\linewidth}{Q F}
Forza&Da 10 a 14\\
Salute&Da 10 a 15\\
Abilità da combattimento su "Fo" (corpo a corpo)&Da 0 a 2\\
Danni&1D10\\
Strato adiposo (livello di armatura)&Da 1 a 3\\
Evasione&Da 0 a 4\\
Difficoltà per addestrarlo&Da 5 a 7\\
\end{tabularx}


\textbf{Lupo}\\
\renewcommand{\arraystretch}{1.2}
\begin{tabularx}{\linewidth}{Q F}
Forza&Da 7 a 10\\
Agilità&Da 9 a 13\\
Salute&Da 6 a 10\\
Abilità da combattimento su "Ag" (corpo a corpo)&Da 1 a 5\\
Danni&1D8\\
Livello armatura&Da 0 a 2\\
Evasione&Da 1 a 4\\
Difficoltà per addestrarlo&Da 5 a 7\\
\end{tabularx}

\end{center}

\clearpage

\paragraph{Animali di Grandi dimensioni}\mbox{}\\
In questa sottocategoria sono presenti tutti gli animali di grandi dimensioni che potrebbero comparire nel corso dell'avventura. Fanno parte di questa categoria animali come: elefanti, mammut, ecc.\\
Queste grandi creature generalmente hanno molti punti vita e infliggono grossi danni, ma a causa delle dimensioni sono piuttosto semplici da colpire.\\

\begin{center}


\textbf{Elefante}\\
\renewcommand{\arraystretch}{1.2}
\begin{tabularx}{\linewidth}{Q F}
Forza&Da 13 a 15\\
Agilità&Da 9 a 13\\
Salute&Da 13 a 20\\
Abilità da combattimento su "Fo" (corpo a corpo)&Da 3 a 5\\
Danni&1D10 + 2\\
Livello armatura&Da 3 a 5\\
Evasione&Da 0 a 2\\
Difficoltà per addestrarlo&Da 4 a 7\\
\end{tabularx}
\end{center}


\subsection{Mostri}
In questa categoria ricadono tutte le creature che non fanno parte della categoria "Animali" o "Umanoidi".\\
Bisogna far notare che alcune creature presenti in questa categoria potrebbero anche essere senzienti (generalmente nelle ambientazioni Fantasy i draghi hanno una coscienza paragonabile a quella umana). Alcune creature mostruose, dipendentemente dall'ambientazione, potrebbero ricadere tra le razze giocabili (come potrebbe avvenire per i minotauri, draconidi ecc).\\
Verrà preso come esempio un classico di tutte le ambientazioni fantasy.


\begin{center}

\textbf{Drago rosso}\\
\renewcommand{\arraystretch}{1.2}
\begin{tabularx}{\linewidth}{Q F}
Forza&Da 12 a 18\\
Intelligenza&Da 11 a 16\\
Salute&Da 25 a 40\\
Abilità da combattimento su "Fo" (corpo a corpo)&Da 4 a 8\\
Stregoneria di attacco su "In" (distruzione)&Da 2 a 8\\
Danni (combattimento)&1D10 + 5\\
Livello armatura&Da 3 a 7\\
Evasione&Da 3 a 5\\
\end{tabularx}

\end{center}


\clearpage

\subsection{Umanoidi}
Questa è la categoria dedicata a tutti i soggetti senzienti (salvo rare eccezioni) di aspetto antropomorfo.\\
Il master potrà comunque decidere di differenziare i rivali che i propri giocatori si troveranno ad affontare, calibrando i valori anche in funzione della razza dei PNG. Come potrete notare, i valori delle abilità hanno una forbice molto ampia, questo per differenziare personaggi di \emph{élite} da \emph{soldati semplici}.\\
Si confida quindi sulla capacità del Master di scegliere i valori che meglio si adattano alla situazione da lui immaginata.\\

\begin{center}

\textbf{Umanoide guerriero}\\
\renewcommand{\arraystretch}{1.2}
\begin{tabularx}{\linewidth}{Q F}
Forza&Da 8 a 13\\
Salute&Da 8 a 15\\
Abilità da combattimento su "Fo" (corpo a corpo)&Da 1 a 7\\
Parare&Da 0 a 5\\
Danni&1D6 +2 \\
Livello armatura&Da 1 a 5\\
Evasione&Da 1 a 6\\
\end{tabularx}

\textbf{Umanoide mago}\\
\renewcommand{\arraystretch}{1.2}
\begin{tabularx}{\linewidth}{Q F}
Forza&Da 5 a 8\\
Intelligenza&Da 8 a 13\\
Salute&Da 7 a 10\\
Abilità da combattimento su "Fo" (corpo a corpo)&Da 0 a 3\\
Abilità magica su "In"&Da 1 a 6\\
Danni (combattimento)&1D4\\
Livello armatura&Da 0 a 1\\
Evasione&Da 1 a 6\\
\end{tabularx}


\textbf{Umanoide ranger}\\
\renewcommand{\arraystretch}{1.2}
\begin{tabularx}{\linewidth}{Q F}
Forza&Da 7 a 10\\
Agilità&Da 8 a 13\\
Salute&Da 8 a 12\\
Abilità da combattimento su "Fo" (corpo a corpo)&Da 0 a 3\\
Abilità da combattimento su "Ag" (a distanza)&Da 1 a 6\\
Danni (combattimento)&1D4\\
Danni (distanza)&1D8\\
Gittata&100 metri\\
Livello armatura&Da 0 a 2\\
Evasione&Da 2 a 6\\
\end{tabularx}

\end{center}


\clearpage
\subsection{Evocazioni note}
In questa sezione sono indicate tutte le evocazioni che vengono citate negli incanti. Le evocazioni prodotte dalle stregonerie vengono descritte e gestite interamente dal giocatore, quindi non avrebbe avuto senso includerle in questa sezione. Le loro caratteristiche però sono ben definite nella descrizione delle Stregonerie di evocazione.\\
L'ambientazione può influire sulla descrizione di queste creature, il master si può sentir libero di allinearli allo stile del mondo in cui i personaggi si muovono. Le evocazioni successivamente indicate possono essere utilizzate dal master come ulteriore esempio nella creazione di NPC e mostri. \\


\begin{center}

\textbf{Cerbero}\\
Il cerbero è un enorme cane a 3 teste, raggiunge l'altezza di un cavallo ma è molto più massiccio. Attacca i nemici dell'incantatore con i suoi denti acuminati. Questa creatura può attaccare fino a 3 volte in un turno.
\renewcommand{\arraystretch}{1.2}
\begin{tabularx}{\linewidth}{Q F}
Forza&12\\
Salute&20\\
Abilità da combattimento su "Fo" (corpo a corpo)&8\\
Danni (combattimento)&1D6 + 1\\
Gittata&30 metri\\
Livello armatura&1\\
Evasione&4\\
\end{tabularx}

\textbf{Demonietto}\\
Questa creatura è un piccolo demone, alto circa un metro e mezzo. Di indole dispettosa e sadica, considera il combattimento come un gioco, trovando divertente infiammare i nemici. Lancia piccole palle di fuoco (considerate per semplicità come armi a distanza).
\renewcommand{\arraystretch}{1.2}
\begin{tabularx}{\linewidth}{Q F}
Forza&4\\
Agilità&11\\
Salute&8\\
Abilità da combattimento su "Fo" (corpo a corpo)&1\\
Abilità da combattimento su "Ag" (a distanza)&5\\
Danni (combattimento)&1D4\\
Danni (distanza)&1D6 + 3\\
Gittata&30 metri\\
Livello armatura&0\\
Evasione&5\\
\end{tabularx}

\textbf{Demone}\\
Il demone è una creatura estremamente pericolosa, solamente gli incantatori esperti sono in grado di controllarla. La sua enorme forza fisica gli permette di cambiare le sorti di uno scontro.
\renewcommand{\arraystretch}{1.2}
\begin{tabularx}{\linewidth}{Q F}
Forza&15\\
Salute&25\\
Abilità da combattimento su "Fo" (corpo a corpo)&5\\
Danni (combattimento)&1D10 + 4\\
Livello armatura&3\\
Evasione&4\\
\end{tabularx}

\clearpage

\textbf{Drago zombi}\\
Questo drago viene richiamato dal regno dei morti per servire l'incantatore. Pur non avendo la possenza di quando era vivo, rimane comunque una delle evocazioni più spaventose e terrificanti. Colpisce i nemici con denti e artigli, inoltre può utilizzare l'incanto "\emph{Soffio del drago}" dell'arte della sapienza (quarto cerchio) una singola volta.
\renewcommand{\arraystretch}{1.2}
\begin{tabularx}{\linewidth}{Q F}
Forza&13\\
Intelligenza&12\\
Salute&25\\
Abilità da combattimento su "Fo" (corpo a corpo)&8\\
Tessimagie (In)&6\\
Danni (combattimento)&1D10 + 4\\
Livello armatura&5\\
Evasione&4\\
\end{tabularx}

\textbf{Grande spirito vendicativo}\\
Si tratta di uno spirito estremamente antico, ancora più consumato dal tempo e distante dal mondo terreno. Tutto ciò lo ha reso più violento con i mortali, rivelandosi un'ottima arma per gli incantatori.
\renewcommand{\arraystretch}{1.2}
\begin{tabularx}{\linewidth}{Q F}
Forza&11\\
Agilità&7\\
Salute&5\\
Abilità da combattimento su "Fo" (corpo a corpo)&8\\
Abilità da combattimento su "Ag" (a distanza)&4\\
Danni (combattimento)&2D6 + 4\\
Danni (distanza)&2D4\\
Gittata&15 metri\\
Livello armatura&2\\
Evasione&6\\
\end{tabularx}



\textbf{Segugio magico}\\
Si tratta di un evocazione dall'aspetto di un cane. Famoso per il suo fiuto, viene utilizzato dagli incantatori nella ricerca di tracce e oggetti come se si trattasse di un vero segugio. Non è in grado di combattere, inoltre subire un qualsiasi attacco lo farebbe svanire.
\renewcommand{\arraystretch}{1.2}
\begin{tabularx}{\linewidth}{Q F}
Sensibilità&13\\
Intelligenza&11\\
Seguire tracce (In)&4\\
Percezione (Se)&4\\
\end{tabularx}

\clearpage

\textbf{Spirito vendicativo}\\
La sua appartenenza a un altro mondo è immediatamente chiara: è una creatura composta di tenebre, senza volto e gambe, con lunghe braccia che terminano in mani sottili dotate di artigli. Lo spirito vendicativo è un'anima in attesa di espiare i crimini commessi in vita, che viene richiamato dagli incantatori votati alla conoscenza della morte. Attacca i nemici dell'incantatore con gli artigli o dardi d'ombra.
\renewcommand{\arraystretch}{1.2}
\begin{tabularx}{\linewidth}{Q F}
Forza&11\\
Agilità&10\\
Salute&5\\
Abilità da combattimento su "Fo" (corpo a corpo)&5\\
Abilità da combattimento su "Ag" (a distanza)&4\\
Danni (combattimento)&1D6\\
Danni (distanza)&1D4\\
Gittata&15 metri\\
Livello armatura&0\\
Evasione&3\\
\end{tabularx}

\textbf{Succube}\\
La succube è una creatura demoniaca, in grado di ammaliare chi la guarda. Pur non eccellendo in capacità combattive, gli incantatori la evocano poichè è in grado di ridurre le capacità offensive dei nemici grazie alla sua presenza. Tutte le creature antropomorfe dotate di volontà propria che vogliono attaccarla tirano 1D6. Con risultato pari a 6 rimangono ammaliate e non riescono a colpirla.
\renewcommand{\arraystretch}{1.2}
\begin{tabularx}{\linewidth}{Q F}
Forza&12\\
Salute&15\\
Abilità da combattimento su "Fo" (corpo a corpo)&5\\
Danni (combattimento)&1D8 + 1\\
Livello armatura&3\\
Evasione&6\\
\end{tabularx}

\end{center}



\end{document}