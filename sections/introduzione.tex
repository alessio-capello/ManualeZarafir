\documentclass[../manuale_main.tex]{subfiles}

\begin{document}


\textbf{Zarafir} è un \textbf{Gioco di Ruolo (GDR)} che permette di creare e vivere una storia attraverso l’interpretazione di un personaggio in un’ambientazione fantasy di qualsiasi tipo. Si tratta di un GDR cartaceo, in cui i personaggi prendono vita in una narrazione raccontata a voce dal Master e dai giocatori stessi.
\vspace{0.5cm}
\noindent
\begin{center}
\rule{\textwidth}{0.4pt} 
\end{center}
\vspace{0.5cm}

\section{Ruolo dei Giocatori e del Master}
I giocatori sono protagonisti attivi all’interno di una trama principale. Attraverso le loro decisioni e azioni, contribuiscono alla storia, descrivendo verbalmente ciò che i loro personaggi fanno, pensano e sentono. 

\begin{itemize}
    \item \textbf{Giocatori:} Interpretano i loro personaggi, prendendo decisioni e affrontando sfide che plasmano la storia. Ogni giocatore ha la libertà di creare e sviluppare il proprio personaggio, esplorandone desideri, paure e motivazioni.
    
    \item \textbf{Il Master:} È il narratore e l’arbitro del gioco. Crea l’ambientazione, interpreta i personaggi non giocanti (PNG) e guida la storia, adattandosi alle scelte dei giocatori. Inoltre, stabilisce il successo o il fallimento delle azioni dei personaggi, spesso utilizzando il tiro dei dadi per determinare l'esito.
\end{itemize}

\vspace{0.3cm}

\section{Un GDR Semplice e Narrativo}
\textbf{Zarafir} non è solo un gioco orientato alla narrazione e all’interpretazione, ma rappresenta il nostro sogno: un sistema di gioco semplice e fluido, dove la storia è guidata dalla fantasia del Master e dei giocatori. 

\begin{itemize}
    \item \textbf{Niente Classi o Livelli Predefiniti:} Ogni giocatore può creare il personaggio che desidera, senza essere vincolato a classi o archetipi fissi.
    
    \item \textbf{Interpretazione al Centro:} Le azioni dei personaggi sono descritte verbalmente dai giocatori, e il successo o il fallimento viene deciso dal Master o tramite il lancio dei dadi.
    
    \item \textbf{Realismo e Magia:} In Zarafir, la verosimiglianza si fonde con la magia del fantasy. I personaggi non sono eroi invincibili, ma possono fallire, soffrire e persino morire, rendendo l’esperienza più realistica e intensa.
\end{itemize}

\vspace{0.3cm}

\section{Un Esempio di Gioco}
Per comprendere meglio come funziona il sistema di Zarafir, ecco un esempio pratico:

\vspace{0.2cm}

\textbf{Esempio:} Un giocatore vuole utilizzare l'abilità \textit{“Convincere”} del proprio personaggio per motivare i compagni, scoraggiati dall’enorme mostro che stanno affrontando. Il giocatore è incoraggiato a descrivere come avviene l'azione:

\begin{itemize}
    \item Potrebbe suonare uno strumento, pronunciando parole epiche per infondere coraggio.
    \item Potrebbe raccontare un'antica leggenda per ricordare ai compagni il loro valore.
    \item Potrebbe esprimere una promessa di vittoria, adattando il suo discorso al carattere del personaggio.
\end{itemize}

Sarà il Master a valutare se l'azione è accettabile o se potrebbe compromettere la trama. Se necessario, il Master potrà chiedere un tiro dei dadi per determinare l'esito dell'azione. 

Zarafir vuole essere un’esperienza coinvolgente e narrativa, dove ogni scelta contribuisce a creare una storia unica e avvincente. La libertà è al centro del gioco:

\begin{itemize}
    \item I giocatori hanno il controllo totale sui loro personaggi, dalla creazione alla crescita.
    \item Il Master ha la flessibilità di adattare la storia alle scelte dei giocatori, creando un mondo vivo e dinamico.
    \item Non ci sono percorsi predefiniti: la storia si sviluppa attraverso l'interazione tra Master e giocatori.
\end{itemize}


\end{document}
