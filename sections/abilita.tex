\documentclass[../manuale_main.tex]{subfiles}



\begin{document}

Nel sistema di \textbf{Zarafir}, le \textbf{Abilità} rappresentano le capacità pratiche e le competenze dei personaggi, definendo cosa sanno fare e quanto sono abili nel farlo. Ogni abilità è associata a una delle Caratteristiche Principali e il livello di ciascuna abilità determina il grado di competenza del personaggio in quella specifica area.

\vspace{0.3cm}

\section*{Migliorare le Abilità}
I personaggi possono migliorare le loro abilità utilizzando i \textbf{Punti Esperienza (PE)}, guadagnati durante l’avventura. Il costo per migliorare un’abilità aumenta progressivamente:

\begin{itemize}
    \item \textbf{Dal livello 4 al 5:} Costa 5 PE.
    \item \textbf{Dal livello 5 al 7:} Costa 13 PE (6 + 7).
    \item \textbf{Apprendere una nuova abilità (0 a 1):} Costa 1 PE e richiede il superamento di un Tiro di Simulazione (TDS) a difficoltà 1.
\end{itemize}

Se il personaggio ha già utilizzato l’abilità durante l’avventura, il Master può scegliere di ignorare il TDS per apprenderla.

\vspace{0.3cm}

\section*{Categorie di Abilità}
Le abilità sono suddivise in diverse categorie per facilitare la comprensione e l’utilizzo durante il gioco. Ogni abilità è associata a una \textbf{Caratteristica Principale} che influenza il risultato dei TDS.

\vspace{0.5cm}
\rule{\textwidth}{0.4pt}
\vspace{0.5cm}

\section{Abilità da Combattimento}

\begin{center}
\textbf{\large{Armi a distanza}}\\ \textit{\textbf{Agilità}}\\
\end{center}
Permette al personaggio di utilizzare con destrezza e precisione armi da lancio, come archi, balestre, giavellotti o pugnali da lancio.

\begin{itemize}
\item \textbf{Esempi Narrativi:} Un ranger che scaglia una freccia verso un lupo in corsa o un ladro che lancia un pugnale per spegnere una candela.
\end{itemize}

\vspace{0.5cm}
\noindent
\begin{center}
\rule{\textwidth}{0.4pt} 
\end{center}
\vspace{0.5cm}

\begin{center}
\textbf{\large{Armi a due mani}}\\ \textit{\textbf{Forza}}\\
\end{center}
Rappresenta la capacità di combattere con armi pesanti e imponenti come spadoni, martelli a due mani o asce da battaglia.

\begin{itemize}
\item \textbf{Esempi Narrativi:} Un barbaro che abbatte una porta con il suo martello o un cavaliere che brandisce uno spadone contro un gruppo di nemici.
\end{itemize}

\vspace{0.5cm}
\noindent
\begin{center}
\rule{\textwidth}{0.4pt} 
\end{center}
\vspace{0.5cm}

\begin{center}
\textbf{\large{Armi a una mano}}\\ \textit{\textbf{Forza}}\\
\end{center}
Permette di utilizzare armi leggere e medie come spade, pugnali, mazze e asce a una mano.

\begin{itemize}
\item \textbf{Esempi Narrativi:} Un guerriero che combatte con una spada e uno scudo o un assassino che colpisce silenziosamente con un pugnale.
\end{itemize}

\vspace{0.5cm}
\noindent
\begin{center}
\rule{\textwidth}{0.4pt} 
\end{center}
\vspace{0.5cm}

\begin{center}
\textbf{\large{Mani Nude}}\\ \textit{\textbf{Forza}}\\
\end{center}
Rappresenta l’abilità di combattere senza armi, utilizzando pugni, calci e prese.

\begin{itemize}
\item \textbf{Esempi Narrativi:} Un monaco che disarma un avversario con una presa o un lottatore che abbatte un bandito con un pugno.
\end{itemize}

\vspace{0.5cm}
\noindent
\begin{center}
\rule{\textwidth}{0.4pt} 
\end{center}
\vspace{0.5cm}

\begin{center}
\textbf{\large{Parare}}\\ \textit{\textbf{Forza}}\\
\end{center}
Questa abilità permette di parare i colpi subiti utilizzando uno scudo.

\begin{itemize}
\item \textbf{Esempi Narrativi:} Un guerriero respinge un fendente riparandosi dietro al suo scudo a torre.
\end{itemize}

\vspace{0.5cm}
\noindent
\begin{center}
\rule{\textwidth}{0.4pt} 
\end{center}
\vspace{0.5cm}

\section{Abilità Magiche}

\begin{center}
\textbf{\large{Resistenza Magica}}\\ \textit{\textbf{Sensibilità}}\\
\end{center}
Permette al personaggio di resistere alle magie che lo bersagliano, riducendone l'efficacia.

\begin{itemize}
\item \textbf{Esempi Narrativi:} Un chierico che impedisce a un mago ostile di alterare le sue percezioni.
\end{itemize}

\vspace{0.5cm}
\noindent
\begin{center}
\rule{\textwidth}{0.4pt} 
\end{center}
\vspace{0.5cm}

\begin{center}
\textbf{\large{Stregoneria di Attacco}}\\ \textit{\textbf{Intelligenza}}\\
\end{center}
Consente di lanciare magie mirate a infliggere danni. L'incantatore può usare queste magie per colpire i propri bersagli o per incantare temporaneamente la propria arma.

\begin{itemize}
\item \textbf{Esempi Narrativi:} Un mago che scaglia magicamente un blocco di pietra su un gruppo di nemici.
\end{itemize}

\vspace{0.5cm}
\noindent
\begin{center}
\rule{\textwidth}{0.4pt} 
\end{center}
\vspace{0.5cm}

\begin{center}
\textbf{\large{Stregoneria di Controllo}}\\ \textit{\textbf{Intelligenza}}\\
\end{center}
Permette di manipolare l’ambiente, influenzare le azioni degli avversari o annullare magie.

\begin{itemize}
\item \textbf{Esempi Narrativi:} Un incantatore che blocca il movimento di un nemico o dissolve una magia nemica.
\end{itemize}

\vspace{0.5cm}
\noindent
\begin{center}
\rule{\textwidth}{0.4pt} 
\end{center}
\vspace{0.5cm}

\begin{center}
\textbf{\large{Stregoneria della Vita}}\\ \textit{\textbf{Intelligenza}}\\
\end{center}
Permette di guarire i bersagli o evocare creature magiche.

\begin{itemize}
\item \textbf{Esempi Narrativi:} Un chierico che cura le ferite di un compagno o un mago che evoca un lupo spettrale.
\end{itemize}

\vspace{0.5cm}
\noindent
\begin{center}
\rule{\textwidth}{0.4pt} 
\end{center}
\vspace{0.5cm}


\begin{center}
\textbf{\large{Tessimagie}}\\ \textit{\textbf{Intelligenza}}\\
\end{center}
Permette di usare gli incanti relativi all'textit{Arte della Sapienza, Arte della Fede, Arte della Morte}.

\begin{itemize}
\item \textbf{Esempi Narrativi:} Un mago lancia una palla di fuoco, oppure un incantatore rianima un compagno caduto grazie alla necromanzia.
\end{itemize}

\vspace{0.5cm}
\noindent
\begin{center}
\rule{\textwidth}{0.4pt} 
\end{center}
\vspace{0.5cm}


\section{Abilità di Sopravvivenza e Competenza}

\begin{center}
\textbf{\large{Acrobazia}}\\ \textit{\textbf{Reattività}}\\
\end{center}
Permette al personaggio di eseguire acrobazie, arrampicarsi, nuotare e compiere azioni atletiche.

\begin{itemize}
\item \textbf{Esempi Narrativi:} Un ladro che si arrampica agilmente su un muro o un monaco che schiva una trappola con una capriola.
\end{itemize}

\vspace{0.5cm}
\noindent
\begin{center}
\rule{\textwidth}{0.4pt} 
\end{center}
\vspace{0.5cm}


\begin{center}
\textbf{\large{Artigianato}}\\ \textit{\textbf{Intelligenza}}\\
\end{center}
Permette al personaggio di creare, riparare e migliorare oggetti, che siano armi, armature, gioielli o manufatti magici.

\begin{itemize}
\item \textbf{Esempi Narrativi:} Un fabbro che forgia una spada incantata o un gioielliere che incastona una gemma magica in un anello.
\end{itemize}


\vspace{0.5cm}
\noindent
\begin{center}
\rule{\textwidth}{0.4pt} 
\end{center}
\vspace{0.5cm}

\begin{center}
\textbf{\large{Comando}}\\ \textit{\textbf{Carisma}}\\
\end{center}
Rappresenta la capacità di guidare e ispirare alleati o subordinati, mantenere la disciplina e organizzare gruppi.

\begin{itemize}
\item \textbf{Esempi Narrativi:} Un capitano che guida i suoi uomini in una carica o un generale che organizza una difesa disperata contro un assedio.
\end{itemize}


\vspace{0.5cm}
\noindent
\begin{center}
\rule{\textwidth}{0.4pt} 
\end{center}
\vspace{0.5cm}

\begin{center}
\textbf{\large{Convincere}}\\ \textit{\textbf{Carisma}}\\
\end{center}
Permette al personaggio di persuadere, intimidire, sedurre o motivare altre persone.

\begin{itemize}
\item \textbf{Esempi Narrativi:} Un mercante che negozia un prezzo migliore o un soldato che calma una folla o negozia un accordo.
\end{itemize}


\vspace{0.5cm}
\noindent
\begin{center}
\rule{\textwidth}{0.4pt} 
\end{center}
\vspace{0.5cm}

\begin{center}
\textbf{\large{Domare Animali}}\\ \textit{\textbf{Sensibilità}}\\
\end{center}
Rappresenta la capacità di comprendere, addestrare e calmare animali selvatici o domestici.

\begin{itemize}
\item \textbf{Esempi Narrativi:} Un druido che calma un orso furioso o un avventuriero che addestra un falco a cacciare per lui.
\end{itemize}


\vspace{0.5cm}
\noindent
\begin{center}
\rule{\textwidth}{0.4pt} 
\end{center}
\vspace{0.5cm}

\begin{center}
\textbf{\large{Erboristeria}}\\ \textit{\textbf{Sensibilità}}\\
\end{center}
Rappresenta la capacità del personaggio di identificare, raccogliere e utilizzare erbe, piante e funghi con effetti curativi, velenosi o altre proprietà.

\begin{itemize}
\item \textbf{Esempi Narrativi:} Un erborista che riconosce un raro fiore curativo nascosto tra i rovi o un alchimista che distilla un potente veleno da un fungo mortale.
\end{itemize}

\vspace{0.5cm}
\noindent
\begin{center}
\rule{\textwidth}{0.4pt} 
\end{center}
\vspace{0.5cm}

\begin{center}
\textbf{\large{Forza di Volontà}}\\ \textit{\textbf{Sensibilità}}\\
\end{center}
Rappresenta la capacità del personaggio di resistere alla paura, al dolore mentale, al controllo magico o a suggestioni emotive. È usata ogni volta che il personaggio deve mantenere lucidità, determinazione o identità in condizioni estreme.

\begin{itemize}
\item \textbf{Esempi Narrativi:} Un condottiero che non cede alla disperazione di fronte a una disfatta annunciata; una prigioniera che resiste a settimane di tortura mentale senza tradire i propri compagni.
\end{itemize}


\vspace{0.5cm}
\noindent
\begin{center}
\rule{\textwidth}{0.4pt} 
\end{center}
\vspace{0.5cm}

\begin{center}
\textbf{\large{Furtività}}\\ \textit{\textbf{Reattività}}\\
\end{center}
Permette al personaggio di nascondersi, muoversi silenziosamente o rimanere inosservato.

\begin{itemize}
\item \textbf{Esempi Narrativi:} Un assassino che si nasconde nell'ombra prima di colpire o una spia che si traveste per infiltrarsi in una festa nobiliare.
\end{itemize}


\vspace{0.5cm}
\noindent
\begin{center}
\rule{\textwidth}{0.4pt} 
\end{center}
\vspace{0.5cm}

\begin{center}
\textbf{\large{Insegnamento}}\\ \textit{\textbf{Carisma}}\\
\end{center}
Rappresenta la capacità del personaggio di trasmettere il proprio sapere ad altri, che si tratti di abilità pratiche, teoria magica, strategie di combattimento o persino filosofia.

\begin{itemize}
\item \textbf{Esempi Narrativi:} Un capitano addestra un gruppo di contadini a formare una milizia o una maga esperta insegna ai suoi apprendisti come lanciare incantesimi di base.
\end{itemize}


\vspace{0.5cm}
\noindent
\begin{center}
\rule{\textwidth}{0.4pt} 
\end{center}
\vspace{0.5cm}

\begin{center}
\textbf{\large{Malaffare}}\\ \textit{\textbf{Reattività}}\\
\end{center}
Rappresenta l’abilità di compiere azioni disoneste come scassinare, borseggiare o barare al gioco.

\begin{itemize}
\item \textbf{Esempi Narrativi:} Un ladro che ruba la borsa di un mercante senza essere notato.
\end{itemize}


\vspace{0.5cm}
\noindent
\begin{center}
\rule{\textwidth}{0.4pt} 
\end{center}
\vspace{0.5cm}

\begin{center}
\textbf{\large{Medicina}}\\ \textit{\textbf{Sensibilità}}\\
\end{center}
Permette al personaggio di diagnosticare e curare ferite, malattie e avvelenamenti.

\begin{itemize}
\item \textbf{Esempi Narrativi:} Un guaritore che fascia una ferita con erbe medicinali o un medico da campo che rimuove una freccia senza causare danni aggiuntivi.
\end{itemize}


\vspace{0.5cm}
\noindent
\begin{center}
\rule{\textwidth}{0.4pt} 
\end{center}
\vspace{0.5cm}

\begin{center}
\textbf{\large{Navigazione}}\\ \textit{\textbf{Sensibilità}}\\
\end{center}
Permette al personaggio di governare navi e imbarcazioni, orientarsi in mare e gestire l’equipaggio.

\begin{itemize}
\item \textbf{Esempi Narrativi:} Un capitano che guida la sua nave attraverso una tempesta o un marinaio che riconosce una corrente favorevole.
\end{itemize}


\vspace{0.5cm}
\noindent
\begin{center}
\rule{\textwidth}{0.4pt} 
\end{center}
\vspace{0.5cm}

\begin{center}
\textbf{\large{Percezione}}\\ \textit{\textbf{Sensibilità}}\\
\end{center}
Rappresenta la capacità di notare dettagli, ascoltare conversazioni sussurrate, vedere in lontananza o fiutare odori sospetti.

\begin{itemize}
\item \textbf{Esempi Narrativi:} Un esploratore che riconosce impronte fresche nel fango o una guardia che nota un'ombra sospetta.
\end{itemize}


\vspace{0.5cm}
\noindent
\begin{center}
\rule{\textwidth}{0.4pt} 
\end{center}
\vspace{0.5cm}

\begin{center}
\textbf{\large{Schivare}}\\ \textit{\textbf{Reattività}}\\
\end{center}
Permette al personaggio di evitare attacchi non diretti, come trappole, oggetti cadenti o frecce vaganti.

\begin{itemize}
\item \textbf{Esempi Narrativi:} Un avventuriero che si lancia a terra per evitare una trappola a scatto o che schiva un colpo improvviso.
\end{itemize}


\vspace{0.5cm}
\noindent
\begin{center}
\rule{\textwidth}{0.4pt} 
\end{center}
\vspace{0.5cm}

\begin{center}
\textbf{\large{Seguire Tracce}}\\ \textit{\textbf{Sensibilità}}\\
\end{center}
Rappresenta la capacità di individuare, seguire e interpretare tracce lasciate da persone, animali o creature.

\begin{itemize}
\item \textbf{Esempi Narrativi:} Un ranger che segue le impronte di un branco di lupi o un cacciatore che identifica le tracce di un'orsa con i suoi cuccioli.
\end{itemize}

\vspace{0.5cm}
\noindent
\begin{center}
\rule{\textwidth}{0.4pt} 
\end{center}
\vspace{0.5cm}

\begin{center}
\textbf{\large{Utilizzare Trappole}}\\ \textit{\textbf{Reattività}}\\
\end{center}
Permette al personaggio di costruire, piazzare, disinnescare o individuare trappole. Questa abilità è essenziale per avventurieri che devono esplorare luoghi pericolosi, ladri che vogliono proteggere i loro nascondigli o cacciatori che desiderano catturare prede.

\begin{itemize}
\item \textbf{Esempi Narrativi:} Un cacciatore che posiziona una trappola per catturare selvaggina, un ladro che disinnesca una trappola su un forziere, o un esploratore che scopre una serie di trappole nascoste in un antico tempio.
\end{itemize}


\vspace{0.5cm}
\noindent
\begin{center}
\rule{\textwidth}{0.4pt} 
\end{center}
\vspace{0.5cm}


 %termine della lista della miscellanea

\end{document}