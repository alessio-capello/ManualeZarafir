\documentclass[../manuale_main.tex]{subfiles}

\begin{document}

In Zarafir, non tutti i compagni dei personaggi sono alleati umani o senzienti. Alcuni potrebbero essere creature selvagge domate con pazienza, altri potrebbero essere spiriti evocati da potenti incantesimi. Questa sezione esplora le regole e le meccaniche per interagire, addestrare e comandare queste creature.

\subsection*{Addestrare una Creatura}

Interagire con una creatura non senziente, come un animale, richiede pazienza, empatia e abilità. Un personaggio può tentare di addestrare una creatura per renderla un alleato fedele, in grado di comprendere e seguire comandi.

\subsubsection{Procedura per Addestrare una Creatura}
\begin{enumerate}
    \item \textbf{Individuare la Creatura:} Il personaggio deve trovare una creatura che desidera addestrare. Questa può essere incontrata casualmente durante l'avventura o cercata intenzionalmente.
    
    \item \textbf{Mostrare Pazienza e Fiducia:} Il personaggio deve interagire pacificamente con la creatura, utilizzando l'abilità \emph{``Domare animali''}. Il Master stabilisce la difficoltà del TDS in base alla natura della creatura:
    \begin{itemize}
        \item \textbf{Creatura Docile (gatto, cane, cavallo):} Difficoltà 2-3.
        \item \textbf{Creatura Selvatica (lupo, falco, cervo):} Difficoltà 4-6.
        \item \textbf{Creatura Aggressiva (orso, leone, aquila gigante):} Difficoltà 7-9.
        \item \textbf{Creatura Mostruosa (drago giovane, grifone, basilisco):} Difficoltà 10 o superiore.
    \end{itemize}
    
    \item \textbf{TDS su ``Domare Animali'':} Il personaggio tira un TDS utilizzando l'abilità \emph{``Domare animali''}. In caso di successo, la creatura comincia a fidarsi e segue il personaggio senza ostilità. In caso di fallimento, la creatura potrebbe fuggire, rifiutarsi di obbedire o persino attaccare.
    
    \item \textbf{Addestramento Prolungato:} Ottenuta la fiducia della creatura, il personaggio può spendere del tempo per addestrarla a eseguire specifici comandi (attaccare, difendere, cercare oggetti, ecc.). Ogni comando richiede un TDS aggiuntivo su \emph{``Domare animali''} con una difficoltà determinata dal Master:
    \begin{itemize}
        \item \textbf{Comandi Semplici (seguire, fermarsi, sedersi):} Difficoltà 3.
        \item \textbf{Comandi Moderati (attaccare, proteggere, cercare oggetti):} Difficoltà 5.
        \item \textbf{Comandi Complessi (combattere in gruppo, esplorare aree pericolose):} Difficoltà 7 o superiore.
    \end{itemize}
    
    \item \textbf{Durata dell'Addestramento:} Il tempo necessario per completare l'addestramento dipende dalla natura della creatura e dalla complessità dei comandi. Il Master può stabilire che siano necessari giorni, settimane o persino mesi per ottenere risultati ottimali.
\end{enumerate}

\subsection*{Creature Evocate}

Le creature evocate sono entità magiche o spirituali richiamate tramite incantesimi o stregonerie. A differenza delle creature addestrate, le creature evocate sono automaticamente fedeli a chi le ha richiamate e obbediscono ai suoi comandi senza esitazione.

\subsubsection{Caratteristiche delle Creature Evocate}
\begin{itemize}
    \item \textbf{Fedeltà Assoluta:} Le creature evocate sono completamente obbedienti all'evocatore. Non è necessario addestrarle o guadagnare la loro fiducia.
    \item \textbf{Durata Limitata:} La maggior parte delle creature evocate esiste solo per un periodo limitato, indicato nella descrizione dell'incantesimo o della stregoneria utilizzata per richiamarle.
    \item \textbf{Vincolo Spirituale:} Se l'evocatore muore o perde conoscenza, la creatura evocata scompare o diventa ostile, a discrezione del Master.
\end{itemize}

\subsection*{Impartire Comandi alle Creature}

Che siano state addestrate con pazienza o evocate tramite la magia, le creature possono eseguire comandi. Tuttavia, la natura della creatura influenza la facilità con cui i comandi vengono eseguiti.

\subsubsection{Creature Addestrate}
\begin{itemize}
    \item \textbf{Comandi Ordinari:} Seguire, restare, attaccare creature ostili. Questi comandi possono essere impartiti senza ulteriori TDS.
    \item \textbf{Comandi Complessi:} Comandi che richiedono precisione, come inseguire un nemico specifico, cercare oggetti nascosti o scappare in modo strategico. Il Master può richiedere un TDS su \emph{``Domare animali''} per far eseguire questi comandi.
    \item \textbf{Limiti Naturali:} Una creatura addestrata mantiene comunque i propri istinti naturali. Un lupo potrebbe avere difficoltà a ignorare la fame, mentre un cavallo potrebbe spaventarsi di fronte al fuoco.
\end{itemize}

\subsubsection{Creature Evocate}
\begin{itemize}
    \item \textbf{Obbedienza Totale:} Le creature evocate obbediscono a qualsiasi comando impartito dall'evocatore, senza bisogno di TDS.
    \item \textbf{Durata Limitata:} Le creature evocate scompaiono automaticamente quando la durata della loro evocazione termina.
    \item \textbf{Immunità ai Timori e alla Fame:} Le creature evocate non sono influenzate dalle emozioni o dai bisogni fisici, a meno che l'incantesimo non preveda esplicitamente queste caratteristiche.
\end{itemize}

\subsection*{Esempi di Creature Addestrabili}

\begin{itemize}
    \item \textbf{Lupo Selvatico:} Una creatura aggressiva che richiede un'abilità \emph{``Domare animali''} elevata. Può essere addestrata a proteggere il padrone o cacciare per lui.
    \item \textbf{Aquila Reale:} Un rapace fiero e intelligente. Può essere addestrato a fungere da esploratore o a intercettare messaggi.
    \item \textbf{Orso delle Montagne:} Una creatura potente ma difficilmente addomesticabile. Può essere utilizzato come guardia personale.
    \item \textbf{Drago Giovane (Evocato):} Una creatura magica richiamata tramite un potente incantesimo. Obbedisce senza esitazione ma esiste solo per un periodo limitato.
\end{itemize}

\subsection*{Le Limitazioni dei Compagni}
Le creature, addestrate o evocate, hanno dei limiti. Non possono comprendere concetti complessi, né agire autonomamente in modo intelligente. Anche i famigli evocati sono limitati dalla loro natura e dal potere dell'incantatore. Il Master ha il compito di interpretare il comportamento delle creature in base alle loro capacità e alla situazione.

\end{document}
