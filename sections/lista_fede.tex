\documentclass[../manuale_main.tex]{subfiles}

\begin{document}
\section{Incantesimi - Arte della Fede}

Gli incantesimi dell'Arte della Fede sono manifestazioni della volontà e della dedizione spirituale dell'incantatore. Incarnano ideali di sacrificio, giustizia, protezione e punizione, permettendo ai devoti di incanalare la propria convinzione in poteri che proteggono gli alleati o annientano gli empi.

 \subsection*{Primo Cerchio - Arte della Fede}

\begin{itemize}


\item \textbf{Benedizione} \\
\textbf{Difficoltà:} 3 \\
\textbf{Durata:} 2 round \\
\textbf{Gittata:} Sé stesso \\
\textbf{Descrizione:} L’incantatore invoca la benedizione della sua fede, ottenendo un bonus di +1 a \textbf{Forza} e una riduzione di -1 ai danni subiti da ogni attacco. Può essere lanciato anche nello stesso turno in cui si attacca con un'arma.

\vspace{0.5cm}\rule{\textwidth}{0.4pt}\vspace{1cm}

\item \textbf{Favore degli Dei} \\
\textbf{Difficoltà:} 2 \\
\textbf{Durata:} Istantaneo \\
\textbf{Gittata:} Contatto \\
\textbf{Descrizione:} Una preghiera sincera dona sollievo a un alleato con cui l’incantatore condivide uno scopo o una fede comune, guarendolo di \textbf{1D6 punti vita}. Il master può richiedere un momento narrativo per attivare l’effetto.

\vspace{0.5cm}\rule{\textwidth}{0.4pt}\vspace{1cm}

\clearpage
\item \textbf{Punizione} \\
\textbf{Difficoltà:} 3 \\
\textbf{Durata:} Istantaneo \\
\textbf{Gittata:} 7 metri \\
\textbf{Descrizione:} L’incantatore condanna un avversario che ha compiuto un atto empio, infliggendo \textbf{1D6 danni}. Il bersaglio deve aver compiuto recentemente un’azione crudele, sacrilega o in contrasto con i voti del personaggio.

\vspace{0.5cm}\rule{\textwidth}{0.4pt}\vspace{1cm}

\item \textbf{Mano Misericordiosa} \\
\textbf{Difficoltà:} 2 \\
\textbf{Durata:} 1 minuto \\
\textbf{Gittata:} Contatto \\
\textbf{Descrizione:} Le mani dell’incantatore si illuminano di una luce calda. Per la durata, può assistere nella cura di ferite lievi o confortare un morente, permettendo test di \textit{Medicina} con -2 al risultato.


\vspace{0.5cm}\rule{\textwidth}{0.4pt}\vspace{1cm}

\end{itemize}


\clearpage
 \subsection*{Secondo Cerchio - Arte della Fede}

\begin{itemize}

\item \textbf{Luce Accecante} \\
\textbf{Difficoltà:} 5 \\
\textbf{Durata:} Istantaneo \\
\textbf{Gittata:} Sé stesso \\
\textbf{Descrizione:} Un bagliore abbagliante esplode attorno all’incantatore. Ogni nemico entro 2 metri con \textbf{Reattività} pari o inferiore a 9 tira un dado: con 5 o più è accecato e salta il turno successivo.

\vspace{0.5cm}\rule{\textwidth}{0.4pt}\vspace{1cm}

\item \textbf{Purificazione} \\
\textbf{Difficoltà:} 3 \\
\textbf{Durata:} Istantaneo \\
\textbf{Gittata:} Contatto \\
\textbf{Descrizione:} L’incantatore estrae il veleno dal corpo di un alleato, a costo di subire parte della corruzione: riceve \textbf{1D3 danni} se il veleno è debole, \textbf{1D6} se è letale.

\vspace{0.5cm}\rule{\textwidth}{0.4pt}\vspace{1cm}

\item \textbf{Occhi della Fede} \\
\textbf{Difficoltà:} 3 \\
\textbf{Durata:} 10 minuti \\
\textbf{Gittata:} Sé stesso \\
\textbf{Descrizione:} L’incantatore ottiene un’aura percettiva che lo aiuta a distinguere l’illusione dalla verità. Ottiene \textbf{-2 alla difficoltà} per i TDS contro illusioni e magie mentali, e può percepire presenze empie o sacrileghe a discrezione del master.


\end{itemize}

\clearpage
 \subsection*{Terzo Cerchio - Arte della Fede}

\begin{itemize}

\item \textbf{Arma Consacrata} \\
\textbf{Difficoltà:} 5 \\
\textbf{Durata:} 2 round \\
\textbf{Gittata:} Sé stesso \\
\textbf{Descrizione:} L’arma dell’incantatore si carica di energia sacra. Per la durata, ogni attacco ottiene \textbf{+1 ai danni} e \textbf{-1 alla difficoltà per colpire}. Può essere lanciato anche durante un turno in cui si attacca.

\vspace{0.5cm}\rule{\textwidth}{0.4pt}\vspace{1cm}

\item \textbf{Chiamata alla Gloria} \\
\textbf{Difficoltà:} 4 \\
\textbf{Durata:} 2 round \\
\textbf{Gittata:} 3 metri \\
\textbf{Descrizione:} L’incantatore solleva il morale degli alleati entro 3 metri, infondendo coraggio sacro. Per la durata, ignorano ogni effetto di paura o terrore.

\vspace{0.5cm}\rule{\textwidth}{0.4pt}\vspace{1cm}

\item \textbf{Guarigione} \\
\textbf{Difficoltà:} 6 \\
\textbf{Durata:} Istantaneo \\
\textbf{Gittata:} Contatto \\
\textbf{Descrizione:} L’incantatore ripristina \textbf{2D6 punti vita} a un bersaglio. Aumentando la difficoltà di +3, può guarire anche cicatrici profonde, fratture o mutilazioni recenti. Richiede concentrazione totale.

\vspace{0.5cm}\rule{\textwidth}{0.4pt}\vspace{1cm}


\end{itemize}

\clearpage
 \subsection*{Quarto Cerchio - Arte della Fede}

\begin{itemize}

\item \textbf{Furia Divina} \\
\textbf{Difficoltà:} 8 \\
\textbf{Durata:} 2 round \\
\textbf{Gittata:} Sé stesso \\
\textbf{Descrizione:} L’incantatore si fa canale del proprio credo, ottenendo \textbf{+2 a tutte le caratteristiche} e \textbf{-1 danni subiti}. Inoltre, tutti i nemici entro 15 metri sono costretti a prenderlo come bersaglio, se in grado.

\vspace{0.5cm}\rule{\textwidth}{0.4pt}\vspace{1cm}

\item \textbf{Luce Sacra} \\
\textbf{Difficoltà:} 6 \\
\textbf{Durata:} Istantaneo \\
\textbf{Gittata:} 7 metri \\
\textbf{Descrizione:} Una luce implacabile travolge i nemici nelle vicinanze, infliggendo \textbf{1D10 danni}.

\vspace{0.5cm}\rule{\textwidth}{0.4pt}\vspace{1cm}

\item \textbf{Sacrificio} \\
\textbf{Difficoltà:} 5 \\
\textbf{Durata:} 2 round \\
\textbf{Gittata:} 5 metri \\
\textbf{Descrizione:} Per la durata dell'incanto, l’incantatore devia su di sè tutti i danni subiti da un personaggio con cui condivide uno scopo o una fede comune.

\vspace{0.5cm}\rule{\textwidth}{0.4pt}\vspace{1cm}

\item \textbf{Esilio del Corrotto} \\
\textbf{Difficoltà:} 7 \\
\textbf{Durata:} Istantaneo \\
\textbf{Gittata:} 10 metri \\
\textbf{Descrizione:} Un bagliore di fede travolge una creatura che ha compiuto atti sacrileghi o crudeli. Se non riesce a superare un TDS su \textit{Resistenza magica}, viene spinta via di 10 metri e salta il suo prossimo turno.

\vspace{0.5cm}\rule{\textwidth}{0.4pt}\vspace{1cm}

\end{itemize}

 \subsection*{Quinto Cerchio - Arte della Fede}

\begin{itemize}

\item \textbf{Assalto Implacabile} \\
\textbf{Difficoltà:} 9 \\
\textbf{Durata:} 1 round (estendibile) \\
\textbf{Gittata:} Sé stesso \\
\textbf{Descrizione:} Per un breve istante, l’incantatore diventa l’incarnazione vivente della sua causa. È immune a terrore, stanchezza e dolore. Le difficoltà per colpire con un’arma sono dimezzate e infligge \textbf{+3 danni}. Può mantenere l'effetto spendendo \textbf{4 punti vita} per ogni round aggiuntivo. Se i suoi PV raggiungono 0, l’effetto termina all’istante. Può essere lanciato anche nello stesso turno in cui si attacca con un'arma.

\vspace{0.5cm}\rule{\textwidth}{0.4pt}\vspace{1cm}
\clearpage

\item \textbf{Grazia Divina} \\
\textbf{Difficoltà:} 8 \\
\textbf{Durata:} Istantaneo \\
\textbf{Gittata:} Contatto \\
\textbf{Descrizione:} L'incantatore riporta in vita una creatura defunta, chiamandone l’anima dalla soglia dell’aldilà. Il corpo deve essere integro, e l’incanto richiede \textbf{una concentrazione totale}.\\

Questo potere richiede un prezzo elevato. L'incantatore subisce \textbf{-4 alla propria soglia massima di Stress Mentale}, mentre il resuscitato perde \textbf{-5 alla propria soglia di Stress Mentale}. Queste ferite dell’anima non possono essere curate normalmente.\\

L’esperienza della morte, unita al legame spirituale con il taumaturgo, può alterare profondamente la percezione del mondo del bersaglio. A discrezione del master, la \textbf{bussola morale del personaggio può venire modificata}, influenzata dalla consapevolezza di ciò che ha vissuto o da ciò che ha intravisto oltre la vita.
Questo incanto può essere usato su \textbf{personaggi giocanti}. In tal caso, il giocatore mantiene il controllo del personaggio, ma dovrà rivedere alcuni aspetti psicologici o motivazionali, anche in collaborazione con il master.

\vspace{0.5cm}\rule{\textwidth}{0.4pt}\vspace{1cm}

\item \textbf{Eco del Giudizio} \\
\textbf{Difficoltà:} 10 \\
\textbf{Durata:} Istantaneo \\
\textbf{Gittata:} 20 metri \\
\textbf{Descrizione:} L’incantatore evoca un’onda spirituale che colpisce tutti i nemici che hanno compiuto atti crudeli, sacrileghi o sleali nelle ultime 24 ore. Ogni bersaglio così identificato subisce \textbf{3D6 danni}. Chi resiste con successo alla magia subisce comunque \textbf{1D6 danni}, riducendo la soglia di Stress Mentale di 1 fino al riposo.







\end{itemize}

\clearpage
 \subsection*{Sesto Cerchio - Arte della Fede}

Le magie del Sesto Cerchio sono estremamente potenti e richiedono uno sforzo mentale enorme per essere lanciate. Per questa ragione, un personaggio non può lanciare più di una magia di Sesto Cerchio al giorno.

\begin{itemize}

\item \textbf{Fulgore Divino} \\
\textbf{Difficoltà:} 13 \\
\textbf{Durata:} Istantaneo \\
\textbf{Gittata:} 20 metri \\
\textbf{Descrizione:} Una colonna di luce divorante si abbatte su un bersaglio. Infligge \textbf{3D6 danni}. Se il bersaglio ha compiuto un atto empio o disonorevole secondo il credo dell’incantatore, i danni aumentano a \textbf{5D6}.

\vspace{0.5cm}\rule{\textwidth}{0.4pt}\vspace{1cm}

\item \textbf{Martire Sacro} \\
\textbf{Difficoltà:} 11 \\
\textbf{Durata:} Istantaneo (1 round di effetto) \\
\textbf{Gittata:} 10 metri \\
\textbf{Descrizione:} L’incantatore assorbe tutti i danni che sarebbero inflitti agli alleati entro gittata per un intero round. Il danno è dimezzato e applicato direttamente a lui. Se l'incantatore muore durante l’effetto, ogni alleato guarisce \textbf{2D6 punti vita} e riduce i loro Stress Mentale a 0. Può essere lanciato anche nello stesso turno in cui si attacca con un'arma.

\vspace{0.5cm}\rule{\textwidth}{0.4pt}\vspace{1cm}
\clearpage
\item \textbf{Aura di Giustizia} \\
\textbf{Difficoltà:} 11 \\
\textbf{Durata:} 2 round \\
\textbf{Gittata:} 15 metri \\
\textbf{Descrizione:} Tutti gli alleati entro gittata infliggono \textbf{+2 danni} e subiscono \textbf{-2 danni} da ogni fonte. Inoltre, ogni attacco dell’incantatore infligge \textbf{+2 danni} addizionali. Chi ha recentemente compiuto un atto crudele o codardo non può beneficiare dell’effetto.

\vspace{0.5cm}\rule{\textwidth}{0.4pt}\vspace{1cm}

\item \textbf{Redivivo} \\
\textbf{Difficoltà:} 12 \\
\textbf{Durata:} 3 round \\
\textbf{Gittata:} Sé stesso \\
\textbf{Descrizione:} Per la durata, l’incantatore non può morire. Anche se i suoi punti vita scendono a zero o subisce mutilazioni, può agire normalmente. Alla fine dell’effetto, subisce tutti i danni accumulati in un colpo solo. Se è ancora vivo, rimane cosciente; altrimenti, cade privo di sensi.

\vspace{0.5cm}\rule{\textwidth}{0.4pt}\vspace{1cm}

\end{itemize}

\end{document}
