\documentclass[../manuale_main.tex]{subfiles}



\begin{document}
\section{Punti Vita, Stress Mentale e Punti Esperienza}

In \textbf{Zarafir}, la sopravvivenza e la crescita del personaggio sono gestite attraverso tre valori fondamentali: \textbf{Punti Vita}, \textbf{Stress Mentale} e \textbf{Punti Esperienza}. Questi rappresentano non solo la salute fisica e mentale del personaggio, ma anche il suo percorso di sviluppo e maturazione durante l'avventura.

\vspace{0.3cm}

\subsection{Punti Vita}

I \textbf{Punti Vita (PV)} rappresentano la salute fisica del personaggio. Sono la misura della sua capacità di resistere ai danni e alle ferite che potrebbe subire durante le avventure.

\subsubsection{Calcolo dei Punti Vita Massimi}
Il numero massimo di Punti Vita di un personaggio è determinato dalla sua \textbf{Costituzione}, secondo la seguente formula:

\[
\textbf{Punti Vita Massimi = Costituzione \times 1.5 + Modificatori}
\]

\vspace{0.3cm}

\subsubsection{Ferite e Conseguenze}
Durante le avventure, il personaggio può subire danni da varie fonti: attacchi fisici, magie, trappole o eventi imprevisti.

\begin{itemize}
    \item \textbf{Sopra lo Zero:} Finché i Punti Vita sono superiori a zero, il personaggio è ancora in piedi, anche se può essere ferito.
    
    \item \textbf{Ferite Gravi:} Alcuni attacchi possono infliggere ferite gravi, come fratture, lacerazioni o ustioni. Queste ferite possono ridurre temporaneamente le capacità del personaggio.
    
    \item \textbf{A Zero Punti Vita:} Il personaggio sviene. È privo di sensi e gravemente ferito. Senza cure, la morte è inevitabile.
    
    \item \textbf{Sotto Zero:} Il personaggio è in fin di vita. Salvarlo sarà molto difficile, e potrebbe richiedere cure immediate.
\end{itemize}

\vspace{0.3cm}

\subsubsection{Recuperare Punti Vita}
I Punti Vita possono essere recuperati in vari modi:
\begin{itemize}
    \item \textbf{Pozioni Curative:} Erbe o elisir magici possono ripristinare una certa quantità di Punti Vita.
    \item \textbf{Magie di Guarigione:} Incantesimi lanciati da un alleato possono guarire ferite istantaneamente.
    \item \textbf{Riposo e Recupero:} Il Master può consentire il recupero di PV durante un periodo di riposo, in base alle condizioni e alla durata del riposo.
    \item \textbf{Medicina:} Un personaggio con l'abilità \textit{Medicina} può curare i compagni, stabilizzare i feriti e trattare ferite gravi.
\end{itemize}

\vspace{0.5cm}
\rule{\textwidth}{0.4pt}
\vspace{0.5cm}

\subsection{Stress Mentale}

Lo \textbf{Stress Mentale} rappresenta la pressione psicologica e la fatica mentale accumulata dai personaggi durante le loro avventure. Non si limita solo agli incantatori, ma può colpire chiunque affronti situazioni estenuanti, pericolose o traumatiche.  

\subsubsection{Accumulo di Stress Mentale}
Lo Stress Mentale può essere accumulato in diversi modi:
\begin{itemize}
    \item \textbf{Utilizzo della Magia:} Ogni volta che un personaggio lancia una magia o utilizza una stregoneria, accumula \textbf{Punti Stress} pari alla difficoltà di lancio della magia. 
    
    \item \textbf{Situazioni di Pericolo o Orrore:} Affrontare creature mostruose, trovarsi in situazioni di terrore, vivere un tradimento o assistere alla morte di un compagno possono causare accumulo di Stress Mentale.
    
    \item \textbf{Fatica Psichica Prolungata:} Privazione del sonno, discussioni intense o sforzi mentali prolungati (come risolvere enigmi complessi) possono causare Stress Mentale.
\end{itemize}

\subsubsection{Soglia di Stress Mentale}
Ogni personaggio ha una soglia massima di Stress Mentale, determinata dalla somma delle sue caratteristiche di:

\[
\textbf{Soglia di Stress Mentale = Sensibilità + Intelligenza}
\]

\subsubsection{Effetti dello Stress Mentale}
Gli effetti dello Stress Mentale possono variare a seconda dell'intensità:

\begin{itemize}
\item \textbf{Lieve Stress:} Distrazione, mal di testa, lievi tremori. Leggera irritabilità.
\item \textbf{Moderato Stress:} Difficoltà di concentrazione, pensieri confusi, ansia crescente, percezione alterata della realtà.
\item \textbf{Grave Stress:} Paranoia, visioni fugaci, tremori intensi, difficoltà a parlare in modo coerente.
\item \textbf{Stress Critico:} Svenimento improvviso, crisi di panico, perdita temporanea della memoria, allucinazioni persistenti.
\item \textbf{Stress Estremo (a discrezione del Master):} Dissociazione, amnesia totale, incubi persistenti, incapacità di distinguere amici da nemici.
\end{itemize}

\subsubsection{Recuperare Stress Mentale}
Lo Stress Mentale può essere alleviato in vari modi:
\begin{itemize}
    \item \textbf{Riposando:} Un periodo di riposo tranquillo riduce una parte dello Stress accumulato.
    \item \textbf{Pozioni e Infusi Calmanti:} Erbe o pozioni speciali possono ridurre lo Stress Mentale.
    \item \textbf{Supporto Sociale:} Parlare con un alleato fidato, ricevere conforto o partecipare a un rito di purificazione.
    \item \textbf{Evasione Creativa:} Attività rilassanti come dipingere, suonare uno strumento o raccontare storie possono ridurre lo Stress Mentale.
\end{itemize}

\vspace{0.5cm}
\rule{\textwidth}{0.4pt}
\vspace{0.5cm}

\subsection{Punti Esperienza}
I \textbf{Punti Esperienza (PE)} rappresentano la crescita del personaggio, la sua maturazione e le lezioni apprese nel corso delle avventure.

\subsubsection{Guadagnare Punti Esperienza}
\begin{itemize}
    \item \textbf{Momenti Significativi:} Superare sfide importanti, fare scelte morali difficili, vivere esperienze intense.
    \item \textbf{Interpretazione del Personaggio:} Giocare il proprio ruolo in modo coerente, sviluppare il background, interagire con altri personaggi.
    \item \textbf{Scoperte e Segreti:} Rivelare misteri, esplorare luoghi nascosti, apprendere nuove abilità.
\end{itemize}

\subsubsection{Utilizzo dei Punti Esperienza}
\begin{itemize}
    \item \textbf{Migliorare Abilità:} Spendere PE per aumentare il livello di un'abilità.
    \item \textbf{Sbloccare Talenti:} Ottenere talenti speciali legati al background del personaggio.
    \item \textbf{Evoluzione del Background:} Approfondire la storia del personaggio, rivelare segreti o ottenere vantaggi narrativi.
\end{itemize}



\end{document}