\documentclass[../manuale_main.tex]{subfiles}

\begin{document}


Nel corso delle loro avventure, i personaggi possono trovarsi in una varietà di condizioni che influenzano le loro capacità fisiche, mentali o emotive. Questa sezione offre una guida per gestire queste condizioni, permettendo al Master di utilizzarle come spunto narrativo e meccanico. Le condizioni qui descritte non sono una lista esaustiva, ma un insieme di esempi per ispirare la gestione degli effetti che possono influenzare i personaggi.

\section{Paura e Terrore}

La paura è un istinto primordiale, una voce nella mente che urla di sopravvivere a ogni costo. Non è un semplice malus numerico, ma una condizione narrativa che colora le decisioni, irrigidisce i muscoli, annebbia la lucidità.

\textbf{Un personaggio impaurito} non è semplicemente meno efficace: esita, osserva i compagni per cercare sicurezza, ha il respiro affannoso e le mani tremanti. Se deve agire contro la fonte della paura (attaccarla, avvicinarsi, affrontarla verbalmente), il Master può chiedere un \textbf{Tiro di Stabilità Mentale} su \textbf{Forza di Volontà} (basata su Sensibilità). La difficoltà dipende dalla natura e dall’intensità della minaccia:

\begin{itemize}
\item la vista di un mostro orrendo potrebbe provocare un’esitazione;
\item l’improvvisa comparsa di uno spettro nel buio, un autentico terrore.
\end{itemize}

\begin{itemize}
\item \textbf{Successo:} il personaggio riesce a gestire la propria paura, ma potrebbe comunque mostrarla: un colpo esitante, uno sguardo distolto, parole più dure del necessario.
\item \textbf{Fallimento:} l’azione riesce, ma con un \textbf{+2 alla difficoltà}, oppure può essere sostituita da un comportamento istintivo (indietreggiare, proteggere un alleato, restare immobile).
\end{itemize}

\textbf{Il Terrore} è un'esperienza diversa. Non ci si limita ad avere paura: ci si convince che non ci sia via d’uscita. L’istinto prende il sopravvento, cancellando ogni razionalità.

All’inizio del turno, un personaggio in preda al terrore deve effettuare un nuovo \textbf{Tiro su Forza di Volontà} (Sensibilità).

\begin{itemize}
\item \textbf{In caso di successo:} il personaggio riesce a controllarsi, ma è ancora scosso: ogni azione avrà un \textbf{+3 alla difficoltà}, e potrebbe comunque mostrare segni di panico.
\item \textbf{In caso di fallimento:} il Master può scegliere se il personaggio fugge in una direzione sicura o resta paralizzato, completamente in balia degli eventi.
\end{itemize}

Le creature capaci di generare terrore (come i demoni, gli spiriti o alcuni rituali) possono usare questa condizione come arma. Tuttavia, anche un evento narrativo particolarmente toccante, come vedere morire un alleato in modo brutale, può scatenarla.

\textbf{Nota:} la paura è anche un’occasione narrativa. Può mostrare vulnerabilità, generare coesione, rivelare traumi. Incoraggia i giocatori a interpretarla, invece che a evitarla.

\vspace{0.5cm}
\noindent
\begin{center}
\rule{\textwidth}{0.4pt} 
\end{center}
\vspace{0.5cm}

\section{Stress Mentale e Crisi Psicologica}

Ogni avventura mette a dura prova non solo il corpo, ma anche la mente. La vista dell’innaturale, il senso di colpa, l’uso ripetuto di poteri oscuri... tutto lascia un segno invisibile. Questo segno è rappresentato dallo \textbf{Stress Mentale}.

\textbf{Lo Stress Mentale} cresce ogni volta che un personaggio:
\begin{itemize}
\item lancia una magia o una stregoneria complessa,
\item è testimone di eventi traumatici (es. un rituale di sangue, la morte di un alleato),
\item viene colpito da effetti mentali (illusioni invasive, visioni, maledizioni).
\end{itemize}

Quando lo Stress Mentale supera la metà della propria soglia massima (pari a \textbf{Intelligenza + Sensibilità}), il personaggio inizia a manifestare segni di \textbf{Affaticamento Mentale}.

\begin{itemize}
\item \textbf{Sintomi lievi:} difficoltà a concentrarsi, memoria fragile, mani tremanti, sonno disturbato.
\item \textbf{Effetti suggeriti:} \textbf{+1 alla difficoltà} di tutte le prove legate a Intelligenza o Sensibilità, ma il Master può anche chiedere al giocatore di interpretare comportamenti paranoici o evasivi.
\end{itemize}

Se lo Stress Mentale raggiunge o supera la soglia, il personaggio subisce una \textbf{Crisi Mentale}.

\begin{itemize}
\item \textbf{Effetti lievi:} smarrimento momentaneo, confusione, tic nervosi.
\item \textbf{Effetti medi:} allucinazioni, senso di persecuzione, incapacità a parlare lucidamente.
\item \textbf{Effetti gravi:} episodi dissociativi, attacchi di panico, confondere compagni e nemici.
\item \textbf{Effetti estremi:} blackout temporanei, rifiuto della realtà, perdita del linguaggio.
\end{itemize}

\textbf{Il Master} è libero di determinare l’intensità della crisi in base al tono della scena. Il giocatore può proporre come interpretare la crisi, magari con flashback, manifestazioni di traumi o reazioni impulsive.

\textbf{Recupero:} Lo Stress Mentale si riduce con il tempo, il riposo e la cura. Alcuni luoghi sacri, rituali o momenti di catarsi (narrazioni condivise, riti funebri, confessioni) possono offrire un recupero immediato, parziale o completo.

\textbf{Nota:} In campagne più psicologiche o investigative, il Master può mantenere traccia anche delle crisi passate, creando un “retaggio mentale” che arricchisce la caratterizzazione del personaggio.
\clearpage
\section{Svenimento e Addormentamento}

\subsection*{Svenimento}
Lo svenimento è la resa del corpo al trauma: dolore, fatica, paura o colpi particolarmente violenti possono causare questa condizione. Un personaggio svenuto è privo di coscienza, indifeso, come una candela spenta in mezzo al buio.

\begin{itemize}
\item \textbf{Effetto:} Il personaggio non può agire, né percepire ciò che accade intorno a lui. Non ha diritto a tiri difensivi o tentativi di evitare colpi.
\item \textbf{Narrativamente:} I compagni possono tentare di soccorrerlo, mentre i nemici possono colpirlo impunemente. Attacchi contro un bersaglio svenuto ignorano ogni bonus difensivo.
\end{itemize}

Il recupero avviene solo per effetto di cure, magia, o dopo un tempo stabilito dal Master in base alla causa del collasso.

\subsection*{Addormentamento}
Il sonno è una sospensione della coscienza, volontaria o imposta da magia, veleni o stanchezza estrema. Un personaggio addormentato non è meno vulnerabile di uno svenuto, ma la sua condizione può essere interrotta più facilmente.

\begin{itemize}
\item \textbf{Effetto:} Il personaggio è incosciente e incapace di agire. È considerato come svenuto a tutti gli effetti.
\item \textbf{Risveglio:} Rumori forti, scosse violente o un attacco diretto lo risvegliano immediatamente, a discrezione del Master.
\item \textbf{Narrativamente:} Un addormentamento può diventare un ottimo espediente per colpi a sorpresa, imboscate o intrusioni furtive.
\end{itemize}
\clearpage
\section{Avvelenamento}

Il veleno è una delle armi più subdole e spietate. Invisibile, lento, insinuante. Può portare alla morte con una goccia, o compromettere le capacità di un avversario con effetti quasi impercettibili. I veleni non hanno un’unica forma: possono essere estratti, distillati, creati magicamente o segretamente mischiati a cibo e bevande.

\textbf{Meccanica:} Ogni veleno ha una durata, un effetto e un metodo di somministrazione. Il Master è libero di crearne di propri, ma alcuni archetipi sono riportati qui come riferimento.

\subsection*{Veleno Debole}
Una sostanza blanda, spesso creata con ingredienti comuni. Può debilitare, ma raramente uccide.

\begin{itemize}
\item \textbf{Effetto:} Infligge 1 danno all’inizio di ogni turno. Il danno è puro e non può essere ridotto da armature o resistenze.
\item \textbf{Durata:} 5 turni.
\item \textbf{Narrativamente:} utile per spossare bersagli in fuga, compromettere un combattente prima di uno scontro o come avvertimento.
\end{itemize}

\subsection*{Veleno Narcotico}
Un composto che agisce sul sistema nervoso o mentale. Ne esistono versioni naturali e magiche, e gli effetti possono variare da uno stordimento lieve all’incoscienza profonda.

\begin{itemize}
\item \textbf{Effetto:} Il personaggio ha un \textbf{+2 alla difficoltà} per ogni azione. In alcuni casi, può causare visioni, allucinazioni o un sonno improvviso.
\item \textbf{Durata:} A discrezione del Master, solitamente da pochi minuti a qualche ora.
\item \textbf{Narrativamente:} Perfetto per rapimenti, interrogatori o sabotaggi.
\end{itemize}

\subsection*{Veleno Letale}
Un veleno progettato per uccidere. Alcuni agiscono rapidamente, altri consumano il corpo lentamente ma inesorabilmente. Alcuni sono di origine alchemica, altri derivano da creature esotiche.

\begin{itemize}
\item \textbf{Effetto:} Infligge danni progressivi ogni turno: 
\[
\textnormal{1D4} \textit{ (turni 1–3)}, \quad \textnormal{1D6} \textit{ (turni 4–6)}, \quad \textnormal{1D8} \textit{ (dal 7° turno in poi)}
\]
\item \textbf{Cura:} Solo un antidoto specifico o una magia molto potente può fermare il decorso. In assenza di trattamento, la morte è certa.
\item \textbf{Narrativamente:} una maledizione avvelenata, il morso di un serpente ultraterreno, o il prezzo per aver tradito una gilda di assassini.
\end{itemize}

\textbf{Nota per il Master:} I veleni possono essere anche usati per generare tensione narrativa. Un personaggio che ha solo poche ore per cercare un antidoto, o che scopre di essere stato avvelenato da un alleato, vive momenti di grande impatto interpretativo.


\end{document}
