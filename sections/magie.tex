\documentclass[../manuale_main.tex]{subfiles}


\begin{document}

\section{Magia e Stregoneria}

La magia è una forza misteriosa e potente che permea il mondo di Zarafir. I personaggi possono utilizzare la magia in due modi distinti: attraverso \textbf{Incantesimi} e \textbf{Stregonerie}. Questi due approcci alla magia offrono stili diversi di manipolazione delle forze arcane:

\begin{itemize}
    \item \textbf{Incantesimi:} Sono formule magiche apprese e studiate, organizzate in scuole specifiche (Sapienza, Fede e Morte). Gli incantesimi offrono effetti precisi e prevedibili, ma richiedono preparazione e concentrazione.
    \item \textbf{Stregonerie:} Sono manifestazioni spontanee di potere, influenzate dalla volontà e dall'immaginazione del personaggio. Le stregonerie sono più flessibili, permettendo una grande varietà di effetti, ma richiedono un maggiore controllo mentale.
\end{itemize}

La magia in Zarafir non utilizza Punti Mana, ma è direttamente legata allo \textbf{Stress Mentale}. Ogni volta che un personaggio lancia una magia, accumula Stress Mentale pari alla \textbf{difficoltà di lancio della magia}. Questo rappresenta la pressione psicologica e la fatica mentale di manipolare forze arcane. Anche in caso di fallimento nel lancio della magia, lo Stress Mentale viene accumulato.

\subsection{Utilizzare una Magia}
Per utilizzare una magia (sia essa un incantesimo o una stregoneria), il personaggio deve dichiarare quale magia desidera lanciare e fare un Tiro di Simulazione (TDS) per verificare il successo. 

\begin{itemize}
    \item \textbf{Dichiarazione del Lancio:} Il giocatore dichiara la magia che il suo personaggio vuole lanciare. Se si tratta di una Stregoneria, specifica anche l'effetto desiderato.
    \item \textbf{Esecuzione del TDS:} Il giocatore esegue un TDS basato sulla propria abilità magica (come Tessimagie per gli Incantesimi, Stregoneria di Attacco per le Stregonerie offensive, ecc.). La difficoltà del TDS è determinata dal livello della magia.
    \item \textbf{Accumulo di Stress Mentale:} Indipendentemente dal risultato del TDS (successo o fallimento), il personaggio accumula Stress Mentale pari alla difficoltà della magia.
\end{itemize}

\subsubsection{Esempio di Utilizzo di una Magia}
Un mago decide di lanciare l'incantesimo \textit{Palla di Fuoco} (difficoltà 4). Ha l'abilità \textbf{Tessimagie} al 5 e \textbf{Intelligenza} a 12.

\begin{itemize}
    \item Il giocatore tira 3D6 e ottiene un risultato totale di 10.
    \item La difficoltà è 4, quindi il risultato necessario è: 
    \[
    12 [\emph{Intelligenza}] + 5 [\emph{Tessimagie}] - 4 [\emph{Difficoltà}] = 13
    \]
    \item 10 è inferiore a 13, quindi l'incantesimo ha successo. La \textbf{Palla di Fuoco} viene lanciata, infliggendo i danni indicati.
    \item Il mago accumula 4 punti di \textbf{Stress Mentale}, pari alla difficoltà dell'incantesimo.
\end{itemize}

\subsection{Resistere a una Magia}
Quando un personaggio viene colpito da una magia ostile, può tentare di resistere utilizzando la sua abilità \textbf{Resistenza Magica} (basata su \textbf{Sensibilità}). Il processo per resistere a una magia segue questi passaggi:

\begin{itemize}
    \item \textbf{Esecuzione del TDS:} Il personaggio bersaglio esegue un TDS utilizzando \textbf{Resistenza Magica}. La difficoltà è pari al valore dell'abilità magica dell'incantatore.
    \item \textbf{Accumulo di Stress Mentale:} Se il personaggio resiste con successo, accumula Stress Mentale pari alla metà della difficoltà di lancio (arrotondato per difetto). Se fallisce, non accumula Stress aggiuntivo.
    \item \textbf{Effetti della Resistenza:}
    \begin{itemize}
        \item \textbf{Magie Offensive:} Se la resistenza ha successo, i danni della magia vengono ridotti di un valore pari a quello dell'effetto della magia (ad esempio, resistere a una \textit{Palla di Fuoco} riduce il danno di 2D6).
        \item \textbf{Magie Non Offensive:} Il Master stabilisce gli effetti di una resistenza riuscita, che possono variare da una riduzione della durata a un annullamento parziale degli effetti.
    \end{itemize}
\end{itemize}

\subsubsection{Esempio di Resistenza a una Magia}
Un personaggio viene colpito da una \textit{Palla di Fuoco} lanciata da un mago (difficoltà 4). Il personaggio ha \textbf{Resistenza Magica} al 3 e \textbf{Sensibilità} a 10.

\begin{itemize}
    \item Il giocatore tira 3D6 e ottiene un risultato di 8.
    \item La difficoltà è 4 (abilità del mago), quindi il risultato necessario è:
    \[
    10 [\emph{Sensibilità}] + 3 [\emph{Resistenza Magica}] - 4 = 9
    \]
    \item 8 è inferiore a 9, quindi la resistenza ha successo.
    \item Il personaggio accumula 2 punti di \textbf{Stress Mentale} (la metà della difficoltà della magia, arrotondata per difetto).
    \item I danni della \textit{Palla di Fuoco} vengono ridotti di 2D6 (come indicato nella descrizione della magia).
\end{itemize}

\subsection{Durata delle Magie}
Le magie possono avere una durata istantanea o prolungata nel tempo. 

\begin{itemize}
    \item \textbf{Magie Istantanee:} Hanno effetto immediato e terminano subito dopo l'attivazione (es. \textit{Palla di Fuoco}).
    \item \textbf{Magie a Durata Prolungata:} Restano attive per un numero di round o minuti indicato nella descrizione della magia. 
\end{itemize}

\subsection{Magia e Equipaggiamento}
Lanciare magie richiede una certa libertà di movimento e concentrazione. L'uso di armature e armi può influenzare la capacità di lanciare incantesimi:

\begin{itemize}
    \item \textbf{Armi e Scudi:} Non è possibile lanciare magie se entrambe le mani sono impegnate.
    \item \textbf{Armature Leggere:} Nessuna penalità al lancio di magie.
    \item \textbf{Armature Medie:} Modificatore di +1 alla difficoltà per ogni pezzo di armatura media indossato.
    \item \textbf{Armature Pesanti:} Non è possibile lanciare magie indossando armature pesanti.
\end{itemize}


\subsection{Gli incanti}
Come già accennato prima, gli incanti sono le magie imparate dai personaggi in modo accademico e rigoroso.
Sono divise in 3 rami ``Sapienza'', ``Fede'' e ``Morte' che possono essere ricondotti alle magie delle tre classi iconiche ``Mago'', ``Paladino'' e ``Necromante'' presenti in altri manuali di gioco di ruolo.\\
Ogni ramo presenta un numero differente di incanti, raggruppati in \emph{cerchi} di ordine crescente. I cerchi di livello più basso contengono le magie più semplici del ramo (ma non per questo inutili in fasi di gioco avanzate); i cerchi di livello più alto racchiudono le magie più difficili ma con effetti molto più incisivi.



\subsubsection{Apprendere un nuovo incanto}
Per apprendere un nuovo incanto è necessario aver eseguito un'azione specifica (per esempio lo studio dello stesso grazie a una pergamena) dipendente dall'ambientazione.

Dipendentemente dall'ambientazione, potrebbe essere possibile per un personaggio apprendere incanti di più di un ramo.
%arte della sapienza
\clearpage
\subfile{lista_sapienza}

\clearpage
\subfile{lista_fede}

\clearpage
\subfile{lista_morte}

\clearpage
\subfile{lista_stregonerie}


\end{document}