\documentclass[./magie.tex]{subfiles}

\begin{document}

\section{Stregonerie}

Le Stregonerie sono un tipo di magia versatile e spontanea, che permette ai personaggi di manipolare le forze del mondo in modo creativo. A differenza degli Incanti, le Stregonerie non richiedono uno studio accademico, ma si basano sulla connessione del personaggio con gli elementi e le energie mistiche. Ogni stregoneria ha una difficoltà di lancio basata sul suo Grado e gli effetti dipendono dalla Potenza, calcolata tramite un dado.
\vspace{0.2cm}
\noindent
\begin{center}
\rule{\textwidth}{0.4pt} 
\end{center}
\vspace{0.2cm}
\subsection{Stregoneria di Attacco}

Le stregonerie di attacco permettono all’incantatore di infliggere danni tramite la manipolazione degli elementi. Ogni effetto dipende dalla volontà del giocatore, che sceglie il \textbf{Grado} (intensità dell'effetto), e da un lancio di \textbf{1D6} che determina la \textbf{Potenza} (manifestazione concreta dell’effetto). Le due forme più comuni sono:
\begin{itemize}
\item \textbf{Distruzione:} vengono inflitti danni ai bersagli sfruttando gli elementi che circondano l'incantatore.
\item \textbf{Incantamento:} l'arma dell'incantatore viene infusa di potere mistico o magico, migliorandone l'efficacia in combattimento.
\end{itemize}

\vspace{0.2cm}
\noindent
\begin{center}
\rule{\textwidth}{0.4pt} 
\end{center}
\vspace{0.2cm}
\clearpage
\subsection*{Tabella della Potenza Narrativa (1D6)}
\begin{itemize}
\item \textbf{1 – Effetto Fiacco:} Il colpo va a segno, ma in modo debole o incerto. L’elemento si disperde parzialmente. Riduci il danno di 1.
\item \textbf{2 – Effetto Standard:} L’effetto ha pieno successo, senza complicazioni o bonus. Il danno rimane pari al Grado (nessun modificatore)
\item \textbf{3 – Efficacia Alta:} L’effetto è più potente del previsto. Aumenta il danno di +1.
\item \textbf{4 – Effetto ambientale:} Oltre al danno, si manifesta un effetto narrativo coerente con l’elemento usato (il fuoco incendia, il ghiaccio rallenta...). Il Master ha l’ultima parola sull’effetto.
\item \textbf{5 – Colpo Crudele:} Il colpo ha un impatto devastante sul bersaglio. Aumenta il danno di +1 e applica anche un effetto secondario.
\item \textbf{6 – Dominio Elementale:} Il potere dell’elemento scelto travolge il bersaglio: il danno ignora armatura e resistenze, e l’effetto narrativo è particolarmente forte.
\end{itemize}


\vspace{0.2cm}
\noindent
\begin{center}
\rule{\textwidth}{0.4pt} 
\end{center}
\vspace{0.2cm}

\subsection*{Distruzione}
\begin{description}
\item \textbf{Difficoltà:} Grado scelto
\item \textbf{Durata:} Istantaneo
\item \textbf{Gittata:} 10 metri per punto in “Stregoneria di Attacco”
\item \textbf{Descrizione:} L’incantatore scaglia un elemento (ombra, fuoco, sale, aculei d’ossa...) contro un bersaglio. Il danno inflitto è pari a \textbf{Grado + modificatori}, influenzato dalla tabella della potenza. L’elemento usato viene descritto liberamente dal giocatore in accordo col Master. Può essere evocato, o richiamato da fonti ambientali (torce, sabbia, sangue...).
\end{description}


\subsection*{Incantamento}
\begin{description}
\item \textbf{Difficoltà:} Grado scelto
\item \textbf{Durata:} Fino al prossimo attacco (entro 5 round)
\item \textbf{Gittata:} Arma impugnata
\item \textbf{Descrizione:} L’incantatore infonde la propria arma con energia mistica. Il prossimo attacco fisico infligge \textbf{Grado + modificatori} danni aggiuntivi. A seconda del risultato del dado, l’arma potrebbe ad esempio:
\begin{itemize}
  \item esplodere in scintille o fiamme al momento dell’impatto,
  \item corrodere l’armatura del bersaglio,
  \item provocare paura o paralisi per effetto dell’energia impiegata.
\end{itemize}
\end{description}

Tira 1D6 e consulta la tabella della potenza narrativa per stabilire gli effetti.\\
\textit{Nota:} L’Incantamento può essere usato anche su armi naturali ottenute tramite Metamorfosi (come artigli, zanne o pungiglioni).

\clearpage
\subsection{Stregoneria di Controllo}

Le stregonerie di controllo permettono all’incantatore di ostacolare direttamente le azioni dei suoi avversari o di dissolvere magie altrui. Le due forme più comuni sono:

\begin{itemize}
\item \textbf{Blocco:} impedisce o rende difficoltosi i movimenti del bersaglio, tramite catene d’ombra, nebbie vincolanti, o radici magiche.
\item \textbf{Annullamento:} dissolve un incantesimo o una stregoneria, negandone gli effetti.
\end{itemize}

Come per tutte le stregonerie, l’incantatore sceglie il \textbf{Grado} (che definisce la difficoltà del TDS) e tira un \textbf{1D6} per determinare la \textbf{Potenza}, che influisce sull’efficacia.

\vspace{0.2cm}
\noindent
\begin{center}
\rule{\textwidth}{0.4pt} 
\end{center}
\vspace{0.2cm}
\subsection*{Tabella della Potenza Narrativa (1D6)}
\begin{itemize}
\item \textbf{1 – Scarto minore:} L’effetto si manifesta, ma è instabile o fiacco. La magia rallenta solo lievemente il bersaglio, o annulla solo parte dell’effetto.
\item \textbf{2 – Esecuzione standard:} La magia ha effetto come previsto, senza complicazioni.
\item \textbf{3 – Concentrazione precisa:} L’effetto è stabile e preciso: il bersaglio è significativamente rallentato, o la magia viene interrotta con precisione.
\item \textbf{4 – Interferenza totale:} Il bersaglio è completamente immobilizzato per 1 round, oppure la magia viene non solo interrotta, ma anche cancellata dal ricordo per un breve periodo (non può essere rilanciata per 2 round).
\item \textbf{5 – Dominio improvviso:} L’incantatore ottiene anche un vantaggio narrativo temporaneo (il bersaglio inciampa, perde tempo, cade…).
\item \textbf{6 – Risonanza assoluta:} L’energia dell’incanto si espande: più bersagli possono essere colpiti, oppure l’effetto è amplificato (es. immobilizzazione per 2 round, dissoluzione totale di ogni effetto secondario).
\end{itemize}

\subsection*{Blocco}
\begin{description}
\item \textbf{Difficoltà:} Grado scelto
\item \textbf{Durata:} In base alla Potenza
\item \textbf{Gittata:} 3 metri per ogni punto nell’abilità “Stregoneria di Controllo”
\item \textbf{Descrizione:} L’incantatore crea un vincolo fisico o magico che rallenta o paralizza il bersaglio. L’effetto può essere descritto liberamente dal giocatore (liane ombrose, catene eteree, gelo mistico...), purché coerente con la scena e l’elemento evocato.
\end{description}


\begin{itemize}
\item Il blocco può influire su movimento, attacco o difesa.
\item La potenza determina se il bersaglio ha solo penalità (+2 alla difficoltà delle azioni) o è completamente immobilizzato.
\item Il Master stabilisce se effetti speciali (es. cadute, disarmo) hanno senso nella scena.
\end{itemize}

\clearpage
\subsection*{Annullamento}
\begin{description}
\item \textbf{Difficoltà:} Pari alla difficoltà del lancio della magia bersaglio
\item \textbf{Durata:} Istantaneo
\item \textbf{Gittata:} 10 metri per ogni punto nell’abilità “Stregoneria di Controllo”
\item \textbf{Descrizione:} L’incantatore dissolve un incantesimo o una stregoneria lanciata da un altro mago. Può essere usato come reazione immediata (se dichiarato al momento del lancio della magia nemica) o su un effetto già attivo.
\end{description}

\begin{itemize}
\item Il Grado dell’annullamento deve essere almeno pari a quello della magia da dissolvere.
\item Il Master può richiedere un elemento narrativo coerente per spiegare come l’annullamento si manifesta (una vibrazione, una contro-ondata di energia, un gesto rituale...).
\item Se il risultato del dado è 4 o più, la magia non può essere rilanciata dallo stesso bersaglio per 2 round.
\end{itemize}

\clearpage
\vspace{0.2cm}
\subsection{Stregoneria della Materia}

Le stregonerie della materia consentono all’incantatore di plasmare il mondo fisico attraverso il potere magico. L’incantatore sceglie un \textbf{Grado} (intensità dello sforzo e difficoltà del TDS) e tira un \textbf{1D6} per determinare la \textbf{Potenza} (manifestazione concreta dell’effetto).

Le due specializzazioni di questa disciplina sono:

\begin{itemize}
\item \textbf{Alterazione:} Manipola la materia già esistente (terra, acqua, pietra, fango, cenere…) per modificarne forma, consistenza o comportamento.
\item \textbf{Metamorfosi:} Trasforma il corpo dell’incantatore, donandogli attributi bestiali o capacità fisiche sovrannaturali.
\end{itemize}

\vspace{0.2cm}
\noindent
\begin{center}
\rule{\textwidth}{0.4pt} 
\end{center}
\vspace{0.2cm}
\subsection*{Tabella della Potenza Narrativa (1D6)}
\begin{itemize}
\item \textbf{1 – Mutamento instabile:} L’effetto ha breve durata o risulta parziale (es. la barriera è fragile, gli artigli sono contorti). Durata: 1 round.
\item \textbf{2 – Trasformazione base:} L’effetto si manifesta in modo completo. Durata: 2 round (o 5 minuti per Alterazione).
\item \textbf{3 – Forma ottimizzata:} L’effetto ha una durata normale e precisione nella trasformazione. Durata: 3 round / 10 minuti.
\item \textbf{4 – Espansione spontanea:} L’effetto si propaga leggermente oltre l’intento originale (es. la parete è più estesa, le zanne più affilate). Durata: 4 round / 15 minuti.
\item \textbf{5 – Fusione perfetta:} L’effetto si armonizza con il contesto. Durata: 5 round / 20 minuti. Il personaggio riceve un bonus narrativo (es. resistenza ambientale, visione speciale).
\item \textbf{6 – Dominio materiale:} L’incantatore piega completamente la materia al proprio volere. Durata: 10 round / 30 minuti. Gli effetti sono amplificati o creativamente spettacolari (es. il corpo diventa interamente d’ombra, una trincea si apre nel terreno).
\end{itemize}

\subsection*{Alterazione}
\begin{description}
\item \textbf{Difficoltà:} Grado scelto
\item \textbf{Durata:} In base alla Potenza (5–30 minuti)
\item \textbf{Gittata:} 5 metri per punto in “Stregoneria della Materia”
\item \textbf{Descrizione:} L’incantatore modifica la materia intorno a sé. L’effetto è limitato alla quantità di materiale accessibile (terra, pietra, metallo, ossa, sabbia, nebbia...), ma può assumere qualsiasi forma: ostacoli, trappole, ripari, aperture, strumenti, ecc.
\end{description}

\textit{Esempi narrativi:}
\begin{itemize}
\item Sollevare una piattaforma di pietra per elevarsi.
\item Incurvare la lama di un nemico prima che colpisca.
\item Solidificare una nube tossica trasformandola in schegge.
\item Creare una gabbia temporanea fatta di cenere compattata.
\end{itemize}

\textit{Nota:} Non è possibile alterare creature viventi o creare materiale dal nulla.

\clearpage
\subsection*{Metamorfosi}
\begin{description}
\item \textbf{Difficoltà:} Grado scelto
\item \textbf{Durata:} In base alla Potenza (1–10 round)
\item \textbf{Gittata:} Sé stesso (o a contatto, se applicata a un altro)
\item \textbf{Descrizione:} L’incantatore trasforma il proprio corpo o parte di esso, assumendo tratti bestiali, vegetali o ultraterreni. Queste trasformazioni garantiscono bonus narrativi o di gioco, a seconda della forma assunta.
\end{description}

\textit{Esempi di trasformazione e benefici:}
\begin{itemize}
\item \textbf{Artigli, zanne, pungiglioni:} Permettono di attaccare a mani nude infliggendo danni pari a \textbf{Grado / 2 + 1D3}, e sono compatibili con la Stregoneria di Attacco (\textbf{Incantamento}).
\item \textbf{Ali membranose o vaporose:} Permettono brevi planate o movimenti verticali facilitati (bonus a Schivare o Mobilità).
\item \textbf{Occhi notturni, naso animale:} Bonus a Percezione, Seguire Tracce o Evasione.
\item \textbf{Pelle coriacea o fusa con corteccia:} Fornisce una \textbf{Armatura} pari a \textbf{Grado / 2}.
\end{itemize}

\textbf{Bonus meccanico base:} Ogni trasformazione offre un beneficio pari a \textbf{Grado / 2} (arrotondato per difetto), salvo diversa indicazione.

\vspace{0.5cm}
\textbf{Interazione con Stregoneria di Attacco:} \\
Una creatura trasformata può utilizzare \textbf{Incantamento} su armi naturali (come artigli o zanne) ottenute tramite Metamorfosi. In questo caso:
\begin{itemize}
\item Il successivo TDS per Incantamento ha \textbf{–1 alla difficoltà}.
\item L’attacco infuso con energia magica segue le regole normali di Incantamento.
\item Il danno base è quello dell’arma naturale (vedi sopra), aumentato dalla Stregoneria.
\end{itemize}

\textit{Nota:} Un personaggio trasformato può continuare a lanciare altre stregonerie, a meno che la trasformazione non ne impedisca fisicamente l’uso (es. bocca sigillata, assenza di mani...). Interrompere l’effetto prima del tempo può generare Stress Mentale, a discrezione del Master.


\clearpage
\vspace{0.2cm}
\subsection{Stregoneria della Mente}

Le stregonerie della mente permettono di ingannare i sensi, alterare le emozioni e piegare la volontà altrui. Sono magie subdole, spesso sottili e destabilizzanti. L’incantatore sceglie il \textbf{Grado} (intensità e difficoltà del TDS) e tira \textbf{1D6} per determinare la \textbf{Potenza}, che ne regola durata, impatto o profondità.

\textbf{Specializzazioni:}
\begin{itemize}
\item \textbf{Illusionismo:} Crea immagini, suoni, sensazioni, odori o presenze completamente fittizie.
\item \textbf{Condizionamento:} Induce emozioni forti o stati mentali alterati (es. colpa, paranoia, coraggio, passività...).
\end{itemize}
\vspace{0.2cm}
\noindent
\begin{center}
\rule{\textwidth}{0.4pt} 
\end{center}
\vspace{0.2cm}

\subsection*{Tabella della Potenza Narrativa (1D6)}
\begin{itemize}
\item \textbf{1 – Illusione instabile / emozione labile:} L’effetto è poco convincente o dura pochissimo (1 minuto o 1 round).
\item \textbf{2 – Effetto normale:} L’illusione inganna i sensi in modo plausibile. L’emozione dura per 5 minuti / 2 round.
\item \textbf{3 – Percezione immersiva / emozione concreta:} L’illusione è convincente e multisensoriale. L’emozione altera anche il comportamento. Durata: 10 minuti / 3 round.
\item \textbf{4 – Suggestione attiva:} Il bersaglio agisce come se l’illusione fosse reale o si comporta in accordo all’emozione (fugge, si arrende, agisce in modo impulsivo). Durata: 15 minuti / 4 round.
\item \textbf{5 – Stato di confusione:} L’effetto è talmente penetrante da confondere la coscienza. Il bersaglio ha un malus di +2 alla difficoltà per tutte le azioni durante l’effetto. Durata: 20 minuti / 5 round.
\item \textbf{6 – Collasso della volontà:} L’illusione altera la realtà percepita in modo traumatico. Il condizionamento impone una condotta. Il bersaglio può perdere il turno, agire in modo incontrollato o credere a qualcosa di falso anche dopo il termine dell’effetto.
\end{itemize}

\clearpage
\subsection*{Illusionismo}
\begin{description}
\item \textbf{Difficoltà:} Grado scelto
\item \textbf{Durata:} In base alla Potenza (1–20 minuti)
\item \textbf{Gittata:} 3 metri per punto in “Stregoneria della Mente”
\item \textbf{Descrizione:} L’incantatore genera una proiezione mentale che colpisce i sensi dei bersagli. Può essere visiva, uditiva, olfattiva, tattile, o una combinazione. L’illusione non infligge danni ma può avere effetti concreti se sfruttata con intelligenza (far sbagliare bersaglio, spaventare, dividere un gruppo…).
\end{description}

\textit{Esempi narrativi:}
\begin{itemize}
\item Un burrone illusorio si apre sotto i piedi del nemico.
\item Una voce familiare grida aiuto da dietro una parete.
\item Una figura vestita di nero cammina minacciosamente nella nebbia.
\item L’incantatore crea un doppione illusorio di sé stesso.
\end{itemize}

\textit{Nota:} Un bersaglio può tentare un TDS su “Resistenza Magica” se ha un forte dubbio o interagisce direttamente con l’illusione.

\clearpage
\subsection*{Condizionamento}
\begin{description}
\item \textbf{Difficoltà:} Grado scelto
\item \textbf{Durata:} In base alla Potenza (1–5 round)
\item \textbf{Gittata:} 2 metri per punto in “Stregoneria della Mente”
\item \textbf{Descrizione:} L’incantatore altera lo stato emotivo o la volontà di un bersaglio. L’effetto può essere diretto (es. “paura”) o indotto (es. “apatia verso la violenza”). La durata e l’intensità dipendono dal tiro di Potenza.
\end{description}

\textit{Esempi narrativi:}
\begin{itemize}
\item Suscitare orrore e far fuggire un nemico.
\item Indurre un guerriero alla compassione verso un prigioniero.
\item Rafforzare la determinazione di un alleato morente.
\item Far provare una colpa schiacciante per azioni passate.
\end{itemize}

\textit{Nota:} Il bersaglio può resistere con un TDS su “Resistenza Magica” se la propria bussola morale è fortemente contraria all’effetto imposto.

\clearpage
\vspace{0.2cm}
\subsection{Stregoneria della Vita}

Le stregonerie della vita canalizzano le forze primordiali dell’esistenza, permettendo all’incantatore di guarire, proteggere o evocare esseri viventi. A differenza degli incanti clericali, queste magie non richiedono devozione, ma un’intima affinità con l’impulso vitale e la sua trasformazione.

\textbf{Specializzazioni:}
\begin{itemize}
\item \textbf{Guarigione:} Ripristina la salute o genera scudi magici che proteggono dai danni.
\item \textbf{Evocazione:} Richiama creature legate all’anima del mago per combattere, esplorare o servire.
\end{itemize}

\vspace{0.2cm}
\noindent
\begin{center}
\rule{\textwidth}{0.4pt} 
\end{center}
\vspace{0.2cm}
\subsection*{Tabella della Potenza Narrativa (1D6)}
\begin{itemize}
\item \textbf{1 – Flusso instabile:} L’effetto è debole. La guarigione è dimezzata, la creatura evocata ha 2 round di durata.
\item \textbf{2 – Risultato modesto:} L’incantatore ottiene un effetto base. Cura \textbf{Grado + 1} punti vita o evoca una creatura per 4 round.
\item \textbf{3 – Vitalità intensa:} Guarisce \textbf{Grado + 3} PV o evoca una creatura con bonus +1 alla difesa o ai danni.
\item \textbf{4 – Scudo vitale / evocazione instancabile:} Oltre alla cura, crea uno scudo temporaneo pari alla metà della cura. La creatura evocata ha armatura +1.
\item \textbf{5 – Risonanza dell’anima:} Il bersaglio guarisce completamente da ferite lievi o riceve un effetto aggiuntivo (rimuove un effetto negativo, o la creatura evocata è immune alla paura).
\item \textbf{6 – Armonia primordiale:} Guarigione totale da ogni danno fisico (entro il valore massimo), oppure la creatura evocata ottiene +2 a tutte le sue statistiche e dura il doppio del tempo.
\end{itemize}

\clearpage

\subsection*{Guarigione}
\begin{description}
\item \textbf{Difficoltà:} Grado scelto
\item \textbf{Durata:} Istantaneo (cura) oppure 1 round per ogni 3 punti curati (scudo)
\item \textbf{Gittata:} 3 metri per punto in “Stregoneria della Vita”
\item \textbf{Descrizione:} L’incantatore riversa energia vitale su un bersaglio, guarendo le sue ferite oppure creando una barriera che assorbe i danni in arrivo.

\textit{Effetti:}
\begin{itemize}
  \item \textbf{Cura:} Ripristina \textbf{Grado + modificatori} punti vita.
  \item \textbf{Scudo Vitale:} In alternativa, lo stesso valore può essere applicato come “scudo” che assorbe danni successivi.
\end{itemize}
\end{description}

\textit{Nota:} A discrezione del Master, una guarigione molto potente può rimuovere effetti fisici temporanei (come stordimento, cecità o emorragie) o permanenti, come mutilazioni gravi o malattie profonde.

\clearpage
\subsection*{Evocazione}
\begin{description}
\item \textbf{Difficoltà:} Grado scelto
\item \textbf{Durata:} In base alla Potenza (4–12 round)
\item \textbf{Gittata:} 2 metri per punto in “Stregoneria della Vita”
\item \textbf{Descrizione:} L’incantatore richiama una creatura magica generata dalla propria volontà o da legami spirituali (spiriti guida, animali totemici, simulacri di anime perdute...).
\end{description}

\textit{Caratteristiche base della creatura evocata:}
\begin{itemize}
\item \textbf{Punti Vita:} Grado + Potenza + 2
\item \textbf{Abilità di Combattimento (caratteristica + abilità):} Grado + 5 + Potenza
\item \textbf{Danni:} Grado / 2 + Potenza + 1D6
\item \textbf{Evasione:} Grado / 2
\item \textbf{Armatura:} Potenza
\item \textbf{Durata:} 2 round per ogni punto di Potenza (minimo 2 round)
\end{itemize}

\textit{Nota:} Le evocazioni scompaiono alla fine della durata, se distrutte, o se allontanate troppo. Non sono veri esseri viventi, ma proiezioni vitali.

\textit{Esempi narrativi:}
\begin{itemize}
\item Una tigre dorata scaturisce dalla pelle dell’incantatore e si lancia contro il nemico.
\item Uno spirito umanoide fluttuante avvolge di luce il mago e lo protegge.
\item Un falco evanescente si libra per esplorare e riportare ciò che vede.
\end{itemize}

\end{document}
