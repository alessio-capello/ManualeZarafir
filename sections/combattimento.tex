\documentclass[../manuale_main.tex]{subfiles}



\begin{document}
Il combattimento è il momento in cui la tensione diventa azione, dove ogni gesto può essere fatale e ogni esitazione può segnare il destino di un personaggio. Le regole che seguono sono pensate per simulare scontri dinamici e strategici, in cui ogni decisione ha un peso reale.


\vspace{0.5cm}
\noindent
\begin{center}
\rule{\textwidth}{0.4pt} 
\end{center}
\vspace{0.5cm}

\section{Iniziativa}
Prima di iniziare un combattimento, è necessario stabilire l'ordine in cui i personaggi agiscono. Questo si determina tramite un tiro di \textbf{Iniziativa}:

\begin{itemize}
    \item Ogni personaggio coinvolto nello scontro tira 1D6.
    \item Al risultato si somma il valore della propria \textbf{Reattività} e ogni eventuale modificatore all'iniziativa.
    \item I personaggi agiscono in ordine decrescente del risultato ottenuto.
\end{itemize}

In alcune situazioni, il Master potrebbe decidere di ignorare l’iniziativa e stabilire arbitrariamente l’ordine delle azioni, per favorire la narrazione o in situazioni di imboscata.


\vspace{0.5cm}
\noindent
\begin{center}
\rule{\textwidth}{0.4pt} 
\end{center}
\vspace{0.5cm}

\section{Fasi del Combattimento}
Il combattimento si svolge seguendo una sequenza chiara e strutturata:

\begin{enumerate}
    \item \textbf{L'attaccante dichiara il suo bersaglio.}
    \item \textbf{L'attaccante esegue il TDS per colpire il bersaglio.}
    \item \textbf{Se l'attacco ha successo, vengono calcolati i danni.}
    \item \textbf{Il difensore dichiara la sua difesa:} può provare a parare o utilizzare un'abilità difensiva (come una magia protettiva).
    \item \textbf{Se il difensore decide di parare, esegue il TDS per la parata.}
\end{enumerate}

Se un'azione fallisce, le fasi successive vengono influenzate (ad esempio, se l'attaccante fallisce l'attacco, non si calcolano i danni).

\textbf{Nota: Tutte le appropssimazioni verranno eseguite per difetto.}

\vspace{0.5cm}
\noindent
\begin{center}
\rule{\textwidth}{0.4pt} 
\end{center}
\vspace{0.5cm}
\clearpage
\section{Calcolo del Valore Offensivo e Difensivo}
Il combattimento in Zarafir è basato su un confronto tra il \textbf{Valore Offensivo} dell'attaccante e il \textbf{Valore Difensivo} del difensore:

\begin{itemize}
    \item \textbf{Valore Offensivo dell'Attaccante:}
    \[
    \textnormal{Valore Offensivo} = \textnormal{Caratteristica} + \textnormal{Abilità} + \textnormal{Modificatori}
    \]
    \item \textbf{Valore Difensivo del Difensore:}
    \[
    \textnormal{Valore Difensivo} = \frac{\textnormal{Reattività}}{2} + \textnormal{Protezione Armatura} + \textnormal{Modificatori}
    \]

\end{itemize}

La \textbf{Protezione Armatura} è calcolata come segue:
\[
\frac{\textnormal{\small{Protezione Elmo + Protezione Corazza + Protezione Schinieri + Protezione Bracciali}}}{2}
\]
La protezione di ciascun pezzo è generalmente determinata dal suo tipo, per esempio:
\begin{itemize}
    \item \textbf{Armature Leggere:} Protezione 1.
    \item \textbf{Armature Medie:} Protezione 2.
    \item \textbf{Armature Pesanti:} Protezione 3.
\end{itemize}
Il Master può cambiare questi bonus, per allinearli all'ambientazione. I valori proposti sono pensato per un'ambientazione ispirata al fantasy classico.

\vspace{0.5cm}
\noindent
\begin{center}
\rule{\textwidth}{0.4pt} 
\end{center}
\vspace{0.5cm}

\section{Esecuzione dell'Attacco}
Per colpire il bersaglio, l'attaccante esegue un TDS. La difficoltà del tiro è pari al \textbf{Valore Difensivo} del bersaglio. 

\textbf{Nota:} In alcuni casi, un attaccante potrebbe colpire quasi automaticamente, fallendo solo con un tiro di \textbf{6 6 6}. Per mantenere il pathos anche negli scontri fortemente sbilanciati, si consiglia di applicare una soglia massima alla differenza tra i valori: si considera automaticamente fallito ogni attacco in cui il risultato dei dadi non sia inferiore a 15. Questo accorgimento introduce un margine di incertezza anche nei duelli più impari, senza alterare l’equilibrio dei combattimenti tra avversari di pari livello.

\begin{itemize}
    \item Se il TDS ha successo, l'attacco colpisce e vengono calcolati i danni.
    \item Se fallisce, l'attacco manca e il turno dell'attaccante termina.
\end{itemize}


\vspace{0.5cm}
\noindent
\begin{center}
\rule{\textwidth}{0.4pt} 
\end{center}
\vspace{0.5cm}

\section{Parata con lo Scudo}
Se il difensore possiede uno scudo, può dichiarare di voler parare l'attacco:

\begin{itemize}
    \item Il difensore esegue un TDS su \textbf{Parare}.
    \item La difficoltà è pari al \textbf{valore dell'abilità di combattimento dell'attaccante}.
    \item Se la parata ha successo, i danni dell'attacco vengono ridotti di un valore determinato dal tipo di scudo:
    \begin{itemize}
        \item \textbf{Scudo Leggero:} Riduzione 1D4.
        \item \textbf{Scudo Medio:} Riduzione 1D6.
        \item \textbf{Scudo Pesante:} Riduzione 1D8.
    \end{itemize}
\end{itemize}


\vspace{0.5cm}
\noindent
\begin{center}
\rule{\textwidth}{0.4pt} 
\end{center}
\vspace{0.5cm}

\section{Calcolo dei Danni}
Se l'attacco va a segno, vengono calcolati i danni:

\begin{itemize}
    \item \textbf{Mani Nude:} 1D3 danni.
    \item \textbf{Armi Leggere:} 1D4 danni.
    \item \textbf{Armi Medie:} 1D6 danni.
    \item \textbf{Armi Pesanti:} 1D8 danni.
\end{itemize}

I danni sono ridotti dallo scudo in caso di parata riuscita o da altri modificatori.

\[
\textnormal{Danni Finali} = \textnormal{Danni Arma} - \textnormal{Riduzione Scudo (se parato)}
\]

\vspace{0.5cm}
\noindent
\begin{center}
\rule{\textwidth}{0.4pt} 
\end{center}
\vspace{0.5cm}

\section{Combattimento con Due Armi}
Un personaggio può impugnare due armi ad una mano. In questo caso:

\begin{itemize}
    \item L'attaccante tira 4D6 per il TDS e considera i solamente i tre dadi con il risultato più alto.
    \item Vengono calcolati i danni della coppia di armi, sommando i danni delle due armi singole.
\end{itemize}

\section{Chiusura del Turno}

Al termine del turno di un personaggio, si risolvono eventuali effetti in sospeso come:
\begin{itemize}
  \item Ferite persistenti o condizioni fisiche (sanguinamento, stordimento).
  \item Effetti magici temporanei.
  \item Bonus o penalità applicabili solo per un singolo turno.
\end{itemize}

Il personaggio può inoltre spostarsi prima o dopo l’azione principale, salvo restrizioni da parte di ambientazione, condizioni o effetti.


\vspace{0.5cm}
\noindent
\begin{center}
\rule{\textwidth}{0.4pt} 
\end{center}
\vspace{0.5cm}

\section*{Esempi di un Turno di Combattimento}
Per chiarire il funzionamento delle regole di combattimento, consideriamo un paio di esempi:

\begin{quote}
Valon, un guerriero con 12 in Forza e 8 in Armi a due mani, attacca un bandito. Il bandito ha 10 in Reattività e indossa una corazza media (2 di protezione).

Il valore offensivo di Valon è:
\[
12 [\emph{Forza}] + 8 [\emph{Armi a due mani}] = 20
\]

Il valore difensivo del bandito è:
\[
\frac{10}{2} [\emph{Contributo Reattività}] + \frac{2}{2} [\emph{Protezione}] = 6
\]

Valon esegue il suo TDS (deve ottenere un risultato inferiore a 14). Tira 3D6 e ottiene 7:
\[
20 - 6 - 7 = 7
\]
Il risultato è maggiore di 0, quindi l’attacco ha successo.

Valon calcola i danni della sua arma, uno spadone pesante:
\[
1D8 \rightarrow 6
\]

Il bandito non tenta di parare. I danni non sono mitigati da alcuna fonte, quindi i danni finali sono:
\[
6
\]
Il bandito subisce sei danni.
\end{quote}


\begin{quote}
Kallist, un guerriero con 11 in Forza e 10 in Armi ad una mano, attacca un soldato con una coppia di spade (pesantezza media). Il soldato indossa: corazza, elmo, schinieri, bracciali e scudo (tutti i pezzi sono leggeri). Il soldato ha 12 in Forza, 9 in Reattività, 8 in Parare.\\
Il valore offensivo di Kallist è:
\[
11 [\emph{Forza}] + 10 [\emph{Armi ad una mano}] = 21
\]
 Il valore difensivo del soldato è:
\[
\frac{9}{2} [\emph{Contributo Reattività}] + \frac{4}{2} [\emph{Protezione}] \approx 6
\]
Kallist esegue il suo TDS (deve ottenere un risultato inferiore a 15). Kallist utilizza 4D6 e scarta il risultato del dado più basso perché combatte con una coppia di armi:
\[
1 4 5 4 \rightarrow 13 
\]
L'attacco ha successo. Kallist calcola i danni:\\
\[
2D6 [\emph{1D6 per ciascula delle due spade}] \rightarrow 5 + 1 = 6
\]
Il soldato prova a parare con lo scudo.
\[
12 [\emph{Forza}] + 8 [\emph{Parare}] - 10 [\emph{Abilità combattimento Kallist}] = 10
\]
Tira 3D6 e ottiene 7, riuscendo a parare. Il soldato mitiga il danno di 1D4.\\
\[
1D4 \rightarrow 4
\]
Vengono quindi calcolati i danni finali:
\[
6 - 4 = 2
\]
Il soldato subisce solamente due danni.
\end{quote}



\end{document}