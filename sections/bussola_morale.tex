\documentclass[../manuale_main.tex]{subfiles}



\begin{document}

\section{La Bussola Morale}

In \textbf{Zarafir}, l’allineamento classico dei personaggi, spesso definito da etichette come “Legale Buono” o “Caotico Malvagio”, viene sostituito da un sistema più flessibile e profondo basato sulla \textbf{Bussola Morale}. Questo approccio permette ai giocatori di definire la personalità e le convinzioni dei loro personaggi con maggiore sfumatura, promuovendo una narrazione più ricca e complessa.

\vspace{0.3cm}

\subsection{Cos'è la Bussola Morale?}
La \textbf{Bussola Morale} è un insieme di principi, convinzioni e valori che guidano le azioni del personaggio. A differenza del classico sistema di allineamenti, la Bussola Morale non è un’etichetta fissa, ma una guida dinamica che può evolversi con il personaggio nel corso delle avventure.

\begin{itemize}
    \item \textbf{Principi Fondamentali:} I valori centrali che il personaggio considera sacri (onestà, lealtà, libertà, giustizia, potere, sopravvivenza...).
    \item \textbf{Limiti e Sacrifici:} Ciò che il personaggio è disposto a fare (o non fare) per proteggere questi valori.
    \item \textbf{Reazione ai Dilemmi Morali:} Come il personaggio reagisce quando è costretto a scegliere tra due principi in conflitto.
    \item \textbf{Rapporto con gli Altri:} Come tratta i deboli, i nemici o gli sconosciuti.
    \item \textbf{Attitudine verso l'Autorità:} La sua opinione su regole, leggi e figure di potere.
\end{itemize}

\vspace{0.3cm}

\subsection{Creare la Bussola Morale}
I giocatori sono incoraggiati a descrivere la Bussola Morale del proprio personaggio rispondendo a queste domande:

\begin{itemize}
    \item \textbf{Quali sono i valori fondamentali del tuo personaggio?} (Onestà, lealtà, libertà, giustizia, potere, sopravvivenza...)
    \item \textbf{Cosa è disposto a fare per proteggere questi valori?}
    \item \textbf{Come reagisce davanti a un dilemma morale?}
    \item \textbf{Come tratta i deboli, i nemici o gli sconosciuti?}
    \item \textbf{Qual è il suo rapporto con l’autorità e le regole?}
    \item \textbf{Quali principi non è disposto a violare?}
\end{itemize}

\vspace{0.5cm}
\rule{\textwidth}{0.4pt}
\vspace{0.5cm}

\subsection{Esempi di Bussole Morali}
Per aiutare i giocatori a creare la propria Bussola Morale, ecco alcuni esempi che mostrano come diversi principi possano dare vita a personaggi unici e complessi:

\vspace{0.3cm}

\textbf{L'Onorevole Protettore}  
Questo personaggio crede che la protezione dei deboli sia il suo dovere più alto. Non abbandona mai un alleato e considera il tradimento un crimine imperdonabile.  

\vspace{0.2cm}

\textbf{L'Ombra Pragmatica}  
Per questo personaggio, la sopravvivenza è la priorità assoluta. È disposto a mentire, rubare o persino uccidere se necessario, ma evita la crudeltà fine a sé stessa.  

\vspace{0.2cm}

\textbf{Il Giudice Implacabile}  
Convinto che il male debba essere punito senza eccezioni, questo personaggio non offre redenzione ai nemici e crede che la giustizia sia più importante della pietà.  

\vspace{0.2cm}

\textbf{Il Cercatore di Libertà}  
La libertà è il valore più alto per questo personaggio. Si oppone a ogni forma di oppressione e si ribella alle autorità che ritiene ingiuste.  

\vspace{0.5cm}
\rule{\textwidth}{0.4pt}
\vspace{0.5cm}

\subsection{Utilizzare la Bussola Morale nel Gioco}
La \textbf{Bussola Morale} del personaggio non è una semplice dichiarazione, ma una guida che il giocatore può seguire per interpretare il proprio personaggio. In caso di dilemmi morali o scelte difficili, il Master può richiamare il giocatore alla sua Bussola Morale, ricordandogli cosa guida il suo personaggio.

\begin{itemize}
    \item \textbf{Scelte Coerenti:} I giocatori possono utilizzare la Bussola Morale per mantenere il loro personaggio coerente anche in situazioni difficili.
    \item \textbf{Sfide Morali:} Il Master può creare situazioni che mettono alla prova i principi del personaggio, offrendo occasioni di crescita e sviluppo narrativo.
    \item \textbf{Evoluzione e Cambiamento:} La Bussola Morale non è immutabile. Eventi significativi, incontri importanti o scoperte personali possono cambiarla.
\end{itemize}

\vspace{0.3cm}

\subsection{Modificare la Bussola Morale}
Nel corso della storia, la Bussola Morale del personaggio può evolversi. Eventi traumatici, scelte difficili o esperienze significative possono trasformare la sua visione del mondo.

\begin{itemize}
    \item \textbf{Cambiamento Drammatico:} Un personaggio che inizia come protettore dei deboli potrebbe diventare un vendicatore spietato dopo aver perso qualcuno di caro.
    \item \textbf{Nuove Prospettive:} Un personaggio che disprezzava l'autorità potrebbe sviluppare un senso di giustizia e responsabilità dopo aver guidato un gruppo di sopravvissuti.
    \item \textbf{Coerenza Narrativa:} Ogni cambiamento dovrebbe essere giustificato dagli eventi vissuti dal personaggio, creando un arco narrativo coerente.
\end{itemize}

\vspace{0.3cm}

\subsection{La Bussola Morale come Strumento Narrativo}
La Bussola Morale non è solo una caratteristica del personaggio, ma uno strumento narrativo che il Master può utilizzare per arricchire la storia. Può essere una fonte di conflitti, una guida per prendere decisioni difficili e un modo per creare momenti di crescita personale.

\begin{itemize}
    \item \textbf{Dilemmi Morali:} Mettere i personaggi di fronte a scelte difficili che mettono alla prova la loro Bussola Morale.
    \item \textbf{Crescita e Sviluppo:} Offrire ai personaggi la possibilità di cambiare e maturare, influenzati dalle loro esperienze.
    \item \textbf{Interazioni Significative:} Utilizzare la Bussola Morale per arricchire i dialoghi tra personaggi, creando momenti di confronto e riflessione.
\end{itemize}

\vspace{0.3cm}

\textbf{In Zarafir, la Bussola Morale non è solo un aspetto del personaggio: è il cuore della sua storia.}


\end{document}