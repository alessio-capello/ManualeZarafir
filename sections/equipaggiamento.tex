\documentclass[../manuale_main.tex]{subfiles}



\begin{document}
Spade, armature, scudi e bastoni incisi non sono solo strumenti di guerra: sono simboli d’identità, eredità familiari, talismani silenziosi. Il modo in cui un personaggio si presenta in battaglia racconta molto della sua storia.

\section{Armi}

Le armi sono suddivise in tre categorie principali, in base alla loro maneggevolezza e potenziale distruttivo. Non esiste una lista completa: le armi sono descritte in modo generico per lasciare spazio all’interpretazione.

\begin{itemize}
  \item \textbf{Armi leggere}: facili da impugnare, adatte a colpi rapidi o al combattimento con due mani. Non ostacolano movimenti agili o uso della magia.\\
  \emph{Esempi}: pugnali, coltelli cerimoniali, mazze corte, spade snelle.

  \item \textbf{Armi medie}: equilibrano potenza e controllo. Spesso sono l’arma principale di chi combatte in prima linea.\\
  \emph{Esempi}: spade lunghe, asce, bastoni da guerra, lance leggere.

  \item \textbf{Armi pesanti}: richiedono forza o addestramento. Possono causare ferite devastanti, ma sono lente o ingombranti.\\
  \emph{Esempi}: martelli a due mani, spadoni, alabarde, asce bipenni.
\end{itemize}

Il tipo di arma può influenzare i tiri del TDS, la portata o il modo in cui si usa un’abilità.

Un personaggio può impugnare:
\begin{itemize}
  \item Un’arma a una mano e uno scudo.
  \item Due armi a una mano (una in ciascuna mano).
  \item Un’arma a due mani.
  \item Un’arma a distanza.
\end{itemize}

L'uso di due armi può offrire vantaggi narrativi o meccanici, ma richiede un'abilità superiore per risultare efficace. Il combattimento con due armi è descritto nel capitolo \textbf{12 - Combattimento}.


\section{Scudi}

\noindent
Gli scudi offrono una protezione aggiuntiva e influenzano direttamente la difesa del personaggio. Possono anche essere usati per manovre difensive o di contatto ravvicinato.

\begin{itemize}
  \item \textbf{Scudo leggero}: piccolo e maneggevole, utile per deviare colpi senza intralciare i movimenti.
  \item \textbf{Scudo medio}: bilancia difesa e ingombro, usato comunemente da guerrieri.
  \item \textbf{Scudo pesante}: grande protezione ma penalizza i movimenti rapidi e ostacola l’uso di magie con gesti complessi.
\end{itemize}

\section{Armature}
Le armature proteggono il corpo ma influenzano anche il modo in cui ci si muove, si reagisce o si lancia magia. Possono essere creazioni tecniche, rituali o artistiche.

\begin{itemize}
  \item \textbf{Armatura leggera}: realizzata in tessuti spessi, cuoio o materiali flessibili. Perfetta per chi preferisce agilità alla resistenza.
  \item \textbf{Armatura media}: include corazze a scaglie, piastre leggere o combinazioni flessibili. Offre buona protezione senza sacrificare troppo la mobilità.
  \item \textbf{Armatura pesante}: piastre rigide, cotte di maglia completa, corazze rituali. Ideale per chi regge la prima linea, ma limita movimenti e gestualità magica.
\end{itemize}

Ogni armatura migliora il Valore Difensivo del personaggio (vedi capitolo  \textbf{12 - Combattimento}). Il Master può imporre malus di Reattività o penalità a incantesimi per armature pesanti.

\section*{Qualità degli Oggetti}

Ogni arma o armatura può essere personalizzata con tratti narrativi o qualità speciali. Alcuni esempi:

\begin{itemize}
  \item \emph{Sacra}, \emph{Corrotta}, \emph{Druida}, \emph{Forgiata nella lava}, \emph{Ereditaria}, \emph{Runica}, \emph{Animale}, \emph{Ultraterrena}, \emph{Intessuta con peli di chimera}.
\end{itemize}

Queste qualità possono non avere impatti meccanici diretti, ma il Master può decidere di conferire bonus o effetti situazionali.

\end{document}