\documentclass[../manuale_main.tex]{subfiles}

\begin{document}


Le trappole non sono semplici meccanismi di danno, ma strumenti narrativi: enigmi, ostacoli ambientali, difese arcane o colpi di scena tesi a suscitare tensione. In \textbf{Zarafir}, le trappole vengono gestite con semplicità e discrezionalità, senza la necessità di un sottosistema rigido.

Il Master stabilisce:
\begin{itemize}
\item \textbf{se una trappola è presente},
\item \textbf{quale minaccia rappresenta},
\item \textbf{quali condizioni ne attivano l’effetto},
\item \textbf{quali prove sono richieste per individuarla, evitarla o disattivarla}.
\end{itemize}

Chi intende creare, disattivare o individuare una trappola effettua un \textbf{TDS su ``Utilizzare Trappole''}, oppure su \textbf{``Percezione''} se il suo intento è soltanto rilevare segnali di pericolo.

Il Master può introdurre modificatori a seconda del tempo a disposizione, della qualità dei materiali o della coerenza della proposta del giocatore.

\vspace{0.4cm}
\noindent\textbf{Esempi di Trappole Narrative}

\begin{itemize}
\item \textbf{Corda di giuramento:} un anello di spago intrecciato che, se calpestato, avvolge le caviglie del bersaglio e gli impone, tramite simboli silenti, di pronunciare ad alta voce il proprio intento.
\item \textbf{Urna delle spine:} un vaso di terracotta sepolto nel terreno. Se calpestato, si frantuma liberando una nube di spine secche, sottili come capelli, che si conficcano nella pelle.
\item \textbf{Serratura insanguinata:} una chiusura magica che si sblocca con il sangue, ma punisce chi forza la serratura con un dolore mentale istantaneo.
\item \textbf{Sigillo del guardiano addormentato:} un simbolo inciso sul pavimento. Se attivato, richiama l’eco spirituale di un guardiano caduto, che veglia silenziosamente per 1 minuto, pronto a punire chi infrange un patto sacro.
\end{itemize}

\vspace{0.4cm}
\noindent\textbf{Esempio di utilizzo da parte di un giocatore}

\begin{quote}
Lira, ladra esperta, vuole tendere un agguato a un inseguitore nella cripta. Il giocatore dice al Master: “Vorrei costruire una trappola usando un’ampolla d’olio, dei frammenti di vetro e il mio coltello. Quando qualcuno calpesta l’area, l’ampolla si rompe, il pavimento si fa scivoloso e chi prova a muoversi si ferisce con i vetri”.

Il Master considera l’idea creativa e fattibile. Stabilisce che servono 10 minuti e un \textbf{TDS su ``Utilizzare Trappole''} con difficoltà 4.

Lira tira i dadi. In caso di successo, il primo nemico che oltrepassa la zona scivolerà e dovrà superare un TDS su “Acrobazia” o cadrà rovinosamente, subendo danni.
\end{quote}

\vspace{0.4cm}
\noindent\textbf{Suggerimento per i Master:} 
Puoi trattare le trappole come incontri ambientali. Alcune possono servire da deterrente, altre rivelare la natura o l’ingegno di chi le ha costruite. Lascia che siano un’opportunità per la narrazione e non solo una fonte di danni.


\end{document}
