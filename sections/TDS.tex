\documentclass[../manuale_main.tex]{subfiles}



\begin{document}

Il \textbf{Tiro di Simulazione (TDS)} è il meccanismo fondamentale di \textbf{Zarafir} per stabilire il successo o il fallimento delle azioni intraprese dai personaggi. Non tutte le azioni richiedono un tiro di dado: le \textbf{azioni banali}, ovvero quelle che il personaggio può eseguire senza difficoltà in condizioni normali, sono considerate riuscite automaticamente.


\vspace{0.5cm}
\noindent
\begin{center}
\rule{\textwidth}{0.4pt} 
\end{center}
\vspace{0.5cm}

\section{Azioni Banalmente Riuscite}
Le azioni banali sono tutte quelle che non presentano alcun rischio di fallimento, a meno che non siano influenzate da fattori esterni. Alcuni esempi comuni includono:

\begin{itemize}
    \item \textbf{Camminare} lungo una strada piana.
    \item \textbf{Raccogliere} un oggetto da terra.
    \item \textbf{Conversare} con un mercante.
\end{itemize}

Se però l'azione banale viene complicata da fattori esterni (come camminare su un terreno scivoloso o raccogliere un oggetto in un ambiente pieno di trappole), diventa necessario un Tiro di Simulazione.


\vspace{0.5cm}
\noindent
\begin{center}
\rule{\textwidth}{0.4pt} 
\end{center}
\vspace{0.5cm}

\section{Come Funziona il TDS}
Il Tiro di Simulazione segue una procedura semplice e intuitiva:

\begin{enumerate}
    \item \textbf{Il Giocatore Descrive l'Azione:} Il giocatore spiega al Master cosa vuole che il suo personaggio faccia.
    
    \item \textbf{Il Master Stabilisce la Difficoltà:} La difficoltà dell'azione viene definita in base alla complessità dell'azione stessa:
    \begin{itemize}
        \item \textbf{0:} Azione estremamente semplice (scassinare una serratura priva di sicurezza).
        \item \textbf{5:} Azione di difficoltà media (scassinare una serratura standard).
        \item \textbf{10:} Azione estremamente complessa (scassinare una serratura magica protetta).
    \end{itemize}
    
    \item \textbf{Il Giocatore Esegue il Tiro:} 
    \begin{itemize}
        \item Lancia 3D6 (tre dadi a sei facce).
        \item Somma il risultato ottenuto ai modificatori applicabili.
        \item Confronta questo totale con la somma di:
        \[
        \textbf{Caratteristica Base + Valore dell’Abilità + Modificatori - Difficoltà}
        \]
    \end{itemize}
    
    \item \textbf{Determinare il Risultato:} 
    \begin{itemize}
        \item Se il risultato del tiro è \textbf{inferiore o uguale} al valore totale calcolato, l'azione ha successo.
        \item Se è \textbf{superiore}, l'azione fallisce.
    \end{itemize}
\end{enumerate}

\vspace{0.3cm}

\section{Esempio di TDS}
Un personaggio vuole saltare da un tetto all'altro:

\begin{itemize}
    \item \textbf{Caratteristica di Riferimento:} Agilità (10).
    \item \textbf{Abilità:} Acrobazia (3).
    \item \textbf{Difficoltà:} 7 (determinata dal Master per la distanza tra i tetti).
\end{itemize}

\[
10 \, (\textnormal{Agilità}) + 3 \, (\textnormal{Acrobazia}) - 7 \, (\textnormal{Difficoltà}) = 6
\]

Il giocatore tira 3D6 e ottiene un 5.

\begin{itemize}
    \item 5 è inferiore a 6 → \textbf{L'azione ha successo:} il personaggio compie il salto con successo.
\end{itemize}

\vspace{0.5cm}
\noindent
\begin{center}
\rule{\textwidth}{0.4pt} 
\end{center}
\vspace{0.5cm}

\section{Successo e Fallimento Critico}
In \textbf{Zarafir}, il tiro di 3D6 può produrre due risultati speciali, indipendentemente dalle caratteristiche e dalla difficoltà:

\begin{itemize}
    \item \textbf{Fallimento Critico:} Se il risultato dei 3D6 è 6, 6, 6, l'azione fallisce automaticamente, anche se il personaggio ha caratteristiche eccezionali.
    \item \textbf{Successo Critico:} Se il risultato dei 3D6 è 1, 1, 1, l'azione ha successo automaticamente, anche se la difficoltà era estremamente alta.
\end{itemize}

\vspace{0.3cm}

\section{Prove Contrapposte}
In alcune situazioni, i personaggi non devono solo superare una difficoltà fissa, ma devono confrontarsi direttamente contro un’altra creatura. Queste situazioni richiedono una \textbf{Prova Contrapposta}.

\begin{enumerate}
    \item \textbf{Entrambi i partecipanti tirano 3D6.}
    \item \textbf{Ciascun partecipante somma il risultato dei dadi alla propria Caratteristica} e, se pertinente, alla propria Abilità.
    \item \textbf{Chi ottiene il risultato più alto vince la prova.}
\end{enumerate}

\textbf{Esempio: Sfida di Braccio di Ferro}

\begin{itemize}
    \item \textbf{Personaggio A:} Forza 12, Abilità Mani Nude 2, Tiro 3D6 = 9.
    \[
    12 + 2 + 9 = 23
    \]
    
    \item \textbf{Personaggio B:} Forza 10, Abilità Mani Nude 4, Tiro 3D6 = 11.
    \[
    10 + 4 + 11 = 25
    \]
    
    \item \textbf{Risultato:} Il Personaggio B vince la sfida di braccio di ferro.
\end{itemize}

\vspace{0.5cm}
\noindent
\begin{center}
\rule{\textwidth}{0.4pt} 
\end{center}
\vspace{0.5cm}

\section{Il Ruolo del Master}
Il Master ha un ruolo fondamentale nel determinare la difficoltà delle azioni e nel gestire le Prove Contrapposte. Per garantire una buona esperienza di gioco:

\begin{itemize}
    \item \textbf{Assegnare Difficoltà Coerenti:} La difficoltà dovrebbe riflettere la complessità reale dell'azione.
    \item \textbf{Valutare le Descrizioni:} Premiate i giocatori che descrivono le loro azioni in modo creativo e dettagliato.
    \item \textbf{Gestire i Fallimenti Creativamente:} Un fallimento non deve essere sempre disastroso, ma può portare a nuove sfide.
\end{itemize}

\vspace{0.3cm}

\section{Il TDS come Strumento Narrativo}
Il Tiro di Simulazione non è solo un meccanismo di gioco, ma uno \textbf{strumento narrativo} per arricchire la storia. Utilizzatelo per:

\begin{itemize}
    \item \textbf{Aumentare la Tensione:} I TDS sono perfetti per momenti di suspense.
    \item \textbf{Gestire l'Imprevisto:} Anche le azioni più semplici possono diventare drammatiche in circostanze difficili.
    \item \textbf{Favorire la Creatività:} I giocatori sono incoraggiati a descrivere le loro azioni per ottenere bonus o vantaggi.
\end{itemize}

\textbf{In Zarafir, il Tiro di Simulazione è un mezzo per rendere l’avventura dinamica e coinvolgente, non una barriera per l’immaginazione.}

\end{document}