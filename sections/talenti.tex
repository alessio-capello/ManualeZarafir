\documentclass[../manuale_main.tex]{subfiles}



\begin{document}

I Talenti in Nightfall sono capacità speciali che arricchiscono il personaggio, rendendolo unico e permettendogli di eccellere in situazioni specifiche. A differenza delle abilità, che rappresentano competenze generali e possono essere migliorate con l’esperienza, i talenti sono doni particolari, vantaggi speciali o capacità straordinarie che il personaggio sviluppa nel corso delle sue avventure.

I talenti possono rappresentare una maestria pratica, una capacità mentale o Spirituale, una qualità sociale unica. In generale sono vantaggi che i personaggi possono avere in situazioni particolari.

L'ottenimento di un talento può essere stabilito dal Master in base a fattori come il background del personaggio, eventi particolari, o acquisiti tramite oggetti magici particolari.


\section{Lista dei Talenti}

La lista dei talenti che segue non è esaustiva, poiché nuovi talenti potrebbero essere aggiunti dal Master, dipendentemente dall'ambientazione e dal tipo di narrazione che vuole adottare.

\begin{center}
\textbf{\large{Accampamento Sicuro}}\\
\end{center}
Permette al personaggio di allestire un accampamento sicuro e nascosto. Chi prova ad individuare l'accampamento subirà una difficoltà aumentata.

\begin{itemize}
\item Esempi Narrativi: Un ranger che nasconde l'accampamento del gruppo tra le rocce, o un druido che utilizza la vegetazione per celarlo alla vista.
\end{itemize}

\vspace{0.5cm}\rule{\textwidth}{0.4pt}\vspace{1cm}

\begin{center}
\textbf{\large{Amico dei Reietti}}\\
\end{center}
Il personaggio ha una naturale affinità con emarginati e diseredati. Può ottenere informazioni o rifugio dai mendicanti, criminali o reietti di una città.

\begin{itemize}
\item Esempi Narrativi: Un avventuriero che ottiene informazioni preziose da un mendicante o che trova rifugio nella rete delle fogne della città.
\end{itemize}

\vspace{0.5cm}\rule{\textwidth}{0.4pt}\vspace{1cm}

\begin{center}
\textbf{\large{Camminare nell’Ombra}}\\
\end{center}
Il personaggio può muoversi inosservato anche in luoghi affollati, evitando l’attenzione indesiderata.

\begin{itemize}
\item Esempi Narrativi: Un ladro che si muove inosservato attraverso una sala da ballo affollata o una spia che si mescola tra la folla senza essere riconosciuta.
\end{itemize}

\vspace{0.5cm}\rule{\textwidth}{0.4pt}\vspace{1cm}

\begin{center}
\textbf{\large{Cambio Rapido}}\\
\end{center}
Il personaggio può passare dall’utilizzo di un’arma a distanza a un’arma da mischia (e viceversa) senza perdere un turno di combattimento.

\begin{itemize}
\item Esempi Narrativi: Un ranger che passa rapidamente dal suo arco alla spada quando i nemici si avvicinano.
\end{itemize}

\vspace{0.5cm}\rule{\textwidth}{0.4pt}\vspace{1cm}

\begin{center}
\textbf{\large{Colpo Impetuoso}}\\
\end{center}
Se il personaggio infligge un colpo critico con un'arma da mischia, può effettuare immediatamente un altro attacco gratuito contro lo stesso bersaglio.

\begin{itemize}
\item Esempi Narrativi: Un barbaro che abbatte un nemico con un colpo devastante e ne ferisce immediatamente un altro.
\end{itemize}

\vspace{0.5cm}\rule{\textwidth}{0.4pt}\vspace{1cm}

\begin{center}
\textbf{\large{Cuoco da Campo}}\\
\end{center}
Il personaggio è in grado di preparare pasti nutrienti anche con ingredienti semplici, riducendo lo Stress Mentale dei suoi alleati dopo ogni pasto.

\begin{itemize}
\item Esempi Narrativi: Un soldato che cucina una zuppa calda per i suoi compagni durante un assedio.
\end{itemize}

\vspace{0.5cm}\rule{\textwidth}{0.4pt}\vspace{1cm}

\begin{center}
\textbf{\large{Diplomatico Abile}}\\
\end{center}
Il personaggio ottiene un bonus alla difficoltà per convincere, negoziare o calmare una folla.

\begin{itemize}
\item Esempi Narrativi: Un ambasciatore che placa una disputa tra due nobili rivali.
\end{itemize}

\vspace{0.5cm}\rule{\textwidth}{0.4pt}\vspace{1cm}

\begin{center}
\textbf{\large{Erborista Naturale}}\\
\end{center}
Quando il personaggio raccoglie erbe, piante o funghi, ottiene sempre una quantità superiore al normale. Può identificare automaticamente le proprietà curative o velenose di una pianta.

\begin{itemize}
\item Esempi Narrativi: Un alchimista che raccoglie foglie curative lungo il sentiero e riconosce subito una pianta velenosa.
\end{itemize}

\vspace{0.5cm}\rule{\textwidth}{0.4pt}\vspace{1cm}

\begin{center}
\textbf{\large{Favore del Popolo}}\\
\end{center}
Il personaggio è naturalmente amato dal popolo. Può ottenere rifugio, cibo o informazioni da persone comuni senza difficoltà.

\begin{itemize}
\item Esempi Narrativi: Un eroe che trova ospitalità in una casa di contadini durante una fuga.
\end{itemize}

\vspace{0.5cm}\rule{\textwidth}{0.4pt}\vspace{1cm}

\begin{center}
\textbf{\large{Guaritore da Campo}}\\
\end{center}
Il personaggio può stabilizzare i feriti anche in condizioni critiche senza bisogno di strumenti. I feriti stabilizzati da lui guariscono più velocemente durante il riposo.

\begin{itemize}
\item Esempi Narrativi: Un guaritore che ferma un’emorragia con nient'altro che fasce di fortuna e erbe.
\end{itemize}

\vspace{0.5cm}\rule{\textwidth}{0.4pt}\vspace{1cm}

\begin{center}
\textbf{\large{Guardia Esperta}}\\
\end{center}
Quando utilizza uno scudo, il personaggio ottiene un bonus di +2 alla difficoltà per essere colpito e può parare attacchi diretti a un alleato adiacente.

\begin{itemize}
\item Esempi Narrativi: Un cavaliere che protegge un mago dalle frecce nemiche con il suo scudo.
\end{itemize}

\vspace{0.5cm}\rule{\textwidth}{0.4pt}\vspace{1cm}

\begin{center}
\textbf{\large{Immune ai Condizionamenti}}\\
\end{center}
Il personaggio è immune a qualsiasi effetto che alteri la sua volontà o percezione (fascinazione, controllo mentale, paura magica).

\begin{itemize}
\item Esempi Narrativi: Un mago che resiste al controllo mentale di un necromante o un guerriero che non viene intimidito da una creatura spaventosa.
\end{itemize}

\vspace{0.5cm}\rule{\textwidth}{0.4pt}\vspace{1cm}

\begin{center}
\textbf{\large{Invisibile all'Occhio Mistico}}\\
\end{center}
L'aura magica del personaggio è completamente nascosta. Non può essere individuato tramite incantesimi di rilevamento magico o sensori mistici.

\begin{itemize}
\item Esempi Narrativi: Un mago che si nasconde da un cacciatore di maghi o un ladro che utilizza un anello magico per passare inosservato tra le guardie.
\end{itemize}

\vspace{0.5cm}\rule{\textwidth}{0.4pt}\vspace{1cm}

\begin{center}
\textbf{\large{Magia Insanguinata}}\\
\end{center}
Il personaggio può scegliere di lanciare una magia utilizzando i propri Punti Vita al posto dei Punti Stress. Ogni 2 punti di Stress Mentale necessario viene speso un Punto Vita. Questa scelta deve essere fatta prima di lanciare la magia e non può far scendere i Punti Vita sotto lo zero.

\begin{itemize}
\item Esempi Narrativi: Un mago che sacrifica parte della sua salute per lanciare un incantesimo devastante in un momento critico.
\end{itemize}

\vspace{0.5cm}\rule{\textwidth}{0.4pt}\vspace{1cm}

\begin{center}
\textbf{\large{Magia Intuitiva}}\\
\end{center}
Il personaggio può lanciare magie senza bisogno di formule verbali o gesti, ma la difficoltà per farlo aumenta di +2.

\begin{itemize}
\item Esempi Narrativi: Un mago che sussurra un incantesimo senza muovere le labbra, o che lancia un dardo magico semplicemente guardando un bersaglio.
\end{itemize}

\vspace{0.5cm}\rule{\textwidth}{0.4pt}\vspace{1cm}

\begin{center}
\textbf{\large{Mente di Ferro}}\\
\end{center}
Il personaggio aumenta la sua soglia di Stress Mentale di +5.

\begin{itemize}
\item Esempi Narrativi: Un mago che mantiene la calma anche sotto un bombardamento mentale o un guerriero che non cede alla paura nemmeno di fronte a un drago.
\end{itemize}

\vspace{0.5cm}\rule{\textwidth}{0.4pt}\vspace{1cm}

\begin{center}
\textbf{\large{Percezione del Menzognero}}\\
\end{center}
Quando qualcuno mente al personaggio, egli ottiene automaticamente indizi che suggeriscono l’inganno (tic nervosi, sguardo sfuggente, cambiamenti nel tono della voce).

\begin{itemize}
\item Esempi Narrativi: Un diplomatico che riconosce subito un traditore durante una trattativa segreta.
\end{itemize}

\vspace{0.5cm}\rule{\textwidth}{0.4pt}\vspace{1cm}

\begin{center}
\textbf{\large{Pellegrino Infaticabile}}\\
\end{center}
Il personaggio può camminare per giorni senza subire penalità per la fatica. Quando guida un gruppo, anche i suoi alleati possono marciare più a lungo senza stancarsi.

\begin{itemize}
\item Esempi Narrativi: Un esploratore che guida una carovana attraverso il deserto senza sosta.
\end{itemize}

\vspace{0.5cm}\rule{\textwidth}{0.4pt}\vspace{1cm}

\begin{center}
\textbf{\large{Poliglotta Naturale}}\\
\end{center}
Il personaggio è in grado di comprendere e parlare rapidamente nuove lingue, anche solo ascoltandole per un breve periodo.

\begin{itemize}
\item Esempi Narrativi: Un mercante che negozia con una tribù sconosciuta dopo aver ascoltato il loro dialetto per pochi minuti.
\end{itemize}

\vspace{0.5cm}\rule{\textwidth}{0.4pt}\vspace{1cm}

\begin{center}
\textbf{\large{Presa di Ferro}}\\
\end{center}
Il personaggio non può essere disarmato.

\begin{itemize}
\item Esempi Narrativi: Un guerriero che mantiene salda la presa sul suo spadone anche sotto l'assalto di numerosi nemici.
\end{itemize}

\vspace{0.5cm}\rule{\textwidth}{0.4pt}\vspace{1cm}

\begin{center}
\textbf{\large{Riparatore da Campo}}\\
\end{center}
Il personaggio può riparare armi, armature o attrezzi anche senza una vera e propria officina, utilizzando materiali improvvisati.

\begin{itemize}
\item Esempi Narrativi: Un fabbro che ripara una spada spezzata con un pezzo di ferro trovato in una caverna.
\end{itemize}

\vspace{0.5cm}\rule{\textwidth}{0.4pt}\vspace{1cm}

\begin{center}
\textbf{\large{Sovraccarico Mentale}}\\
\end{center}
Il personaggio lancia le sue magie con un modificatore di -3 alla difficoltà, ma ogni volta che fallisce subisce un incremento di 5 punti Stress Mentale.

\begin{itemize}
\item Esempi Narrativi: Un mago che canalizza un potere arcano incontrollabile, ottenendo effetti devastanti a rischio della sua sanità mentale.
\end{itemize}

\vspace{0.5cm}\rule{\textwidth}{0.4pt}\vspace{1cm}

\begin{center}
\textbf{\large{Vibrosensi}}\\
\end{center}
Il personaggio può vedere e sentire tramite gli occhi e le orecchie (o organi di senso affini) delle sue evocazioni.

\begin{itemize}
\item Esempi Narrativi: Un evocatore che utilizza uno spirito come esploratore, vedendo attraverso i suoi occhi.
\end{itemize}

\vspace{0.5cm}\rule{\textwidth}{0.4pt}\vspace{1cm}

\begin{center}
\textbf{\large{Voce del Leader}}\\
\end{center}
Il personaggio emana un carisma naturale che ispira gli altri. Quando guida un gruppo o tiene un discorso, i suoi alleati ottengono un bonus di +1 ai loro TDS per il prossimo conflitto o situazione critica.

\begin{itemize}
\item Esempi Narrativi: Un comandante che sprona i suoi soldati prima di una battaglia decisiva.
\end{itemize}

\vspace{0.5cm}\rule{\textwidth}{0.4pt}\vspace{1cm}


\end{document}