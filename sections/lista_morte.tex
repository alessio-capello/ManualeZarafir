\documentclass[./magie.tex]{subfiles}

\begin{document}


\section{Incantesimi - Arte della Morte}

Gli incantesimi dell'Arte della Morte attingono alle forze oscure della decomposizione, dell’anima e dell’oltretomba. Offrono potere attraverso la corruzione, la sofferenza o la manipolazione dei defunti, permettendo all'incantatore di piegare la morte stessa alla propria volontà.

\vspace{0.2cm}
{\zarafirtitlefont\Large\bfseries Primo Cerchio - Arte della Morte}

\begin{itemize}

\item \textbf{Lingua Infausta} \\
\textbf{Difficoltà:} 2 \\
\textbf{Durata:} 3 round \\
\textbf{Gittata:} 5 metri \\
\textbf{Descrizione:} L’incantatore maledice la voce del bersaglio. Per la durata dell’incanto (3 round), ogni ordine impartito, formula magica o tentativo di persuasione ha effetto contrario a quanto desiderato. Se tenta di lanciare una magia, \textbf{la difficoltà aumenta di +3}

\vspace{0.5cm}\rule{\textwidth}{0.4pt}\vspace{1cm}

\item \textbf{Bomba Necrotica} \\
\textbf{Difficoltà:} 3 \\
\textbf{Durata:} Istantaneo \\
\textbf{Gittata:} 10 metri \\
\textbf{Descrizione:} L'incantatore lancia un globo denso di miasmi mortali, che esplode in un’area di 2 metri. I bersagli subiscono \textbf{1D4 danni}. Se colpisce un solo bersaglio, i danni diventano \textbf{1D6}.

\vspace{0.5cm}\rule{\textwidth}{0.4pt}\vspace{1cm}
\clearpage
\item \textbf{Dolore} \\
\textbf{Difficoltà:} 2 \\
\textbf{Durata:} 3 round \\
\textbf{Gittata:} 5 metri \\
\textbf{Descrizione:} L’incantatore schiaccia la mente del bersaglio con un’ondata di sofferenza illusoria. Il bersaglio (con \textbf{Sensibilità ≤ 9}) subisce \textbf{1D3 danni} ogni round. Al termine dell’incanto, recupera immediatamente tutti i punti vita persi in questo modo.

\vspace{0.5cm}\rule{\textwidth}{0.4pt}\vspace{1cm}

\item \textbf{Arte Rigenerativa} \\
\textbf{Difficoltà:} 2 \\
\textbf{Durata:} Istantaneo \\
\textbf{Gittata:} Contatto \\
\textbf{Descrizione:} L'incantatore impone una rigenerazione dolorosa e innaturale a un bersaglio. Questo recupera \textbf{1D4 punti vita}, ma per l'ora successiva all'incanto ogni azione fisica subisce \textbf{+1 alla difficoltà}, a causa del dolore lancinante.

\end{itemize}

\clearpage
\vspace{0.2cm}
{\zarafirtitlefont\Large\bfseries Secondo Cerchio - Arte della Morte}

\begin{itemize}

\item \textbf{Armatura di Nebbia} \\
\textbf{Difficoltà:} 3 \\
\textbf{Durata:} 2 round \\
\textbf{Gittata:} 5 metri \\
\textbf{Descrizione:} Il corpo del bersaglio viene avvolto da vapori sepolcrali. Per la durata, ha \textbf{+1 alla difficoltà per essere colpito in mischia} e \textbf{+2 per essere colpito da attacchi a distanza}. Se colpito, la nebbia si dissolve in un lamento.

\vspace{0.5cm}\rule{\textwidth}{0.4pt}\vspace{1cm}

\item \textbf{Spirito Vendicativo} \\
\textbf{Difficoltà:} 4 \\
\textbf{Durata:} 3 round \\
\textbf{Gittata:} 1 metro \\
\textbf{Descrizione:} L'incantatore evoca uno spirito fatto di rancore e ombra. Scompare se distrutto o al termine della durata. Le caratteristiche dell'evocazione sono descritte nell capitolo 17 - Bestiario, nella sezione delle evocazioni note.

\vspace{0.5cm}\rule{\textwidth}{0.4pt}\vspace{1cm}

\item \textbf{Occhi della Putrefazione} \\
\textbf{Difficoltà:} 5 \\
\textbf{Durata:} 2 round \\
\textbf{Gittata:} 10 metri \\
\textbf{Descrizione:} Gli occhi del bersaglio si riempiono di visioni di corpi divorati e carni marcite. Per la durata, ogni prova che richieda concentrazione, precisione o mira subisce \textbf{+2 alla difficoltà}.

\vspace{0.5cm}\rule{\textwidth}{0.4pt}\vspace{1cm}

\item \textbf{Conoscenza della Morte} \\
\textbf{Difficoltà:} 2 \\
\textbf{Durata:} 20 minuti \\
\textbf{Gittata:} Sé stesso \\
\textbf{Descrizione:} Concentrandosi su un cadavere, l'incantatore ottiene una fugace visione degli ultimi istanti di vita della vittima. Può determinare la causa della morte e percepire eventuali emozioni residue (paura, dolore, vendetta).

\end{itemize}

\clearpage
\vspace{0.2cm}
{\zarafirtitlefont\Large\bfseries Terzo Cerchio - Arte della Morte}

\begin{itemize}

\item \textbf{Finta Morte} \\
\textbf{Difficoltà:} 5 \\
\textbf{Durata:} Un'ora \\
\textbf{Gittata:} Sé stesso \\
\textbf{Descrizione:} Il cuore rallenta, il sangue si raffredda. Il corpo dell'incantatore simula una morte perfetta: nessun battito, nessuna reazione. Durante questo stato l'incantatore non può agire e resta del tutto vulnerabile. Ideale per ingannare nemici, nascondersi dai sensi magici o sfuggire a un'esecuzione imminente.

\vspace{0.5cm}\rule{\textwidth}{0.4pt}\vspace{1cm}

\item \textbf{Visione Terrificante} \\
\textbf{Difficoltà:} 6 \\
\textbf{Durata:} 2 round \\
\textbf{Gittata:} 10 metri \\
\textbf{Descrizione:} Il bersaglio assiste a una visione straziante della propria morte o del crollo delle persone care.  Per la durata, ogni volta che il bersaglio tenta di attaccare o lanciare una magia, deve superare un TDS su \textbf{Forza di Volontà} (difficoltà 5) o perdere automaticamente l'azione, consumato da presagi funesti e voci interiori. Se fallisce due volte consecutive, fugge alla cieca per 1 round.

\vspace{0.5cm}\rule{\textwidth}{0.4pt}\vspace{1cm}
\clearpage
\item \textbf{Rituale} \\
\textbf{Difficoltà:} 0 \\
\textbf{Durata:} Permanente \\
\textbf{Gittata:} Contatto \\
\textbf{Descrizione:} L’incantatore rimuove il cuore da una creatura senziente appena defunta e vi sigilla la sua anima, creando una \textbf{Reliquia}. Il cuore continua a pulsare debolmente fino all’utilizzo o alla distruzione, il resto corpo si riduce in polvere. Questo rituale richiede concentrazione assoluta e non può essere interrotto. Una Reliquia può essere consumata per uno dei seguenti effetti:
\begin{itemize}
  \item Recuperare \textbf{6 Punti Vita}.
  \item Recuperare \textbf{4 punti di Stress Mentale}.
  \item Velenizzare 1 litro d'acqua, rendendola letale.
  \item Servire come catalizzatore per \emph{Evocazione Blasfema}.
\end{itemize}
Ogni utilizzo consuma la reliquia. L'incanto richiede concentrazione totale.
Ogni azione richiede un’intera azione da parte dell’incantatore (eccetto la quarta, inglobata nell’incanto).

\vspace{0.5cm}\rule{\textwidth}{0.4pt}\vspace{1cm}

\item \textbf{Teschio di Morte} \\
\textbf{Difficoltà:} 7 \\
\textbf{Durata:} Istantaneo \\
\textbf{Gittata:} 20 metri \\
\textbf{Descrizione:} Un teschio incandescente vola in linea retta dal palmo dell'incantatore. Ogni creatura attraversata subisce \textbf{2D6 danni} che ignorano armatura e resistenza magica. Se almeno due bersagli muoiono, il teschio esplode, infliggendo \textbf{1D6 danni} aggiuntivi in un raggio di 2 metri.

\end{itemize}

\clearpage
\vspace{0.2cm}
{\zarafirtitlefont\Large\bfseries Quarto Cerchio - Arte della Morte}

\begin{itemize}

\item \textbf{Animare i Morti} \\
\textbf{Difficoltà:} 10 \\
\textbf{Durata:} Permanente \\
\textbf{Gittata:} Contatto \\
\textbf{Descrizione:} L'incantatore restituisce una parvenza di vita a un essere defunto, purché il suo teschio sia integro. Il corpo si rigenera magicamente, assumendo un aspetto spettrale o cadaverico, e la creatura torna a muoversi con coscienza.\\

La creatura rianimata mantiene frammenti della propria personalità, memorie sbiadite, e una forma alterata della propria volontà: la sua \textbf{bussola morale può essere deviata} dal trauma della morte e dalla manipolazione magica. Sebbene resti legata all'incantatore e costretta a obbedirgli, può provare disagio, esitazione o persino rifiuto se gli ordini violano la sua nuova etica.\\

Le sue \textbf{caratteristiche vengono ridotte di 1 punto} rispetto ai valori originari (con un minimo di 4 per ogni statistica). Il corpo non può recuperare punti vita naturalmente, ma può essere curato tramite magia. La creatura è immune a malattie, veleno e paura.\\

Questo incanto può essere usato per riportare in vita un \textbf{personaggio giocante}. In tal caso, il giocatore conserva il controllo del personaggio, ma dovrà \textbf{rivedere la propria bussola morale e ridefinire alcuni tratti psicologici}, d'accordo con il master. L’effetto è permanente e può essere rimosso solo tramite incanti di livello superiore o con un rituale specifico legato all’anima del defunto.\\

Richiede una \textbf{concentrazione massima} e può essere lanciato solo una volta al giorno.


\vspace{0.5cm}\rule{\textwidth}{0.4pt}\vspace{1cm}

\item \textbf{Gli Occhi del Defunto} \\
\textbf{Difficoltà:} 7 \\
\textbf{Durata:} Un’ora \\
\textbf{Gittata:} 10 chilometri \\
\textbf{Descrizione:} L’incantatore entra in comunione con un teschio da lui posseduto, vedendo e udendo tutto ciò che accade attorno a esso. Il collegamento si dissolve al termine dell’effetto, bruciando il teschio in una fiamma verdastra.

\vspace{0.5cm}\rule{\textwidth}{0.4pt}\vspace{1cm}

\item \textbf{Tocco della Fame} \\
\textbf{Difficoltà:} 4 \\
\textbf{Durata:} 4 round \\
\textbf{Gittata:} 15 metri \\
\textbf{Descrizione:} Il bersaglio viene colto da una fame e sete insaziabili. Per la durata non può beneficiare di cure né guarigioni, ed è distratto da un bisogno viscerale. Tutte le sue azioni subiscono \textbf{+1 alla difficoltà}. Non ha effetto sui non-morti.

\vspace{0.5cm}\rule{\textwidth}{0.4pt}\vspace{1cm}
\clearpage
\item \textbf{La Mente è Superiore al Corpo} \\
\textbf{Difficoltà:} 0 \\
\textbf{Durata:} Permanente \\
\textbf{Gittata:} Sé stesso \\
\textbf{Descrizione:} L'incantatore rinuncia al vigore del corpo per rafforzare la propria mente. Perde \textbf{1 punto in Forza, Agilità, Costituzione e Reattività} e ottiene \textbf{+1 in Intelligenza, Sensibilità e Carisma}. Nessuna caratteristica può scendere sotto 3. L’effetto è irreversibile e spaventosamente lucido. A discrezione del Master, come catalizzatore del rituale, l'incantatore potrebbe dover sacrificare una parte del proprio corpo.

\vspace{0.5cm}\rule{\textwidth}{0.4pt}\vspace{1cm}

\item \textbf{Infettare} \\
\textbf{Difficoltà:} 8 \\
\textbf{Durata:} Progressiva \\
\textbf{Gittata:} Contatto \\
\textbf{Descrizione:} L'incantatore contamina il sangue di un bersaglio con un morbo magico. Dopo un’ora, se non curata, l’infezione si diffonde: il bersaglio comincia a fallire azioni fisiche (+2 alla difficoltà) e perde 1 punto vita ogni ora. A discrezione del Master, l'infezione potrebbe essere fermata solo con magie di purificazione o antidoti rari.

\vspace{0.5cm}\rule{\textwidth}{0.4pt}\vspace{1cm}

\end{itemize}

\clearpage
\vspace{0.2cm}
{\zarafirtitlefont\Large\bfseries Quinto Cerchio - Arte della Morte}

\begin{itemize}

\item \textbf{Notte Maledetta} \\
\textbf{Difficoltà:} 9 \\
\textbf{Durata:} 3 round \\
\textbf{Gittata:} Sé stesso \\
\textbf{Descrizione:} L’incantatore estende un’ombra viva in un raggio di 1 chilometro: la luce scompare, e ogni creatura viva percepisce voci sussurrate, ricordi altrui e presenze invisibili.\\
Durante la durata:
\begin{itemize}
\item Tutte le creature \textbf{devono superare un TDS su Forza di Volontà (difficoltà 8)} ogni round per agire normalmente.
\item Chi fallisce, agisce con \textbf{+2 difficoltà} o perde il turno (a discrezione del master).
\item Le evocazioni dell’incantatore ottengono \textbf{+2 ai danni} e appaiono più terrificanti del normale.
\end{itemize}

\vspace{0.5cm}\rule{\textwidth}{0.4pt}\vspace{1cm}

\item \textbf{Scheletro di Spine} \\
\textbf{Difficoltà:} 9 \\
\textbf{Durata:} Istantaneo \\
\textbf{Gittata:} 20 metri \\
\textbf{Descrizione:} Le ossa del bersaglio si contorcono e lacerano il corpo dall’interno. Infligge \textbf{3D6 danni} che ignorano armatura e resistenza magica. Se il danno è letale, il corpo si libera della carne e si rianima come \textbf{scheletro senz’anima} sotto il controllo dell’incantatore per \textbf{4 round}, agendo con un comportamento predatorio e semplice.


\vspace{0.5cm}\rule{\textwidth}{0.4pt}\vspace{1cm}
\clearpage
\item \textbf{Sia Fatta la Mia Volontà} \\
\textbf{Difficoltà:} 10 \\
\textbf{Durata:} Permanente \\
\textbf{Gittata:} 2 metri \\
\textbf{Descrizione:} L’incantatore tenta di soggiogare la volontà di un bersaglio. Si effettua una \textbf{Prova Contrapposta} tra la sua abilità \emph{Tessimagie} e la \emph{Resistenza Magica} del bersaglio (che ha un malus di -2). Se l’incantatore vince, ottiene il controllo mentale della creatura. Le creature con \textbf{Sensibilità ≥ 15} sono immuni. Il vincolo può essere spezzato solo tramite un rituale di purificazione o morte.

\vspace{0.5cm}\rule{\textwidth}{0.4pt}\vspace{1cm}

\item \textbf{Evocazione Blasfema} \\
\textbf{Difficoltà:} 10 \\
\textbf{Durata:} 6 round \\
\textbf{Gittata:} 10 metri \\
\textbf{Descrizione:} Consumando una \textbf{Reliquia}, l’incantatore richiama una creatura da un altro piano. Lancia \textbf{1D6} per determinare quale evocazione si manifesta:
\begin{itemize}
\item \textbf{1} – Demonietto: crudele e agile, ama giocare con il fuoco.
\item \textbf{2} – Grande Spirito Vendicativo: spettro violento e instabile.
\item \textbf{3} – Cerbero: feroce bestia a tre teste.
\item \textbf{4} – Mietitore Abissale: un'entità silenziosa, ma letale.
\item \textbf{5} – Demone: creatura d’odio e distruzione.
\item \textbf{6} – Drago Zombi: la carne morta di un antico predatore.
\end{itemize}
L'evocazione agisce al turno successivo a quello dell'incantatore. Le caratteristiche dell'evocazione sono descritte nell capitolo 17 - Bestiario, nella sezione delle evocazioni note.
\end{itemize}


\clearpage
\vspace{0.2cm}
{\zarafirtitlefont\Large\bfseries Sesto Cerchio - Arte della Morte}

\begin{itemize}

\item \textbf{Maledizione Eterna} \\
\textbf{Difficoltà:} 11 \\
\textbf{Durata:} Permanente \\
\textbf{Gittata:} Contatto \\
\textbf{Descrizione:} Il tocco dell’incantatore imprime un marchio magico sulla vittima. Il bersaglio subisce un \textbf{-2 permanente a tutte le Caratteristiche}. La maledizione può essere rimossa solo con un potente rituale o una magia superiore, decisa dal Master.

\vspace{0.5cm}\rule{\textwidth}{0.4pt}\vspace{1cm}

\item \textbf{Ascensione del Lich} \\
\textbf{Difficoltà:} 12 \\
\textbf{Durata:} Permanente \\
\textbf{Gittata:} Sé stesso \\
\textbf{Descrizione:} L’incantatore compie un rito oscuro, sigillando la propria anima in un oggetto noto come \textbf{filatterio}. Da quel momento, egli perde ogni traccia di vita e si risveglia come \textbf{Lich}, ottenendo immunità a veleno, malattie, paura e controllo mentale.\\
Se il filatterio viene distrutto, l’incantatore viene disgregato definitivamente. Il filatterio \textbf{deve essere portato addosso} ma non può essere protetto da magia comune. Questo incanto è irreversibile, e chi lo lancia è considerato \textbf{non-morto}. Il Lich infatti non può recuperare punti vita naturalmente, ma può essere curato tramite magia. Il lich è inoltre immune a fame, sete, malattie e veleno.\\

\vspace{0.5cm}\rule{\textwidth}{0.4pt}\vspace{1cm}
\clearpage
\item \textbf{Lamento delle Anime} \\
\textbf{Difficoltà:} 11 \\
\textbf{Durata:} Istantaneo \\
\textbf{Gittata:} 30 metri (cono) \\
\textbf{Descrizione:} L’incantatore evoca un’onda di lamenti spettrali. Tutte le creature nemiche nell’area devono effettuare un TDS su \textbf{Forza di volontà (difficoltà 10)}. In caso di fallimento, subiscono \textbf{4D6 danni}. In caso di successo, la metà. Le vittime sono tormentate da visioni di coloro che hanno tradito, ucciso o dimenticato.

\vspace{0.5cm}\rule{\textwidth}{0.4pt}\vspace{1cm}

\item \textbf{Frantuma Volontà} \\
\textbf{Difficoltà:} 12 \\
\textbf{Durata:} 3 round \\
\textbf{Gittata:} 15 metri \\
\textbf{Descrizione:} L’incantatore frammenta la mente del bersaglio con visioni di morte, rimorso e decadenza. Ogni round, il bersaglio deve superare un TDS su \textbf{Forza di Volontà (difficoltà 9)} o:
\begin{itemize}
\item \textbf{1° fallimento:} ha \textbf{+3 alla difficoltà} per ogni azione.
\item \textbf{2° fallimento:} perde l’azione del turno, tremando o urlando.
\item \textbf{3° fallimento:} cade in uno stato catatonico per 1D4 +1 round.
\end{itemize}
Creature immuni al terrore non subiscono effetti. Un solo successo interrompe l’effetto.

\vspace{0.5cm}\rule{\textwidth}{0.4pt}\vspace{1cm}
\clearpage
\item \textbf{Nido dell’Abisso} \\
\textbf{Difficoltà:} 11 \\
\textbf{Durata:} 10 minuti \\
\textbf{Gittata:} 100 metri \\
\textbf{Descrizione:} L’incantatore apre un varco tra il mondo dei vivi e quello dei morti, sovrapponendoli. Tutte le creature morte entro l’area appaiono come \textbf{ombre persistenti}, visibili e interrogabili da chiunque abbia Sensibilità ≥ 9.\\
Ogni incanto dell’Arte della Morte lanciato nell’area riceve \textbf{-2 alla difficoltà}. 

\end{itemize}



\end{document}
