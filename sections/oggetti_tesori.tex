\documentclass[../manuale_main.tex]{subfiles}



\begin{document}
Nel corso di un’avventura, ogni personaggio si troverà a interagire con una moltitudine di oggetti: strumenti comuni, armamenti, reliquie dimenticate, artefatti carichi di potere. In questo sistema, ogni oggetto può avere un ruolo funzionale, simbolico o narrativo.

\section{Oggetti Comuni}

Gli oggetti comuni sono quelli che i personaggi possono acquistare, creare o trovare con una certa facilità: corde, lanterne, pergamene, pugnali, libri, utensili. Non è necessario elencarli tutti. I giocatori e il Master sono liberi di descrivere l'equipaggiamento con ragionevolezza e gusto narrativo. Un oggetto comune non offre modifiche meccaniche, ma può essere usato creativamente per ottenere vantaggi contestuali.

Un oggetto comune può concedere un vantaggio se utilizzato in modo intelligente e contestuale. Il Master può richiedere una prova di abilità per determinarne l’efficacia.

\section{Valore degli Oggetti}

Il valore di un oggetto è espresso in tre possibili unità:
\begin{itemize}
  \item \textbf{Comune} – facilmente reperibile o replicabile.
  \item \textbf{Raro} – difficile da trovare o costruire, spesso con valore culturale o tecnico.
  \item \textbf{Unico} – esemplare irripetibile, dotato di valore storico, simbolico o magico.
\end{itemize}

\textbf{Regola opzionale:} Il valore può influenzare la difficoltà di reperimento, la resistenza, e il tipo di materiali necessari alla riparazione. In caso di vendita, il valore può essere moltiplicato per 10 da Comune a Raro, e da Raro a Unico.

\section{Oggetti Rotti e Deperibili}

Ogni oggetto può deteriorarsi con l’uso, gli urti o le intemperie. Un oggetto può essere:
\begin{itemize}
  \item \textbf{Integro} – utilizzabile senza limitazioni.
  \item \textbf{Danneggiato} – funziona, ma subisce penalità (a discrezione del Master: svantaggi, modificatori ridotti, rischio di rottura).
  \item \textbf{Rotto} – inutilizzabile finché non riparato.
\end{itemize}

\textbf{Regola opzionale:} Quando un personaggio fallisce un tiro di attacco o utilizza un oggetto in condizioni critiche, il Master può chiedere un tiro di Reattività o Costituzione per evitare che l’oggetto diventi danneggiato. Un secondo fallimento potrebbe causarne la rottura.

\section{Riparare un Oggetto}

\noindent
Per riparare un oggetto rotto è necessaria una prova di Abilità pertinente (Es. Artigianato, Forgia, Meccanica). La difficoltà dipende dalla complessità dell’oggetto e dal livello di danno.

\begin{itemize}
  \item Oggetti \textbf{Comuni}: difficoltà bassa, materiali reperibili.
  \item Oggetti \textbf{Rari}: difficoltà media, componenti specifici.
  \item Oggetti \textbf{Unici o magici}: difficoltà elevata, condizioni speciali richieste.
\end{itemize}

\subsection{Regola opzionale dettagliata:}
\begin{itemize}
  \item Un oggetto danneggiato può essere riparato con una prova a difficoltà 2--3.
  \item Un oggetto rotto richiede una prova a difficoltà 4--6 e almeno un’ora di lavoro.
  \item Un fallimento peggiora lo stato (da danneggiato a rotto o richiede nuovi materiali).
\end{itemize}

\section{Oggetti Magici}

Un oggetto magico è una reliquia, arma, strumento o manufatto dotato di un potere attivo o passivo che trascende la normalità. Può avere una funzione semplice o un’influenza profonda sulla narrazione.

\medskip
\textbf{Esempi di Oggetti Magici:}
\begin{itemize}
  \item \textbf{Anello di Cenere}: chi lo indossa può dissolversi per un istante in un turbine di fumo, evitando un colpo al giorno.
  \item \textbf{Lama del Ricordo}: ogni volta che infligge un colpo mortale, sussurra al portatore il nome dell’anima uccisa.
  \item \textbf{Maschera dei Tre Volti}: permette di modificare l’aspetto fisico a piacimento, ma ogni uso riduce temporaneamente l’Empatia.
  \item \textbf{Ampolla del Sussurro}: chi beve il liquido può parlare direttamente alla mente di una persona, ovunque essa sia.
\end{itemize}

\textbf{Uso al Tavolo:} 
\begin{itemize}
  \item Oggetti magici possono avere effetti \textbf{attivabili} (1 volta al giorno, con tiro su un'abilità o senza tiro);
  \item effetti \textbf{passivi} (es. bonus costanti a una caratteristica);
  \item oppure \textbf{effetti narrativi} (malus segreti, condizioni morali, mutazioni).
\end{itemize}

\end{document}