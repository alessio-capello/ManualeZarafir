\documentclass[../manuale_main.tex]{subfiles}

\begin{document}
\section*{Tiro di Simulazione (TDS)}
\begin{itemize}
  \item \textbf{Formula}: Caratteristica + Abilit\`a vs Difficolt\`a + 3D6
  \item \textbf{Critico}: \emph{Successo} con somma < 4; \emph{Fallimento} con somma > 17
  \item \textbf{Prove Contrapposte}: Caratteristica + Abilit\`a + 3D6 \rightarrow Vince chi ha lo scarto maggiore
\end{itemize}

\section*{Punti Vita (PV)}
\begin{itemize}
  \item \textbf{PV Max} = Costituzione $\times$ 1.5 + modificatori
  \item \textbf{A 0 PV}: si sviene, si \`e in pericolo
  \item \textbf{Sotto 0 PV}: in fin di vita, servono cure immediate
\end{itemize}

\section*{Stress Mentale}
\begin{itemize}
  \item \textbf{Accumulo}: Durante il lancio di magie: pari alla difficolt\`a della magia; aumenta anche in situazioni mentalmente ed emotivamente sfidanti
  \item \textbf{Soglia} = Sensibilit\`a + Intelligenza
  \item \textbf{Effetti}: da lieve disagio ad allucinazioni e dissociazione
  \item \textbf{Recupero}: con riposo, conforto, o rituali
\end{itemize}

\section*{Magia e Resistenza}
\begin{itemize}
  \item \textbf{Lancio}: TDS con abilità magica + caratteristica; difficoltà stabilita dall'incanto
  \item \textbf{Stress Mentale}: sempre inflitto in base alla difficolt\`a
  \item \textbf{Resistere}: TDS con Resistenza Magica (Sensibilit\`a); la difficolt\`a \`e il livello magico dell'incantatore
\end{itemize}

\section*{Abilit\`a e Caratteristiche}
\begin{multicols}{2}
\begin{itemize}
  \item Armi a distanza -- Agilit\`a
  \item Armi a due mani -- Forza
  \item Armi a una mano -- Forza
  \item Mani nude -- Forza
  \item Parare -- Forza
  \item Resistenza Magica -- Sensibilit\`a
  \item Stregoneria di Attacco -- Intelligenza
  \item Stregoneria di Controllo -- Intelligenza
  \item Stregoneria della Vita -- Intelligenza
  \item Tessimagie -- Intelligenza
  \item Acrobazia -- Reattivit\`a
  \item Artigianato -- Intelligenza
  \item Comando -- Carisma
  \item Convincere -- Carisma
  \item Domare Animali -- Sensibilit\`a
  \item Erboristeria -- Sensibilit\`a
  \item Forza di Volont\`a -- Sensibilit\`a
  \item Furtivit\`a -- Reattivit\`a
  \item Insegnamento -- Carisma
  \item Malaffare -- Reattivit\`a
  \item Medicina -- Sensibilit\`a
  \item Navigazione -- Sensibilit\`a
  \item Percezione -- Sensibilit\`a
  \item Schivare -- Reattivit\`a
  \item Seguire Tracce -- Sensibilit\`a
  \item Utilizzare Trappole -- Reattivit\`a
\end{itemize}
\end{multicols}

\section*{Valori di Combattimento}
\begin{itemize}
  \item \textbf{Valore Offensivo} = Caratteristica + Abilit\`a d'attacco
  \item \textbf{Armatura} = Somma delle protezioni offerte dai singoli pezzi di armatura (Corazza, Elmo, Guanti e Schinieri) /2
  \item \textbf{Valore Difensivo} = Reattivit\`a/2 + Armatura
\end{itemize}

\section*{Combattimento: Riepilogo}
\begin{itemize}
  \item \textbf{Iniziativa}: determinata dal Master o da un tiro di Reattivit\`a
  \item \textbf{Attacco}: l'attaccante effettua un TDS contro il Valore Difensivo del bersaglio
  \item \textbf{Parata}: il difensore pu\`o effettuare un TDS con l'abilit\`a Parare, usando come difficoltà il valore dell'abilità usata dall'attaccante
  \item \textbf{Calcolo danni}: Quando l'attacco va a segno, i danni sono quelli previsti dall'arma + modificatori ridotti in caso di parata con successo
  \item \textbf{Combattimenti sbilanciati}: per evitare situazioni in cui si colpisce automaticamente, si considera fallito ogni attacco con 3D6 \(\geq 15\)
\end{itemize}

\section*{Equipaggiamento}
\begin{itemize}
  \item Oggetti: Comune, Raro, Unico
  \item Stato: Integro, Danneggiato, Rotto
  \item Riparazione: con TDS su abilità pertinente (Artigianato)
\end{itemize}

\subsection*{Valori indicativi di armi e scudi}
\begin{itemize}
  \item \textbf{Mani Nude}: 1D3 \quad \textbf{Armi Leggere}: 1D4 \quad \textbf{Armi Medie}: 1D6 \\ \textbf{Armi Pesanti}: 1D8
  \item \textbf{Scudi Leggeri}: 1D4 \quad \textbf{Scudi Medi}: 1D6 \quad \textbf{Scudi Pesanti}: 1D8
\end{itemize}


\end{document}
