\documentclass[../manuale_main.tex]{subfiles}



\begin{document}
Questa sezione contiene una serie di condizioni di esempio nelle quali i personaggi potrebbero trovarsi.\\
Non è una lista esaustiva in quanto vengono considerati principalmente gli status che potrebbero essere inflitti usando le magie presenti in questo manuale. Considerate questi esempi come una guida di riferimento per sapere come calibrare i modificatori e trarre spunto per la gestione di alcune meccaniche condizioni che non sono descritte, come un personaggio affaticato o basso di morale.\\

\subsection{Paura e Terrore} 
\paragraph{Paura}\mbox{}\\
Un personaggio impaurito sarà restio a compiere un’azione collegata a ciò che gli ha causato la paura. Per rappresentare questo comportamento, prima di svolgere una di quelle azioni, il personaggio deve fare un TDS su “Forza di volontà”. In caso di fallimento, nello svolgere l’azione ha un modificatore di +2 alla difficoltà. La difficoltà del TDS è decisa dal master.
\paragraph{Terrore}\mbox{}\\
Il personaggio è atterrito. È incapace di ragionare e di agire lucidamente. Per rappresentare questo comportamento, all’inizio del suo turno, il personaggio deve fare un TDS su “Forza di volontà”. In caso di fallimento, a discrezione del master, il personaggio tenterà di fuggire da ciò che lo terrorizza oppure salterà il suo turno, rimanendo paralizzato dalla paura. In caso di successo avrà un modificatore di +3 alla difficoltà per ogni azione che vorrà svolgere. La difficoltà del TDS su “Forza di volontà” è decisa dal master.
\subsection{Svenimento e Addormentamento} 
\paragraph{Svenimento}\mbox{}\\
Un personaggio svenuto è completamente inerme e senza possibilità di ribellarsi fino a che non viene fatto rinvenire. Egli è quindi vittima di ogni attacco che gli venga sferrato contro (in questo caso verranno ignorati tutti i bonus dati dalla capacità di “Evitare” del personaggio). Lo svenimento può essere causato dalla perdita di punti vita o mana, da un particolare veleno o un evento troppo traumatico per il personaggio stesso.
\paragraph{Addormentamento}\mbox{}\\
Un personaggio addormentato non potrà agire finché non tornerà cosciente; questo può avvenire a causa di fattori esterni (essere colpito da un attacco, la presenza di forti rumori, ecc) o autonomamente. Mentre è addormentato un personaggio è sottoposto alle stesse regole dello svenimento.
\clearpage
\subsection{Avvelenamento} 
Il personaggio è stato avvelenato, dipendentemente dal tipo di veleno si distinguono casistiche differenti.\\
Verranno mostrati alcuni tipi di veleni di esempio per farvi capire quali possono essere gli effetti sul personaggio.
\begin{itemize}
\item Debole: questo tipo di veleno è molto semplice da produrre ma i suoi effetti non sono letali per l’organismo. Questo veleno infligge 1 danno all’inizio di ogni turno del personaggio (non può essere ridotto). Il danno inflitto in questo modo non può essere letale. Questo veleno dopo breve tempo viene neutralizzato dal corpo della vittima, per questo motivo può infliggere al massimo 5 danni e può essere utilizzato nuovamente sullo stesso bersaglio dopo qualche settimana.
\item Narcotico: questo tipo di sostanza non ha effetti particolarmente dannosi per il corpo della vittima, ma lo espone a conseguenze potenzialmente letali. Dipendentemente dalla concentrazione e dal modo in cui è stato ottenuto, avrà effetti più o meno importanti. In linea generale i movimenti e le capacità di reazione della vittima saranno rallentati, fino a causarne lo svenimento o la sonnolenza in caso di grossi quantitativi di veleno. 
\item Letale: dipendentemente dall’ambientazione, questo veleno è considerato letale: una volta somministrato, se non viene utilizzato l’antidoto, la morte è certa. A seconda del tipo di veleno questa può avvenire più o meno velocemente.
\end{itemize}


\end{document}