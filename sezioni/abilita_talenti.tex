\documentclass[../manuale_main.tex]{subfiles}



\begin{document}

Questo manuale utilizza un sistema di abilità e specializzazioni per la costruzione del proprio personaggio.

L'abilità e punteggio relativo ad essa indicano la capacità del personaggio di svolgere un'azione (relativamente generica); la specializzazione indica in quale attività il personaggio sia più ferrato.

 \emph{L'abilità ``Armi a una mano” offre l'indicazione che il personaggio sia un combattente che si destreggia con armi a una mano, ma non offre informazioni aggiuntive; la specializzazione ``Uso spade” è molto più restrittiva, ma offre dei bonus importanti quando il personaggio userà una spada anziché un'altra arma a una mano.}

\subsection{Migliorare le proprie abilità}
Per alzare il livello delle proprie abilità vengono utilizzati i punti esperienza assegnati dal master al termine di una o più sessioni.\\
Per alzare di un livello una propria abilità è necessario spendere tanti punti esperienza pari al livello che si vuole raggiungere.\\
\emph{Per portare l'abilità ``Armi a una mano” dal libello 5 al 6, sarà necessario spendere 6 punti esperienza. Conseguentemente per innalzare la stessa abilità dal livello 5 al livello 7 sarà necessario spendere 13 punti esperienza (6+7).}\\
Quando si desidera apprendere una nuova abilità (ovvero portandola dal livello 0 all'1) è necessario inoltre superare un TDS a difficoltà 1 sull'abilità che si intende imparare oltre a spendere 1 punto esperienza. \\
Il master potrebbe far ignorare il TDS purché il personaggio abbia svolto un'azione legata all'abilità da apprendere.\\
\emph{Per imparare ad utilizzare armi a una mano, è necessario superare un TDS su ``Armi ad una mano” nel momento in cui si investe il punto abilità. Indipendentemente dal risultato, il punto sarà stato consumato. Solamente se si avrà avuto successo si sarà ottenuta l'abilità all'1.}

\subsection{Specializzazioni e Talenti}
Ogni abilità ha diverse specializzazioni e talenti; le specializzazioni migliorano un determinato utilizzo dell'abilità, il talento offre dei bonus superiori relativi all'abilità stessa.

Le specializzazioni sono divise in due livelli ciascuna, i bonus ottenuti con il livello 1 di una specializzazione non vanno perduti ottenendo il livello 2, ma si sommano.
I talenti invece non hanno livelli, ma possono essere ottenuti solamente possedendo una specializzazione al livello 2.

Il livello 1 della specializzazione indica (a livello generale) che il personaggio conosce i rudimenti per svolgere una specifica azione o nell'utilizzo di un determinato tipo di armi.

Non possedere una specializzazione non implica il non poter svolgere l'azione relativa (a meno che non sia indicato diversamente nella descrizione dell'abilità), ma non si avranno a disposizione i bonus (e in alcuni casi si potrebbero avere dei malus).

Anche dove non è esplicitamente indicato, i talenti sono riferiti solamente ad azioni relative all'oggetto o abilità con la quale il talento è stato ottenuto: il talento ``Presa di ferro” relativo ad ``Armi a una mano”, impedirà di essere disarmati laddove si usi un'arma coerente con l'abilità; non impedirà di essere disarmati imbracciando un altro tipo di arma (per esempio, un arco o una lancia).

Si ottengono in totale 3 punti da assegnare alle specializzazioni o talenti dell'abilità portandola dallo 0 al 10:
\begin{itemize}
\item il primo punto si guadagna ottenendo l'abilità stessa (portandola quindi dallo 0 all'1);
\item il secondo punto si ottiene portandola al 6;
\item il terzo punto si ottiene portandola al 10.
\end{itemize}
I punti specializzazione ottenuti in un'abilità possono essere spesi solamente nelle specializzazioni o talenti relativi a quell'abilità.

Conseguentemente sarà possibile (per ogni abilità si potrà optare per una combinazione diversa, a discrezione del giocatore) avere:
\begin{itemize}
\item una specializzazione al 2 e un talento;
\item una specializzazione al 2 e una seconda specializzazione all'1;
\item tre specializzazioni all'1.
\end{itemize}
I punti Specializzazione devono essere investiti all'ottenimento del punto stesso, non possono essere accumulati.

Non è possibile riassegnare i punti specializzazione (a meno che il master non ve lo permetta a seguito di eventi specifici), per cui scegliete bene in cosa investirli.


\subsection{Lista abilità e talenti}

In questa sezione verranno elecante tutte le abilità con relative specializzazioni e talenti.
Al fianco del nome dell'abilità è indicata la caratteristica di riferimento. Dove è indicato \emph{Passiva} significa che l'abilità non ha un uso attivo (non verranno eseguiti TDS su di essa), ma fornisce bonus permanenti al personaggio.\\
È segnata la descrizione dell'abilità, corredata con alcuni esempi d'uso per le abilità meno intuitive.
Le specializzazioni e i talenti sono indicati sotto a ogni abilità, le specializzazioni sono distinte nei due livelli che si possono ottenere.


%abilità da combattimento
\subsubsection{Abilità da combattimento}


\renewcommand{\arraystretch}{1.2}
\begin{tabularx}{\linewidth}{m|s|b}
\hline
           \textbf{ Armi a distanza}      &     \textit{\textbf{Agilità}}      &      Questa abilità si utilizza per colpire i nemici con armi da distanza come archi, balestre o giavellotti.     \\
\hline
\multicolumn{3}{c}{\textbf{Specializzazioni}}           \\
\hline
\multirow{2}{*}{\textit{Archi}} &1 &     Questo tipo di arma non è più considerata come arma improvvisata.    \\
                  & 2&            Il personaggio ha ottenuto una buona maestria con questo tipo di armi, avendo un modificatore di -1 alla difficoltà.   \\\hline
\multirow{2}{*}{\textit{Balestre}} &  1  &   Questo tipo di arma non è più considerata come arma improvvisata.      \\
                  &  2    &          Il personaggio ha ottenuto una buona maestria con questo tipo di armi, avendo un modificatore di -1 alla difficoltà.   \\ \hline
\multirow{2}{*}{\textit{Armi da lancio}} &  1  &   Questo tipo di arma non è più considerata come arma improvvisata.      \\
                  &  2    &          Il personaggio ha ottenuto una buona maestria con questo tipo di armi, avendo un modificatore di -1 alla difficoltà.   \\ 
\hline
\multicolumn{3}{c}{\textbf{Talenti}}           \\
\hline
      \multicolumn{2}{l}{  \textit{Disarmare}  } & Il personaggio può disarmare un bersaglio colpito con armi a distanza. \\\hline
          \multicolumn{2}{l}{  \textit{Punte Acuminate}   } &Il personaggio ignora ogni armatura con grado di protezione pari o inferiore a 2 se utilizza armi a distanza. \\\hline
         \multicolumn{2}{l}{   \textit{Cambio Rapido}      } &Passare dall'utilizzo di armi a distanza a quelle per il corpo a corpo (e viceversa) non richiede di perdere un turno di combattimento. \\\hline
         \multicolumn{2}{l}{   \textit{Armi Pesanti}     }&Il personaggio non ha difficoltà addizionale (data dall'arma) se utilizza un'arma pesante a distanza.\\\hline
          \multicolumn{2}{l}{  \textit{Esperto}      }&Per personaggio nessuna arma da distanza (che fa parte delle specializzazioni di questa abilità) è considerata come arma improvvisata. Per le armi delle quali non possiede la specializzazione, ha un modificatore di -1 al danno.\\
\hline
\end{tabularx}

\begin{tabularx}{\linewidth}{m|s|b}
\hline
           \textbf{ Armi a due mani}      &     \textit{\textbf{Forza}}      &      Questa abilità si utilizza per colpire i nemici in combattimento con armi a due mani come Spadoni, Martelli, Asce ecc.    \\
\hline
\multicolumn{3}{c}{\textbf{Specializzazioni}}           \\
\hline
\multirow{2}{*}{\textit{Spadoni}} &1 &     Questo tipo di arma non è più considerata come arma improvvisata.    \\
                  & 2&            Il personaggio ha ottenuto una buona maestria con questo tipo di armi, avendo un modificatore di -1 alla difficoltà.   \\\hline
\multirow{2}{*}{\textit{Martelli o Mazze}} &  1  &   Questo tipo di arma non è più considerata come arma improvvisata.      \\
                  &  2    &          Il personaggio ha ottenuto una buona maestria con questo tipo di armi, avendo un modificatore di -1 alla difficoltà.   \\ \hline
\multirow{2}{*}{\textit{Asce}} &  1  &   Questo tipo di arma non è più considerata come arma improvvisata.      \\
                  &  2    &          Il personaggio ha ottenuto una buona maestria con questo tipo di armi, avendo un modificatore di -1 alla difficoltà.   \\ 
\hline
\multicolumn{3}{c}{\textbf{Talenti}}           \\
\hline
      \multicolumn{2}{l}{  \textit{Presa di ferro}  } & Il personaggio non può essere disarmato quando utilizza armi a due mani. \\\hline
          \multicolumn{2}{l}{  \textit{Colpire alle mani}   } &Il personaggio ignora ogni armatura con grado di protezione pari o inferiore a 2 se utilizza armi a distanza. \\\hline
         \multicolumn{2}{l}{   \textit{Danneggia armatura}      } &Se si infliggono più di 8 danni a un bersaglio (calcolati prima della riduzione data dall'armatura) utilizzando armi a due mani, l'armatura presente sulla zona colpita abbassa il suo livello di protezione di 1 punto (permanentemente). \\\hline
         \multicolumn{2}{l}{   \textit{Attacco roteante}     }&Il personaggio ha la possibilità di infliggere la metà dei danni, colpendo però tutti i bersagli che lo circondano (e che possono essere colpiti dalla sua arma), alleati inclusi. Per colpire verrà utilizzata la difficoltà più alta data dai bersagli, il danno viene calcolato una singola volta e ridotto dalle eventuali protezioni di ciascun bersaglio. Utilizzabile solo quando il personaggio utilizza armi a due mani.\\\hline
          \multicolumn{2}{l}{  \textit{Esperto}      }&Per personaggio nessuna arma a due mani(che fa parte delle specializzazioni di questa abilità) è considerata come arma improvvisata. Per le armi delle quali non possiede la specializzazione, ha un modificatore di -1 al danno.\\
\hline
\end{tabularx}

\begin{tabularx}{\linewidth}{m|s|b}
\hline
 \textbf{ Armi a una mano}      &     \textit{\textbf{Forza}}      &     Questa abilità si utilizza per colpire i nemici in combattimento con armi bianche a una mano come spade, pugnali e mazze.  \\
\hline
\multicolumn{3}{c}{\textbf{Specializzazioni}}           \\
\hline
\multirow{2}{*}{\textit{Spade}} &1 &     Questo tipo di arma non è più considerata come arma improvvisata.    \\
                  & 2&            Il personaggio ha ottenuto una buona maestria con questo tipo di armi, avendo un modificatore di -1 alla difficoltà.   \\\hline
\multirow{2}{*}{\textit{Pugnali o Daghe}} &  1  &   Questo tipo di arma non è più considerata come arma improvvisata.      \\
                  &  2    &          Il personaggio ha ottenuto una buona maestria con questo tipo di armi, avendo un modificatore di -1 alla difficoltà.   \\ \hline
\multirow{2}{*}{\textit{Asce}} &  1  &   Questo tipo di arma non è più considerata come arma improvvisata.      \\
                  &  2    &          Il personaggio ha ottenuto una buona maestria con questo tipo di armi, avendo un modificatore di -1 alla difficoltà.   \\ \hline
\multirow{2}{*}{\textit{Mazze e martelli (a una mano)}} &  1  &   Questo tipo di arma non è più considerata come arma improvvisata.      \\
                  &  2    &          Il personaggio ha ottenuto una buona maestria con questo tipo di armi, avendo un modificatore di -1 alla difficoltà.   \\ 
\hline
\multicolumn{3}{c}{\textbf{Talenti}}           \\
\hline
      \multicolumn{2}{l}{  \textit{Presa di ferro}  } & Il personaggio non può essere disarmato quando utilizza armi a una mano. \\\hline
          \multicolumn{2}{l}{  \textit{Colpire alle mani}   } &Il personaggio ignora ogni armatura con grado di protezione pari o inferiore a 2 se utilizza armi a distanza. \\\hline
         \multicolumn{2}{l}{   \textit{Attacco furtivo}      } &Se il personaggio attacca un bersaglio con un'arma leggera a una mano quando è nascosto, ha un bonus di -3 alla difficoltà per colpirlo. \\\hline
         \multicolumn{2}{l}{   \textit{Armi gemelle}     }& da fare\\\hline
          \multicolumn{2}{l}{  \textit{Esperto}      }&Per personaggio nessuna arma a una mano (che fa parte delle specializzazioni di questa abilità) è considerata come arma improvvisata. Per le armi delle quali non possiede la specializzazione, ha un modificatore di -1 al danno.\\
\hline
\end{tabularx}


\begin{tabularx}{\linewidth}{m|s|b}
\hline
 \textbf{ Armi ad asta}      &     \textit{\textbf{Agilità}}      &    Questa abilità si utilizza per colpire i nemici in combattimento con armi ad asta come Lance, Alabarde, Picche ecc.  \\
\hline
\multicolumn{3}{c}{\textbf{Specializzazioni}}           \\
\hline
\multirow{2}{*}{\textit{Lance o Picche}} &1 &     Questo tipo di arma non è più considerata come arma improvvisata.    \\
                  & 2&            Il personaggio ha ottenuto una buona maestria con questo tipo di armi, avendo un modificatore di -1 alla difficoltà.   \\\hline
\multirow{2}{*}{\textit{Alabarde}} &  1  &   Questo tipo di arma non è più considerata come arma improvvisata.      \\
                  &  2    &          Il personaggio ha ottenuto una buona maestria con questo tipo di armi, avendo un modificatore di -1 alla difficoltà.   \\ 
\hline
\multicolumn{3}{c}{\textbf{Talenti}}           \\
\hline
      \multicolumn{2}{l}{  \textit{Presa di ferro}  } &  Il personaggio non può essere disarmato quando utilizza armi ad asta.\\\hline
          \multicolumn{2}{l}{  \textit{Affondo letale}   } &Il personaggio ignora ogni armatura con grado di protezione pari o inferiore a 3 quando utilizza armi ad asta. \\\hline
         \multicolumn{2}{l}{   \textit{Disarcionare}      } & Attaccando un bersaglio a cavallo utilizzando armi ad asta si ha la possibilità di disarcionarlo, infliggendogli i danni di un normale attacco più quelli della caduta.\\\hline
         \multicolumn{2}{l}{   \textit{Sanguinamento}     }& Colpire un bersaglio in un punto scoperto dall'armatura con un'arma ad asta gli causerà ferite profonde. Subisce 1 danno ogni turno finché la ferita non viene curata.\\\hline
          \multicolumn{2}{l}{  \textit{Esperto}      }&Per personaggio nessuna arma ad asta (che fa parte delle specializzazioni di questa abilità) è considerata come arma improvvisata. Per le armi delle quali non possiede la specializzazione, ha un modificatore di -1 al danno.\\
\hline
\end{tabularx}



\begin{tabularx}{\linewidth}{m|s|b}
\hline
           \textbf{Arte della Guerra}      &     \textit{\textbf{Passiva}}      &   Questa abilità passiva permette di aumentare i danni inflitti con le armi da corpo a corpo. Ogni tre punti investiti in questa abilità si ha modificatore di +1 danni in combattimento corpo a corpo quando si utilizza un'arma da mischia (o a mani nude con l'apposita specializzazione).  \\
\hline
\multicolumn{3}{c}{\textbf{Specializzazioni}}           \\
\hline
\multirow{2}{*}{\textit{Armi leggere}} &1 &   Modificatore di +1 danni utilizzando armi leggere.   \\
                  & 2&       Modificatore di +1 danni utilizzando armi leggere. \\\hline
\multirow{2}{*}{\textit{Armi medie}} &  1  &    Modificatore di +1 danni utilizzando armi medie. \\
                  &  2    &    Modificatore di +1 danni utilizzando armi medie.\\ \hline
\multirow{2}{*}{\textit{Armi pesanti}} &  1  &    Modificatore di +1 danni utilizzando armi pesanti.\\
                  &  2    &   Modificatore di +1 danni utilizzando armi pesanti.\\ \hline
\multirow{2}{*}{\textit{Mani nude}} &  1  &    I modificatori ai danni di arte della guerra vengono applicati anche quando si combatte a mani nude..\\
                  &  2    &    Modificatore di +1 danni combattendo a mani nude.\\ 
\hline
\multicolumn{3}{c}{\textbf{Talenti}}           \\
\hline
      \multicolumn{2}{l}{  \textit{Furia cieca}  } & Se il personaggio attacca un bersaglio con un'arma pesante e non indossa armatura, ha un modificatore di +3 danni.\\\hline
          \multicolumn{2}{l}{  \textit{Attacco a sorpresa}   } &  Se il personaggio attacca un bersaglio con un'arma non pesante quando è nascosto, ha un modificatore di +2 danni. \\\hline
         \multicolumn{2}{l}{   \textit{Brutalità}      } &  Il personaggio ha un modificatore di +2 danni se colpisce il bersaglio in un punto non protetto dall'armatura.\\\hline
         \multicolumn{2}{l}{   \textit{Guerriero esperto}     }& Il personaggio ha un modificatore di +1 danni quando utilizza armi medie.\\\hline
         \multicolumn{2}{l}{   \textit{Pugile}     }& Il personaggio ha un modificatore di +2 danni se combatte a mani nude.\\
\hline
\end{tabularx}


\begin{tabularx}{\linewidth}{m|s|b}
\hline
           \textbf{Evitare}      &     \textit{\textbf{Passiva}}      &     Determina la difficoltà necessaria per essere colpiti durante un combattimento. Ogni due punti investiti in quest'abilità si ottiene un modificatore di +1 alla difficoltà per essere colpiti.  \\
\hline
\multicolumn{3}{c}{\textbf{Specializzazioni}}           \\
\hline
\multirow{2}{*}{\textit{Leggiadro}} &1 &   Modificatore di +1 alla difficoltà per essere colpito in combattimento.    \\
                  & 2&       Modificatore di +1 alla difficoltà per essere colpito in combattimento. \\\hline
\multirow{2}{*}{\textit{Statico}} &  1  &    Modificatore di -1 alla difficoltà per essere colpiti in combattimento.  \\
                  &  2    &    Danni subiti in combattimento ridotti di 1.\\ 
\hline
\multicolumn{3}{c}{\textbf{Talenti}}           \\
\hline
      \multicolumn{2}{l}{  \textit{Evasivo}  } &  Il personaggio ottiene un modificatore di +3 alla difficoltà per essere colpito in combattimento.\\\hline
          \multicolumn{2}{l}{  \textit{Estremamente fortunato}   } &  Se il personaggio è stato colpito in un combattimento, tira un D6. Con un risultato pari a 6 potrà ignorare l'attacco. \\\hline
         \multicolumn{2}{l}{   \textit{Corazzato}      } &   Il personaggio non può essere colpito in zone critiche (punteggio pari a 5 sul tipo per determinare la direzione del colpo), ma ottiene un bonus di -5 alla difficoltà per essere colpito in combattimento.\\\hline
         \multicolumn{2}{l}{   \textit{Re delle risse}     }& Il personaggio ottiene un bonus di +1 alla difficoltà per essere colpito per ogni avversario che si trova in combattimento (corpo a corpo) con lui.\\
\hline
\end{tabularx}



\begin{tabularx}{\linewidth}{m|s|b}
\hline
           \textbf{Mani nude}      &     \textit{\textbf{Forza}}      &      Questa abilità si utilizza per colpire i nemici il corpo a corpo senza utilizzare un'arma.  \\
\hline
\multicolumn{3}{c}{\textbf{Specializzazioni}}           \\
\hline
\multirow{2}{*}{\textit{Colpi forti}} &1 &    Il personaggio ha un modificatore di +1 ai danni inflitti combattendo a mani nude.    \\
                  & 2&         Il personaggio ha un modificatore di +1 ai danni inflitti combattendo a mani nude.   \\\hline
\multirow{2}{*}{\textit{Colpi precisi}} &  1  &   Il personaggio ha un modificatore di -1 alla difficoltà combattendo a mani nude.   \\
                  &  2    &     Il personaggio ha un modificatore di -1 alla difficoltà combattendo a mani nude.\\ 
\hline
\multicolumn{3}{c}{\textbf{Talenti}}           \\
\hline
      \multicolumn{2}{l}{  \textit{Colpire alle mani}  } &  Se il bersaglio dell'attacco viene disarmato, subisce anche i danni di un normale attacco (conta come attacco alle braccia). \\\hline
          \multicolumn{2}{l}{  \textit{Picchiatore preciso}   } &   Il personaggio può scegliere dove colpire quando attacca a mani nude (tra Gambe, Busto, Braccia, Testa). \\\hline
         \multicolumn{2}{l}{   \textit{Scudiero}      } &  Il modificatore di -2 danni quando una delle due mani è impegnata è ridotto a -1.\\\hline
         \multicolumn{2}{l}{   \textit{Signore delle risse}     }& Il personaggio ottiene un modificatore di -1 alla difficoltà per colpire per ogni avversario che si trova in combattimento (corpo a corpo) con lui.\\
\hline
\end{tabularx}


\begin{tabularx}{\linewidth}{m|s|b}
\hline
           \textbf{Parare}      &     \textit{\textbf{Forza}}      &      Questa abilità permette di parare i colpi subiti utilizzando uno scudo.  \\
\hline
\multicolumn{3}{c}{\textbf{Specializzazioni}}           \\
\hline
\multirow{2}{*}{\textit{Scudi leggeri}} &1 &    Modificatore di -1 alla difficoltà per parare utilizzando scudi leggeri.    \\
                  & 2&           Modificatore di -1 alla difficoltà per parare utilizzando scudi leggeri.   \\\hline
\multirow{2}{*}{\textit{Scudi medi}} &  1  &   Modificatore di -1 alla difficoltà per parare utilizzando scudi medi.    \\
                  &  2    &        La protezione offerta da uno scudo medio ha un modificatore di +1. \\ \hline
\multirow{2}{*}{\textit{Scudi pesanti}} &  1  &  La protezione offerta da uno scudo pesante ha un modificatore di +1.    \\
                  &  2    &       La protezione offerta da uno scudo pesante ha un modificatore di +1.   \\ 
\hline
\multicolumn{3}{c}{\textbf{Talenti}}           \\
\hline
      \multicolumn{2}{l}{  \textit{Guardiano}  } &  Modificatore di -2 alla difficoltà per parare. Il personaggio ha un modificatore di -2 danni inflitti con armi e magie. \\\hline
          \multicolumn{2}{l}{  \textit{Protettore}   } &  Il personaggio può parare un attacco non magico rivolto a un bersaglio al suo fianco senza difficoltà addizionale.  \\\hline
         \multicolumn{2}{l}{   \textit{Contrattacco}      } &  Il personaggio ha un modificatore di +1 danni contro un bersaglio dopo aver parato un suo attacco.\\\hline
         \multicolumn{2}{l}{   \textit{Difensore}     }&   Il personaggio non subisce i malus dati dall'utilizzo di uno scudo pesante.\\
\hline
\end{tabularx}



\begin{tabularx}{\linewidth}{m|s|b}
\hline
           \textbf{Precisione}      &     \textit{\textbf{Passiva}}      &      Questa abilità permette di aumentare i danni inflitti con le armi a distanza. Ogni tre punti investiti in questa abilità si ha un modificatore di +1 danno utilizzando questo tipo di armi.   \\
\hline
\multicolumn{3}{c}{\textbf{Specializzazioni}}           \\
\hline
\multirow{2}{*}{\textit{Armi leggere}} &1 &    Modificatore di +1 danni utilizzando armi leggere.    \\
                  & 2&           Modificatore di +1 danni utilizzando armi leggere.   \\\hline
\multirow{2}{*}{\textit{Armi medie}} &  1  &   Modificatore di +1 danni utilizzando armi medie.    \\
                  &  2    &        Modificatore di +1 danni utilizzando armi medie. \\ \hline
\multirow{2}{*}{\textit{Armi pesanti}} &  1  &  Modificatore di +1 danni utilizzando armi pesanti.     \\
                  &  2    &        Modificatore di +1 danni utilizzando armi pesanti.   \\ 
\hline
\multicolumn{3}{c}{\textbf{Talenti}}           \\
\hline
      \multicolumn{2}{l}{  \textit{Tiratore leggero}  } &  Se il personaggio non indossa armatura quando attacca un bersaglio, ha un modificatore di +2 danni. \\\hline
          \multicolumn{2}{l}{  \textit{Tiratore a cavallo}   } & Il personaggio può utilizzare armi a distanza mentre è a cavallo senza difficoltà extra.  \\\hline
         \multicolumn{2}{l}{   \textit{Tiratore scelto}      } & Il personaggio ha un modificatore di +2 danni se colpisce il bersaglio in un punto non protetto dall'armatura.\\\hline
         \multicolumn{2}{l}{   \textit{Colpi efficaci}     }& Il personaggio ha un modificatore di +1 danno.\\
\hline
\end{tabularx}



\begin{tabularx}{\linewidth}{m|s|b}
\hline
           \textbf{Salute}      &     \textit{\textbf{Passiva}}      &    Questa abilità aumenta il numero di punti vita massimi a disposizione del giocatore. Ogni due punti investiti in quest'abilità il massimale dei propri punti vita aumenta di 1.    \\
\hline
\multicolumn{3}{c}{\textbf{Specializzazioni}}           \\
\hline
\multirow{2}{*}{\textit{Resistenza migliorata}} &1 &   Massimale dei propri punti vita aumentato ulteriormente di 1.    \\
                  & 2&           Massimale dei propri punti vita aumentato ulteriormente di 2.  \\\hline
\multirow{2}{*}{\textit{Pelle spessa}} &  1  &   Massimale dei propri punti vita aumentato ulteriormente di 1.    \\
                  &  2    &         Danni subiti ridotti di 1. \\ 
\hline
\multicolumn{3}{c}{\textbf{Talenti}}           \\
\hline
      \multicolumn{2}{l}{  \textit{Fisico bestiale}  } & Massimale dei propri punti vita aumentato ulteriormente di 3.\\\hline
          \multicolumn{2}{l}{  \textit{Ottima resistenza al dolore}   } &Il personaggio ha una soglia del dolore così alta che è in grado di resistere allo svenimento quando sembra praticamente impossibile. Un personaggio con questo talento non sverrà se i suoi punti vita dovessero scendere a 0 o meno di zero a causa di un attacco fisico. Conseguentemente però tutte le azioni che il personaggio deve svolgere hanno un modificatore di +2 alla difficoltà. Questo talento non previene la morte.   \\\hline
         \multicolumn{2}{l}{   \textit{Ottima resistenza al veleno}      } &Massimale dei propri punti vita aumentato ulteriormente di 1. Il personaggio riduce i danni subiti da ogni veleno non letale di 1.\\\hline
         \multicolumn{2}{l}{   \textit{Incutere timore}     }& Massimale dei propri punti vita aumentato ulteriormente di 1. Il personaggio ha un modificatore di -3 alla difficoltà sui TDS di \emph{Persuasione}.\\
\hline
\end{tabularx}


\begin{tabularx}{\linewidth}{m|s|b}
\hline
           \textbf{Riflessi}      &     \textit{\textbf{Reattività}}      &      Questa abilità regola la velocità del personaggio nel prepararsi per il combattimento e la sua iniziativa.    \\
\hline
\multicolumn{3}{c}{\textbf{Specializzazioni}}           \\
\hline
\multirow{2}{*}{\textit{spec1}} &1 &    desc11    \\
                  & 2&           desc12   \\\hline
\multirow{2}{*}{\textit{spec2}} &  1  &   spec21    \\
                  &  2    &         spec22 \\ \hline
\multirow{2}{*}{\textit{spec3}} &  1  &   spec31     \\
                  &  2    &        spec32   \\ 
\hline
\multicolumn{3}{c}{\textbf{Talenti}}           \\
\hline
      \multicolumn{2}{l}{  \textit{Allerta}  } &  Il personaggio ha un modificatore di +2 all'iniziativa. Inoltre ottiene un modificatore di -1 alla difficoltà per ogni TDS su ``\emph{Percezione}''. \\\hline
          \multicolumn{2}{l}{  \textit{tal2}   } &dt2   \\\hline
         \multicolumn{2}{l}{   \textit{tal3}      } &dt3 \\\hline
         \multicolumn{2}{l}{   \textit{tal4}     }&dt4\\\hline
          \multicolumn{2}{l}{  \textit{tal5}      }&dt5\\
\hline
\end{tabularx}



%abilità magiche
\subsubsection{Abilità legate alla magia}


\begin{tabularx}{\linewidth}{m|s|b}
\hline
           \textbf{Meditare}      &     \textit{\textbf{Sensibilità}}      &     Questa abilità riduce il costo in mana di ogni magia di un ammontare pari al livello di questa abilità (il costo minimo raggiungibile da una magia è comunque fissato a 1). Il personaggio può scegliere di cadere in trance (non potendo quindi svolgere nessuna azione) per recuperare 1 punto mana per ogni ora di meditazione. Non è possibile recuperare punti mana indossando un'armatura pesante. \\
\hline
\multicolumn{3}{c}{\textbf{Specializzazioni}}           \\
\hline
\multirow{2}{*}{\textit{Meditazione migliorata}} &1 &  +2 punti mana recuperati ogni ora durante la meditazione.  \\
                  & 2&     +3 punti mana recuperati ogni ora durante la meditazione.  \\\hline
\multirow{2}{*}{\textit{Concentrazione assoluta}} &  1  &  Il personaggio può recuperare punti mana indossando un'armatura pesante. \\
                  &  2    &   +2 punti mana recuperati ogni ora durante la meditazione.  \\ 
\hline
\multicolumn{3}{c}{\textbf{Talenti}}           \\
\hline
      \multicolumn{2}{l}{  \textit{Ascetismo}  } & +5 punti mana recuperati ogni ora durante la meditazione. \\\hline
          \multicolumn{2}{l}{  \textit{Comunione con l'energia}   } & Il personaggio recupera 2 punti mana ogni ora.   \\\hline
         \multicolumn{2}{l}{   \textit{Riserva di mana}      } & Il personaggio ha il massimale dei punti mana aumentato di 4.\\\hline
         \multicolumn{2}{l}{   \textit{Magia insanguinata}     }& Il personaggio può scegliere di pagare il costo in mana di una magia con i punti vita anziché con il mana. Non è possibile pagare parte del costo della magia con i punti mana e il rimanente con i punti vita. Non è possibile pagare in punti vita il costo di una magia che ha come effetto quello di restituire punti vita. Questo costo non potrà far scendere i punti vita sotto zero. \\\hline

\end{tabularx}

\begin{tabularx}{\linewidth}{m|s|b}
\hline
           \textbf{Resistenza magica}      &     \textit{\textbf{Sensibilità}}      &      Questa abilità permette al personaggio di provare a resistere a una magia che lo bersaglia. In caso di successo potrà ridurre l'efficacia o i danni subiti. Le specializzazioni sono riferite ai diversi comportamenti che il personaggio può adottare contro magie offensive o non offensive.    \\
\hline
\multicolumn{3}{c}{\textbf{Specializzazioni}}           \\
\hline
\multirow{2}{*}{\textit{spec1}} &1 &    desc11    \\
                  & 2&           desc12   \\\hline
\multirow{2}{*}{\textit{spec2}} &  1  &   spec21    \\
                  &  2    &         spec22 \\ \hline
\multirow{2}{*}{\textit{spec3}} &  1  &   spec31     \\
                  &  2    &        spec32   \\ 
\hline
\multicolumn{3}{c}{\textbf{Talenti}}           \\
\hline
      \multicolumn{2}{l}{  \textit{tal1}  } &dt1 \\\hline
          \multicolumn{2}{l}{  \textit{tal2}   } &dt2   \\\hline
         \multicolumn{2}{l}{   \textit{tal3}      } &dt3 \\\hline
         \multicolumn{2}{l}{   \textit{tal4}     }&dt4\\\hline
          \multicolumn{2}{l}{  \textit{tal5}      }&dt5\\
\hline
\end{tabularx}


\begin{tabularx}{\linewidth}{m|s|b}
\hline
           \textbf{Stregoneria di attacco}      &     \textit{\textbf{Intelligenza}}      &    Questa abilità è utilizzata dal personaggio per lanciare stregonerie mirate infliggere danni ai bersagli o a incrementare i danni inflitti con un'arma.  \\
\hline
\multicolumn{3}{c}{\textbf{Specializzazioni}}           \\
\hline
\multirow{2}{*}{\textit{Distruzione}} &1 &    Il personaggio può utilizzare le stregonerie di Distruzione con un grado massimo pari a 6.    \\
                  & 2&          Per le stregonerie di Distruzione il personaggio utilizza 1D6 come dado per determinare la potenza anzichè 1D4   \\\hline
\multirow{2}{*}{\textit{Incantamento}} &  1  &    Il personaggio può utilizzare le stregonerie di Incantamento con un grado massimo pari a 6.    \\
                  & 2&          Per le stregonerie di Incantamento il personaggio utilizza 1D6 come dado per determinare la potenza anzichè 1D4   \\\hline
\hline
\multicolumn{3}{c}{\textbf{Talenti}}           \\
\hline
      \multicolumn{2}{l}{  \textit{Follia}  } & Le stregonerie di Distruzione del personaggio infliggono 3 danni in più. Se il personaggio fallisce il TDS per lanciare una stregoneria di Distruzione, subisce 4 danni (non possono essere ridotti dall'armatura).\\\hline
            \multicolumn{2}{l}{  \textit{Fretta}  } & Le stregonerie di Distruzione del personaggio hanno un modificatore di -1 alla difficoltà per essere lanciate, ma possono essere \textit{annullate} con un modificatore di -2 alla difficoltà.  \\\hline
          \multicolumn{2}{l}{  \textit{Enfasi}   } &Quando il personaggio attacca con un'arma incantata ha un modificatore di -1 alla difficoltà.\\\hline
         \multicolumn{2}{l}{   \textit{aa}      } &aaaaa \\
\hline
\end{tabularx}

\begin{tabularx}{\linewidth}{m|s|b}
\hline
           \textbf{Stregoneria di controllo}      &     \textit{\textbf{Intelligenza}}      &    Questa abilità è utilizzata dal personaggio per lanciare stregonerie mirate a negare il lancio di altre magie oppure per ostacolare i movimenti di un bersaglio.  \\
\hline
\multicolumn{3}{c}{\textbf{Specializzazioni}}           \\
\hline
\multirow{2}{*}{\textit{Blocco}} &1 &    Il personaggio può utilizzare le stregonerie di Blocco con un grado massimo pari a 6.    \\
                  & 2&          Per le stregonerie di Blocco il personaggio utilizza 1D6 come dado per determinare la potenza anzichè 1D4   \\\hline
\multirow{2}{*}{\textit{Antimagia}} &  1  &    Il personaggio può utilizzare le stregonerie di Antimagia con un grado massimo pari a 6.    \\
                  & 2&          Per le stregonerie di Antimagia il personaggio utilizza 1D6 come dado per determinare la potenza anzichè 1D4   \\\hline
\hline
\multicolumn{3}{c}{\textbf{Talenti}}           \\
\hline
      \multicolumn{2}{l}{  \textit{aaaa}  } &aaaa \\\hline
            \multicolumn{2}{l}{  \textit{aaaai}  } &aaaa  \\\hline
          \multicolumn{2}{l}{  \textit{Beffa}   } & Se il personaggio \textit{annulla} con successo una magia, colui che ha utilizzato quella magia perde 1 punto vita.\\\hline
         \multicolumn{2}{l}{   \textit{aa}      } &aaaaa \\
\hline
\end{tabularx}

\begin{tabularx}{\linewidth}{m|s|b}
\hline
           \textbf{Stregoneria della materia}      &     \textit{\textbf{Intelligenza}}      &    Questa abilità è utilizzata dal personaggio per lanciare stregonerie mirate ad alterare sè stesso o gli elementi naturali che lo circondano.  \\
\hline
\multicolumn{3}{c}{\textbf{Specializzazioni}}           \\
\hline
\multirow{2}{*}{\textit{Alterazione}} &1 &    Il personaggio può utilizzare le stregonerie di Alterazione con un grado massimo pari a 6.    \\
                  & 2&          Per le stregonerie di Alterazione il personaggio utilizza 1D6 come dado per determinare la potenza anzichè 1D4   \\\hline
\multirow{2}{*}{\textit{Metamorfosi}} &  1  &    Il personaggio può utilizzare le stregonerie di Metamorfosi con un grado massimo pari a 6.    \\
                  & 2&          Per le stregonerie di Metamorfosi il personaggio utilizza 1D6 come dado per determinare la potenza anzichè 1D4   \\\hline
\hline
\multicolumn{3}{c}{\textbf{Talenti}}           \\
\hline
      \multicolumn{2}{l}{  \textit{aaaa}  } & Il personaggio può lanciare le stregonerie di Alterazione al doppio della gittata. \\\hline
            \multicolumn{2}{l}{  \textit{aaaai}  } &aaaa  \\\hline
          \multicolumn{2}{l}{  \textit{aa}   } &aaaa\\\hline
         \multicolumn{2}{l}{   \textit{aa}      } &aaaaa \\
\hline
\end{tabularx}


\begin{tabularx}{\linewidth}{m|s|b}
\hline
           \textbf{Stregoneria della mente}      &     \textit{\textbf{Intelligenza}}      &    Questa abilità è utilizzata dal personaggio per lanciare stregonerie mirate a creare illusioni o a manipolare le emozioni dei propri bersagli.  \\
\hline
\multicolumn{3}{c}{\textbf{Specializzazioni}}           \\
\hline
\multirow{2}{*}{\textit{Illusionismo}} &1 &    Il personaggio può utilizzare le stregonerie di Illusionismo con un grado massimo pari a 6.    \\
                  & 2&          Per le stregonerie di Illusionismo il personaggio utilizza 1D6 come dado per determinare la potenza anzichè 1D4   \\\hline
\multirow{2}{*}{\textit{Condizionamento}} &  1  &    Il personaggio può utilizzare le stregonerie di Condizionamento con un grado massimo pari a 6.    \\
                  & 2&          Per le stregonerie di Condizionamento il personaggio utilizza 1D6 come dado per determinare la potenza anzichè 1D4   \\\hline
\hline
\multicolumn{3}{c}{\textbf{Talenti}}           \\
\hline
      \multicolumn{2}{l}{  \textit{Ingannatore}  } & Le illusioni costruite personaggio hanno un modificatore di +2 alla difficoltà per essere smascherate. \\\hline
            \multicolumn{2}{l}{  \textit{Illusioni resistenti}  } &Le stregonerie di Illusionismo utilizzate dal personaggio hanno una durata minima pari a 5 minuti.  \\\hline
          \multicolumn{2}{l}{  \textit{Emozioni profonde}   } &Le stregonerie di Condizionamento utilizzate dal personaggio hanno una durata raddoppiata. \\\hline
         \multicolumn{2}{l}{   \textit{Impulso}      } & I bersagli condizionati dalle stregonerie del personaggio manifestano maggiormente le loro reazioni. \\
\hline
\end{tabularx}


\begin{tabularx}{\linewidth}{m|s|b}
\hline
           \textbf{Stregoneria della vita}      &     \textit{\textbf{Intelligenza}}      &      Questa abilità è utilizzata dal personaggio per lanciare stregonerie mirate a curare i propri bersagli o evocare delle creature alleate.    \\
\hline
\multicolumn{3}{c}{\textbf{Specializzazioni}}           \\
\hline
\multirow{2}{*}{\textit{Guarigione}} &1 &    Il personaggio può utilizzare le stregonerie di Guarigione con un grado massimo pari a 6.    \\
                  & 2&          Per le stregonerie di Guarigione il personaggio utilizza 1D6 come dado per determinare la potenza anzichè 1D4   \\\hline
\multirow{2}{*}{\textit{Evocazione}} &  1  &    Il personaggio può utilizzare le stregonerie di Evocazione con un grado massimo pari a 6.    \\
                  & 2&          Per le stregonerie di Evocazione il personaggio utilizza 1D6 come dado per determinare la potenza anzichè 1D4   \\\hline
\hline
\multicolumn{3}{c}{\textbf{Talenti}}           \\
\hline
      \multicolumn{2}{l}{  \textit{Guarigione eccessiva}  } &Quando il personaggio utilizza la stregoneria di guarigione per guarire uno o più personaggi, se il valore della cura dovesse portare l'ammontare di punti vita sopra al valore massimo, converti la cura in eccesso in uno scudo magico (segue le regole degli scudi della stregoneria della vita). \\\hline
            \multicolumn{2}{l}{  \textit{Guarigione perfetta}  } &Le stregonerie di Guarigione del personaggio non possono essere \emph{annullate}. \\\hline
          \multicolumn{2}{l}{  \textit{Maestro evocatore}   } & Le evocazioni del personaggio eseguono sempre i comandi impartiti.   \\\hline
         \multicolumn{2}{l}{   \textit{Vibrosensi}      } & Il personaggio può vedere e sentire tramite gli occhi e orecchie (o organi di senso affini) delle sue evocazioni. \\
\hline
\end{tabularx}



\begin{tabularx}{\linewidth}{m|s|b}
\hline
           \textbf{Tessimagie}      &     \textit{\textbf{Intelligenza}}      &     Questa abilità viene utilizzata per lanciare gli incanti di "Arte della Sapienza","Arte della Fede" e "Arte della Morte". Le specializzazioni regolano quali incanti possono essere utilizzati in base al livello del cerchio. \\
\hline
\multicolumn{3}{c}{\textbf{Specializzazioni}}           \\
\hline
\multirow{2}{*}{\textit{Sapienza}} &1 &  Il personaggio può utilizzare gli incanti di ``Arte della sapienza'' fino al terzo cerchio.  \\
                  & 2&     Il personaggio può utilizzare gli incanti di ``Arte della sapienza'' fino al terzo cerchio. \\\hline
\multirow{2}{*}{\textit{Fede}} &  1  &  Il personaggio può utilizzare gli incanti di ``Arte della fede'' fino al secondo cerchio. \\
                  &  2    &  Il personaggio può utilizzare gli incanti di ``Arte della fede'' fino al quinto cerchio.   \\ \hline
\multirow{2}{*}{\textit{Morte}} &  1  &  Il personaggio può utilizzare gli incanti di ``Arte della morte'' fino al secondo cerchio.   \\
                  &  2    &    Il personaggio può utilizzare gli incanti di ``Arte della morte'' fino al quinto cerchio.   \\ 
\hline
\multicolumn{3}{c}{\textbf{Talenti}}           \\
\hline
      \multicolumn{2}{l}{  \textit{Magie Instabili}  } & Gli incanti che infliggono danno del personaggio infliggono inoltre la metà del danno (il valore della magia, calcolato una sola volta) anche a tutti i bersagli vicini. Ciascun bersaglio ridurrà il danno in base ai propri modificatori (come l'armatura). \\\hline
          \multicolumn{2}{l}{  \textit{Sovraccarico}   } & Gli incanti del personaggio hanno un modificatore di -3 alla difficoltà per essere lanciati. In caso di fallimento nel lancio di un incanto, il personaggio perde per il mese successivo 1 punto in Intelligenza.   \\\hline
         \multicolumn{2}{l}{   \textit{Durata aumentata}      } &  Il personaggio può aumentare di 1 la difficoltà necessaria lanciare un incanto non istantaneo per aumentarne la durata di 1 round.\\\hline
         \multicolumn{2}{l}{   \textit{Pazienza}     }& Il personaggio può attendere un round addizionale tra il momento in cui si prepara a lanciare un incanto e quando questa viene effettivamente lanciato. Se lo fa e riesce a lanciarlo con successo, quell'incanto non può essere \textit{annullato}. Essere colpiti tra il turno di la preparazione dell'incanto e il lancio effettivo causerà il fallimento automatico nel lancio. \\\hline

\end{tabularx}


 %termine abilità magiche

\subsubsection{Abilità miscellanee}
 %inizio tab abilità miscellanee 



\begin{tabularx}{\linewidth}{m|s|b}
\hline
           \textbf{Acrobazia}      &     \textit{\textbf{Costituzione}}      &     Questa abilità viene usata dal personaggio di compiere attività come arrampicarsi, saltare ostacoli e nuotare e correre. \\
\hline
\multicolumn{3}{c}{\textbf{Specializzazioni}}           \\
\hline
\multirow{2}{*}{\textit{Marciare e Correre}} &1 &  Il personaggio ha un modificatore di -1 alla difficoltà per ogni TDS quando deve compiere scatti o camminare per lunghe distanze.  \\
                  & 2&        Il personaggio ha un modificatore di -1 alla difficoltà per ogni TDS quando deve compiere scatti o camminare per lunghe distanze.   \\\hline
\multirow{2}{*}{\textit{Nuotare}} &  1  &  Il personaggio ha un modificatore di -1 alla difficoltà per ogni TDS per nuotare.  \\
                  &  2    &     Il personaggio ha un modificatore di -1 alla difficoltà per ogni TDS per nuotare. \\ \hline
\multirow{2}{*}{\textit{Arrampicarsi}} &  1  &   Il personaggio ha un modificatore di -1 alla difficoltà per ogni TDS su per arrampicarsi.  \\
                  &  2    &      Il personaggio ha un modificatore di -1 alla difficoltà per ogni TDS su per arrampicarsi.   \\ 
\hline
\multicolumn{3}{c}{\textbf{Talenti}}           \\
\hline
      \multicolumn{2}{l}{  \textit{Istinto}  } & Se il personaggio subisce un modificatore alla difficoltà nei TDS su "\emph{Acrobazia}" compreso tra 1 e 4 (inclusi) a causa del terreno accidentato, ignora quel modificatore. \\\hline
          \multicolumn{2}{l}{  \textit{Allenamento}   } &  Il personaggio ha un ottima capacità fisiche, resistendo più a lungo della media allo sforzo fisico. L'aumento della difficoltà a causa della fatica è sensibilmente ridotto.   \\\hline
         \multicolumn{2}{l}{   \textit{Fisico eccezionale}      } & Il personaggio ha un modificatore di -1 alla difficoltà per ogni TDS su \emph{Acrobazia}. \\\hline
         \multicolumn{2}{l}{   \textit{tal4}     }&dt4\\\hline
          \multicolumn{2}{l}{  \textit{tal5}      }&dt5\\
\hline
\end{tabularx}



\begin{tabularx}{\linewidth}{m|s|b}
\hline
           \textbf{Cavalcare}      &     \textit{\textbf{Reattività}}      &     Questa abilità permette al personaggio di rimanere in sella a una cavalcatura senza problemi. Generalmente stare un groppa a un cavallo al passo non richiede particolari difficoltà. \\
\hline
\multicolumn{3}{c}{\textbf{Specializzazioni}}           \\
\hline
\multirow{2}{*}{\textit{a}} &1 &   a  \\
                  & 2&          a   \\\hline
\multirow{2}{*}{\textit{li}} &  1  &  a  \\
                  &  2    &        a \\ \hline
\multirow{2}{*}{\textit{e}} &  1  &   a  \\
                  &  2    &        spec32   \\ 
\hline
\multicolumn{3}{c}{\textbf{Talenti}}           \\
\hline
      \multicolumn{2}{l}{  \textit{tal1}  } &dt1 \\\hline
          \multicolumn{2}{l}{  \textit{tal2}   } &dt2   \\\hline
         \multicolumn{2}{l}{   \textit{tal3}      } &dt3 \\\hline
         \multicolumn{2}{l}{   \textit{tal4}     }&dt4\\\hline
          \multicolumn{2}{l}{  \textit{tal5}      }&dt5\\
\hline
\end{tabularx}



\begin{tabularx}{\linewidth}{m|s|b}
\hline
           \textbf{Comandare Animale / Evocazione}      &     \textit{\textbf{Carisma}}      &     Questa abilità permette al personaggio di impartire istruzioni a un proprio compagno animale o evocazione. \\
\hline
\multicolumn{3}{c}{\textbf{Specializzazioni}}           \\
\hline
\multirow{2}{*}{\textit{a}} &1 &   a  \\
                  & 2&          a   \\\hline
\multirow{2}{*}{\textit{li}} &  1  &  a  \\
                  &  2    &        a \\ \hline
\multirow{2}{*}{\textit{e}} &  1  &   a  \\
                  &  2    &        spec32   \\ 
\hline
\multicolumn{3}{c}{\textbf{Talenti}}           \\
\hline
      \multicolumn{2}{l}{  \textit{tal1}  } &dt1 \\\hline
          \multicolumn{2}{l}{  \textit{tal2}   } &dt2   \\\hline
         \multicolumn{2}{l}{   \textit{tal3}      } &dt3 \\\hline
         \multicolumn{2}{l}{   \textit{tal4}     }&dt4\\\hline
          \multicolumn{2}{l}{  \textit{tal5}      }&dt5\\
\hline
\end{tabularx}




\begin{tabularx}{\linewidth}{m|s|b}
\hline
           \textbf{Costruire / Posizionare trappole}      &     \textit{\textbf{Intelligenza}}      &     Questa abilità permette al personaggio di costruire e posizionare una trappola. Per costruire un tipo di trappola, non è necessario avere la specializzazione relativa. In questo caso però non si avranno a disposizione i bonus dati dalla specializzazione stessa.   \\
\hline
\multicolumn{3}{c}{\textbf{Specializzazioni}}           \\
\hline
\multirow{2}{*}{\textit{Trappole Offensive}} &1 &    Le trappole offensive costruite dal personaggio hanno un modificatore di +1 al danno.  \\
                  & 2&          Il personaggio ha un modificatore di -1 alla difficoltà per costruire e posizionare trappole offensive.   \\\hline
\multirow{2}{*}{\textit{Trappole Bloccanti}} &  1  &  Le trappole bloccanti costruite dal personaggio hanno un modificatore di +1 alla difficoltà per liberarsi.  \\
                  &  2    &       Il personaggio ha un modificatore di -1 alla difficoltà per costruire e posizionare trappole bloccanti. \\ \hline

\multicolumn{3}{c}{\textbf{Talenti}}           \\
\hline
      \multicolumn{2}{l}{  \textit{Maestro delle trappole}  } &  Il livello massimo delle trappole costruite dal personaggio è aumentato a 12. \\\hline
          \multicolumn{2}{l}{  \textit{Posizionamento rapido}   } & Il personaggio impiega meno tempo a costruire e posizionare una trappola.  \\\hline
         \multicolumn{2}{l}{   \textit{Posizionamento strategico}      } & Il personaggio posiziona le trappole in modo tattico per renderne più difficoltosa l’individuazione. Per questo motivo le trappole posizionate dal personaggio hanno una difficoltà minima per essere individuate pari a 2 (che potrà essere migliorata con “Occultare Trappole”). \\\hline
         \multicolumn{2}{l}{   \textit{Meccanismi complicati}     }& Le trappole costruite dal personaggio possono essere disattivate con difficoltà pari a: Livello trappola + 2. \\\hline

\end{tabularx}



\begin{tabularx}{\linewidth}{m|s|b}
\hline
           \textbf{Cultura generale}      &     \textit{\textbf{Intelligenza}}      &     Indica la e la capacità di conoscere e riconoscere culture, lingue, religioni. Azioni come il provare a riconoscere un vessillo, un particolare tipo di architettura, oggettistica o effigie religiosa dipendono da questa abilità.  \\
\hline
\multicolumn{3}{c}{\textbf{Specializzazioni}}           \\
\hline
\multirow{2}{*}{\textit{Conoscenze geografiche}} &1 &    Il personaggio ha una discreta conoscenza degli elementi naturali. Per questo motivo può riconoscere con maggior facilità punti di interesse come fiumi o montagne. Per queste azioni ha un modificatore di -1 alla difficoltà.  \\
                  & 2&          a   \\\hline
\multirow{2}{*}{\textit{Conoscenze linguistiche}} &1 &    Il personaggio ha una discreta familiarità con lo studio delle diverse lingue. Per lui impararne una nuova è più semplice.  \\
                  & 2&          a   \\\hline
\multirow{2}{*}{\textit{Conoscenze culturali}} &  1  &  Il personaggio ha una discreta conoscenza delle differenti culture della zona. Per questo motivo può riconoscere con maggior facilità l'appartenenza di un particolare oggetto (come un vessillo o un'arma) coerente con la zona di appartenenza del personaggio. Per queste azioni ha un modificatore di -1 alla difficoltà.  \\
                  &  2    &        a \\ \hline
\multirow{2}{*}{\textit{Conoscenze religiose}} &  1  &    Il personaggio ha una discreta conoscenza delle differenti religioni della zona. Per questo motivo può riconoscere con maggior facilità un particolare rito o oggetto religioso coerente con la zona di appartenenza del personaggio. Per queste azioni ha un modificatore di -1 alla difficoltà.     \\
                  &  2    &        spec32   \\ 
\hline
\multicolumn{3}{c}{\textbf{Talenti}}           \\
\hline
      \multicolumn{2}{l}{  \textit{tal1}  } &dt1 \\\hline
          \multicolumn{2}{l}{  \textit{tal2}   } &dt2   \\\hline
         \multicolumn{2}{l}{   \textit{tal3}      } &dt3 \\\hline
         \multicolumn{2}{l}{   \textit{tal4}     }&dt4\\\hline
          \multicolumn{2}{l}{  \textit{tal5}      }&dt5\\
\hline
\end{tabularx}




\begin{tabularx}{\linewidth}{m|s|b}
\hline
           \textbf{Disattivare trappole}      &     \textit{\textbf{Reattività}}      &     Questa abilità permette al personaggio di disattivare una trappola preventivamente individuata.   \\
\hline
\multicolumn{3}{c}{\textbf{Specializzazioni}}           \\
\hline
\multirow{2}{*}{\textit{a}} &1 &    a  \\
                  & 2&          a   \\\hline
\multirow{2}{*}{\textit{a}} &  1  &  a    \\
                  &  2    &        a \\ \hline
\multirow{2}{*}{\textit{spec3}} &  1  &   spec31     \\
                  &  2    &        spec32   \\ 
\hline
\multicolumn{3}{c}{\textbf{Talenti}}           \\
\hline
      \multicolumn{2}{l}{  \textit{tal1}  } &dt1 \\\hline
          \multicolumn{2}{l}{  \textit{tal2}   } &dt2   \\\hline
         \multicolumn{2}{l}{   \textit{tal3}      } &dt3 \\\hline
         \multicolumn{2}{l}{   \textit{tal4}     }&dt4\\\hline
          \multicolumn{2}{l}{  \textit{tal5}      }&dt5\\
\hline
\end{tabularx}



\begin{tabularx}{\linewidth}{m|s|b}
\hline
           \textbf{Domare animali}      &     \textit{\textbf{Sensibilità}}      &      Questa abilità permette al personaggio di domare un animale o una creatura. Questo potrebbe permettergli di stringere un legame con essa e di averlo come compagno animale.     \\
\hline
\multicolumn{3}{c}{\textbf{Specializzazioni}}           \\
\hline
\multirow{2}{*}{\textit{Intuizione}} &1 &    Il personaggio osservando l’animale o creatura può capire la sua dieta (carnivoro/erbivoro/onnivoro).    \\
                  & 2&           Il personaggio osservando l’animale o creatura può intuire costa stia provando (paura/tranquillità/ecc).   \\\hline
\multirow{2}{*}{\textit{Fiducia}} &  1  &   Il personaggio ha un modificatore di -1 alla difficoltà per domare un animale o creatura già conosciuta.1    \\
                  &  2    &         Il personaggio ha un modificatore di -1 alla difficoltà su ogni tiro per domare un animale o creatura. \\ \hline
\multirow{2}{*}{\textit{spec3}} &  1  &   spec31     \\
                  &  2    &        spec32   \\ 
\hline
\multicolumn{3}{c}{\textbf{Talenti}}           \\
\hline
      \multicolumn{2}{l}{  \textit{tal1}  } &dt1 \\\hline
          \multicolumn{2}{l}{  \textit{tal2}   } &dt2   \\\hline
         \multicolumn{2}{l}{   \textit{tal3}      } &dt3 \\\hline
         \multicolumn{2}{l}{   \textit{tal4}     }&dt4\\\hline
          \multicolumn{2}{l}{  \textit{tal5}      }&dt5\\
\hline
\end{tabularx}




\begin{tabularx}{\linewidth}{m|s|b}
\hline
           \textbf{Erboristeria}      &     \textit{\textbf{Sensibilità}}      &       Questa abilità permette al personaggio di cercare o coltivare erbe, piante e funghi. \\
\hline
\multicolumn{3}{c}{\textbf{Specializzazioni}}           \\
\hline
\multirow{2}{*}{\textit{Effetti curativi}} &1 &   Il personaggio dedica particolare attenzione allo studio e alla conoscenza di erbe, piante o funghi con proprietà curative. La ricerca selettiva di questo tipo di piante ha un modificatore di -1 alla difficoltà.    \\
                  & 2&         Il personaggio conosce a tal punto i tipi di piante aventi effetti curativi che la difficoltà per coltivarle ha un modificatore di -2.   \\\hline
\multirow{2}{*}{\textit{Effetti curativi}} &  1  &  Il personaggio dedica particolare attenzione allo studio e alla conoscenza di erbe, piante o funghi con proprietà velenose. La ricerca selettiva di questo tipo di piante ha un modificatore di -1 alla difficoltà.   \\
                  &  2    &        Il personaggio conosce a tal punto i tipi di piante aventi effetti velenosi che la difficoltà per coltivarle ha un modificatore di -1. \\ \hline
\multirow{2}{*}{\textit{Piante commestibili}} &  1  &   Il personaggio dedica particolare attenzione allo studio e alla conoscenza di erbe, piante o funghi commestibili. La ricerca selettiva di questo tipo di piante ha un modificatore di -1 alla difficoltà.    \\
                  &  2    &      Il personaggio conosce a tal punto i tipi di piante commestibili che la difficoltà per coltivarle ha un modificatore di -3.   \\ 
\hline
\multicolumn{3}{c}{\textbf{Talenti}}           \\
\hline
      \multicolumn{2}{l}{  \textit{Farmacista}  } &  Le pozioni curative prodotte dal personaggio sono più efficaci; ogni pozione di questo tipo permette a chi la beve di recuperare un punto vita addizionale. \\\hline
          \multicolumn{2}{l}{  \textit{Avvelenatore}   } & I veleni preparati dal personaggio sono più difficili da individuare. Egli ha inoltre sviluppato una discreta resistenza ai veleni, per questo motivo riduce i danni subiti da ogni veleno non letale di 1.   \\\hline
         \multicolumn{2}{l}{   \textit{Crescita rigogliosa}      } &Il personaggio nel coltivare dedica una cura maniacale ai dettagli, facendo crescere la pianta più rigogliosa (e aumentando quindi le risorse ottenute da essa). \\\hline
         \multicolumn{2}{l}{   \textit{Cercare erbe}     }&Il personaggio ha una vasta conoscenza di molti tipi di erbe e li riconosce con facilità. Il personaggio ha un modificatore alla difficoltà di -3 per Cercare erbe.\\\hline

\end{tabularx}

\begin{tabularx}{\linewidth}{m|s|b}
\hline
           \textbf{Forza di volontà}      &     \textit{\textbf{Sensibilità}}      &      Questa abilità permette al personaggio di mantenere la calma e il controllo sulle sue azioni in condizioni di paura e terrore o altri stati d'animo che potrebbero impedirgli di ragionare e agire lucidamente.  \\
\hline
\multicolumn{3}{c}{\textbf{Specializzazioni}}           \\
\hline
\multirow{2}{*}{\textit{spec1}} &1 &    desc11    \\
                  & 2&           desc12   \\\hline
\multirow{2}{*}{\textit{spec2}} &  1  &   spec21    \\
                  &  2    &         spec22 \\ \hline
\multirow{2}{*}{\textit{spec3}} &  1  &   spec31     \\
                  &  2    &        spec32   \\ 
\hline
\multicolumn{3}{c}{\textbf{Talenti}}           \\
\hline
      \multicolumn{2}{l}{  \textit{tal1}  } &dt1 \\\hline
          \multicolumn{2}{l}{  \textit{tal2}   } &dt2   \\\hline
         \multicolumn{2}{l}{   \textit{tal3}      } &dt3 \\\hline
         \multicolumn{2}{l}{   \textit{tal4}     }&dt4\\\hline
          \multicolumn{2}{l}{  \textit{tal5}      }&dt5\\
\hline
\end{tabularx}


\begin{tabularx}{\linewidth}{m|s|b}
\hline
           \textbf{Individuare trappole}      &     \textit{\textbf{Sensibilità}}      &      Questa abilità permette al personaggio di individuare e se possibile di identificare una trappola. La difficoltà dipende dal modo in cui la trappola è stata occultata alla vista.    \\
\hline
\multicolumn{3}{c}{\textbf{Specializzazioni}}           \\
\hline
\multirow{2}{*}{\textit{spec1}} &1 &    desc11    \\
                  & 2&           desc12   \\\hline
\multirow{2}{*}{\textit{spec2}} &  1  &   spec21    \\
                  &  2    &         spec22 \\ \hline
\multirow{2}{*}{\textit{spec3}} &  1  &   spec31     \\
                  &  2    &        spec32   \\ 
\hline
\multicolumn{3}{c}{\textbf{Talenti}}           \\
\hline
      \multicolumn{2}{l}{  \textit{tal1}  } &dt1 \\\hline
          \multicolumn{2}{l}{  \textit{tal2}   } &dt2   \\\hline
         \multicolumn{2}{l}{   \textit{tal3}      } &dt3 \\\hline
         \multicolumn{2}{l}{   \textit{tal4}     }&dt4\\\hline
          \multicolumn{2}{l}{  \textit{tal5}      }&dt5\\
\hline
\end{tabularx}

\begin{tabularx}{\linewidth}{m|s|b}
\hline
           \textbf{Leadership}      &     \textit{\textbf{Carisma}}      &   Questa abilità determina chi all’interno di un gruppo sia più influente nel proporre le proprie idee o decisioni. Può essere utilizzata anche per motivare un gruppo di persone.   \\
\hline
\multicolumn{3}{c}{\textbf{Specializzazioni}}           \\
\hline
\multirow{2}{*}{\textit{Influenza superiore}} &1 &   Il personaggio risulta particolarmente carismatico: il valore di influenza è aumentato di 1.   \\
                  & 2&      Il personaggio è rispettato e considerato all’interno del proprio gruppo in modo molto superiore alla media. Il valore di influenza è aumentato di 2.  \\\hline
\multirow{2}{*}{\textit{Comandante}} &  1  & Il personaggio non si tira indietro quando si tratta di motivare i propri compagni. Utilizzare la propria leadership per motivare i propri compagni ha un modificatore di -1 alla difficoltà.  \\
                  &  2    &      Il personaggio è considerato come un riferimento nei momenti di difficoltà. Utilizzare la propria leadership per motivare i propri compagni ha un modificatore di -2 alla difficoltà.\\ \hline
\multicolumn{3}{c}{\textbf{Talenti}}           \\
\hline
      \multicolumn{2}{l}{  \textit{Presenza ispiratrice}  } &Il personaggio riesce a trasmettere parte delle sue conoscenze agli altri membri del suo gruppo. I personaggi del suo gruppo hanno un modificatore di -2 alla difficoltà nei TDS relativi alle abilità in comune con il personaggio che possiede questo talento. Gli effetti multipli di questo talento non sono cumulabili. \\\hline
          \multicolumn{2}{l}{  \textit{Sedare gli animi}   } & Il personaggio può utilizzare la propria abilità “Leadership” per provare a evitare ribellioni, sedare il malcontento o fermare una rissa.   \\\hline
         \multicolumn{2}{l}{   \textit{Dare conforto}      } & Se il personaggio utilizza la propria leadership per far ritornare in sé un personaggio terrorizzato/spaventato/ecc ha un modificatore di -3 alla difficoltà. \\\hline
         \multicolumn{2}{l}{   \textit{Chiamata alle armi}     }& Il personaggio infonde la sua foga negli alleati, conferendo a loro e a sé stesso un bonus di +3 a “Cavalcare” e “Marciare” durante una carica\\\hline

\end{tabularx}



\begin{tabularx}{\linewidth}{m|s|b}
\hline
           \textbf{Medicina}      &     \textit{\textbf{Sensibilità}}      &      Questa abilità permette al personaggio di curare un ferito. Il tipo di medicazione necessaria determina la difficoltà e le risorse necessarie.    \\
\hline
\multicolumn{3}{c}{\textbf{Specializzazioni}}           \\
\hline
\multirow{2}{*}{\textit{Medicina da campo}} &1 &    Il personaggio è abituato all'uso di strumenti non ottimali per la cura delle ferite (purchè siano utilizzabili per lo scopo). Il personaggio non subisce malus se utilizza strumenti di questo tipo.   \\
                  & 2&            Il personaggio impiega meno tempo per curare una ferita.   \\\hline
\multirow{2}{*}{\textit{Cure approfondite}} &  1  &   Il personaggio ha ottime capacità di diagnosi, ottiene quindi un modificatore di -2 alla difficoltà per ogni TDS su "Medicina" mirato al riconoscimento di ferite, avvelenamenti o malattie di un soggetto umanoide.   \\
                  &  2    &         Il personaggio ha un modificatore di -1 alla difficoltà nei TDS su "Medicina" per curare e stabilizzare le ferite di un soggetto umanoide.\\ \hline
\multirow{2}{*}{\textit{Veterinaria}} &  1  &    Il personaggio conosce l'anatomia degli animali  e creature più comuni: ottiene quindi un modificatore di -2 alla difficoltà per ogni TDS su "Medicina" mirato al riconoscimento di ferite, avvelenamenti o malattie di un animale o creatura non umanoide.    \\
                  &  2    &        Il personaggio ha un modificatore di -1 alla difficoltà nei TDS su "Medicina" per curare le ferite di un animale o creatura non umanoide.   \\ 
\hline
\multicolumn{3}{c}{\textbf{Talenti}}           \\
\hline
      \multicolumn{2}{l}{  \textit{Antropologia forense}  } &Studiando un corpo morto, per il personaggio è possibile provare a capire le cause del decesso più probabili. \\\hline
          \multicolumn{2}{l}{  \textit{tal2}   } &dt2   \\\hline
         \multicolumn{2}{l}{   \textit{tal3}      } &dt3 \\\hline
         \multicolumn{2}{l}{   \textit{tal4}     }&dt4\\\hline
          \multicolumn{2}{l}{  \textit{tal5}      }&dt5\\
\hline
\end{tabularx}


\begin{tabularx}{\linewidth}{m|s|b}
\hline
           \textbf{Muoversi furtivamente / Nascondersi}      &     \textit{\textbf{Reattività}}      &      Questa abilità permette al personaggio di o nascondersi muoversi rendendo molto più difficile essere individuato. La difficoltà dipende dalla situazione.    \\
\hline
\multicolumn{3}{c}{\textbf{Specializzazioni}}           \\
\hline
\multirow{2}{*}{\textit{Silenzioso}} &1 &    Il personaggio è estremamente silenzioso quando si muove; provare ad ascoltarlo mentre si muove furtivamente ha un modificatore di +1 alla difficoltà per individuarlo.    \\
                  & 2&           desc12   \\\hline
\multirow{2}{*}{\textit{spec2}} &  1  &   spec21    \\
                  &  2    &         spec22 \\ \hline
\multirow{2}{*}{\textit{spec3}} &  1  &   spec31     \\
                  &  2    &        spec32   \\ 
\hline
\multicolumn{3}{c}{\textbf{Talenti}}           \\
\hline
      \multicolumn{2}{l}{  \textit{tal1}  } &dt1 \\\hline
          \multicolumn{2}{l}{  \textit{tal2}   } &dt2   \\\hline
         \multicolumn{2}{l}{   \textit{tal3}      } &dt3 \\\hline
         \multicolumn{2}{l}{   \textit{tal4}     }&dt4\\\hline
          \multicolumn{2}{l}{  \textit{tal5}      }&dt5\\
\hline
\end{tabularx}



\begin{tabularx}{\linewidth}{m|s|b}
\hline
           \textbf{Navigazione}      &     \textit{\textbf{Intelligenza}}      &   Questa abilità indica quanto il personaggio sia in grado di governare una barca di medie e piccole dimensioni. Azioni come: seguire una rotta, spiegare all'equipaggio come svolgere semplici azioni (come \textit{ ammainare le vele}) ricadono in questa abilità.    \\
\hline
\multicolumn{3}{c}{\textbf{Specializzazioni}}           \\
\hline
\multirow{2}{*}{\textit{Governare la Nave}} &1 &   Il personaggio ha un modificatore di -1 alla difficoltà per governare la nave in modo da mantenere correttamente la rotta.   \\
                  & 2&          Il personaggio ha un modificatore di -2 alla difficoltà per governare la nave in modo da mantenere correttamente la rotta. \\\hline
\multirow{2}{*}{\textit{Scegliere la Rotta}} &  1  & Il personaggio ha un modificatore di -1 alla difficoltà per scegliere la rotta corretta per evitare i pericoli.    \\
                  &  2    &      Il personaggio ha un modificatore di -1 alla difficoltà per scegliere la rotta corretta per evitare i pericoli.  \\ 
\hline
\multirow{2}{*}{\textit{Gestire la ciurma}} &  1  &Coloro che ricevono le indicazioni dal personaggio, hanno un modificatore di -1 alla difficoltà per i TDS su "Navigare".   \\
                  &  2    &    Coloro che ricevono le indicazioni dal personaggio, hanno un modificatore di -1 alla difficoltà per i TDS su "Navigare".  \\ 
\hline
\multicolumn{3}{c}{\textbf{Talenti}}           \\
\hline
      \multicolumn{2}{l}{  \textit{Avanti tutta}  } &Le rotte identificate dal personaggio saranno le più rapide possibile. Egli è inoltre in grado di spingere l'imbarcazione al suo massimo ottenendo una velocità di crociera fuori dal comune. \\\hline
          \multicolumn{2}{l}{  \textit{Razionamento}   } &Il personaggio è esperto della vita a bordo, sa come gestire al meglio le provviste di cibo e acqua. Inoltre la raccolta di cibo a bordo della nave (pescare o raccogliere alghe) è facilitata. \\\hline
         \multicolumn{2}{l}{   \textit{Navigare nella tempesta}      } &Il personaggio sa riconoscere il clima e i suoi cambiamenti con facilità. Anche se sorpreso da una tempesta improvvisa sa come gestire l'imbarcazione e minimizza i danni che possono essere subiti.\\\hline
         \multicolumn{2}{l}{   \textit{Conoscenza strutturale}     }& Il personaggio conosce estremamente bene l'imbarcazione. Ogni tentativo di ripararla, se fatto sotto la sua supervisione, ha un modificatore di -3 alla difficoltà.\\\hline
\end{tabularx}


\begin{tabularx}{\linewidth}{m|s|b}
\hline
           \textbf{Occultare Trappole}      &     \textit{\textbf{Sensibilità}}      &      Questa abilità permette al personaggio di nascondere alla vista una trappola già posizionata, rendendola più difficile da individuare.    \\
\hline
\multicolumn{3}{c}{\textbf{Specializzazioni}}           \\
\hline
\multirow{2}{*}{\textit{Occultamento Migliorato}} &1 &    Le trappole nascoste dal personaggio hanno un modificatore di +1 alla difficoltà per essere individuate.    \\
                  & 2& Le trappole nascoste dal personaggio hanno un modificatore di +1 alla difficoltà per essere individuate. \\\hline
\multirow{2}{*}{\textit{Occultamento Facilitato}} &  1  &  Il personaggio ha un modificatore di -1 alla difficoltà per nascondere una trappola.   \\
                  &  2    &        Il personaggio ha un modificatore di -1 alla difficoltà per nascondere una trappola. \\ \hline

\hline
\multicolumn{3}{c}{\textbf{Talenti}}           \\
\hline
      \multicolumn{2}{l}{  \textit{Improvvisare}  } & Il personaggio trova più facilmente materiali adatti a nascondere una trappola.\\\hline
          \multicolumn{2}{l}{  \textit{Maldestro}   } & Fallire un tiro su “Occultare trappole” non causerà mai l’innesco della trappola stessa.   \\\hline
         \multicolumn{2}{l}{   \textit{Camuffamento}      } &Le trappole nascoste dal personaggio hanno un +2 di difficoltà per essere individuate. \\\hline
         \multicolumn{2}{l}{   \textit{Trucchi del Mestriere}     }&Il personaggio ha un modificatore di +1 alla difficoltà per occultare le trappole, ma queste non verranno innescate da personaggi hanno assistito all’occultamento.\\\hline
\end{tabularx}



\begin{tabularx}{\linewidth}{m|s|b}
\hline
           \textbf{Percezione}      &     \textit{\textbf{Sensibilità}}      &      Questa azione permette al personaggio di cogliere dettagli, comprendere al meglio le situazioni che lo circondano utilizzando i propri sensi.    \\
\hline
\multicolumn{3}{c}{\textbf{Specializzazioni}}           \\
\hline
\multirow{2}{*}{\textit{Fisionomista}} &1 &   Il personaggio ha un modificatore di -1 alla difficoltà se utilizza \emph{Percezione} per cercare una persona in una folla.    \\
                  & 2&             Il personaggio ha un modificatore di -2 alla difficoltà se utilizza \emph{Percezione} per cercare una persona in una folla. \\\hline
\multirow{2}{*}{\textit{Cinesica}} &  1  &  Il personaggio ha un modificatore di -1 alla difficoltà se utilizza \emph{Percezione} per provare a capire le intenzioni di una persona.   \\
                  &  2    &        Il personaggio ha un modificatore di -1 alla difficoltà se utilizza \emph{Percezione} per provare a capire le intenzioni di una persona.  \\ \hline
\multirow{2}{*}{\textit{Attenzione ai particolari}} &  1  & Il personaggio ha un modificatore di -1 alla difficoltà se utilizza \emph{Percezione} per cercare piccoli dettagli in un ambiente.    \\
                  &  2    &     Il personaggio ha un modificatore di -1 alla difficoltà se utilizza \emph{Percezione} per cercare piccoli dettagli in un ambiente.  \\ 
\hline
\multicolumn{3}{c}{\textbf{Talenti}}           \\
\hline
      \multicolumn{2}{l}{  \textit{tal1}  } & Il personaggio ha un modificatore di -1 alla difficoltà se utilizza \emph{Percezione} per cercare delle tracce. \\\hline
          \multicolumn{2}{l}{  \textit{Vista superiore alla norma}   } &  Il personaggio ha un modificatore di -1 alla difficoltà se utilizza \emph{Percezione} per osservare.  \\\hline
         \multicolumn{2}{l}{   \textit{Udito superiore alla norma}      } & Il personaggio ha un modificatore di -1 alla difficoltà se utilizza \emph{Percezione} per ascoltare. \\\hline
         \multicolumn{2}{l}{   \textit{tal4}     }&dt4\\\hline
          \multicolumn{2}{l}{  \textit{tal5}      }&dt5\\
\hline
\end{tabularx}



\begin{tabularx}{\linewidth}{m|s|b}
\hline
           \textbf{Riparare Equipaggiamento}      &     \textit{\textbf{Forza}}      &        Questa abilità permette al personaggio di riparare gli oggetti usati come equipaggiamento, come armi scudi e armature.   \\
\hline
\multicolumn{3}{c}{\textbf{Specializzazioni}}           \\
\hline
\multirow{2}{*}{\textit{Riparare Scudi}} &1 &   Il personaggio ha un bonus di -1 alla difficoltà per riparare gli scudi.   \\
                  & 2&          Il personaggio ha un bonus di -1 alla difficoltà per riparare gli scudi. \\\hline
\multirow{2}{*}{\textit{Riparare Armature}} &  1  &    Il personaggio ha un bonus di -1 alla difficoltà per riparare le armature.  \\
                  &  2    &      Il personaggio ha un bonus di -1 alla difficoltà per riparare le armature.\\ \hline
\multirow{2}{*}{\textit{Riparare Armi}} &  1  &   Il personaggio ha un bonus di -1 alla difficoltà per riparare le armi.   \\
                  &  2    &      Il personaggio ha un bonus di -1 alla difficoltà per riparare le armi. \\ 
\hline
\multicolumn{3}{c}{\textbf{Talenti}}           \\
\hline
      \multicolumn{2}{l}{  \textit{Tempra}  } & Gli oggetti riparati o migliorati dal personaggio sono particolarmente resistenti all’usura. \\\hline
          \multicolumn{2}{l}{  \textit{Cannone di Vetro}   } & Il personaggio può usare le sue conoscenze permigliorare le caratteristiche di un oggetto, ma questo risulterà estremamente fragile e se danneggiato non può essere nuovamente riparato. La fragilità verrà stabilita dal master. L’aumento di efficacia dipende dalla difficoltà (scelta dal giocatore).  \\\hline
         \multicolumn{2}{l}{   \textit{Maldestro}      } & Fallire un tiro su ``\emph{Riparare Equipaggiamento}'' causa con minor probabilità la distruzione dell'oggetto. \\\hline
         \multicolumn{2}{l}{   \textit{Riciclare}     }& Il personaggio può ricavare un piccolo ammontare di risorse anche da oggetti rotti.\\\hline

\end{tabularx}



\begin{tabularx}{\linewidth}{m|s|b}
\hline
           \textbf{Riparare meccanismi}      &     \textit{\textbf{Intelligenza}}      &       Questa abilità permette al personaggio di riparare trappole danneggiate, congegni di varia natura (come le serrature) o piccoli oggetti come i monili.    \\
\hline
\multicolumn{3}{c}{\textbf{Specializzazioni}}           \\
\hline
\multirow{2}{*}{\textit{Riparare Trappole}} &1 &    Il personaggio ha un bonus di -1 alla difficoltà per riparare trappole.    \\
                  & 2&           Il personaggio ha un bonus di -1 alla difficoltà per riparare trappole. \\\hline
\multirow{2}{*}{\textit{Riparare Congegni}} &  1  &   Il personaggio ha un bonus di -1 alla difficoltà per riparare congegni.   \\
                  &  2    &        Il personaggio ha un bonus di -1 alla difficoltà per riparare congegni. \\ \hline
\multirow{2}{*}{\textit{Riparare Monili}} &  1  &   Il personaggio ha un bonus di -1 alla difficoltà per riparare monili.     \\
                  &  2    &       Il personaggio ha un bonus di -1 alla difficoltà per riparare monili.  \\ 
\hline
\multicolumn{3}{c}{\textbf{Talenti}}           \\
\hline
      \multicolumn{2}{l}{  \textit{tal1}  } &desc \\\hline
          \multicolumn{2}{l}{  \textit{tal2}   } &dt2   \\\hline
         \multicolumn{2}{l}{   \textit{tal3}      } &dt3 \\\hline
         \multicolumn{2}{l}{   \textit{tal4}     }&dt4\\\hline
          \multicolumn{2}{l}{  \textit{tal5}      }&dt5\\
\hline
\end{tabularx}



\begin{tabularx}{\linewidth}{m|s|b}
\hline
           \textbf{Malaffare}      &     \textit{\textbf{Reattività}}      &      Questa abilità gestisce tutte le attività disoneste che il personaggio può compiere: per esempio viene usata per scassinare serrature, sottrarre un oggetto senza essere notato o barare al gioco d'azzardo. Il valore di difficoltà dipende dalla situazione dal tipo di serratura da scassinare o dalle dimensioni dell'oggetto che si intende rubare).    \\
\hline
\multicolumn{3}{c}{\textbf{Specializzazioni}}           \\
\hline
\multirow{2}{*}{\textit{Scassinare}} &1 &    Il personaggio può provare a riconoscere il tipo di serratura e stimare la difficoltà necessaria per aprirla.    \\
                  & 2&           Il personaggio ha un modificatore di -1 alla difficoltà per provare a scassinare una serratura.   \\\hline
\multirow{2}{*}{\textit{Borseggiare}} &  1  &    Il personaggio può provare a riconoscere il tipo di oggetti contenuto in tasche, fodere, borselli, ecc e stimare la difficoltà necessaria per sottrarli.    \\
                  &  2    &         Il personaggio ha un modificatore di -1 alla difficoltà per provare a rubare un oggetto. \\ \hline
\multirow{2}{*}{\textit{Gioco d'azzardo}} &  1  &   Il personaggio ha un modificatore di -1 alla difficoltà per provare a barare al gioco d'azzardo.    \\
                  &  2    &       Il personaggio ha un modificatore di -1 alla difficoltà per provare a barare al gioco d'azzardo.  \\ 
\hline
\multicolumn{3}{c}{\textbf{Talenti}}           \\
\hline
      \multicolumn{2}{l}{  \textit{Ricettatore}  } & Il personaggio può provare a capire con maggior accuratezza il valore degli oggetti. Ha inoltre un modificatore di -1 alla difficoltà nei tiri su \emph{Persuasione} se prova a vendere un oggetto. \\\hline
          \multicolumn{2}{l}{  \textit{Azzardopatia}   } & Il personaggio ha un modificatore di -1 alla difficoltà nei tiri su \emph{Persuasione} se prova a convincere qualcuno a gicoare d'azzardo con lui. Ha anche un ulteriore modificatore di -1 alla difficoltà per provare a barare al gioco d'azzardo. \\\hline
         \multicolumn{2}{l}{   \textit{Mercato nero}      } & Il personaggio ha un modificatore di -2 alla difficoltà quando prova a ottenere informazioni relative a eventuali commerci illeciti presenti in città. \\\hline
         \multicolumn{2}{l}{   \textit{Rapidità di mano}     }&  Il personaggio ha un modificatore di -1 alla difficoltà per ogni TDS su Malaffare \\
\hline
\end{tabularx}



\begin{tabularx}{\linewidth}{m|s|b}
\hline
           \textbf{Persuasione}      &     \textit{\textbf{Carisma}}      &     Questa abilità permette al personaggio di usare la propria capacità oratoria o le proprie doti seduttive per persuadere o raggirare. Può anche essere utlizzata per portare avanti una trattativa. In caso di fallimento, non è possibile riprovare a convincere la stessa persona con le finalità appena fallite.   \\
\hline
\multicolumn{3}{c}{\textbf{Specializzazioni}}           \\
\hline
\multirow{2}{*}{\textit{a}} &1 &    a  \\
                  & 2&          a   \\\hline
\multirow{2}{*}{\textit{a}} &  1  &  a    \\
                  &  2    &        a \\ \hline
\multirow{2}{*}{\textit{spec3}} &  1  &   spec31     \\
                  &  2    &        spec32   \\ 
\hline
\multicolumn{3}{c}{\textbf{Talenti}}           \\
\hline
      \multicolumn{2}{l}{  \textit{tal1}  } &dt1 \\\hline
          \multicolumn{2}{l}{  \textit{tal2}   } &dt2   \\\hline
         \multicolumn{2}{l}{   \textit{tal3}      } &dt3 \\\hline
         \multicolumn{2}{l}{   \textit{tal4}     }&dt4\\\hline
          \multicolumn{2}{l}{  \textit{tal5}      }&dt5\\
\hline
\end{tabularx}


\begin{tabularx}{\linewidth}{m|s|b}
\hline
           \textbf{Schivare}      &     \textit{\textbf{Reattività}}      &      Questa abilità permette al personaggio di provare a schivare tutto ciò che non non lo bersaglia direttamente. Viene quindi utilizzata per definire se il personaggio venga colpito da una trappola, da un oggetto in caduta o da una freccia scagliata contro il suo gruppo (ma non indirizzata direttamente a lui). In caso di riuscita, l'effetto dato dall'aver schivato la minaccia sarà stabilito dal master in base alla situazione.    \\
\hline
\multicolumn{3}{c}{\textbf{Specializzazioni}}           \\
\hline
\multirow{2}{*}{\textit{Prontezza}} &1 &    Il personaggio ha un modificatore di -1 alla difficoltà per ogni TDS su Schivare.    \\
                  & 2&           Il personaggio ha un modificatore di -1 alla difficoltà per ogni TDS su Schivare.   \\\hline
\multirow{2}{*}{\textit{Per un pelo}} &  1  &   Il personaggio ha un modificatore di -2 alla difficoltà per ogni TDS su Schivare quando prova a evitare le conseguenze date dall'aver fatto scattare una trappola.   \\
                  &  2    &         Il personaggio ha un modificatore di -1 ai danni subiti dalle trappole. \\ \hline
\multirow{2}{*}{\textit{spec3}} &  1  &   spec31     \\
                  &  2    &        spec32   \\ 
\hline
\multicolumn{3}{c}{\textbf{Talenti}}           \\
\hline
      \multicolumn{2}{l}{  \textit{Evasivo}  } &  Il personaggio ottiene un modificatore di -2 alla difficoltà per ogni TDS su Schivare.\\\hline
          \multicolumn{2}{l}{  \textit{Estremamente fortunato}   } &Se il personaggio ha fallito un TDS su Schivare, tira un D6. Con un risultato pari a 6 il tiro verrà considerato come superato.  \\\hline
         \multicolumn{2}{l}{   \textit{Leggerezza}      } & Se il personaggio non indossa pezzi di armatura media o pesante, ha un modificatore di -2 ai danni subiti dalle trappole. \\\hline
         \multicolumn{2}{l}{   \textit{Limitare i danni}     }& Se il personaggio fallisce un TDS su Schivare, può scegliere su quale parte del corpo subire gli effetti.\\
\hline
\end{tabularx}



\begin{tabularx}{\linewidth}{m|s|b}
\hline
           \textbf{Seguire tracce}      &     \textit{\textbf{Intelligenza}}      &      Questa abilità permette al personaggio, una volta individuate delle tracce, di seguirle. Alternativamente, può essere usata per provare a capire quale sia l'origine delle tracce stesse.   \\
\hline
\multicolumn{3}{c}{\textbf{Specializzazioni}}           \\
\hline
\multirow{2}{*}{\textit{Tracce Animali}} &1 &    Il personaggio ha un modificatore di -1 alla difficoltà nel riconoscere tracce di animali a lui noti.    \\
                  & 2&           Rinvenute le tracce, il personaggio ha un modificatore di -1 alla difficoltà per capire quanto le tracce siano recenti, identificare con maggior precisione il tipo di creatura (se la conosce).   \\\hline
\multirow{2}{*}{\textit{Tracce Umane}} &  1  &   Il personaggio ha un modificatore di -1 alla difficoltà nel riconoscere tracce appartenenti a un soggetto umanoide.    \\
                  &  2    &         Rinvenute le tracce, il personaggio ha un modificatore di -1 alla difficoltà per capire quanto le tracce siano recenti, identificare con una buona precisione il peso di chi le ha lasciate e capire (se possibile) come sia equipaggiato. \\ \hline
\multirow{2}{*}{\textit{spec3}} &  1  &   spec31     \\
                  &  2    &        spec32   \\ 
\hline
\multicolumn{3}{c}{\textbf{Talenti}}           \\
\hline
      \multicolumn{2}{l}{  \textit{Scaltrezza}  } &Il personaggio conosce i trucchi per depistare gli inseguitori, per lui sarà molto più difficile essere ingannato. \\\hline
          \multicolumn{2}{l}{  \textit{Tracce Persistenti}   } &Il personaggio è in grado seguire le tracce del suo bersaglio anche in condizioni avverse. Se subisce un modificatore alla difficoltà nei TDS su "Seguire tracce" compreso tra 1 e 4 (inclusi) a causa di condizioni sfavorevoli, ignora quel modificatore.   \\\hline
         \multicolumn{2}{l}{   \textit{Indagine}      } & Il personaggio vedendo un'insieme di tracce può provare a ricostruire gli eventi che le hanno causate. Se lo fa, ha un modificatore di -2 alla difficoltà. Esempio: \textit{vedendo una finestra rotta all'interno di una stanza, si può provare a capire con maggior facilità se il soggetto stesse entrando o uscendo, come sia passato ecc ecc.} \\\hline
         \multicolumn{2}{l}{   \textit{Identificazione}     }& Se il personaggio prova a riconoscere le caratteristiche fisiche e il tipo di creatura/personaggio che ha lasciato delle tracce ha un modificatore di -3 alla difficoltà\\
\hline
\end{tabularx}

\begin{tabularx}{\linewidth}{m|s|b}
\hline
           \textbf{Sopravvivenza}      &     \textit{\textbf{Sensibilità}}      &    Il personaggio utilizza questa abilità per provare a sopravvivere in ambienti non urbani. In particolare può essere utilizzata per orientarsi, cercare cibo, costruire un accampamento, accendere un fuoco o nascondere le tracce lasciate dal proprio accampamento.    \\
\hline
\multicolumn{3}{c}{\textbf{Specializzazioni}}           \\
\hline
\multirow{2}{*}{\textit{Razionamento}} &1 &    desc11    \\
                  & 2&           desc12   \\\hline
\multirow{2}{*}{\textit{spec2}} &  1  &   spec21    \\
                  &  2    &         spec22 \\ \hline
\multirow{2}{*}{\textit{spec3}} &  1  &   spec31     \\
                  &  2    &        spec32   \\ 
\hline
\multicolumn{3}{c}{\textbf{Talenti}}           \\
\hline
      \multicolumn{2}{l}{  \textit{tal1}  } &dt1 \\\hline
          \multicolumn{2}{l}{  \textit{tal2}   } &dt2   \\\hline
         \multicolumn{2}{l}{   \textit{tal3}      } &dt3 \\\hline
         \multicolumn{2}{l}{   \textit{tal4}     }&dt4\\\hline
          \multicolumn{2}{l}{  \textit{tal5}      }&dt5\\
\hline
\end{tabularx}


%template per l'abilità nuova
\begin{tabularx}{\linewidth}{m|s|b}
\hline
           \textbf{nome}      &     \textit{\textbf{stat}}      &      desc    \\
\hline
\multicolumn{3}{c}{\textbf{Specializzazioni}}           \\
\hline
\multirow{2}{*}{\textit{spec1}} &1 &    desc11    \\
                  & 2&           desc12   \\\hline
\multirow{2}{*}{\textit{spec2}} &  1  &   spec21    \\
                  &  2    &         spec22 \\ \hline
\multirow{2}{*}{\textit{spec3}} &  1  &   spec31     \\
                  &  2    &        spec32   \\ 
\hline
\multicolumn{3}{c}{\textbf{Talenti}}           \\
\hline
      \multicolumn{2}{l}{  \textit{tal1}  } &dt1 \\\hline
          \multicolumn{2}{l}{  \textit{tal2}   } &dt2   \\\hline
         \multicolumn{2}{l}{   \textit{tal3}      } &dt3 \\\hline
         \multicolumn{2}{l}{   \textit{tal4}     }&dt4\\\hline
          \multicolumn{2}{l}{  \textit{tal5}      }&dt5\\
\hline
\end{tabularx}

 %termine della lista della miscellanea

\end{document}