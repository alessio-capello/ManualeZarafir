\documentclass[../manuale_main.tex]{subfiles}



\begin{document}

Questo manuale utilizza un sistema di abilità e specializzazioni per la costruzione del proprio personaggio: l'abilità e il suo punteggio indicano la capacità del personaggio di svolgere un'azione, mentre la specializzazione indica in quale attività specifica il personaggio sia più ferrato.

\subsection{Migliorare le proprie abilità}
Per alzare il livello delle proprie abilità vengono utilizzati i punti esperienza assegnati dal master al termine di una o più sessioni.\\
Per aumentare di un livello una propria abilità è necessario spendere tanti punti esperienza pari al livello che si vuole raggiungere.\\
\emph{Per portare l'abilità ``Armi a una mano” dal livello 4 al 5, sarà necessario spendere 5 punti esperienza. Conseguentemente per portarla dal livello 5 al livello 7 sarà necessario spendere 13 punti esperienza (6+7).}\\
Quando si desidera apprendere una nuova abilità (ovvero portarla dal valore 0 all'1) è necessario inoltre superare un TDS a difficoltà 1 sull'abilità che si intende imparare oltre a spendere 1 punto esperienza. Il master potrebbe far ignorare il TDS purché il personaggio abbia svolto un'azione legata all'abilità da apprendere.\\
\emph{Per imparare ad utilizzare armi a una mano, è necessario spendere un punto esperienza superare un TDS su ``Armi ad una mano”.  In caso di riuscita si ottiene l'abilità all'1.}

\subsection{Specializzazioni e Talenti}
Ogni abilità ha diverse specializzazioni e talenti; le specializzazioni migliorano un determinato utilizzo dell'abilità, il talento offre dei bonus superiori.

Le specializzazioni sono divise in due livelli ciascuna, i bonus ottenuti con il livello 1 di una specializzazione non vengono persi ottenendo il livello 2, ma si sommati.
I talenti invece non hanno livelli, ma possono essere ottenuti solamente avendo una specializzazione al livello 2 o due specializzazioni all'1.

In termini generali, il livello 1 della specializzazione indica che il personaggio conosce i rudimenti per svolgere una specifica azione, il livello 2 indica una competenza superiore alla media.

\textbf{Non possedere una specializzazione non implica il non poter svolgere l'azione relativa (a meno che non sia indicato diversamente nella descrizione dell'abilità), ma non si avranno a disposizione i bonus.}

Anche se non è esplicitamente indicato, i talenti sono riferiti solamente ad azioni relative all'oggetto o abilità con la quale il talento è stato ottenuto: il talento ``Presa di ferro” relativo ad ``Armi a una mano”, impedisce di essere disarmati quando si usa un'arma una mano; non impedisce di essere disarmati usando un altro tipo di arma (per esempio, un arco o una lancia).

Si ottengono in totale 3 punti da assegnare alle specializzazioni o talenti dell'abilità portandola dallo 0 al 10:
\begin{itemize}
\item il primo punto si guadagna ottenendo l'abilità stessa (portandola quindi dallo 0 all'1);
\item il secondo punto si ottiene portandola al 6;
\item il terzo punto si ottiene portandola al 10.
\end{itemize}
I punti specializzazione ottenuti in un'abilità possono essere spesi solamente nelle specializzazioni o talenti relativi a quell'abilità.

Conseguentemente, per ogni abilità al 10, sarà possibile avere una delle seguenti:
\begin{itemize}
\item una specializzazione al 2 e un talento;
\item una specializzazione al 2 e una seconda specializzazione all'1;
\item due specializzazioni all'1 e un talento.
\item tre specializzazioni all'1.
\end{itemize}
I punti Specializzazione devono essere investiti all'ottenimento del punto stesso, non possono essere accumulati.

Non è possibile riassegnare le specializzazioni o talenti (a meno che il master non ve lo permetta a seguito di eventi specifici).

\clearpage
\subsection{Lista abilità e talenti}

In questa sezione sono elencate tutte le abilità con relative specializzazioni e talenti.
Sotto al nome dell'abilità è indicata la caratteristica di riferimento. Se un'abilità è indicata come \emph{Passiva} significa non viene usata per i TDS su di essa, ma fornisce bonus  al personaggio. Per queste abilità viene comunque indicata la statistica più appropriata, per permettere al master di dotarle di un uso attivo.\\
È segnata la descrizione dell'abilità, in alcuni casi sono presenti degli esempi d'uso per le abilità meno intuitive.
Le specializzazioni e i talenti sono indicati sotto a ogni abilità, le specializzazioni sono distinte nei due livelli che si possono ottenere.


%abilità da combattimento
\subsubsection{Abilità da combattimento}


\renewcommand{\arraystretch}{1.2}

\begin{center}
\textbf{ \large{Armi a distanza}}\\ \textit{\textbf{Agilità}}\\
\end{center}

 Questa abilità si utilizza per colpire i nemici con armi da distanza come archi, balestre o giavellotti.    


\begin{tabularx}{\linewidth}{|m|s|b|}
\hline
\multicolumn{3}{|c|}{\textbf{Specializzazioni}}           \\
\hline
\multirow{2}{*}{\textit{Archi}} &1 &     Questo tipo di arma non è più considerata come arma improvvisata.    \\
                  & 2&            Il personaggio ha ottenuto una buona maestria con questo tipo di armi, avendo un modificatore di -1 alla difficoltà.   \\\hline
\multirow{2}{*}{\textit{Balestre}} &  1  &   Questo tipo di arma non è più considerata come arma improvvisata.      \\
                  &  2    &          Il personaggio ha ottenuto una buona maestria con questo tipo di armi, avendo un modificatore di -1 alla difficoltà.   \\ \hline
\multirow{2}{*}{\textit{Armi da lancio}} &  1  &   Questo tipo di arma non è più considerata come arma improvvisata.      \\
                  &  2    &          Il personaggio ha ottenuto una buona maestria con questo tipo di armi, avendo un modificatore di -1 alla difficoltà.   \\ 
\hline
\multicolumn{3}{|c|}{\textbf{Talenti}}           \\
\hline
       \textit{Disarmare}  & \multicolumn{2}{k|}{Il personaggio può disarmare un bersaglio colpito con armi a distanza.} \\\hline
       \textit{Punte Acuminate} &\multicolumn{2}{k|}{Il personaggio ignora ogni armatura con grado di protezione pari o inferiore a 2 se utilizza armi a distanza.} \\\hline
       \textit{Cambio Rapido}  &\multicolumn{2}{k|}{Passare dall'utilizzo di armi a distanza a quelle per il corpo a corpo (e viceversa) non richiede di perdere un turno di combattimento.} \\\hline
      \textit{Armi Pesanti} &\multicolumn{2}{k|}{Il personaggio non ha difficoltà addizionale (data dall'arma) se utilizza un'arma pesante a distanza.}\\\hline
      \textit{Esperto} &\multicolumn{2}{k|}{Per personaggio nessuna arma da distanza (che fa parte delle specializzazioni di questa abilità) è considerata come arma improvvisata. .}\\
\hline
\end{tabularx}

\clearpage

\begin{center}
\textbf{ \large{Armi a due mani}}\\ \textit{\textbf{Forza}}\\
\end{center}

Questa abilità si utilizza per colpire i nemici in combattimento con armi a due mani come Spadoni, Martelli, Asce. 


\begin{tabularx}{\linewidth}{|m|s|b|}
\hline
\multicolumn{3}{|c|}{\textbf{Specializzazioni}}           \\
\hline
\multirow{2}{*}{\textit{Spadoni}} &1 &     Questo tipo di arma non è più considerata come arma improvvisata.    \\
                  & 2&            Il personaggio ha ottenuto una buona maestria con questo tipo di armi, avendo un modificatore di -1 alla difficoltà.   \\\hline
\multirow{2}{*}{\textit{Martelli o Mazze}} &  1  &   Questo tipo di arma non è più considerata come arma improvvisata.      \\
                  &  2    &          Il personaggio ha ottenuto una buona maestria con questo tipo di armi, avendo un modificatore di -1 alla difficoltà.   \\ \hline
\multirow{2}{*}{\textit{Asce}} &  1  &   Questo tipo di arma non è più considerata come arma improvvisata.      \\
                  &  2    &          Il personaggio ha ottenuto una buona maestria con questo tipo di armi, avendo un modificatore di -1 alla difficoltà.   \\ 
\hline
\multicolumn{3}{|c|}{\textbf{Talenti}}           \\
\hline
     \textit{Presa di ferro}  &\multicolumn{2}{k|}{Il personaggio non può essere disarmato quando utilizza armi a due mani. }\\\hline
       \textit{Colpire alle mani}  &\multicolumn{2}{k|}{Il personaggio ignora ogni armatura con grado di protezione pari o inferiore a 2 se utilizza armi a distanza.} \\\hline
     \textit{Danneggia armatura}  & \multicolumn{2}{k|}{ Se si infliggono più di 8 danni a un bersaglio (calcolati prima della riduzione data dall'armatura) utilizzando armi a due mani, l'armatura presente sulla zona colpita abbassa il suo livello di protezione di 1 punto (permanentemente). }\\\hline
       \textit{Attacco roteante} &  \multicolumn{2}{k|}{Il personaggio ha la possibilità di infliggere la metà dei danni, colpendo però tutti i bersagli che lo circondano (e che possono essere colpiti dalla sua arma), alleati inclusi. Per colpire verrà utilizzata la difficoltà più alta data dai bersagli, il danno viene calcolato una singola volta e ridotto dalle eventuali protezioni di ciascun bersaglio. Utilizzabile solo quando il personaggio utilizza armi a due mani.}\\\hline
        \textit{Esperto}     & \multicolumn{2}{k|}{Per personaggio nessuna arma a due mani(che fa parte delle specializzazioni di questa abilità) è considerata come arma improvvisata. }\\
\hline
\end{tabularx}


\clearpage

\begin{center}
\textbf{ \large{Armi a una mano}}\\ \textit{\textbf{Forza}}\\
\end{center}

Questa abilità si utilizza per colpire i nemici in combattimento con armi bianche a una mano come spade, pugnali e mazze.

\begin{tabularx}{\linewidth}{|m|s|b|}
\hline

\multicolumn{3}{|c|}{\textbf{Specializzazioni}}           \\
\hline
\multirow{2}{*}{\textit{Spade}} &1 &     Questo tipo di arma non è più considerata come arma improvvisata.    \\
                  & 2&            Il personaggio ha ottenuto una buona maestria con questo tipo di armi, avendo un modificatore di -1 alla difficoltà.   \\\hline
\multirow{2}{*}{\textit{Pugnali o Daghe}} &  1  &   Questo tipo di arma non è più considerata come arma improvvisata.      \\
                  &  2    &          Il personaggio ha ottenuto una buona maestria con questo tipo di armi, avendo un modificatore di -1 alla difficoltà.   \\ \hline
\multirow{2}{*}{\textit{Asce}} &  1  &   Questo tipo di arma non è più considerata come arma improvvisata.      \\
                  &  2    &          Il personaggio ha ottenuto una buona maestria con questo tipo di armi, avendo un modificatore di -1 alla difficoltà.   \\ \hline
\multirow{2}{*}{\textit{Mazze e martelli (a una mano)}} &  1  &   Questo tipo di arma non è più considerata come arma improvvisata.      \\
                  &  2    &          Il personaggio ha ottenuto una buona maestria con questo tipo di armi, avendo un modificatore di -1 alla difficoltà.   \\ 
\hline
\multicolumn{3}{|c|}{\textbf{Talenti}}           \\
\hline
    \textit{Presa di ferro}& \multicolumn{2}{k|}{  Il personaggio non può essere disarmato quando utilizza armi a una mano. }\\\hline
      \textit{Colpire alle mani}&\multicolumn{2}{k|}{  Il personaggio ignora ogni armatura con grado di protezione pari o inferiore a 2 se utilizza armi a distanza. }\\\hline
      \textit{Attacco furtivo}  &\multicolumn{2}{k|}{  Se il personaggio attacca un bersaglio con un'arma leggera a una mano quando è nascosto, ha un bonus di -3 alla difficoltà per colpirlo. }\\\hline
       \textit{Armi gemelle} &\multicolumn{2}{k|}{   DA FARE   }	\\\hline
      \textit{Esperto}&\multicolumn{2}{k|}{  Per personaggio nessuna arma a una mano (che fa parte delle specializzazioni di questa abilità) è considerata come arma improvvisata. }\\
\hline
\end{tabularx}

\clearpage


\begin{center}
\textbf{ \large{Armi ad asta}}\\ \textit{\textbf{Agilità}}\\
\end{center}
 Questa abilità si utilizza per colpire i nemici in combattimento con armi ad asta come Lance, Alabarde, Picche. 

\begin{tabularx}{\linewidth}{|m|s|b|}
\hline

\multicolumn{3}{|c|}{\textbf{Specializzazioni}}           \\
\hline
\multirow{2}{*}{\textit{Lance o Picche}} &1 &     Questo tipo di arma non è più considerata come arma improvvisata.    \\
                  & 2&            Il personaggio ha ottenuto una buona maestria con questo tipo di armi, avendo un modificatore di -1 alla difficoltà.   \\\hline
\multirow{2}{*}{\textit{Alabarde}} &  1  &   Questo tipo di arma non è più considerata come arma improvvisata.      \\
                  &  2    &          Il personaggio ha ottenuto una buona maestria con questo tipo di armi, avendo un modificatore di -1 alla difficoltà.   \\ 
\hline
\multicolumn{3}{|c|}{\textbf{Talenti}}           \\
\hline
       \textit{Presa di ferro} & \multicolumn{2}{k|}{   Il personaggio non può essere disarmato quando utilizza armi ad asta.}\\\hline
       \textit{Affondo letale} &\multicolumn{2}{k|}{  Il personaggio ignora ogni armatura con grado di protezione pari o inferiore a 3 quando utilizza armi ad asta. }\\\hline
         \textit{Disarcionare} &\multicolumn{2}{k|}{   Attaccando un bersaglio a cavallo utilizzando armi ad asta si ha la possibilità di disarcionarlo, infliggendogli i danni di un normale attacco più quelli della caduta.}\\\hline
         \textit{Sanguinamento}  &\multicolumn{2}{k|}{   Colpire un bersaglio in un punto scoperto dall'armatura con un'arma ad asta gli causerà ferite profonde. Subisce 1 danno ogni turno finché la ferita non viene curata.}\\\hline
         \textit{Esperto} &\multicolumn{2}{k|}{  Per personaggio nessuna arma ad asta (che fa parte delle specializzazioni di questa abilità) è considerata come arma improvvisata.}\\
\hline
\end{tabularx}


\clearpage


\begin{center}
\textbf{ \large{Arte della guerra}}\\ \textit{\textbf{ Forza  - Passiva}}
\\
\end{center}
Questa abilità passiva permette di aumentare i danni inflitti con le armi da mischiao. Ogni tre punti investiti in questa abilità si ha modificatore di +1 danni in combattimento corpo a corpo quando si utilizza un'arma da mischia (o a mani nude con l'apposita specializzazione). 

\begin{tabularx}{\linewidth}{|m|s|b|}
\hline

\multicolumn{3}{|c|}{\textbf{Specializzazioni}}           \\
\hline
\multirow{2}{*}{\textit{Armi leggere}} &1 &   Modificatore di +1 danni utilizzando armi da mischia leggere.   \\
                  & 2&       Modificatore di +1 danni utilizzando armi leggere. \\\hline
\multirow{2}{*}{\textit{Armi medie}} &  1  &    Modificatore di +1 danni utilizzando armi da mischia medie. \\
                  &  2    &    Modificatore di +1 danni utilizzando armi medie.\\ \hline
\multirow{2}{*}{\textit{Armi pesanti}} &  1  &    Modificatore di +1 danni utilizzando armi da mischia pesanti.\\
                  &  2    &   Modificatore di +1 danni utilizzando armi pesanti.\\ \hline
\multirow{2}{*}{\textit{Mani nude}} &  1  &    I modificatori ai danni di arte della guerra vengono applicati anche quando si combatte a mani nude..\\
                  &  2    &    Modificatore di +1 danni combattendo a mani nude.\\ 
\hline
\multicolumn{3}{|c|}{\textbf{Talenti}}           \\
\hline
     \textit{Furia cieca}  &\multicolumn{2}{k|}{  Se il personaggio attacca un bersaglio con un'arma da mischia pesante e non indossa armatura, ha un modificatore di +3 danni.}\\\hline
      \textit{Attacco a sorpresa}   &  \multicolumn{2}{k|}{ Se il personaggio attacca un bersaglio con un'arma da mischia non pesante quando è nascosto, ha un modificatore di +2 danni. }\\\hline
         \textit{Brutalità}     & \multicolumn{2}{k|}{  Il personaggio ha un modificatore di +2 danni se colpisce il bersaglio in un punto non protetto dall'armatura.}\\\hline
       \textit{Guerriero esperto}     & \multicolumn{2}{k|}{ Il personaggio ha un modificatore di +1 danni quando utilizza armi da mischia medie.}\\\hline
        \textit{Pugile}     &\multicolumn{2}{k|}{  Il personaggio ha un modificatore di +2 danni se combatte a mani nude.}\\
\hline
\end{tabularx}


\begin{center}
\textbf{ \large{Evitare}}\\ \textit{\textbf{ Reattività  - Passiva}}
\\
\end{center}
 Condiziona l'\emph{evasione} del personaggio. Ogni due punti investiti in quest'abilità si ottiene un modificatore di +1 all'evasione.

\begin{tabularx}{\linewidth}{|m|s|b|}
\hline

\multicolumn{3}{|c|}{\textbf{Specializzazioni}}           \\
\hline
\multirow{2}{*}{\textit{Leggiadro}} &1 &   Modificatore di +1 all'evasione.    \\
                  & 2&       Modificatore di +1 all'evasione.\\\hline
\multirow{2}{*}{\textit{Statico}} &  1  &    Modificatore di -1 all'evasione. \\
                  &  2    &    Danni subiti in combattimento ridotti di 1.\\ 
\hline
\multicolumn{3}{|c|}{\textbf{Talenti}}           \\
\hline
       \textit{Evasivo}  &\multicolumn{2}{k|}{   Il personaggio ottiene un modificatore di +3 all'evasione.}\\\hline
        \textit{Estremamente fortunato}   &  \multicolumn{2}{k|}{ Se il personaggio è stato colpito in un combattimento, tira un D6. Con un risultato pari a 6 potrà ignorare l'attacco.} \\\hline
        \textit{Corazzato}      & \multicolumn{2}{k|}{   Il personaggio non può essere colpito in zone critiche (punteggio pari a 5 sul tipo per determinare la direzione del colpo), ma ottiene un modificatore di -5 all'evasione.}\\\hline
        \textit{Re delle risse}   &\multicolumn{2}{k|}{  Il personaggio ottiene un bonus di +1 all'evasione per ogni avversario che si trova in combattimento (corpo a corpo) con lui.}\\
\hline
\end{tabularx}


\clearpage


\begin{center}
\textbf{ \large{Mani nude}}\\ \textit{\textbf{ Forza}}
\\
\end{center}
Questa abilità si utilizza per colpire i nemici il corpo a corpo senza utilizzare un'arma.

\begin{tabularx}{\linewidth}{|m|s|b|}
\hline

\multicolumn{3}{|c|}{\textbf{Specializzazioni}}           \\
\hline
\multirow{2}{*}{\textit{Colpi forti}} &1 &    Il personaggio ha un modificatore di +1 ai danni inflitti combattendo a mani nude.    \\
                  & 2&         Il personaggio ha un modificatore di +1 ai danni inflitti combattendo a mani nude.   \\\hline
\multirow{2}{*}{\textit{Colpi precisi}} &  1  &   Il personaggio ha un modificatore di -1 alla difficoltà combattendo a mani nude.   \\
                  &  2    &     Il personaggio ha un modificatore di -1 alla difficoltà combattendo a mani nude.\\ 
\hline
\multicolumn{3}{|c|}{\textbf{Talenti}}           \\
\hline
      \textit{Colpire alle mani} &  \multicolumn{2}{k|}{  Se il bersaglio dell'attacco viene disarmato, subisce anche i danni di un normale attacco (conta come attacco alle braccia). }\\\hline
       \textit{Picchiatore preciso}  & \multicolumn{2}{k|}{    Il personaggio può scegliere dove colpire quando attacca a mani nude (tra Gambe, Busto, Braccia, Testa). }\\\hline
       \textit{Scudiero}     & \multicolumn{2}{k|}{   Il modificatore di -2 danni quando una delle due mani è impegnata è ridotto a -1.}\\\hline
      \textit{Signore delle risse}   & \multicolumn{2}{k|}{  Il personaggio ottiene un modificatore di -1 alla difficoltà per colpire per ogni avversario che si trova in combattimento (corpo a corpo) con lui.}\\
\hline
\end{tabularx}


\begin{center}
\textbf{ \large{Parare}}\\ \textit{\textbf{ Forza}}
\\
\end{center}
 Questa abilità permette di parare i colpi subiti utilizzando uno scudo.

\begin{tabularx}{\linewidth}{|m|s|b|}
\hline

\multicolumn{3}{|c|}{\textbf{Specializzazioni}}           \\
\hline
\multirow{2}{*}{\textit{Scudi leggeri}} &1 &    Modificatore di -1 alla difficoltà per parare utilizzando scudi leggeri.    \\
                  & 2&           Modificatore di -1 alla difficoltà per parare utilizzando scudi leggeri.   \\\hline
\multirow{2}{*}{\textit{Scudi medi}} &  1  &   Modificatore di -1 alla difficoltà per parare utilizzando scudi medi.    \\
                  &  2    &        La protezione offerta da uno scudo medio ha un modificatore di +1. \\ \hline
\multirow{2}{*}{\textit{Scudi pesanti}} &  1  &  La protezione offerta da uno scudo pesante ha un modificatore di +1.    \\
                  &  2    &       La protezione offerta da uno scudo pesante ha un modificatore di +1.   \\ 
\hline
\multicolumn{3}{|c|}{\textbf{Talenti}}           \\
\hline
      \textit{Guardiano} & \multicolumn{2}{k|}{  Modificatore di -2 alla difficoltà per parare. Il personaggio ha un modificatore di -2 danni inflitti con armi e magie.} \\\hline
         \textit{Protettore}& \multicolumn{2}{k|}{  Il personaggio può parare un attacco non magico rivolto a un bersaglio al suo fianco senza difficoltà addizionale.}  \\\hline
           \textit{Contrattacco}   &\multicolumn{2}{k|}{   Il personaggio ha un modificatore di +1 danni contro un bersaglio dopo aver parato un suo attacco.}\\\hline
         \textit{Difensore}   & \multicolumn{2}{k|}{   Il personaggio non subisce i malus dati dall'utilizzo di uno scudo pesante.}\\
\hline
\end{tabularx}

\clearpage

\begin{center}
\textbf{ \large{Precisione}}\\ \textit{\textbf{  Passiva - Agilità}}
\\
\end{center}
Questa abilità permette di aumentare i danni inflitti con le armi a distanza. Ogni tre punti investiti in questa abilità si ha un modificatore di +1 danno utilizzando questo tipo di armi.

\begin{tabularx}{\linewidth}{|m|s|b|}
\hline

\multicolumn{3}{|c|}{\textbf{Specializzazioni}}           \\
\hline
\multirow{2}{*}{\textit{Armi leggere}} &1 &    Modificatore di +1 danni utilizzando armi a distanza leggere.    \\
                  & 2&           Modificatore di +1 danni utilizzando armi leggere.   \\\hline
\multirow{2}{*}{\textit{Armi medie}} &  1  &   Modificatore di +1 danni utilizzando armi a distanza medie.    \\
                  &  2    &        Modificatore di +1 danni utilizzando armi medie. \\ \hline
\multirow{2}{*}{\textit{Armi pesanti}} &  1  &  Modificatore di +1 danni utilizzando arm a distanzai pesanti.     \\
                  &  2    &        Modificatore di +1 danni utilizzando armi pesanti.   \\ 
\hline
\multicolumn{3}{|c|}{\textbf{Talenti}}           \\
\hline
        \textit{Tiratore leggero}  &  \multicolumn{2}{k|}{ Se il personaggio non indossa armatura quando attacca un bersaglio con un'arma a distanza, ha un modificatore di +2 danni. }\\\hline
         \textit{Tiratore a cavallo}  & \multicolumn{2}{k|}{ Il personaggio può utilizzare armi a distanza mentre è a cavallo senza difficoltà extra. } \\\hline
        \textit{Tiratore scelto}     &\multicolumn{2}{k|}{  Il personaggio ha un modificatore di +2 danni se colpisce il bersaglio con un'arma a distanza in un punto non protetto dall'armatura.}\\\hline
          \textit{Colpi efficaci}   &\multicolumn{2}{k|}{  Il personaggio ha un modificatore di +1 danno quando utilizza armi a distanza.}\\
\hline
\end{tabularx}



\begin{center}
\textbf{ \large{Salute}}\\ \textit{\textbf{  Passiva - Costituzione}}
\\
\end{center}
Questa abilità aumenta il numero di punti vita massimi a disposizione del giocatore. Ogni due punti investiti in quest'abilità il massimale dei propri punti vita aumenta di 1. 

\begin{tabularx}{\linewidth}{|m|s|b|}
\hline

\multicolumn{3}{|c|}{\textbf{Specializzazioni}}           \\
\hline
\multirow{2}{*}{\textit{Resistenza migliorata}} &1 &   Massimale dei propri punti vita aumentato ulteriormente di 1.    \\
                  & 2&           Massimale dei propri punti vita aumentato ulteriormente di 2.  \\\hline
\multirow{2}{*}{\textit{Pelle spessa}} &  1  &   Massimale dei propri punti vita aumentato ulteriormente di 1.    \\
                  &  2    &         Danni subiti ridotti di 1. \\ 
\hline
\multicolumn{3}{|c|}{\textbf{Talenti}}           \\
\hline
     \textit{Fisico bestiale} & \multicolumn{2}{k|}{  Massimale dei propri punti vita aumentato ulteriormente di 3.}\\\hline
        \textit{Ottima resistenza al dolore} & \multicolumn{2}{k|}{ Il personaggio ha una soglia del dolore così alta che è in grado di resistere allo svenimento quando sembra praticamente impossibile. Un personaggio con questo talento non sverrà se i suoi punti vita dovessero scendere a 0 o meno di zero a causa di un attacco fisico. Conseguentemente però tutte le azioni che il personaggio deve svolgere hanno un modificatore di +2 alla difficoltà. Questo talento non previene la morte. }  \\\hline
      \textit{Ottima resistenza al veleno} & \multicolumn{2}{k|}{ Massimale dei propri punti vita aumentato ulteriormente di 1. Il personaggio riduce i danni subiti da ogni veleno non letale di 1.}\\\hline

\end{tabularx}

\clearpage


%abilità magiche
\subsubsection{Abilità legate alla magia}


\begin{center}
\textbf{ \large{Meditare}}\\ \textit{\textbf{  Sensibilità}}
\\
\end{center}
Questa abilità riduce il costo in mana di ogni magia di un ammontare pari al livello di questa abilità (il costo minimo raggiungibile da una magia è comunque fissato a 1). Il personaggio può scegliere di cadere in trance (non potendo quindi svolgere nessuna azione) per recuperare 1 punto mana per ogni ora di meditazione. Non è possibile recuperare punti mana indossando un'armatura pesante.

\begin{tabularx}{\linewidth}{|m|s|b|}
\hline
\multicolumn{3}{|c|}{\textbf{Specializzazioni}}           \\
\hline
\multirow{2}{*}{\textit{Meditazione migliorata}} &1 &  +2 punti mana recuperati ogni ora durante la meditazione.  \\
                  & 2&     +3 punti mana recuperati ogni ora durante la meditazione.  \\\hline
\multirow{2}{*}{\textit{Concentrazione assoluta}} &  1  &  Il personaggio può recuperare punti mana indossando un'armatura pesante. \\
                  &  2    &   +2 punti mana recuperati ogni ora durante la meditazione.  \\ 
\hline
\multicolumn{3}{|c|}{\textbf{Talenti}}           \\
\hline
      \textit{Ascetismo}  & \multicolumn{2}{k|}{ +5 punti mana recuperati ogni ora durante la meditazione. }\\\hline
     \textit{Comunione con l'energia}   &\multicolumn{2}{k|}{  Il personaggio recupera 2 punti mana ogni ora. }  \\\hline
        \textit{Riserva di mana}      &\multicolumn{2}{k|}{  Il personaggio ha il massimale dei punti mana aumentato di 4.}\\\hline
       \textit{Magia insanguinata}    & \multicolumn{2}{k|}{ Il personaggio può scegliere di pagare il costo in mana di una magia con i punti vita anziché con il mana. Non è possibile pagare parte del costo della magia con i punti mana e il rimanente con i punti vita. Non è possibile pagare in punti vita il costo di una magia che ha come effetto quello di restituire punti vita. Questo costo non potrà far scendere i punti vita sotto zero.} \\\hline

\end{tabularx}


\begin{center}
\textbf{ \large{Resistenza magica}}\\ \textit{\textbf{  Sensibilità}}
\\
\end{center}
 Questa abilità permette al personaggio di provare a resistere a una magia che lo bersaglia. In caso di successo potrà ridurre l'efficacia o i danni subiti. Le specializzazioni sono riferite ai diversi comportamenti che il personaggio può adottare contro magie offensive o non offensive. La spiegazione approfondita sul funzionamento è nella sezione dedicata alla magia.

\begin{tabularx}{\linewidth}{|m|s|b|}
\hline
\multicolumn{3}{|c|}{\textbf{Specializzazioni}}           \\
\hline
\multirow{2}{*}{\textit{Magie offensive}} &1 &  Il personaggio riduce di 1 i danni subiti dalle magie offensive.    \\
                  & 2&          Il personaggio riduce di 1 i danni subiti dalle magie offensive.    \\\hline
\multirow{2}{*}{\textit{Magie non offensive}} &  1  &   La difficoltà per lanciare una magia non offensiva contro il personaggio ha un modificatore di +1   \\
                  &  2    &         La difficoltà per lanciare una magia non offensiva contro il personaggio ha un modificatore di +1  \\ \hline

\multicolumn{3}{|c|}{\textbf{Talenti}}           \\
\hline
      \textit{Immunità magica}  & \multicolumn{2}{k|}{ Il personaggio è immune a tutte le magie aventi difficoltà di lancio pari o inferiore a 3.} \\\hline
          \textit{Invisibile all'occhio magico} & \multicolumn{2}{k|}{ Il personaggio rende invisibile la sua aura magica, per questo motivo non può essere individuato grazie ad essa.}   \\\hline
          \textit{Immune ai condizionamenti}   & \multicolumn{2}{k|}{ Il personaggio è immune a ogni forma di condizionamento. Su di lui le maledizioni e benedizioni non avranno effetto.} \\\hline
         \textit{Refrattario alle maledizioni}   & \multicolumn{2}{k|}{ Il personaggio riduce ulteriormente gli effetti negativi delle magie non offensive di un round.} \\\hline

\end{tabularx}


\begin{center}
\textbf{ \large{Stregoneria di attacco}}\\ \textit{\textbf{  Intelligenza}}
\\
\end{center}
 Questa abilità è utilizzata dal personaggio per lanciare stregonerie mirate infliggere danni ai bersagli o a incrementare i danni inflitti con un'arma.

\begin{tabularx}{\linewidth}{|m|s|b|}
\hline
\multicolumn{3}{|c|}{\textbf{Specializzazioni}}           \\
\hline
\multirow{2}{*}{\textit{Distruzione}} &1 &    Il personaggio può utilizzare le stregonerie di Distruzione con un grado massimo pari a 6.    \\
                  & 2&         Il personaggio può utilizzare le stregonerie di Distruzione con un grado massimo pari a 10.  Per le stregonerie di Distruzione il personaggio utilizza 1D6 per determinare la potenza anzichè 1D4.   \\\hline
\multirow{2}{*}{\textit{Incantamento}} &  1  &    Il personaggio può utilizzare le stregonerie di Incantamento con un grado massimo pari a 6.    \\
                  & 2&        Il personaggio può utilizzare le stregonerie di Incantamento con un grado massimo pari a 10.   Per le stregonerie di Incantamento il personaggio utilizza 1D6 per determinare la potenza anzichè 1D4.   \\\hline
\hline
\multicolumn{3}{|c|}{\textbf{Talenti}}           \\
\hline
       \textit{Follia}  & \multicolumn{2}{k|}{ Le stregonerie di Distruzione del personaggio infliggono 3 danni in più. Se il personaggio fallisce il TDS per lanciare una stregoneria di Distruzione, subisce 4 danni (non possono essere ridotti dall'armatura).}\\\hline
       \textit{Fretta}  & \multicolumn{2}{k|}{ Le stregonerie di Distruzione del personaggio hanno un modificatore di -2 alla difficoltà per essere lanciate, ma possono essere \textit{annullate} con un modificatore di -2 alla difficoltà.}  \\\hline
       \textit{Enfasi}   & \multicolumn{2}{k|}{Quando il personaggio attacca con un'arma incantata ha un modificatore di -1 alla difficoltà.}\\\hline
         \textit{Colpo concatenato}    & \multicolumn{2}{k|}{ Le stregonerie di Distruzione del personaggio propagano parte del loro effetto anche a coloro che sono vicini. Esse infatti infliggono 2 danni a tutti i coloro entro dieci metri dal bersaglio. Se il bersaglio della stregoneria muore, coloro subiranno invece 4 danni. Ciascun bersaglio riduce il danno in base ai propri modificatori.} \\
\hline
\end{tabularx}

\clearpage

\begin{center}
\textbf{ \large{Stregoneria di controllo}}\\ \textit{\textbf{  Intelligenza}}
\\
\end{center}
 Questa abilità è utilizzata dal personaggio per lanciare stregonerie mirate a negare il lancio di altre magie oppure per ostacolare i movimenti di un bersaglio.

\begin{tabularx}{\linewidth}{|m|s|b|}
\hline
\multicolumn{3}{|c|}{\textbf{Specializzazioni}}           \\
\hline
\multirow{2}{*}{\textit{Blocco}} &1 &    Il personaggio può utilizzare le stregonerie di Blocco con un grado massimo pari a 6.    \\
                  & 2&         Il personaggio può utilizzare le stregonerie di Blocco con un grado massimo pari a 10.   Per le stregonerie di Blocco il personaggio utilizza 1D6 per determinare la potenza anzichè 1D4.   \\\hline
\multirow{2}{*}{\textit{Antimagia}} &  1  &    Il personaggio può utilizzare le stregonerie di Antimagia con un grado massimo pari a 6.    \\
                  & 2&         Il personaggio può utilizzare le stregonerie di Antimagia con un grado massimo pari a 10.  Per le stregonerie di Antimagia il personaggio utilizza 1D6 per determinare la potenza anzichè 1D4.   \\\hline
\hline
\multicolumn{3}{|c|}{\textbf{Talenti}}           \\
\hline
     \textit{Magia silenziosa}  &\multicolumn{2}{k|}{ Il personaggio occulta il modo in cui utilizza le Stregonerie di Controllo, rendendo estremamente difficile capire quando le utilizzi.} \\\hline
      \textit{Aura di annientamento}   &\multicolumn{2}{k|}{Tutti coloro che si trovano entro 30 metri dal personaggio hanno un modificatore di +2 alla difficoltà per lanciare ogni magia.}  \\\hline
      \textit{Beffa}    &\multicolumn{2}{k|}{ Se il personaggio \textit{annulla} con successo una magia, colui che ha utilizzato quella magia perde 2 punti vita.}\\\hline
        \textit{Incatenare}       &\multicolumn{2}{k|}{Le Stregonerie di Blocco del personaggio durano il doppio dei round.} \\
\hline
\end{tabularx}

\clearpage

\begin{center}
\textbf{ \large{Stregoneria della materia}}\\ \textit{\textbf{  Intelligenza}}
\\
\end{center}
Questa abilità è utilizzata dal personaggio per lanciare stregonerie mirate ad alterare sè stesso o gli elementi naturali che lo circondano. 

\begin{tabularx}{\linewidth}{|m|s|b|}
\hline
\multicolumn{3}{|c|}{\textbf{Specializzazioni}}           \\
\hline
\multirow{2}{*}{\textit{Alterazione}} &1 &    Il personaggio può utilizzare le stregonerie di Alterazione con un grado massimo pari a 6.    \\
                  & 2&        Il personaggio può utilizzare le stregonerie di Alterazione con un grado massimo pari a 10.   Per le stregonerie di Alterazione il personaggio utilizza 1D6 per determinare la potenza anzichè 1D4.   \\\hline
\multirow{2}{*}{\textit{Metamorfosi}} &  1  &    Il personaggio può utilizzare le stregonerie di Metamorfosi con un grado massimo pari a 6.    \\
                  & 2&         Il personaggio può utilizzare le stregonerie di Metamorfosi con un grado massimo pari a 10.  Per le stregonerie di Metamorfosi il personaggio utilizza 1D6 per determinare la potenza anzichè 1D4.   \\\hline
\hline
\multicolumn{3}{|c|}{\textbf{Talenti}}           \\
\hline
    \textit{Lunga distanza}   &\multicolumn{2}{k|}{  Il personaggio può lanciare le Stregonerie di Alterazione al doppio della gittata.} \\\hline
           \textit{Cangiante}  &\multicolumn{2}{k|}{ Le Stregonerie di Metamorfosi lanciate dal personaggio durano 1 round in più.}  \\\hline
        \textit{Maestro degli elementi}    &\multicolumn{2}{k|}{Le Stregonerie di Alterazione lanciate dal personaggio durano il doppio.} \\\hline
       \textit{Manipolazione superiore}    &\multicolumn{2}{k|}{ Le stregonerie diella Materia del personaggio hanno un modificatore di -1 alla difficoltà per essere lanciate.} \\
\hline
\end{tabularx}


\clearpage

\begin{center}
\textbf{ \large{Stregoneria della mente}}\\ \textit{\textbf{  Intelligenza}}
\\
\end{center}
  Questa abilità è utilizzata dal personaggio per lanciare stregonerie mirate a creare illusioni o a manipolare le emozioni dei propri bersagli.

\begin{tabularx}{\linewidth}{|m|s|b|}
\hline
\multicolumn{3}{|c|}{\textbf{Specializzazioni}}           \\
\hline
\multirow{2}{*}{\textit{Illusionismo}} &1 &    Il personaggio può utilizzare le stregonerie di Illusionismo con un grado massimo pari a 6.    \\
                  & 2&        Il personaggio può utilizzare le stregonerie di Illusionismo con un grado massimo pari a 10.  Per le stregonerie di Illusionismo il personaggio utilizza 1D6 per determinare la potenza anzichè 1D4.   \\\hline
\multirow{2}{*}{\textit{Condizionamento}} &  1  &    Il personaggio può utilizzare le stregonerie di Condizionamento con un grado massimo pari a 6.    \\
                  & 2&        Il personaggio può utilizzare le stregonerie di Condizionamento con un grado massimo pari a 10.    Per le stregonerie di Condizionamento il personaggio utilizza 1D6 per determinare la potenza anzichè 1D4.   \\\hline
\hline
\multicolumn{3}{|c|}{\textbf{Talenti}}           \\
\hline
       \textit{Ingannatore}  &\multicolumn{2}{k|}{  Le illusioni costruite personaggio hanno un modificatore di +2 alla difficoltà per essere smascherate.} \\\hline
       \textit{Illusioni resistenti}  &\multicolumn{2}{k|}{ Le stregonerie di Illusionismo utilizzate dal personaggio hanno una durata minima pari a 5 minuti.}  \\\hline
        \textit{Emozioni profonde}   &\multicolumn{2}{k|}{ Le stregonerie di Condizionamento utilizzate dal personaggio hanno una durata raddoppiata.} \\\hline
        \textit{Impulso}     & \multicolumn{2}{k|}{ I bersagli condizionati dalle stregonerie del personaggio manifestano maggiormente le loro reazioni.} \\
\hline
\end{tabularx}

\clearpage

\begin{center}
\textbf{ \large{Stregoneria della vita}}\\ \textit{\textbf{  Intelligenza}}
\\
\end{center}
Questa abilità è utilizzata dal personaggio per lanciare stregonerie mirate a curare i propri bersagli o evocare delle creature alleate.

\begin{tabularx}{\linewidth}{|m|s|b|}
\hline
\multicolumn{3}{|c|}{\textbf{Specializzazioni}}           \\
\hline
\multirow{2}{*}{\textit{Guarigione}} &1 &    Il personaggio può utilizzare le stregonerie di Guarigione con un grado massimo pari a 6.    \\
                  & 2&        Il personaggio può utilizzare le stregonerie di Guarigione con un grado massimo pari a 10.   Per le stregonerie di Guarigione il personaggio utilizza 1D6 per determinare la potenza anzichè 1D4.   \\\hline
\multirow{2}{*}{\textit{Evocazione}} &  1  &    Il personaggio può utilizzare le stregonerie di Evocazione con un grado massimo pari a 6.    \\
                  & 2&        Il personaggio può utilizzare le stregonerie di Evocazione con un grado massimo pari a 10.  Per le stregonerie di Evocazione il personaggio utilizza 1D6 per determinare la potenza anzichè 1D4.   \\\hline
\hline
\multicolumn{3}{|c|}{\textbf{Talenti}}           \\
\hline
       \textit{Guarigione eccessiva}   &\multicolumn{2}{k|}{  Quando il personaggio utilizza la stregoneria di guarigione per guarire uno o più personaggi, se il valore della cura dovesse portare l'ammontare di punti vita sopra al valore massimo, converti la cura in eccesso in uno scudo magico (segue le regole degli scudi della stregoneria della vita). }\\\hline
        \textit{Guarigione perfetta}  &\multicolumn{2}{k|}{ Le stregonerie di Guarigione del personaggio non possono essere \emph{annullate}. }\\\hline
      \textit{Maestro evocatore}   &\multicolumn{2}{k|}{   Le evocazioni del personaggio durano 1 round in più. }  \\\hline
      \textit{Vibrosensi}      &\multicolumn{2}{k|}{   Il personaggio può vedere e sentire tramite gli occhi e orecchie (o organi di senso affini) delle sue evocazioni. }\\
\hline
\end{tabularx}

\clearpage

\begin{center}
\textbf{ \large{Tessimagie}}\\ \textit{\textbf{  Intelligenza}}
\\
\end{center}
Questa abilità viene utilizzata per lanciare gli incanti di "Arte della Sapienza","Arte della Fede" e "Arte della Morte". Le specializzazioni regolano quali incanti possono essere utilizzati in base al livello del cerchio. 

\begin{tabularx}{\linewidth}{|m|s|b|}
\hline
\multicolumn{3}{|c|}{\textbf{Specializzazioni}}           \\
\hline
\multirow{2}{*}{\textit{Sapienza}} &1 &  Il personaggio può utilizzare gli incanti di ``Arte della sapienza'' fino al terzo cerchio.  \\
                  & 2&     Il personaggio può utilizzare gli incanti di ``Arte della sapienza'' fino al sesto cerchio. \\\hline
\multirow{2}{*}{\textit{Fede}} &  1  &  Il personaggio può utilizzare gli incanti di ``Arte della fede'' fino al secondo cerchio. \\
                  &  2    &  Il personaggio può utilizzare gli incanti di ``Arte della fede'' fino al quinto cerchio.   \\ \hline
\multirow{2}{*}{\textit{Morte}} &  1  &  Il personaggio può utilizzare gli incanti di ``Arte della morte'' fino al secondo cerchio.   \\
                  &  2    &    Il personaggio può utilizzare gli incanti di ``Arte della morte'' fino al quinto cerchio.   \\ 
\hline
\multicolumn{3}{|c|}{\textbf{Talenti}}           \\
\hline
     \textit{Magie Instabili}  &\multicolumn{2}{k|}{ Gli incanti che infliggono danno del personaggio infliggono inoltre la metà del danno (il valore della magia, calcolato una sola volta) anche a tutti i bersagli vicini. Ciascun bersaglio riduce il danno in base ai propri modificatori.} \\\hline
      \textit{Sovraccarico}&\multicolumn{2}{k|}{ Gli incanti del personaggio hanno un modificatore di -3 alla difficoltà per essere lanciati. In caso di fallimento nel lancio di un incanto, il personaggio perde per il mese successivo 1 punto in Intelligenza. }  \\\hline
     \textit{Durata aumentata}     &\multicolumn{2}{k|}{  Il personaggio può aumentare di 1 la difficoltà necessaria lanciare un incanto non istantaneo per aumentarne la durata di 1 round.}\\\hline
       \textit{Pazienza}    &\multicolumn{2}{k|}{ Il personaggio può attendere un round addizionale tra il momento in cui si prepara a lanciare un incanto e quando questa viene effettivamente lanciato. Se lo fa e riesce a lanciarlo con successo, quell'incanto non può essere \textit{annullato}. Essere colpiti tra il turno di la preparazione dell'incanto e il lancio effettivo causerà il fallimento automatico nel lancio.} \\\hline

\end{tabularx}


 %termine abilità magiche


\clearpage

\subsubsection{Abilità miscellanee}
 %inizio tab abilità miscellanee 



\begin{center}
\textbf{ \large{Acrobazia}}\\ \textit{\textbf{ Reattività}}
\\
\end{center}
  Questa abilità viene usata dal personaggio per tutte le azioni atletiche che richiedono delle buone capacità fisiche.Si tratta quindi di attività come arrampicarsi, marciare, nuotare e correre. Anche cavalcare ricade in questa attività. 

\begin{tabularx}{\linewidth}{|m|s|b|}
\hline
\multicolumn{3}{|c|}{\textbf{Specializzazioni}}           \\
\hline
\multirow{2}{*}{\textit{Marciare e Correre}} &1 &  Il personaggio ha un modificatore di -1 alla difficoltà per ogni TDS quando deve compiere scatti o camminare per lunghe distanze.  \\
                  & 2&        Il personaggio ha un modificatore di -1 alla difficoltà per ogni TDS quando deve compiere scatti o camminare per lunghe distanze.   \\\hline
\multirow{2}{*}{\textit{Nuotare}} &  1  &  Il personaggio ha un modificatore di -1 alla difficoltà per ogni TDS per nuotare.  \\
                  &  2    &     Il personaggio ha un modificatore di -1 alla difficoltà per ogni TDS per nuotare. \\ \hline
\multirow{2}{*}{\textit{Arrampicarsi}} &  1  &   Il personaggio ha un modificatore di -1 alla difficoltà per ogni TDS su per arrampicarsi.  \\
                  &  2    &      Il personaggio ha un modificatore di -1 alla difficoltà per ogni TDS su per arrampicarsi.   \\ \hline
\multirow{2}{*}{\textit{Cavalcare}} &  1  &  Il personaggio ha un modificatore di -1 alla difficoltà per ogni TDS per cavalcare.  \\
                  &  2    &     Il personaggio ha un modificatore di -1 alla difficoltà per ogni TDS per cavalcare. \\ \hline
\multicolumn{3}{|c|}{\textbf{Talenti}}           \\
\hline
      \textit{Istinto}  & \multicolumn{2}{k|}{Se il personaggio subisce un modificatore alla difficoltà nei TDS su "\emph{Acrobazia}" compreso tra 1 e 3 (inclusi) a causa del terreno accidentato, ignora quel modificatore.} \\\hline
        \textit{Allenamento} &\multicolumn{2}{k|}{  Il personaggio ha un ottima capacità fisiche, resistendo più a lungo della media allo sforzo fisico. L'aumento della difficoltà a causa della fatica è sensibilmente ridotto. }  \\\hline
      \textit{Fisico eccezionale}    &\multicolumn{2}{k|}{ Il personaggio ha un modificatore di -1 alla difficoltà per ogni TDS su \emph{Acrobazia}.} \\\hline
      \textit{Cavallerizzo provetto}   &\multicolumn{2}{k|}{Il personaggio ha un ottimo controllo del cavallo, quindi disarcionarlo diventa estremamente difficile. Egli è anche in grado di calmare la sua cavalcatura con maggior facilità.}\\\hline
        \textit{Caduta morbida}    &\multicolumn{2}{k|}{Il personaggio può cadere da un'altezza pari o inferiore a 5 metri senza subire danni. Anche cadendo da altezze superiori subirà sempre meno danni della norma.}\\
\hline
\end{tabularx}


\clearpage


\begin{center}
\textbf{ \large{Convincere}}\\ \textit{\textbf{ Carisma}}
\\
\end{center}
  Questa abilità viene usata dal personaggio per tutte le azioni atletiche che richiedono delle buone capacità fisiche.Si tratta quindi di attività come arrampicarsi, marciare, nuotare e correre. Anche cavalcare ricade in questa attività. 

\begin{tabularx}{\linewidth}{|m|s|b|}
\hline
\multicolumn{3}{|c|}{\textbf{Specializzazioni}}           \\
\hline
\multirow{2}{*}{\textit{Persuadere}} &1 & Il personaggio ha un modificatore di -1 alla difficoltà quando prova a persuadere un qualcuno.\\
                  & 2&   Il personaggio ha un modificatore di -1 alla difficoltà quando prova a persuadere un qualcuno. \\\hline
\multirow{2}{*}{\textit{Incutere timore}} &  1  &  Se utilizzata per spaventare, ottieni un modificatore di -1 alla difficoltà ogni  due punti di ``Costituzione'' sopra al 10.   \\
                  &  2    &    Se utilizzata per spaventare, ottieni invece un modificatore di -1 alla difficoltà per ogni punto di ``Costituzione'' sopra al 10.   \\ \hline
                  \multirow{2}{*}{\textit{Comandante}} &  1  & Il personaggio non si tira indietro quando si tratta di motivare i propri compagni. Utilizzare la propria leadership per motivare i propri compagni ha un modificatore di -1 alla difficoltà.  \\
                  &  2    &     Quando il personaggio utilizza la propria leadership per motivare i propri compagni ha un modificatore di -1 alla difficoltà.\\ 
                  \hline
                  
\multicolumn{3}{|c|}{\textbf{Talenti}}           \\
\hline
      \textit{Mercanteggiare}  &\multicolumn{2}{k|}{Il personaggio ha un modificatore di -1 alla difficoltà quando prova a condurre una trattativa. }\\\hline
         \textit{Seduzione}   &\multicolumn{2}{k|}{ Il personaggio ha un modificatore di -2 alla difficoltà quando utilizza le proprie abilità di seduzione su qualcuno.} \\\hline
         \textit{Terrore}     & \multicolumn{2}{k|}{Questo talento raddoppia il modificatore alla difficoltà dato dalla specializzazione ``Incutere timore''.}\\\hline
      \textit{Dialettica}    &\multicolumn{2}{k|}{Il personaggio ha un modificatore di -1 alla difficoltà quando prova a convincere un'altra persona di un qualcosa.}\\\hline
    \textit{Alle armi}    &\multicolumn{2}{k|}{ Il personaggio infonde la sua foga negli alleati, conferendo a sé stesso  e a loro un modificatore di -3alla difficoltà per i tiri su "\emph{Acrobazia}" per cavalcare o marciare durante una carica.}\\\hline
\end{tabularx}


\clearpage

\begin{center}
\textbf{ \large{Domare animali}}\\ \textit{\textbf{ Sensibilità}}
\\
\end{center}
Questa abilità permette al personaggio di domare un animale o una creatura. Questo potrebbe permettergli di stringere un legame con essa e di averlo come compagno animale. 

\begin{tabularx}{\linewidth}{|m|s|b|}
\hline
\multicolumn{3}{|c|}{\textbf{Specializzazioni}}           \\
\hline
\multirow{2}{*}{\textit{Intuizione}} &1 &    Il personaggio osservando l’animale o creatura può capire la sua dieta (carnivoro/erbivoro/onnivoro).    \\
                  & 2&           Il personaggio osservando l’animale o creatura può intuire cosa stia provando (paura/tranquillità/ecc).   \\\hline
\multirow{2}{*}{\textit{Fiducia}} &  1  &   Il personaggio ha un modificatore di -1 alla difficoltà per domare un animale o creatura già conosciuta.   \\
                  &  2    &         Il personaggio ha un modificatore di -1 alla difficoltà su ogni tiro per domare un animale o creatura. \\ \hline
\multirow{2}{*}{\textit{Simpatia}} &  1  &    Il personaggio ha un modificatore di -1 alla difficoltà su ogni tiro per domare un animale o creatura. \\
                  &  2    &      Il personaggio è in grado di familiarizzare più velocemente con una creatura. Il tempo per domarela diminuisce  \\ 
\hline
\multicolumn{3}{|c|}{\textbf{Talenti}}           \\
\hline
      \textit{Empatia}  &\multicolumn{2}{k|}{  Il personaggio è in grado di domare una creatura molto più velocemente della norma.}\\\hline
        \textit{Calma} &\multicolumn{2}{k|}{ Il personaggio può calmare un \emph{animale} inferocito senza fare alcun tiro. L'animale non sarà però domato.  }  \\\hline
        \textit{Creature risolute} &\multicolumn{2}{k|}{  Le creature domate dal personaggio sono immuni a tutte le magie di condizionamento mentale.  }  \\\hline
        \textit{Domatore esperto} &\multicolumn{2}{k|}{  Il personaggio ha un modificatore di -2 alla difficoltà quando prova a domare una creatura  }  \\\hline
\end{tabularx}



\clearpage



\begin{center}
\textbf{ \large{Erboristeria}}\\ \textit{\textbf{ Sensibilità}}
\\
\end{center}
  Questa abilità permette al personaggio di cercare o coltivare erbe, piante e funghi.

\begin{tabularx}{\linewidth}{|m|s|b|}
\hline
\multicolumn{3}{|c|}{\textbf{Specializzazioni}}           \\
\hline
\multirow{2}{*}{\textit{Effetti curativi}} &1 &   Il personaggio dedica particolare attenzione allo studio e alla conoscenza di erbe, piante o funghi con proprietà curative. La ricerca selettiva di questo tipo di piante ha un modificatore di -1 alla difficoltà.    \\
                  & 2&         Il personaggio conosce a tal punto i tipi di piante aventi effetti curativi che la difficoltà per coltivarle ha un modificatore di -2.   \\\hline
\multirow{2}{*}{\textit{Effetti velenosi}} &  1  &  Il personaggio dedica particolare attenzione allo studio e alla conoscenza di erbe, piante o funghi con proprietà velenose. La ricerca selettiva di questo tipo di piante ha un modificatore di -1 alla difficoltà.   \\
                  &  2    &        Il personaggio conosce a tal punto i tipi di piante aventi effetti velenosi che la difficoltà per coltivarle ha un modificatore di -1. \\ \hline
\multirow{2}{*}{\textit{Piante commestibili}} &  1  &   Il personaggio dedica particolare attenzione allo studio e alla conoscenza di erbe, piante o funghi commestibili. La ricerca selettiva di questo tipo di piante ha un modificatore di -1 alla difficoltà.    \\
                  &  2    &      Il personaggio conosce a tal punto i tipi di piante commestibili che la difficoltà per coltivarle ha un modificatore di -3.   \\ 
\hline
\multicolumn{3}{|c|}{\textbf{Talenti}}           \\
\hline
    \textit{Farmacista}  &\multicolumn{2}{k|}{   Le pozioni curative prodotte dal personaggio sono più efficaci; ogni pozione di questo tipo permette a chi la beve di recuperare un punto vita addizionale. }\\\hline
    \textit{Avvelenatore} & \multicolumn{2}{k|}{ I veleni preparati dal personaggio sono più difficili da individuare. Egli ha inoltre sviluppato una discreta resistenza ai veleni, per questo motivo riduce i danni subiti da ogni veleno non letale di 1. }  \\\hline
         \textit{Crescita rigogliosa}&\multicolumn{2}{k|}{ Il personaggio nel coltivare dedica una cura maniacale ai dettagli, facendo crescere la pianta più rigogliosa (e aumentando quindi le risorse ottenute da essa).} \\\hline
       \textit{Cercare erbe}&\multicolumn{2}{k|}{ Il personaggio ha una vasta conoscenza di molti tipi di erbe e li riconosce con facilità. Il personaggio ha un modificatore alla difficoltà di -3 per Cercare erbe.}\\\hline

\end{tabularx}



\clearpage


\begin{center}
\textbf{ \large{Malaffare}}\\ \textit{\textbf{ Reattività}}
\\
\end{center}

 Questa abilità gestisce tutte le attività disoneste che il personaggio può compiere: per esempio viene usata per scassinare serrature, sottrarre un oggetto senza essere notato o barare al gioco d'azzardo. Il valore di difficoltà dipende dalla situazione dal tipo di serratura da scassinare o dalle dimensioni dell'oggetto che si intende rubare). 

\begin{tabularx}{\linewidth}{|m|s|b|}
\hline
\multicolumn{3}{|c|}{\textbf{Specializzazioni}}           \\
\hline
\multirow{2}{*}{\textit{Scassinare}} &1 &    Il personaggio può provare a riconoscere il tipo di serratura e stimare la difficoltà necessaria per aprirla.    \\
                  & 2&           Il personaggio ha un modificatore di -1 alla difficoltà per provare a scassinare una serratura.   \\\hline
\multirow{2}{*}{\textit{Borseggiare}} &  1  &    Il personaggio può provare a riconoscere il tipo di oggetti contenuto in tasche, fodere, borselli, ecc e stimare la difficoltà necessaria per sottrarli.    \\
                  &  2    &         Il personaggio ha un modificatore di -1 alla difficoltà per provare a rubare un oggetto. \\ \hline
\multirow{2}{*}{\textit{Gioco d'azzardo}} &  1  &   Il personaggio ha un modificatore di -1 alla difficoltà per provare a barare al gioco d'azzardo.    \\
                  &  2    &       Il personaggio ha un modificatore di -1 alla difficoltà per provare a barare al gioco d'azzardo.  \\ 
\hline
\multicolumn{3}{|c|}{\textbf{Talenti}}           \\
\hline
      \textit{Ricettatore} &\multicolumn{2}{k|}{ Il personaggio può provare a capire con maggior accuratezza il valore degli oggetti. Ha inoltre un modificatore di -1 alla difficoltà nei tiri su \emph{Convincere} se prova a vendere un oggetto. }\\\hline
         \textit{Azzardopatia}  &\multicolumn{2}{k|}{ Il personaggio ha un modificatore di -1 alla difficoltà nei tiri su \emph{Convincere} se prova a convincere qualcuno a gicoare d'azzardo con lui. Ha anche un ulteriore modificatore di -1 alla difficoltà per provare a barare al gioco d'azzardo. }\\\hline
       \textit{Mercato nero}    &\multicolumn{2}{k|}{ Il personaggio ha un modificatore di -2 alla difficoltà quando prova a ottenere informazioni relative a eventuali commerci illeciti presenti in città.} \\\hline
         \textit{Rapidità di mano}   &\multicolumn{2}{k|}{  Il personaggio ha un modificatore di -1 alla difficoltà per ogni TDS su Malaffare.} \\
\hline
\end{tabularx}



\clearpage


\begin{center}
\textbf{ \large{Medicina}}\\ \textit{\textbf{ Sensibilità}}
\\
\end{center}
Questa abilità permette al personaggio di curare un ferito. Il tipo di medicazione necessaria determina la difficoltà e le risorse necessarie.  

\begin{tabularx}{\linewidth}{|m|s|b|}
\hline
\multicolumn{3}{|c|}{\textbf{Specializzazioni}}           \\
\hline
\multirow{2}{*}{\textit{Medicina da campo}} &1 &    Il personaggio è abituato all'uso di strumenti non ottimali per la cura delle ferite (purchè siano utilizzabili per lo scopo). Il personaggio non subisce malus se utilizza strumenti di questo tipo.   \\
                  & 2&            Il personaggio impiega meno tempo per curare una ferita.   \\\hline
\multirow{2}{*}{\textit{Cure approfondite}} &  1  &   Il personaggio ha ottime capacità di diagnosi, ottiene quindi un modificatore di -2 alla difficoltà per ogni TDS su "Medicina" mirato al riconoscimento di ferite, avvelenamenti o malattie di un soggetto umanoide.   \\
                  &  2    &         Il personaggio ha un modificatore di -1 alla difficoltà nei TDS su "Medicina" per curare e stabilizzare le ferite di un soggetto umanoide.\\ \hline
\multirow{2}{*}{\textit{Veterinaria}} &  1  &    Il personaggio conosce l'anatomia degli animali  e creature più comuni: ottiene quindi un modificatore di -2 alla difficoltà per ogni TDS su "Medicina" mirato al riconoscimento di ferite, avvelenamenti o malattie di un animale o creatura non umanoide.    \\
                  &  2    &        Il personaggio ha un modificatore di -1 alla difficoltà nei TDS su "Medicina" per curare le ferite di un animale o creatura non umanoide.   \\ 
\hline
\multicolumn{3}{|c|}{\textbf{Talenti}}           \\
\hline
    \textit{Antropologia forense} &\multicolumn{2}{k|}{ Studiando un corpo morto, per il personaggio è possibile provare a capire le cause del decesso più probabili. }\\\hline
     \textit{Primo soccorso} &\multicolumn{2}{k|}{  Il personaggio ha le competenze necessarie per non far morire una persona ferita gravemente. Potendo disporre di sufficiente tempo e concentrazione, è in grado di stabilizzare un soggetto senza eseguire alcun TDS. Questa operazione non fa recuperare punti vita, ma permette al paziente di sopravvivere.  }  \\\hline
        \textit{Diagnosi} &\multicolumn{2}{k|}{  Il personaggio esaminando una persona infetta o avvelenata, può riconoscere il tipo di infezione o veleno che l'ha colpita.  }  \\\hline
        \textit{Cura rapida} &\multicolumn{2}{k|}{  Il personaggio ha la possibilità di spendere il suo turno (anche durante un combattimento) per curare un alleato con cui è a contatto senza eseguire un TDS. Se lo fa, l'alleato recupera 2 punti vita. }  \\\hline
\end{tabularx}


\clearpage


\begin{center}
\textbf{ \large{Navigazione}}\\ \textit{\textbf{ Intelligenza}}
\\
\end{center}
 Questa abilità indica quanto il personaggio sia in grado di governare una barca di medie e piccole dimensioni. Azioni come: seguire una rotta, spiegare all'equipaggio come svolgere semplici azioni (come \textit{ ammainare le vele}) ricadono in questa abilità. 

\begin{tabularx}{\linewidth}{|m|s|b|}
\hline
\multicolumn{3}{|c|}{\textbf{Specializzazioni}}           \\
\hline
\multirow{2}{*}{\textit{Governare la Nave}} &1 &   Il personaggio ha un modificatore di -1 alla difficoltà per governare la nave in modo da mantenere correttamente la rotta.   \\
                  & 2&          Il personaggio ha un modificatore di -2 alla difficoltà per governare la nave in modo da mantenere correttamente la rotta. \\\hline
\multirow{2}{*}{\textit{Scegliere la Rotta}} &  1  & Il personaggio ha un modificatore di -1 alla difficoltà per scegliere la rotta corretta per evitare i pericoli.    \\
                  &  2    &      Il personaggio ha un modificatore di -1 alla difficoltà per scegliere la rotta corretta per evitare i pericoli.  \\ 
\hline
\multirow{2}{*}{\textit{Gestire la ciurma}} &  1  &Coloro che ricevono le indicazioni dal personaggio, hanno un modificatore di -1 alla difficoltà per i TDS su "Navigare".   \\
                  &  2    &    Coloro che ricevono le indicazioni dal personaggio, hanno un modificatore di -1 alla difficoltà per i TDS su "Navigare".  \\ 
\hline
\multicolumn{3}{|c|}{\textbf{Talenti}}           \\
\hline
      \textit{Avanti tutta} &\multicolumn{2}{k|}{  Le rotte identificate dal personaggio saranno le più rapide possibile. Egli è inoltre in grado di spingere l'imbarcazione al suo massimo ottenendo una velocità di crociera fuori dal comune.} \\\hline
     \textit{Razionamento}  &\multicolumn{2}{k|}{  Il personaggio è esperto della vita a bordo, sa come gestire al meglio le provviste di cibo e acqua. Inoltre la raccolta di cibo a bordo della nave (pescare o raccogliere alghe) è facilitata.} \\\hline
     \textit{Navigare nella tempesta}  &\multicolumn{2}{k|}{  Il personaggio sa riconoscere il clima e i suoi cambiamenti con facilità. Anche se sorpreso da una tempesta improvvisa sa come gestire l'imbarcazione e minimizza i danni che possono essere subiti.}\\\hline
      \textit{Conoscenza strutturale}  &\multicolumn{2}{k|}{   Il personaggio conosce estremamente bene l'imbarcazione. Ogni tentativo di ripararla, se fatto sotto la sua supervisione, ha un modificatore di -3 alla difficoltà.}\\\hline
\end{tabularx}


\clearpage

\begin{center}
\textbf{ \large{Occultarsi}}\\ \textit{\textbf{ Reattività}}
\\
\end{center}
  Questa abilità permette al personaggio di nascondersi o muoversi silenziosamente rendendone molto più difficile l'individuazione. La difficoltà dipende dalla situazione. 

\begin{tabularx}{\linewidth}{|m|s|b|}
\hline
\multicolumn{3}{|c|}{\textbf{Specializzazioni}}           \\
\hline
\multirow{2}{*}{\textit{Muoversi furtivamente}} &1 &    Il personaggio è estremamente silenzioso quando si muove; provare ad ascoltarlo mentre si muove furtivamente ha un modificatore di +1 alla difficoltà.    \\
                  & 2&  Quando il personaggio si muove furtivamente non produce rumori addizionali a causa dell'armatura. \\\hline
\multirow{2}{*}{\textit{Nascondersi}} &  1  & Il personaggio ha un modificatore di -1 alla difficoltà quando prova a nascondersi. \\
                  &  2    &   Il personaggio può coprire il proprio odore sfruttando elementi a lui vicini. Provare a individuarlo ha un modificatore di +1 alla difficoltà.   \\ \hline
\multirow{2}{*}{\textit{Incognito}} &  1  &  Se il personaggio utilizza questa abilità per fingersi un'altra persona, ha un modificatore di -1 alla difficoltà per i TDS basati su ``\emph{Convincere}'' gli altri sulla sua identità. \\
                  &  2    &  Il personaggio sfrutta al meglio le sue capacità oratorie e di prestidigitazione qundo qualcuno prova a perquisirlo. Trovare un oggetto o arma piccola nascosti tra i vestiti avrà un modificatore di +2 alla difficoltà. \\ 
\hline
\multicolumn{3}{|c|}{\textbf{Talenti}}           \\
\hline
    \textit{Mimetismo}&\multicolumn{2}{k|}{Il personaggio ha un modificatore di -2 alla difficoltà quando prova a nascondersi in un ambiente naturale.} \\\hline
       \textit{Senza tracce} &\multicolumn{2}{k|}{Quando si muove furtivamente, il personaggio non lascia tracce di alcun tipo. }\\\hline
         \textit{Adattamento}   &\multicolumn{2}{k|}{Il personaggio impara in modo estremamente veloce come adattare il proprio modo di comportarsi e parlare per immergersi perfettamente nell'ambiente in cui si trova. Quando è in incognito non attira l'attenzione e gli altri non sospettano della sua identità.} \\\hline
\end{tabularx}


\clearpage

\begin{center}
\textbf{ \large{Percezione}}\\ \textit{\textbf{ Sensibilità}}
\\
\end{center}
   Questa azione permette al personaggio di cogliere dettagli, comprendere al meglio le situazioni che lo circondano utilizzando i propri sensi. 

\begin{tabularx}{\linewidth}{|m|s|b|}
\hline
\multicolumn{3}{|c|}{\textbf{Specializzazioni}}           \\
\hline
\multirow{2}{*}{\textit{Fisionomista}} &1 &   Il personaggio ha un modificatore di -1 alla difficoltà se utilizza \emph{Percezione} per cercare una persona in una folla.    \\
                  & 2&             Il personaggio ha un modificatore di -2 alla difficoltà se utilizza \emph{Percezione} per cercare una persona in una folla. \\\hline
\multirow{2}{*}{\textit{Cinesica}} &  1  &  Il personaggio ha un modificatore di -1 alla difficoltà se utilizza \emph{Percezione} per provare a capire le intenzioni di una persona.   \\
                  &  2    &        Il personaggio ha un modificatore di -1 alla difficoltà se utilizza \emph{Percezione} per provare a capire le intenzioni di una persona.  \\ \hline
\multirow{2}{*}{\textit{Attenzione ai particolari}} &  1  & Il personaggio ha un modificatore di -1 alla difficoltà se utilizza \emph{Percezione} per cercare piccoli dettagli in un ambiente.    \\
                  &  2    &     Il personaggio ha un modificatore di -1 alla difficoltà se utilizza \emph{Percezione} per cercare piccoli dettagli in un ambiente.  \\ 
\hline
\multicolumn{3}{|c|}{\textbf{Talenti}}           \\
\hline
    \textit{Fiuto superiore alla norma} &\multicolumn{2}{k|}{ Il personaggio ha un modificatore di -1 alla difficoltà se utilizza \emph{Percezione} per cercare delle tracce.} \\\hline
        \textit{Vista superiore alla norma}   & \multicolumn{2}{k|}{ Il personaggio ha un modificatore di -1 alla difficoltà se utilizza \emph{Percezione} per osservare. } \\\hline
        \textit{Udito superiore alla norma}    &\multicolumn{2}{k|}{ Il personaggio ha un modificatore di -1 alla difficoltà se utilizza \emph{Percezione} per ascoltare.} \\\hline
\end{tabularx}





\clearpage

\begin{center}
\textbf{ \large{Schivare}}\\ \textit{\textbf{ Reattività}}
\\
\end{center}

  Questa abilità permette al personaggio di provare a schivare tutto ciò che non non lo bersaglia direttamente. Viene quindi utilizzata per definire se il personaggio venga colpito da una trappola, da un oggetto in caduta o da una freccia scagliata contro il suo gruppo (ma non indirizzata direttamente a lui). In caso di riuscita, l'effetto dato dall'aver schivato la minaccia sarà stabilito dal master in base alla situazione.  

\begin{tabularx}{\linewidth}{|m|s|b|}
\hline
\multicolumn{3}{|c|}{\textbf{Specializzazioni}}           \\
\hline
\multirow{2}{*}{\textit{Prontezza}} &1 &    Il personaggio ha un modificatore di -1 alla difficoltà per ogni TDS su Schivare.    \\
                  & 2&           Il personaggio ha un modificatore di -1 alla difficoltà per ogni TDS su Schivare.   \\\hline
\multirow{2}{*}{\textit{Per un pelo}} &  1  &   Il personaggio ha un modificatore di -2 alla difficoltà per ogni TDS su Schivare quando prova a evitare le conseguenze date dall'aver fatto scattare una trappola.   \\
                  &  2    &         Il personaggio ha un modificatore di -1 ai danni subiti dalle trappole. \\ \hline

\multicolumn{3}{|c|}{\textbf{Talenti}}           \\
\hline
      \textit{Evasivo}  &\multicolumn{2}{k|}{  Il personaggio ottiene un modificatore di -2 alla difficoltà per ogni TDS su Schivare.}\\\hline
       \textit{Estremamente fortunato}  &\multicolumn{2}{k|}{Se il personaggio ha fallito un TDS su Schivare, tira un D6. Con un risultato pari a 6 il tiro verrà considerato come superato.} \\\hline
         \textit{Leggerezza}    & \multicolumn{2}{k|}{Se il personaggio non indossa pezzi di armatura media o pesante, ha un modificatore di -2 ai danni subiti dalle trappole. }\\\hline
        \textit{Limitare i danni}   & \multicolumn{2}{k|}{Se il personaggio fallisce un TDS su Schivare, può scegliere su quale parte del corpo subire gli effetti.}\\
\hline
\end{tabularx}

\clearpage

\begin{center}
\textbf{ \large{Seguire tracce}}\\ \textit{\textbf{ Intelligenza}}
\\
\end{center}

  Questa abilità permette al personaggio, una volta individuate delle tracce, di seguirle. Alternativamente, può essere usata per provare a capire quale sia l'origine delle tracce stesse. 

\begin{tabularx}{\linewidth}{|m|s|b|}
\hline
\multicolumn{3}{|c|}{\textbf{Specializzazioni}}           \\
\hline
\multirow{2}{*}{\textit{Tracce Animali}} &1 &    Il personaggio ha un modificatore di -2 alla difficoltà nel riconoscere tracce di animali a lui noti.    \\
                  & 2&           Rinvenute le tracce, il personaggio ha un modificatore di -1 alla difficoltà per capire quanto le tracce siano recenti, identificare con maggior precisione il tipo di creatura (se la conosce).   \\\hline
\multirow{2}{*}{\textit{Tracce Umane}} &  1  &   Il personaggio ha un modificatore di -1 alla difficoltà nel riconoscere tracce appartenenti a un soggetto umanoide.    \\
                  &  2    &         Rinvenute le tracce, il personaggio ha un modificatore di -1 alla difficoltà per capire quanto le tracce siano recenti, identificare con una buona precisione il peso di chi le ha lasciate e capire (se possibile) come sia equipaggiato. \\ \hline
\multirow{2}{*}{\textit{Segugio}} &  1  &   Il personaggio ha un modificatore di -1 alla difficoltà nel seguire le tracce. \\
                  &  2    &    Il personaggio ha un modificatore di -1 alla difficoltà nel seguire le tracce. \\ 
\hline
\multicolumn{3}{|c|}{\textbf{Talenti}}           \\
\hline
      \textit{Scaltrezza}  &\multicolumn{2}{k|}{ Il personaggio conosce i trucchi per depistare gli inseguitori, per lui sarà molto più difficile essere ingannato.} \\\hline
        \textit{Tracce Persistenti} &\multicolumn{2}{k|}{ Il personaggio è in grado seguire le tracce del suo bersaglio anche in condizioni avverse. Se subisce un modificatore alla difficoltà nei TDS su "Seguire tracce" compreso tra 1 e 4 (inclusi) a causa di condizioni sfavorevoli, ignora quel modificatore. }  \\\hline
         \textit{Indagine}   & \multicolumn{2}{k|}{ Se il personaggio vedendo un'insieme di tracce prova a ricostruire gli eventi che le hanno causate, ha un modificatore di -2 alla difficoltà.}\\\hline
        \textit{Identificazione}     & \multicolumn{2}{k|}{ Se il personaggio prova a riconoscere le caratteristiche fisiche e il tipo di creatura/personaggio che ha lasciato delle tracce ha un modificatore di -3 alla difficoltà.}\\
\hline
\end{tabularx}

\clearpage

\begin{center}
\textbf{ \large{Utilizzare trappole}}\\ \textit{\textbf{ Reattività}}
\\
\end{center}
Questa abilità è utilizzata per tutte le azioni legate al posizionamento e sabotaggio di trappole. Permette infatti di costruire, posizionare, occultare, individuare e disinnescare una trappola. Le regole specifiche sono segnate nell'apposita sezione del manuale.

\begin{tabularx}{\linewidth}{|m|s|b|}
\hline
\multicolumn{3}{|c|}{\textbf{Specializzazioni}}           \\
\hline
\multirow{2}{*}{\textit{Costruire trappole}} &1 &    Il personaggio ha un modificatore di -1 alla difficoltà nel costruire e posizionare una trappola.    \\
                  & 2&          Il personaggio ha un modificatore di -1 alla difficoltà nel costruire e posizionare una trappola.    \\\hline
\multirow{2}{*}{\textit{Occultare trappole}} &  1  &  Il personaggio ha un modificatore di -1 alla difficoltà nel nascondere una trappola alla vista.   \\
                  &  2    &      Il personaggio ha un modificatore di -1 alla difficoltà nel nascondere una trappola alla vista. \\ \hline
\multirow{2}{*}{\textit{Individuare trappole}} &  1  &   Il personaggio ha un modificatore di -1 alla difficoltà nell'individuare trappole in un ambiente. \\
                  &  2    &     Il personaggio ha un modificatore di -1 alla difficoltà nell'individuare trappole in un ambiente. \\ \hline
\multirow{2}{*}{\textit{Disattivare trappole}} &  1  &   Il personaggio ha un modificatore di -1 alla difficoltà per disattivare una trappola. \\
                  &  2    &    Il personaggio ha un modificatore di -1 alla difficoltà per disattivare una trappola. \\ \hline
\multicolumn{3}{|c|}{\textbf{Talenti}}           \\
\hline
      \textit{Meccanismi complicati}  &\multicolumn{2}{k|}{Le trappole costruite dal personaggio possono essere disattivate con difficoltà pari a: Livello trappola + 2. } \\\hline
        \textit{Improvvisare} &\multicolumn{2}{k|}{Il personaggio trova più facilmente materiali adatti a nascondere una trappola. }  \\\hline
         \textit{Delicatezza}   & \multicolumn{2}{k|}{Fallire un tiro per provare a disattivare una trappola, non causerà mai l’innesco della trappola stessa. }\\\hline
        \textit{Artificiere}     & \multicolumn{2}{k|}{ Il personaggio è esperto nel trovare e disattivare trappole. Ha un modificatore di -1 alla difficoltà per individuarle e disattivarle.}\\
\hline
\end{tabularx}

 %termine della lista della miscellanea

\end{document}