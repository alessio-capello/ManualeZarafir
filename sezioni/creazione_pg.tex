\documentclass[../manuale_main.tex]{subfiles}



\begin{document}
Prima di vedere come procedere per la creazione di un nuovo personaggio è importate ricordare quella che è considerata la ``Regola 0'' di tutti i giochi di ruolo: il master ha la possibilità di alterare le regole del manuale per un qualsiasi motivo (si sconsiglia comunque l'abuso di questa \textit{capacità}), in altre parole: il master ha sempre ragione.

Questa regola è importantissima dato che il seguente manuale non prevede un sistema di ``Classi'' standard ma un sistema basato sull'intreccio di ``Caratteristiche'' ed ``Abilità''. 
Il master avrà quindi il compito di valutare ogni richiesta relativa a  un nuovo personaggio e decidere se le abilità e le caratteristiche scelte dai giocatori siano congruenti con la storia, con l'ambientazione e se riflettano il tipo di personaggio che vorrebbero interpretare. Eventuali modifiche dovranno essere stabilite assieme al giocatore in questione.
\subsection{La scheda del personaggio}
Creare un nuovo personaggio é piuttosto semplice, basta fotocopiare la Scheda Personaggio presente in questo manuale (ultima pagina), in alternativa si possono scrivere su un foglio di carta i seguenti campi (più avanti vedremo anche come riempirli):\\
\begin{center}
\renewcommand{\arraystretch}{1.2}
\begin{tabular}{ l l }
Nome & Cognome \textit{(se l'ambientazione lo prevede)} \\
Età  & Razza \\
Allineamento & Sesso \\
Peso & Altezza \\

Punti Vita &\\
Punti Mana &\\
Punti Esperienza &\\

Forza (Fo) &\\
Agilità (Ag) &\\
Costituzione (Co) &\\
Intelligenza (In) &\\
Sensibilità (Se) &\\
Reattività (Re) &\\
Carisma (Ca) &\\

Abilità, Specializzazioni e Talenti &\\
\end{tabular}
\end{center}

L'inventario e l'equipaggiamento devono essere segnati sulla scheda, ma verranno trattati nei capitoli successivi.\\
Si consiglia di dedicare una sezione della scheda a segnare tutti i modificatori del personaggio, per evitare di ricalcolarli ogni volta.


\clearpage

\subsubsection{Caratteristiche}

\paragraph{Significato delle caratteristiche}\mbox{}\\
Le caratteristiche sono il principale elemento caratterizzante del vostro personaggio. Esse sono:

\begin{center}
\renewcommand{\arraystretch}{1.2}
\begin{tabularx}{\linewidth}{ l X }
Forza (Fo) &Rappresenta non solo la forza fisica del personaggio, ma anche la capacità di controllarla e di dosarla. Si usa ad esempio per attaccare in mischia. Inoltre condiziona la capacità del personaggio di trasportare oggetti pesanti.\\
&\\
Agilità (Ag)&Rappresenta la capacità del personaggio di compiere movimenti rapidi e coordinati. Si usa principalmente per attaccare con armi a distanza.\\
&\\
Costituzione (Co)&Rappresenta la conformazione fisica del personaggio. Condiziona la capacità di resistere a sforzi fisici prolungati, di sopportare le ferite e le malattie.\\
&\\
Intelligenza (In)&Rappresenta la memoria, la capacità di capire e di fare ragionamenti complessi, di apprendere e di insegnare nozioni. Condiziona la capacità del personaggio di utilizzare la magia.\\
&\\
Sensibilità (Se)&Rappresenta la saggezza, il buon senso, la capacità di percepire e capire i significati non immediati delle cose, di entrare in profonda empatia.\\
&\\
Reattività (Re)&Rappresenta la capacità di pensare ed agire rapidamente. Condiziona azioni come il cavalcare, rubare o utilizzare trappole.\\
&\\
Carisma (Ca)& Rappresenta come gli altri vedono il personaggio. Condiziona le capacità diplomatiche del personaggio.\\
\end{tabularx}
\end{center}

\clearpage
\paragraph{Assegnazione delle caratteristiche}\mbox{}\\
Le caratteristiche servono per i tiri di simulazione (chiamati TDS) che regola ogni meccanica di gioco.

Alla creazione del personaggio, il giocatore assegna ad ogni caratteristica un valore tra quelli disponibili, che sono i seguenti: 

\textbf{[6] [7] [8] [9] [10] [11] [12]}

Piú avanti vedremo che questi valori sono la base al quale aggiungere (o sottrarre) i modificatori dati dalla razza e dalla competenza.



Vediamo adesso qualche esempio di caratteristiche con i valori appena assegnati (e dunque senza considerare i malus ed i bonus che vedremo piú avanti):

Ad esempio, un ``guerriero'' probabilmente sceglierà queste caratteristiche (o simili):\\\mbox{}\\
\renewcommand{\arraystretch}{1.2}
\begin{tabular}{| l | l |}
\hline
Forza (Fo)&12\\
Agilità (Ag)&6\\
Costituzione (Co)&11\\
Intelligenza (In)&7\\
Sensibilità (Se)&8\\
Reattività (Re)&10\\
Carisma (Ca)&9\\
\hline
\end{tabular}

Un ``mago'', invece, potrebbe preferire assegnare i valori in questo modo:\\\mbox{}\\
\begin{tabular}{| l |l |}
\hline
Forza (Fo)&6\\
Agilità (Ag)&7\\
Costituzione (Co)&8\\
Intelligenza (In)&12\\
Sensibilità (Se)&11\\
Reattività (Re)&10\\
Carisma (Ca)&9\\
\hline
\end{tabular}




\subsubsection{Razze}
Il sistema di razze che vi proponiamo è completamente aperto in modo che non sia vincolato all'ambientazione.
In base all’ambientazione che si utilizza si può identificare una razza in uno dei seguenti insiemi. Considerate quindi i modificatori del gruppo del quale la razza scelta fa parte.

\begin{tabular}{l l l}

Razze intelligenti&Fo: -1&In: +1\\

Razze acquatiche&Co: -1&Re: +1\\

 Razze alte ed agili&Ag: +1&Co: -1\\

Razze forti e goffe&Fo: +1&Re: -1\\

Razze piccole&Fo: -1&Ag: +1\\

Razze umane&Ag: -1&Ca: +1\\

Razze stupide e forti&Fo: +1&In: -1\\

\end{tabular}

\clearpage

\subsubsection{Competenza}

Il valore base delle caratteristiche viene alterato dalla razza e dalla competenza: la prima indica una differente distribuzione delle statistiche data dalle differenze fisiche tra le varie razze che popolano la vostra ambientazione, la seconda indica un'inclinazione particolare del personaggio, che lo distingue dagli altri.

Il personaggio può avere una particolare dote in una caratteristica ed essere meno incline in un'altra. 
Ciò comporta la possibilità di assegnare un modificatore di +1 ad una caratteristica e conseguentemente -1 ad un’altra. 
È possibile decidere di non dare nessuna competenza al vostro personaggio.

\textbf{La competenza non può portare il valore di una caratteristica sopra il 13 (considerando anche i modificatori dati dalla razza) e sotto il 5}. \\

Immaginiamo che il guerriero preso come esempio precedentemente sia un umano (-1 Ag +1 Ca) molto incline alla forza bruta (+1 a Fo) ma meno carismatico (-1 a Ca). 
Le sue caratteristiche quindi diventano:

\begin{tabular}{| l | l |}
\hline
Forza (Fo)&\textbf{13}\\
Agilità (Ag)&\textbf{5}\\
Costituzione (Co)&11\\
Intelligenza (In)&7\\
Sensibilità (Se)&8\\
Reattività (Re)&10\\
Carisma (Ca)&\textbf{8}\\
\hline
\end{tabular}

\subsubsection{Allineamenti}

Altra parte importante della creazione personaggio è la scelta di un allineamento, gli allineamenti sono un elemento principalmente narrativo che serve per rendere più ``reale” il vostro personaggio.

Ogni allineamento è caratterizzato da dei principi basilari, che lasciano ampio spazio allo sviluppo di una personalità individuale. Si tratta quindi di un elemento che servirà per definire il comportamento di un personaggio davanti ai dilemmi morali nei quali si imbatterà nel corso della sua carriera. 

\renewcommand{\arraystretch}{1.5}
\begin{tabularx}{\linewidth}{ l X }
Legale&Gli avventurieri di questo allineamento credono fermamente nella legge e nell'ordine, nella verità e nella giustizia. Lottano per far trionfare il bene sulla malvagità e sulla menzogna. Un paladino che mette in gioco la propria vita per difendere i suoi compagni senza chiedere loro niente in cambio è un esempio perfetto di avventuriero legale.\\

Neutrale&I personaggi Neutrali (la variante più comune) vivono, più o meno inconsapevolmente, nel sottile equilibrio tra il bene e il male. Per questo motivo, a seconda dei casi, compiono sia gesti eroici che riprovevoli. Vivono, per essere più chiari, secondo la popolare massima “il fine giustifica i mezzi”. Un buon esempio di un personaggio neutrale è un ladro che non si fa scrupoli per rubare un oggetto al mercante, ma che è disposto a mettere a repentaglio la propria vita per salvare un amico da una trappola mortale.\\

Caotico&Gli avventurieri caotici son diametralmente opposti a quelli legali. Sono profondamente malvagi e spesso antepongono, egoisticamente, i loro interessi e il loro benessere a tutto e a tutti. Non si fermano davanti a niente per portare a termine i loro piani. Non ci si può fidare di loro. Credono fermamente che sia giusto avvantaggiarsi di ogni opportunità che si presenti e molti di loro sono convinti che gli eventi si susseguano in modo caotico, che non esista un ordine prestabilito per essi. Un mago necromante ossessionato dal desiderio e dal potere è un perfetto esempio di questo allineamento.\\
\end{tabularx}

\subsubsection{Abilità e specializzazioni iniziali}
Alla creazione del personaggio, il giocatore più assegnare liberamente 15 punti abilità, con l'unico vincolo di non mettere più di 3 punti per ogni abilità.\\
La lista delle abilità e il loro funzionamento è descritto nel dettaglio nella prossima sezione di questo manuale.


\end{document}