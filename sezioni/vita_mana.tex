\documentclass[../manuale_main.tex]{subfiles}



\begin{document}


Nella scheda personaggio sono presenti dei valori che non abbiamo ancora trattato approfonditamente, che danno una chiara (seppur sommaria) rappresentazione delle condizioni in cui si trova un personaggio.

\subsection{Punti vita}
Probabilmente sono il valore al quale presterete più attenzione poiché è il discriminante tra la vita e la morte del personaggio stesso.
Il numero massimo di punti vita di un personaggio è pari a: ``\emph{Costituzione}'' + livello abilità ``\emph{Salute}'' + modificatori dati da talenti, specializzazioni ed oggetti.
Per ciò che riguarda il valore attuale, dipende. Se siete stati feriti da un’arma, il valore sarà verosimilmente più basso del massimale; nella sezione ``Combattimento'' il calcolo dei danni è trattato nel dettaglio. Anche per le magie il calcolo dei danni è equivalente, le uniche differenze nascono dal fatto che si può provare a resistere ai danni di una magia e in alcuni le armature si comportano in modo leggermente differente; nella sezione relativa comunque è tutto spiegato, non preoccupatevi.
Il nostro obiettivo è di definire le regole principali relative ai punti vita, alle ferite e alla morte.
\subsubsection{Perdita di punti vita}
Finché i punti vita rimangono sopra allo zero va quasi tutto bene. Perché quasi? E abbastanza intuitivo: le ferite.
Saranno proprio queste la preoccupazione che dovreste avere quando affrontate in duello qualcuno: i duelli all’ultimo sangue sono piuttosto rari, essi infatti violano ogni buonsenso e principio di conservazione. Molto più probabile è che i duelli siano al primo sangue, ma questa riflessione si applica più spesso di quanto crediate. 
\begin{framed}
\textbf{\textit{Nota per i master}}\\
Vi invito a non trasformare gli incontri casuali che i vostri giocatori potrebbero avere con banditi o animali (anche loro apprezzano la vita) in scontri all’ultimo sangue dove il vincitore è l’ultimo a essere rimasto in piedi. Salvo rari casi, non ha senso che questa cosa avvenga: nessun brigante, dopo aver capito che l’avversario è troppo forte per lui, continuerebbe ad affrontarlo. In particolar modo dopo averne avuto una prova tangibile: ferite più o meno gravi. 
\end{framed}
Terminata questo breve (ma secondo me doverosa) introduzione, possiamo parlare di come funzionano le \textbf{ferite}:
Se un attacco qualsiasi, infligge al personaggio un ammontare di danni pari o superiore alla metà dei suoi punti vita, questo rimarrà ferito. La gravità della ferita dipende dal tipo di attacco e da quanto il valore ha superato la soglia. La ferita in questione può anche essere qualcosa di superiore a un taglio, come la rottura di un arto o la sua mutilazione.\\

Quando i punti vita raggiungono lo 0, il personaggio sviene (talenti o altre eccezioni possono invalidare questa regola). 
Il personaggio ha ferite diffuse e non stabilizzate, se non verrà curato entro breve tempo, la sua morte sarà certa.
Un personaggio svenuto a causa della perdita di punti vita potrà essere ucciso senza problemi, sferrandogli colpi mortali senza difficoltà addizionale e senza considerare i danni (non è rilevante la quantità di attacchi poiché il bersaglio non ha modo di ribellarsi).\\


La \textbf{morte} di un personaggio non avviene quando i suoi punti vita precipitano a 0, ma quando i suoi punti vita raggiungono il valore negativo pari al valore dell’abilità ``\emph{Salute}''. Se il personaggio subisce un ammontare di danni tale da mandarlo direttamente sotto a quel valore, la morte sarà istantanea (quindi senza svenimento e possibilità si guarigioni successive).
Esempio: Un personaggio che abbia l’abilità ``\emph{Salute}'' al 7, al raggiungimento di -7 punti vita morirà.

\subsubsection{Recuperare punti vita}
I punti vita possono essere recuperati in molti modi differenti: il modo più comune è tramite pozioni, unguenti o erbe curative, ma è possibile anche sfruttare magie o altri modi a discrezione del master.


\subsection{Punti mana}

I punti mana sono la “risorsa” che viene utilizzata per provare a lanciare una magia.\\
Il valore dei punti mana (massimi) di un giocatore è pari a: “Sensibilità” + modificatori dati da talenti, specializzazioni ed oggetti.\\

Una magia può portare il valore dei punti mana a 0, ma non sotto. Se un personaggio si trovasse ad avere 0 punti mana disponibili, sverrebbe. Le regole dello svenimento sono simili a quelle relative allo svenimento a causa della perdita di punti vita, ma in questo caso non sono presenti ferite che devono essere stabilizzate, quindi il personaggio non è in pericolo di vita.\\
I punti mana possono essere recuperati meditando, utilizzando particolari oggetti o grazie a dei talenti.\\
Poiché un personaggio svenuto non è in grado di meditare, starà al master decidere quando egli sarà in grado di riprendersi (tipicamente facendo passare del tempo, ma potrebbe essere richiesto un intervento di altro tipo, come la somministrazione di pozioni o erbe).


\subsection{Punti esperienza}
I punti esperienza sono la "risorsa" utilizzata per migliorare le abilità del proprio personaggio. Vengono distribuiti dal master al termine di ogni sessione in una misura che varia tra i 5 e 10. Un master potrebbe anche scegliere di assegnare i punti solamente al termine delle varie missioni(per calibrare meglio le sfide durante il loro svolgimento).


\end{document}