\documentclass[../manuale_main.tex]{subfiles}



\begin{document}



“Nightfall” è un Gioco Di Ruolo (GDR) che permette di creare e vivere una storia mediante l'interpretazione di un personaggio in un’ambientazione fantasy di qualsiasi tipo. Si tratta di un GDR cartaceo in cui i personaggi si sviluppano in una storia interamente descritta a voce dal master e dai giocatori. I giocatori sono coloro che si trovano ad agire all'interno di una narrazione primaria tramite azioni descritte verbalmente da loro stessi. Il Master è ideatore della trama principale dell'avventura a cui spetta l'approvazione delle scelte e azioni che vorrebbero compiere i giocatori.
In questo manuale le scelte dei personaggi e l'utilizzo di molte loro abilità è lasciato nella maggior parte dei casi a discrezione della fantasia del giocatore o del Master. Il fallimento o il successo di quest'ultime viene quindi deciso dal master o in base al tiro dei dadi.
“Nightfall”  non è solo la creazione di un GDR maggiormente incentrato sulla narrazione e interpretazione della storia, è la realizzazione del nostro sogno di scrivere un manuale che permettesse di realizzare il gioco di ruolo che abbiamo sempre desiderato: un  GDR semplice e fluido, dove tutto il mondo ruota intorno alla pura interpretazione dei giocatori e la narrazione è in mano alla fantasia del Master e dei giocatori, in cui la verosimiglianza si fonde con la magia del fantasy, dove non esistono livelli o classi predefinite ed ogni giocatore può creare il personaggio che desidera e caratterizzarlo come lo ha sempre immaginato. Un mondo in cui i personaggi non sono obbligatoriamente supereroi ma eroi che possono anche morire.
Il nostro obiettivo all'interno di questo manuale è di fornire un insieme di regole per gestire la maggior parte delle situazioni. Abbiamo deciso di non scendere nel dettaglio di ogni situazione per evitare di aumentare significativamente la quantità di regole, e correre il rischio di fermare più volte le sessioni  per la necessità di  consultare il manuale; rischiando inoltre di vincolare le decisioni non più al Master e giocatori, ma al manuale stesso.

Esempio di utilizzo di abilità:
supponiamo che un giocatore voglia far utilizzare al suo personaggio l'abilità  “Convincere” per rinvigorire i compagni, spaventati dall'enorme mostro che stanno affrontando. 
Il giocatore è incentivato a descrivere il modo in cui ciò avviene: può farlo suonando uno strumento, pronunciando parole epiche o ogni altra idea coerente con il personaggio che gli venga in mente in quella situazione. Sarà il Master che, qualora quell'azione possa compromettere la storia o eventi collegati, potrà riservarsi la possibilità di dire “Non puoi farlo" o legare la riuscita o meno dell'azione al tiro dei dadi. 
Questo breve esempio non è sufficiente a far comprendere in modo esaustivo il manuale, ma è una semplicissima introduzione al suo funzionamento e approccio.


\end{document}