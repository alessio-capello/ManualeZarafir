\documentclass[../manuale_main.tex]{subfiles}



\begin{document}

Questo manuale contiene una piccola lista di creature che potrete utilizzare durante le vostre avventure. Non è e non potrebbe essere una lista esaustiva: non conoscendo l'ambientazione che andrete ad utilizzare, sarebbe impossibile definire i tipi e le capacità di ogni creatura del vostro mondo. Abbiamo deciso di integrare questa sezione principalmente per fornire un riferimento al master nella costruzione delle creature in modo che siano coerenti con la \emph{sua} ambientazione.\\
Le creature che verranno mostrate appartengono a un'ambientazione fantasy classica.
Nella descrizione delle statistiche o abilità non sono presenti talenti o specializzazioni, per mantenere i parametri il più generico possibile. Sarà compito del master deciderli, per fornire loro una caratterizzazione migliore e rendere le sfide sempre diverse.

\subsection{Animali}
Fanno parte di questa categoria tutte le creature appartenenti all'ambientazione non contaminate dalla magia in alcun modo (che potrebbero quindi appartenere a un mondo realistico e non "Fantasy"). Gli animali sono considerati tutti come "Non senzienti".

\paragraph{Animali di Piccole dimensioni o inoffensivi}\mbox{}\\ 
Questa categoria racchiude tutti gli animali non pericolosi per un avventuriero che possono essere incontrati. Fanno parte di questa categoria conigli, cani, gatti, cavalli, ecc. Queste creature vengono considerate innocue, pertanto ogni azione svolta contro di esse (in condizioni ottimali) sarà considerata sempre riuscita. Per la stessa ragione il loro quantitativo di Punti Vita viene considerato pari a 1. Nota bene: possano esistere versioni degli animali sopracitati particolarmente aggressive o addestrate per esserlo (vedi cani da guerra); in questo caso si consiglia al Master di elaborare le statistiche relative da una creatura che possa essere coerente con la situazione che si desidera rappresentare.
\paragraph{Animale di Medie dimensioni}\mbox{}\\ 
In questa sottocategoria ricadono gli animali di medie dimensioni che in determinate situazioni potrebbero aggredire gli avventurieri (per fame, se minacciati, ecc). Fanno parte di questa categoria orsi, tigri, lupi, ecc. Queste creature, sebbene più piccole in dimensioni, si possono rivelare più feroci e pericolose rispetto ad animali di grandi dimensioni. Sono tipicamente predatori carnivori, in grado di ferire a morte animali più grandi di loro.\\
Anche in questo caso le linee guida generali per strutturare queste creature è di capire concettualmente cosa devono rappresentare; un grosso predatore potrebbe avere un buon quantitativo di punti vita e danni base, ma non colpire di frequente con i suoi attacchi (le sue dimensioni lo rendono più lento di altre creature in questa categoria); un predatore più piccolo, ma molto feroce potrebbe avere alti danni, buona abilità di combattimento ma pochi punti vita.\\
Seguendo questi due archetipi vengono proposti due esempi di animali di medie dimensioni.\\

\textbf{Orso bruno}\\
\begin{tabular}{l l}
Forza&Da 10 a 14\\
Costituzione&Da 10 a 15\\
Abilità da combattimento su "Forza" (in corpo a corpo)&Da 0 a 2\\
Artigli possenti (abilità "Arte della Guerra")&Da 6 a 10\\
Strato adiposo (livello di armatura)&Da 1 a 3\\
Difficolta per colpirlo&Da 0 a 4\\
Riflessi (complessivo)& Da 12 a 15\\
Difficoltà per addestrarlo&Da 4 a 6\\
\end{tabular}\\
\textbf{Lupo}
\begin{itemize}
\item Agilità: Da 9 a 13
\item Costituzione: Da 5 a 7
\item Abilità da combattimento su "Agilità" (in corpo a corpo): Da 1 a 5
\item Artigli possenti (abilità "Arte della Guerra") : Da 4 a 8
\item Livello armatura: Da 0 a 2
\item Difficolta per colpirlo: Da 1 a 3
\item Difficoltà per addestrarlo: Da 3 a 5
\end{itemize}

\paragraph{Animali di Grandi dimensioni}\mbox{}\\
In questa sottocategoria ricadono tutti gli animali di grandi dimensioni che potrebbero comparire nel corso dell'avventura. Fanno parte di questa categoria animali come: elefanti, mammut, ecc.\\
Queste grandi creature generalmente hanno molti punti vita, fanno grossi danni ma a causa delle dimensioni, sono imprecisi in corpo a corpo.

\textbf{Elefante}
\begin{itemize}
\item Forza : Da 13 a 17
\item Costituzione: Da 10 a 18
\item Abilità da combattimento su "Forza" (in corpo a corpo): Da 0 a 2
\item Difficolta per colpirlo: Da -2 a 2
\item Forza sovraumana (abilità "Arte della Guerra") : Da 6 a 10
\item Pelle dura (livello di armatura) : Da 3 a 5
\item Difficoltà per addestrarlo: Da 3 a 6
\end{itemize}




\subsection{Mostri}
In questa categoria ricadono tutte le creature che non fanno parte della categoria "Animali" o "Umanoidi".\\
Bisogna far notare che alcune creature presenti in questa categoria potrebbero anche essere senzienti (generalmente nelle ambientazioni Fantasy i draghi hanno una coscienza paragonabile a quella umana). Alcune creature mostruose, dipendentemente dall'ambientazione, potrebbero ricadere tra le razze giocabili (come potrebbe avvenire per i minotauri, draconidi ecc).\\
Verrà preso come esempio un classico di tutte le ambientazioni fantasy.

\textbf{Drago Rosso}
\begin{itemize}
\item Intelligenza : Da 14 a 16
\item Costituzione: Da 12 a 18
\item Forza : Da 8 a 16
\item Magia Distruttiva: Da 2 a 8
\item Evitare : Da 6 a 10
\item Salute : Da 8 a 10 (+10 di vita extra per i draghi adulti)
\item Schivare : Da 3 a 5
\item Abilità da combattimento su "Forza" (in corpo a corpo): Da 3 a 7
\item Forza sovraumana (abilità "Arte della Guerra") : Da 6 a  10
\end{itemize}



\subsection{Umanoidi}
Questa è la categoria dedicata a tutti i soggetti senzienti (salvo rare eccezioni) di aspetto antropomorfo.\\
Non verranno indicate statistiche specifiche per razza: trattandosi di soggetti allenati e addestrati e scelti per un ruolo, le loro caratteristiche saranno naturalmente orientate alla loro funzione. Il ruolo che avrebbe il modificatore dato dalla classe èmarginale e quindi non viene indicato negli esempi che seguiranno.\\
Il master potrà comunque decidere di differenziare i rivali che i propri giocatori si troveranno ad affontare, calibrando i valori anche in funzione della razza dei PNG. Come potrete notare, i valori delle abilità hanno una forbice molto ampia, questo per differenziare personaggi di \emph{élite} da \emph{soldati semplici}.\\
Si confida quindi sulla capacità del Master di scegliere i valori che meglio si associno alla situazione da lui immaginata.
\clearpage
\textbf{Umanoide Guerriero}
\begin{itemize}
\item Forza : Da 8 a 13
\item Costituzione: Da 8 a 12
\item Abilità da combattimento su "Forza" (in corpo a corpo): Da 1 a 7
\item Parare: da 0 a 5
\item Evitare : Da 2 a 10
\item Salute : Da 0 a 8
\item Schivare : Da 1 a 6
\item Arte della Guerra : Da 0 a 10
\item Armatura : Da 1 a 5
\end{itemize}

\textbf{Umanoide Mago}
\begin{itemize}
\item Intelligenza : Da 10 a 13
\item Costituzione: Da 7 a 10
\item Abilità magica: Da 1 a 6
\item Evitare : Da 1 a 6
\item Salute : Da 0 a 6
\item Schivare : Da 0 a 4
\item Armatura : Da 0 a 1
\end{itemize}


\textbf{Umanoide Ranger}
\begin{itemize}
\item Agilità : Da 8 a 13
\item Costituzione: Da 7 a 10
\item Abilità da combattimento su "Agilità" (a distanza): Da 1 a 7
\item Gittata : Da 15 a 150 metri
\item Evitare : Da 3 a 8
\item Salute : Da 0 a 6
\item Schivare : Da 0 a 6
\item Arte della Guerra : Da 0 a 10
\item Armatura : Da 0 a 2
\end{itemize}


\begin{tabularx}{\textwidth}{|l|c|}
\hline
\multicolumn{2}{|c|}{} \\
\multicolumn{2}{|X|}{} \\
\hline
          &          \\
          &          \\
          &          \\
          &          \\
\hline
\end{tabularx}


\end{document}