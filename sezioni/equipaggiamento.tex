\documentclass[../manuale_main.tex]{subfiles}



\begin{document}

Il personaggio può indossare molti più oggetti di quelli segnati nella sezione equipaggiamento: oggetti come  vestiti e monili non sono indicati poiché non hanno delle regole precise che li regolino.
Si confida sul buonsenso del giocatore e del master per la loro gestione.

\subsection{Equipaggiamento}
L'equipaggiamento utilizzabile da un personaggio è composto da:\\
\renewcommand{\arraystretch}{1.2}
\begin{tabular}{|l l|}
\hline
Elmo&Protezione della testa.\\
Corazza&Protezione del busto.\\
Schinieri&Protezione delle gambe.\\
Bracciali e Guanti&Protezione delle braccia e mani.\\
\hline
\end{tabular}
\\\mbox{}\\
Inoltre, possono essere tenuti fino a uno o due oggetti in mano (dipendentemente dalla tipologia):\\
\begin{tabular}{|l|}
\hline
Arma bianca a una mano e scudo.\\
Coppia di armi bianche.\\
Arma bianca a due mani.\\
Arma da distanza (occupa entrambe le mani).\\
\hline
\end{tabular}\\\mbox{}\\
Chiaramente è anche possibile tenere una o entrambe le mani libere.

Non vengono fornite indicazioni sul numero massimo di monili (collane, anelli ecc) utilizzabili, in questo caso sarà compito del master decidere in quantitativo massimo utilizzabile.
\subsubsection{Tipi di oggetti}
Ciascuno degli oggetti sopracitati è caratterizzato da un livello di protezione/capacità di infliggere danni:
\begin{itemize}
\item \textbf{Oggetto leggero}: ricadono sotto questa categoria gli oggetti facilmente trasportabili grazie alle dimensioni contenute e un peso ridotto. Alcuni esempi sono:
\begin{itemize}
\item le armi corte come pugnali, daghe, gladi, archi corti;
\item le armature in cuoio;
\item gli scudi piccoli (tipicamente in legno e cuoio).
\end{itemize}
\item \textbf{Oggetto medio}: questa categoria comprende gli oggetti che offrono prestazioni migliori del corrispettivo leggero (maggiore protezione o danni) a discapito della trasportabilità. Alcuni esempi sono:
\begin{itemize}
\item armi più ingombranti (senza essere a due mani) come spade, mazze, archi lunghi;
\item armature come la cotta di maglia;
\item scudi di buone dimensioni (tipicamente in metallo).
\end{itemize}
\item \textbf{Oggetto pesante}: questa categoria comprende gli oggetti più scomodi da trasportare e da utilizzare, ma che sono in grado di offrire ottime difese e capacità di infliggere danni. Alcuni esempi sono:
\begin{itemize}
\item armi a due mani come spadoni, martelli, asce;
\item armature a piastre;
\item scudi di grandi dimensioni.
\end{itemize}
\end{itemize}

Gli oggetti pesanti offrono una protezione e capacità di infliggere danni superiore rispetto a quelli leggeri o medi ma il loro uso limita i movimenti del personaggio:\\
\textbf{ogni abilità basata su ``\emph{Agilità}'' ha un modificatore di +1 alla difficoltà per ogni oggetto pesante indossato dal personaggio, mentre la difficoltà per provare a colpirlo ha un modificatore di -1 alla difficoltà per ogni oggetto pesante indossato.}\\
\textit{Per esempio, un personaggio che utilizzi un'armatura pesante completa (per un totale di 4 oggetti) e un'arma a due mani pesante, avrà un modificatore di -5 per essere colpito.}

\subsubsection{Caratteristiche medie dell'equipaggiamento}
Considerando le tipologie sopra descritte, queste sono le caratteristiche base che ciascuna arma o protezione può avere:\\
\renewcommand{\arraystretch}{1.2}
\begin{tabular}{|l l| }

\hline
\multirow{2}{8em}{\textbf{Arma leggera}}&D6 danni \\&Modificatore +0 alla difficoltà nel colpire con l'arma\\
\hline
\multirow{2}{8em}{\textbf{Arma media}}&D8 danni \\&Modificatore +1 alla difficoltà nel colpire con l'arma\\
\hline
\multirow{2}{8em}{\textbf{Arma pesante}}&D10 danni \\&Modificatore +2 alla difficoltà nel colpire con l'arma\\
\hline

\multirow{2}{8em}{\textbf{Armatura leggera}}&+1 di riduzione danni\\&\\
\hline
\multirow{2}{8em}{\textbf{Armatura media}}&+2 di riduzione danni\\&\\
\hline
\multirow{2}{8em}{\textbf{Armatura pesante}}&+4 di riduzione danni\\&\\
\hline

\multirow{2}{8em}{\textbf{Scudo leggero}}&+2 di riduzione danni\\&\\
\hline
\multirow{2}{8em}{\textbf{Scudo medio}}&+3 di riduzione danni\\&\\
\hline
\multirow{2}{8em}{\textbf{Scudo pesante}}&+4 di riduzione danni\\&Modificatore +1 alla difficoltà per parare\\
\hline
\end{tabular}
\mbox{}\\

Ogni oggetto può variare le sue statistiche in base alla condizione di usura, a come è stato costruito ed eventuali incantamenti o maledizioni impresse sull'oggetto stesso.



\subsubsection{Oggetti ben realizzati o usurati}

Un oggetto può essere realizzato in modo particolarmente pregiato utilizzando un materiale migliore o tecniche più ricercate. Conseguentemente le caratteristiche potrebbero migliorare rispetto a quelle base date dalla categoria: un'arma potrebbe avere un bonus migliore ai danni o una difficoltà ridotta nell'utilizzo, un'armatura o scudo potrebbero offrire una miglior protezione o ostacolare meno i movimenti.\\
Contrariamente, l'usura di un oggetto peggiora le sue caratteristiche, con il passare del tempo infatti l'oggetto diverrà quasi inutilizzabile fino a che non sarà riparato.

\subsubsection{Oggetti rotti}
Come già accennato, gli oggetti rotti sono praticamente inutilizzabili.\\
In questo paragrafo considereremo solamente quelli relativi all'equipaggiamento, per gli oggetti che non ne fanno parte sarà il master a decidere se quell'oggetto sia inservibile o possa ancora essere usato (magari per eventuali proprietà magiche).\\
\renewcommand{\arraystretch}{1.5}
\begin{tabularx}{\linewidth}{l X}
\textbf{Armature}&Un'armatura rotta non offre alcun tipo di protezione alla parte che dovrebbe proteggere.\\
\textbf{Scudi}&Uno scudo rotto non può essere usato per parare.\\
\textbf{Armi}&Un'arma rotta può essere usata in combattimento, ma viene considerata come Arma improvvisata.\\
\end{tabularx}

\subsubsection{Riparare un oggetto}
Gli oggetti possono essere riparati da un personaggio che possieda l'abilità adatta (``\emph{Riparare Congegni}'' o ``\emph{Riparare equipaggiamento}'').\\
Se l'azione riesce con successo, sarà il master a stabilire quanto la riparazione sia stata efficace, ovvero se l'oggetto sia tornato perfettamente utilizzabile e senza aver perso le proprietà che lo contraddistinguevano oppure se qualcosa sia andato storto. Un oggetto al seguito della riparazione potrebbe risultare più fragile o meno efficace, anche laddove la riparazione sia stata eseguita correttamente.\\
In caso di fallimento nel tentativo di riparazione, è possibile che l'oggetto sia andato completamente distrutto, impedendo ogni altro tentativo di riparazione.


\subsubsection{Oggetti magici}

Si definisce oggetto magico un oggetto che è stato incantato da un mago e grazie a questo procedimento ha ottenuto proprietà magiche.\\
Possono esistere oggetti che pur avendo proprietà magiche, non sono stati incantati da un mago. Un esempio può essere un artefatto di origine divina. Le regole generali tra oggetti magici e artefatti sono le stesse, ma sono presenti alcune differenze. Le proprietà magiche di un artefatto non possono essere manipolate né rimosse dalla magia di un comune mago.\\

Un oggetto magico può averne caratteristiche anche molto diverse da quelle base.
Un bonus assegnato a un oggetto magico potrebbe essere un +1 alla caratteristica principale (danni bonus nel caso di armi, protezione bonus nel caso di scudi e armature) ed eventualmente un effetto magico.
Dipendentemente dalla rarità e dal valore di quell'oggetto, i bonus potrebbero aumentare sensibilmente.
Un oggetto potrebbe essere magico ma maledetto, avendo quindi caratteristiche peggiori di quelle base di un oggetto di quel tipo.

\subsection{Armi improvvisate}

Il generale ogni arma propriamente detta può essere ricondotta a una macro categoria, come evidenziato dalle specializzazioni di utilizzo relative alle varie abilità.
Questo discorso non vale però per l'utilizzo di oggetti trovati nello scenario di gioco che non nascono come armi, ma possono essere usati come tali. Un giocatore potrebbe voler utilizzare una pietra per ferire un bersaglio: in questo caso si parla di arma improvvisata.
Non esiste una specializzazione in armi improvvisate, ma sarà possibile ricondurre all'abilità grazie al modo in cui vengono utilizzate. Consideriamo i seguenti esempi:
\begin{itemize}
\item Pietre, frammenti di vetro e spade corte spezzate possono essere considerate armi a una mano;
\item Forconi e vanghe possono essere considerate come armi ad asta;
\item Grosse pietre e in generale oggetti ingombranti che richiedono due mani per essere utilizzati possono essere considerati come armi a due mani.
\item Oggetti acuminati scagliati contro un bersaglio possono essere considerati come armi a distanza.
\item Quando si utilizza un'arma nella quale non si è specializzati, questa verrà contata come arma improvvisata.
\end{itemize}
Regole per le armi improvvisate:\\
\renewcommand{\arraystretch}{1.5}
\begin{tabularx}{\linewidth}{X}
I talenti e le specializzazioni relativi al combattimento non contano per le armi improvvisate.\\
Quando si usa un'arma improvvisata si ha un +1 alla difficoltà per colpire.\\
I modificatori al danno dati da “\emph{Arte della guerra}” e “\emph{Precisione}” non contano per le armi improvvisate.\\
\end{tabularx}

\subsection{Mani nude}

Un personaggio potrebbe essere nella condizione di non voler (o poter) utilizzare un'arma. Il combattimento a mani nude differisce da quello con armi sotto alcuni punti di vista.\\\mbox{}\\
\renewcommand{\arraystretch}{1.5}
\begin{tabularx}{\linewidth}{|X|}
\hline
Se si combatte a mani nude, per calcolare i danni verrà usato un D4, al quale verranno sommati tutti i modificatori.\\
L'abilità che viene utilizzata per colpire è “\emph{Mani nude}”.\\
Senza la specializzazione relativa (\emph{Arte della guerra - Combattimento a mani nude}), l'abilità “\emph{Arte della guerra}” non fornirà danni bonus.\\
Se una delle due mani fosse impegnata (per esempio da uno scudo), si avrà un modificatore di -2 al danno (ridotto a -1 con il talento \emph{Mani nude - Scudiero}).\\
\hline
\end{tabularx}


\subsection{Esempi di effetti magici}

Per aiutare il master nella “costruzione” di oggetti magici da rendere disponibili durante l’avventura, viene qua riportata una lista con alcuni effetti di esempio (quindi privati dei parametri) da usare come spunto.\\
Sentitevi liberi di combinare questi effetti (positivi e negativi) e di inventarne di nuovi, in modo da proporre sempre oggetti diversi tra loro. Un ulteriore spunto sugli effetti può derivare dalla lista delle magie presente in questo manuale.\\
Al fianco di ogni effetti è segnato il tipo di oggetto che meglio si adatta allo stesso. Considerate questa indicazione come una guida, non come un obbligo.

\textbf{Effetti positivi}
\begin{itemize}
\item Danni aumentati (armi/amuleti)
\item Difficoltà ridotta su certi TDS (amuleti/tutti)
\item Non puoi essere disarmato utilizzando quest'arma (armi)
\item Visione notturna (armature/amuleti)
\item Ignora armatura (armi)
\item Oggetto molto leggero (armature/armi)
\item Bonus statistiche (amuleti/tutti)
\item Bonus abilità (tutti)
\item Resistenza al freddo (armature/amuleti)
\item Resistenza al fuoco (armature/amuleti)
\item Causa terrore (armature/tutti)
\item Permette di muoversi silenziosamente (armature/amuleti)
\item Acceca l'attaccante se la parata è stata eseguita con successo (scudi)
\item Attira frecce (scudi)
\item Deviazione frecce (scudi)
\end{itemize}
\textbf{Effetti negativi}
\begin{itemize}
\item Aumento dei danni subiti (armature/amuleti)
\item L'oggetto può essere funziona solo una parte delle volte (regolato da un dado) (tutti)
\item Perdita permanente di punti vita o mana ogni utilizzo (tutti)
\item Perdita permanente di punti vita o mana per ogni TDS fallito (tutti)
\item Modificatore negativo statistiche (tutti)
\item Modificatore negativo abilità (tutti)
\item Difficoltà aumentata ogni utilizzo (tutti)
\item Una volta indossato non può essere rimosso (armature/amuleti)
\end{itemize}

\end{document}