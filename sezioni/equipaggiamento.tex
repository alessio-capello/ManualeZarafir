\documentclass[../manuale_main.tex]{subfiles}



\begin{document}



\subsection{Equipaggiamento}
Per equipaggiamento si intende ogni oggetto che deve essere indossato dal personaggio per far sì che il suo effetto sia sfruttato.
\subsubsection{Tipologie di equipaggiamento}
In base alla loro funzione, esistono tre tipi possibili di equipaggiamento che un personaggio può utilizzare:
\paragraph{Protezioni}\mbox{}\\
In questa categoria ricadono gli oggetti che hanno come scopo principale quello di ridurre i danni subiti dal personaggio. Si tratta delle parti di armatura e degli scudi.
In particolare di differenziano in:\\\mbox{}\\
\renewcommand{\arraystretch}{1.2}
\begin{tabular}{|l l|}
\hline
Elmo&Protezione della testa.\\
Corazza&Protezione del busto.\\
Schinieri&Protezione delle gambe.\\
Bracciali e Guanti&Protezione delle braccia e mani.\\
Scudo&Una protezione aggiuntiva che consente al personaggio di \emph{parare} un attacco subito.\\
\hline
\end{tabular}
\\\mbox{}\\
\paragraph{Armi}\mbox{}\\
Questa è la categoria degli oggetti che hanno come obiettivo quello di infliggere danni. Nella maggior parte dei casi il personaggio potrà utilizzare un'arma alla volta.
Un personaggio può quindi equipaggiare:\\\mbox{}\\
\begin{tabular}{|l|}
\hline
Arma bianca a una mano e scudo.\\
Coppia di armi bianche.\\
Arma bianca a due mani.\\
Arma da distanza (occupa entrambe le mani).\\
\hline
\end{tabular}\\\mbox{}\\
Chiaramente è anche possibile tenere una o entrambe le mani libere.

\paragraph{Vestiti e monili}\mbox{}\\
Il personaggio può indossare più oggetti di quelli segnati nella sezione equipaggiamento della scheda: un esempio sono monili e vestiti.
Non vengono fornite indicazioni sul numero massimo di monili (collane, anelli ecc) e vestiti utilizzabili, in questo caso sarà compito del master decidere come gestirli.


\subsubsection{Tipi di oggetti}
Le armi e le protezioni posseggono caratteristiche differenti tra loro, in particolare per ogni categoria si possono individuare tre tipologie di oggetti differenti.\\
Per ogni categoria sono indicate le caratteristiche medie. Ogni oggetto può infatti variare le sue statistiche in base alla condizione di usura, a come è stato costruito ed eventuali incantamenti o maledizioni impresse sull'oggetto stesso.

\paragraph{Oggetto leggero}: ricadono sotto questa categoria gli oggetti facilmente trasportabili grazie alle dimensioni contenute e un peso ridotto. Alcuni esempi sono:
\begin{itemize}
\item Armi corte come pugnali, daghe, gladi;
\item Armi a distanza come archi corti o balestre leggere;
\item Armature in cuoio;
\item Scudi piccoli (tipicamente in legno e cuoio).
\end{itemize}

\renewcommand{\arraystretch}{1.2}
\begin{tabular}{|l l| }
\hline
\multicolumn{2}{|c|}{\textbf{Oggetti Leggeri - Caratteristiche}}\\
\hline
\hline
\multirow{2}{8em}{\textbf{Arma leggera}}&D6 danni \\&Modificatore +0 alla difficoltà nel colpire con l'arma\\
\hline
\multirow{2}{8em}{\textbf{Armatura leggera}}&+1 di riduzione danni\\&\\
\hline
\multirow{2}{8em}{\textbf{Scudo leggero}}&+2 di riduzione danni\\&\\
\hline
\end{tabular}
\mbox{}\\


\paragraph{Oggetto medio}: questa categoria comprende gli oggetti che offrono prestazioni migliori del corrispettivo leggero (maggiore protezione o danni) a discapito della trasportabilità. Alcuni esempi sono:
\begin{itemize}
\item Armi più ingombranti (ma non a due mani) come spade, mazze;
\item Armi a distanza come archi lunghi o balestre di medie dimensioni;
\item Armature come la cotta di maglia;
\item Scudi di buone dimensioni (tipicamente in metallo).
\end{itemize}

\renewcommand{\arraystretch}{1.2}
\begin{tabular}{|l l| }
\hline
\multicolumn{2}{|c|}{\textbf{Oggetti Medi - Caratteristiche}}\\
\hline
\hline
\multirow{2}{8em}{\textbf{Arma media}}&D8 danni \\&Modificatore +1 alla difficoltà nel colpire con l'arma\\
\hline
\multirow{2}{8em}{\textbf{Armatura media}}&+2 di riduzione danni\\&\\
\hline
\multirow{2}{8em}{\textbf{Scudo medio}}&+3 di riduzione danni\\&\\
\hline
\end{tabular}
\mbox{}\\


\paragraph{Oggetto pesante}: questa categoria comprende gli oggetti più scomodi da trasportare e da utilizzare, ma che sono in grado di offrire ottime difese e capacità di infliggere danni. Alcuni esempi sono:
\begin{itemize}
\item Armi a due mani come spadoni, martelli, asce;
\item Armi a distanza come le balestre di grandi dimensioni;
\item Armature a piastre;
\item Scudi di grandi dimensioni.
\end{itemize}

\renewcommand{\arraystretch}{1.2}
\begin{tabular}{|l l| }
\hline
\multicolumn{2}{|c|}{\textbf{Oggetti Pesanti - Caratteristiche}}\\
\hline
\hline
\multirow{2}{8em}{\textbf{Arma pesante}}&D10 danni \\&Modificatore +2 alla difficoltà nel colpire con l'arma\\
\hline

\multirow{2}{8em}{\textbf{Armatura pesante}}&+4 di riduzione danni\\&\\
\hline

\multirow{2}{8em}{\textbf{Scudo pesante}}&+4 di riduzione danni\\&Modificatore +1 alla difficoltà per parare\\
\hline
\end{tabular}
\mbox{}\\

Gli oggetti pesanti offrono una protezione e capacità di infliggere danni superiore rispetto a quelli leggeri o medi ma il loro uso limita i movimenti del personaggio:\\
\textbf{Il personaggio ha un -1 in \emph{evasione} per ogni protezione pesante indossata.}\\
\textit{Per esempio, un personaggio che utilizzi un'armatura pesante completa (per un totale di 4 oggetti) e un'arma a due mani pesante, avrà un modificatore di -4 all'evasione dato dall'armatura.}

\clearpage

\subsubsection{Oggetti ben realizzati o usurati}

Un oggetto può essere realizzato in modo particolarmente pregiato utilizzando un materiale migliore o tecniche più ricercate. Conseguentemente le caratteristiche potrebbero migliorare rispetto a quelle base date dalla categoria: un'arma potrebbe avere un bonus migliore ai danni o una difficoltà ridotta nell'utilizzo, un'armatura o scudo potrebbero offrire una miglior protezione o ostacolare meno i movimenti.\\
Contrariamente, l'usura peggiora le caratteristiche, con il passare del tempo infatti un oggetto diverrà quasi inutilizzabile fino a quando non sarà riparato.

\subsubsection{Oggetti rotti}
Come già accennato, gli oggetti rotti sono praticamente inutilizzabili.\\
In questo paragrafo considereremo solamente quelli relativi all'equipaggiamento, per gli oggetti che non ne fanno parte sarà il master a decidere se quell'oggetto sia inservibile o possa ancora essere usato (magari per eventuali proprietà magiche).\\
\renewcommand{\arraystretch}{1.5}
\begin{tabularx}{\linewidth}{l X}
\textbf{Armature}&Un'armatura rotta non offre alcun tipo di protezione alla parte che dovrebbe proteggere.\\
\textbf{Scudi}&Uno scudo rotto non può essere usato per parare.\\
\textbf{Armi}&Un'arma rotta può essere usata in combattimento, ma viene considerata come Arma improvvisata.\\
\end{tabularx}

\subsubsection{Riparare un oggetto}
Gli oggetti possono essere riparati da un personaggio che possieda l'abilità adatta (``\emph{Riparare Congegni}'' o ``\emph{Riparare equipaggiamento}'').\\
Se l'azione riesce con successo, sarà il master a stabilire quanto la riparazione sia stata efficace, ovvero se l'oggetto sia tornato perfettamente utilizzabile e senza aver perso le proprietà che lo contraddistinguevano oppure se qualcosa sia andato storto. Un oggetto al seguito della riparazione potrebbe risultare più fragile o meno efficace, anche laddove la riparazione sia stata eseguita correttamente.\\
In caso di fallimento nel tentativo di riparazione, è possibile che l'oggetto sia completamente distrutto, impedendo ogni altro tentativo di riparazione.


\subsubsection{Oggetti magici}

Si definisce oggetto magico un oggetto che è stato incantato da un mago e grazie a questo procedimento ha ottenuto proprietà magiche.\\
Possono esistere oggetti che pur avendo proprietà magiche, non sono stati incantati da un mago. Un esempio può essere un artefatto di origine divina. Le regole generali tra oggetti magici e artefatti sono le stesse, ma sono presenti alcune differenze. Le proprietà magiche di un artefatto non possono essere manipolate né rimosse dalla magia di un comune mago.\\

Un oggetto magico può averne caratteristiche anche molto diverse da quelle base.
Un bonus assegnato a un oggetto magico potrebbe essere un +1 alla caratteristica principale (danni bonus nel caso di armi, protezione bonus nel caso di scudi e armature) ed eventualmente un effetto magico.
Dipendentemente dalla rarità e dal valore di quell'oggetto, i bonus potrebbero aumentare sensibilmente.
Un oggetto potrebbe essere magico ma maledetto, avendo quindi caratteristiche peggiori di quelle base di un oggetto di quel tipo.


\clearpage
\subsection{Armi improvvisate}

Il generale ogni arma propriamente detta può essere ricondotta a una macro categoria, come evidenziato dalle specializzazioni di utilizzo relative alle varie abilità.
Questo discorso non vale però per l'utilizzo di oggetti trovati nello scenario di gioco che non nascono come armi, ma possono essere usati come tali. Un giocatore potrebbe voler utilizzare una pietra per ferire un bersaglio: in questo caso si parla di arma improvvisata.
Non esiste una specializzazione in armi improvvisate, ma sarà possibile ricondurre all'abilità grazie al modo in cui vengono utilizzate. Consideriamo i seguenti esempi:
\begin{itemize}
\item Pietre, frammenti di vetro e spade corte spezzate possono essere considerate armi a una mano;
\item Forconi e vanghe possono essere considerate come armi ad asta;
\item Grosse pietre e in generale oggetti ingombranti che richiedono due mani per essere utilizzati possono essere considerati come armi a due mani.
\item Oggetti acuminati scagliati contro un bersaglio possono essere considerati come armi a distanza.
\item Quando si utilizza un'arma nella quale non si è specializzati, questa verrà contata come arma improvvisata.
\end{itemize}
Regole per le armi improvvisate:\\
\renewcommand{\arraystretch}{1.5}
\begin{tabularx}{\linewidth}{X}
I talenti e le specializzazioni relativi al combattimento non contano per le armi improvvisate.\\
Quando si usa un'arma improvvisata si ha un +1 alla difficoltà per colpire.\\
I modificatori al danno dati da “\emph{Arte della guerra}” e “\emph{Precisione}” non contano per le armi improvvisate.\\
\end{tabularx}

\subsection{Mani nude}

Un personaggio potrebbe essere nella condizione di non voler (o poter) utilizzare un'arma. Il combattimento a mani nude differisce da quello con armi sotto alcuni punti di vista.\\\mbox{}\\
\renewcommand{\arraystretch}{1.5}
\begin{tabularx}{\linewidth}{|X|}
\hline
Se si combatte a mani nude, per calcolare i danni verrà usato un D4, al quale verranno sommati tutti i modificatori.\\
L'abilità che viene utilizzata per colpire è “\emph{Mani nude}”.\\
Senza la specializzazione relativa (\emph{Arte della guerra - Combattimento a mani nude}), l'abilità “\emph{Arte della guerra}” non fornirà danni bonus.\\
Se una delle due mani fosse impegnata (per esempio da uno scudo), si avrà un modificatore di -2 al danno (ridotto a -1 con il talento \emph{Mani nude - Scudiero}).\\
\hline
\end{tabularx}

\clearpage
\subsection{Esempi di effetti magici}

Per aiutare il master nella “costruzione” di oggetti magici da rendere disponibili durante l’avventura, viene qua riportata una lista con alcuni effetti di esempio (quindi privati dei parametri) da usare come spunto.\\
Sentitevi liberi di combinare questi effetti (positivi e negativi) e di inventarne di nuovi, in modo da proporre sempre oggetti diversi tra loro. Un ulteriore spunto sugli effetti può derivare dalla lista delle magie presente in questo manuale.\\
Al fianco di ogni effetti è segnato il tipo di oggetto che meglio si adatta allo stesso. Considerate questa indicazione come una guida, non come un obbligo.

\begin{tabular}{l}
\textbf{Effetti positivi} \\
    Danni aumentati      \\
Non puoi essere disarmato utilizzando quest'arma\\
Il personaggio vede al buio \\
Ignora le protezioni \\
Oggetto molto leggero\\
Bonus statistiche\\ 
Bonus abilità \\
Resistenza al freddo \\
Resistenza al fuoco\\
Permette di muoversi silenziosamente \\
Acceca l'attaccante se la parata è stata eseguita con successo\\

\textbf{Effetti negativi} \\
Aumento dei danni subiti \\
 Perdita permanente di punti vita o mana ogni utilizzo\\
Perdita permanente di punti vita o mana per ogni TDS fallito\\
 Modificatore negativo statistiche\\
 Modificatore negativo abilità \\
Difficoltà aumentata ogni utilizzo\\
Una volta indossato non può essere rimosso\\
             
\end{tabular}



\end{document}