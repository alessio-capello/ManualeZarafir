\documentclass[../manuale_main.tex]{subfiles}



\begin{document}


``Nightfall” si propone di essere un valido manuale per i novizi, per coloro che vogliono un manuale rapido con cui giocare subito e per coloro che desiderano creare una propria ambientazione e ampliarla tra le varie sessioni, aggiungendo piccoli dettagli per renderla unica e condivisa tra tutti i giocatori. 
Alcuni consigli per gestire con successo un’avventura lasciando spazio all’improvvisazione è necessario:
\begin{itemize}
\item Conoscere bene le regole del manuale, per garantire una risoluzione rapida di azioni, conflitti ed eventi, affinchè si possa gestire con semplicità ed efficacia nelle diverse ambientazioni.
\item Avere un vasto ventaglio di idee di base da cui poter attingere: geografia del mondo (anche abbozzata), capitali, maggiori luoghi d'interesse per la trama, i suoi abitanti e gli eventi principali che hanno segnato la sua storia.
\item Ragionare sulle proposte dei giocatori prima di rifiutarle. Questo aiuta lo sviluppo dell’avventura, che punta ad intrecciarsi tra volere del master e giocatori. Potete applicare al mondo le idee dei giocatori  e/o prendere spunto dall’azione proposta da loro per dare ulteriore sviluppo alla trama. Nel caso ci siano delle ovvie difficoltà trasformate il “non si può fare” in un “sì, ma per farlo devi/ è necessario che...”. \\
Esempio: il giocatore, siccome il gruppo dovrà mettersi in marcia per una lunga missione lontano dalle zone abitate, vuole spendere del tempo a cercare una biblioteca per trovare nozioni su oggetti magici presenti in zona da cercare durante il viaggio per essere di supporto al gruppo. Potrete utilizzare questo per inserire missioni secondarie alla ricerca di questi oggetti.
\item Aiutare i giocatori a creare dei personaggi. Creazione di una storia/background del personaggio che possa evolversi nella trama principale e una motivazione per cui il party stia assieme. Questo sarà un buon aiuto nella creazione dell’avventura grazie alle interazioni tra i diversi personaggi del party e con i PNG che presenterete loro durante la storia.
Sono questi dettagli a rendere la storia meno banale e più personale.
\item Lasciate la possibilità ai giocatori di intervenire concretamente nelle azioni, sebbene comporti modifiche alla storia che avete immaginato a priori: non trasformate l’avventura nella narrazione di un libro nel quale i giocatori hanno solamente un ruolo passivo. Se avete la sensazione che i giocatori siano “persi” , intervenite per riportarli al centro dell’azione; un semplice trucco può essere quello di chiedere ai propri giocatori: “che scegliete di fate adesso?”. In questo modo saranno loro a portare avanti la storia grazie alle loro scelte. Se esitassero ancora, inserite eventi che li portino ad agire (dal rumore di passi oltre l’angolo ad un incontro con degli altri personaggi, all’arrivo di un folto gruppo di inseguitori etc).
\end{itemize}

\clearpage

Ricordatevi che l'ambientazione deve lasciare spazio per la scelta umana, per le sofferenze e per le ricompense,
deve essere epica in modo da permettere forti cambiamenti nei personaggi dei protagonisti e dovrebbe essere avventurosa, con eventi che siano spettacolari e sorprendenti.\\
Non abbiate fretta di spiegare tutto e subito, date le informazioni necessarie in modo che altri dettagli possano essere aggiunti dai giocatori quando creano i propri personaggi e da voi Master quando svilupperete le situazioni da giocare.\\
Ricordatevi che descrizioni troppo lunghe, tendono a essere dimenticate facilmente, dettagli più concisi ma d'impatto rimarranno impressi nella memoria più facilmente.\\
Siate giusti ed onesti, senza creare forzature, mantenete un’atmosfera appassionante, date fiducia ai vostri giocatori ed alla loro creatività e cercate di dare spazio a tutti, anche in relazione ai loro personaggi.\\

Nota bene: il mondo (e quindi l'ambientazione) è il ``contenitore'' all'interno del quale i personaggi si muovono, un quadro di cui essi fanno parte ma al quale possono (e devono) aggiungere i propri tocchi e pennellate. 
Se volete scrivere la vostra ambientazione, il vostro compito è quindi quello di pensare alle razze che popoleranno il vostro mondo, alle principali forme di governo, ai più importanti eventi storici ed alle religioni che dominano il cuore dei fedeli.\\

Attenzione: non vincolatevi troppo a eventi del mondo magico e divino: ci siamo resi conto (in anni di playtest) che la verosimiglianza è importante per rendere il gioco divertente. Un modo in cui gli eventi seguono una logica assurda può sembrare divertente a priori, ma per i giocatori capire come gestire le situazioni potrebbe diventare impossibile, rendendo il gioco frustrante. Verosimile però non deve diventare ``vita vera”: se nel tentativo di rendere gli eventi verosimili vi trovate ad eliminare tutto ciò che rende divertente il fantasy, forse state sbagliando qualcosa. \\
Per questo motivo vi sconsigliamo di evitare di far incontrare divinità (o esseri paragonabili a esse) ai giocatori, inoltre anche la magia (in particolare quella più pericolosa) dovrebbe essere in mano a pochi. Meno elementi magici vi saranno e più sarà semplice per il master gestire la situazione, inoltre meno cose magiche ci saranno al mondo e più i giocatori saranno invogliati a trovarle e scoprirle. Vi consigliamo quindi di puntare molto sui piccoli eventi e sulle piccole avventure, ogni abitante del vostro mondo deve poter assegnare una o più piccole (o grandi) missioni affinché i giocatori abbiano sempre la possibilità di decidere cosa fare e in che direzione indirizzare la storia.\\
P.S.: Scrivetevi tutto! Se state creando un'ambientazione passo dopo passo allora non potete permettervi di dimenticare nulla di ciò che avete già detto!\\

Considerate questo breve capitolo come una linea guida, che potete infrangere se lo vorrete, la fantasia è il vostro unico limite!

\end{document}