\documentclass[../manuale_main.tex]{subfiles}



\begin{document}

Un personaggio nel corso della sua vita apprende nozioni e conoscenze; Il loro utilizzo ha all'interno del gioco una forte componente narrativa, per questo motivo si è scelto di non definire una o più abilità a riguardo e di lasciarle quindi slegate dai sistemi basati sul tiro del dado.
Sono stati individuati 4 livelli differenti per rappresentare l'effettiva competenza di un personaggio per ogni argomento.
\begin{itemize}
\item Assenti: il personaggio non ha conoscenze nell'argomento.
\item Discrete: il personaggio possiede delle conoscenze rudimentali dell'argomento che gli permettono di districarsi nelle situazioni più comuni in cui sono richieste.
\item Buone: il personaggio è abbastanza competente sull'argomento, pur non essendo un esperto.
\item Ottime: il personaggio un esperto sull'argomento, salvo rare eccezioni, è sempre in grado di sfruttare le sue conoscenze quando sono necessarie.
\end{itemize}

Per chiarire al meglio questa divisione considerate questo esempio relativo alle conoscenze geografiche: 

\begin{itemize}
\item Assenti: il personaggio non conosce la geografia della regione, uscito dalla sua città non conosce i luoghi di interesse e punti di riferimento della regione.
\item Discrete: il personaggio conosce i punti di riferimento e luoghi di interesse più importanti della regione.
\item Buone: il personaggio conosce la maggior parte dei punti d'interesse e di riferimento della regione, non avendo conoscenza solamente per quelli situati nei luoghi più remoti.
\item Ottime: il personaggio conosce perfettamente la geografia della regione, riuscendo a riconoscere tutti i punti di interesse e di riferimento che trova.
\end{itemize}
Le competenze possedute ai livelli indicati devono servire come linea guida poiché dipendono fortemente dall'ambientazione e in generale dal modo in cui ogni master vuole gestire la propria campagna.


\paragraph{Tipi di conoscenza}\mbox{}\\
Per quello che riguarda un'ambientazione di stampo fantasy classico, possono essere individuate le seguenti aree di conoscenza:
\begin{itemize}
\item Geografiche: Racchiude l'insieme delle nozioni legate agli elementi naturali, punti di interesse e di riferimento come fiumi, montagne o costruzioni note.
\item Storiche e culturali: Sono le conoscenze degli avvenimenti storici più importanti per la storia delle maggiori civiltà e della loro influenza nell'arte e architettura.
\item Magiche: Il sapere di stampo magico che può essere compreso anche da un non utilizzatore della magia. La possibilità di riconoscere un incantesimo ascoltando la formula magica o riconoscendo la gestualità a esso legata, è parte di queste conoscenze.
\item Araldica: La storia, effigi, segni e motti delle casate e regni presenti.
\item Linguistiche: Sono tutte le conoscenze legate all'apprendimento di una o più lingue diverse da quella parlata comunemente dal personaggio.
\item Religiose: Sono le conoscenze legate all'aspetto teologico, alle varie religioni presenti, ai riti e simboli che le contraddistinguono.
\item Sopravvivenza: Sono le conoscenze legate alla vita dell'avventuriero, come individuare fonti di cibo o acqua o saper costruire un accampamento.
\end{itemize}


\paragraph{Migliorare le proprie conoscenze}\mbox{}\\
Le conoscenze di un personaggio possono derivare dal suo background o migliorate nel corso dell'avventura tramite lo studio o altre esperienze. Le modalità sono stabilite dal master di volta in volta.

\end{document}