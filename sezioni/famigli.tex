\documentclass[../manuale_main.tex]{subfiles}



\begin{document}
\subsection{Addestrare una creatura}
Può capitare che i personaggi abbiano interesse a interagire con una creatura non senziente (come un animale) al fine di simpatizzare con esso e trarne dei vantaggi; in alcuni casi la creatura potrebbe diventare un compagno del personaggio stesso, seguendo i suoi ordini e restandogli fedele.\\
Per familiarizzare con una creatura è necessario superare uno o più TDS su \emph{``Addestrare animali''}. Il personaggio, mostrandosi amichevole, riesce ad entrare in simpatia con essa. Con il passare del tempo, sempre grazie all'abilità \emph{``Addestrare animali''} sarà possibile addestrare la creatura in modo che segua i propri comandi.\\
La difficoltà necessaria per empatizzare con la creatura ed il tempo e la difficoltà necessari per addestrarla saranno decisi dal master e dipendono dal tipo di creatura: provare ad addestrare un animale aggressivo e indomabile avrà alte difficoltà associate, mentre una creatura più mansueta dovrebbe essere più semplice da addestrare. 

\subsection{Creature evocate}
Quando una creatura è stata evocata, non obbedirà direttamente alle istruzioni date dall'evocatore, ma questi avrà comunque dei vantaggi:
\begin{itemize}
\item La creatura evocata è sempre considerata \emph{addestrata}. Non sarà possibile per lei tornare allo stato selvaggio.
\item L'evocatore ha un modificatore di -3 alla difficoltà per impartire i propri comandi alla sue evocazioni.
\end{itemize}

\subsection{Impartire i propri comandi}
Dal momento che una creatura è stata \emph{addestrata}, per colui che è stato riconosciuto (dalla creatura stessa) come il \emph{padrone} sarà possibile impartirle i propri comandi.\\
Per farlo dovrà superare con successo un TDS su \emph{``Comandare Animale / Evocazione''}. La creatura eseguirà quindi l'azione richiesta.\\
In caso di insuccesso del TSD, il risultato verrà deciso dal master: la creatura potrebbe rifiutarsi di fare ciò che le è stato richiesto, oppure potrebbe provare a farlo, fallendo (con conseguenze più o meno gravi).\\
La difficoltà del TDS dipende dal tipo di azione da eseguire: quanto sia complesso svolgerla correttamente per la creatura di riflette direttamente sulla difficoltà necessaria per darle quell'istruzione.\\
Una serie di fallimenti o azioni spericolate che mettano in pericolo la vita della creatura stessa potrebbero portarla a fuggire, tornando allo stato selvaggio.
\end{document}