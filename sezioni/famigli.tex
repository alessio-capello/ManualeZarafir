\documentclass[../manuale_main.tex]{subfiles}



\begin{document}
\subsection{Addestrare una creatura}
I personaggi possono essere interessati ad interagire con creature non senzienti (come un animale) per simpatizzare con esse e trarne dei vantaggi; in alcuni casi la creatura potrebbe diventare un compagno del personaggio, seguendo i suoi ordini e restandogli fedele.\\
Per familiarizzare con una creatura è necessario superare uno o più TDS su \emph{``Addestrare animali''}. Il personaggio, mostrandosi amichevole, riesce ad entrare in simpatia con essa. 
Con il passare del tempo, grazie alla stessa abilità sarà possibile addestrare la creatura in modo che segua i propri comandi.\\
Il tempo e la difficoltà necessari per addestrarla saranno decisi dal master e dipendono dal tipo di creatura: provare ad addestrare un animale aggressivo e indomabile avrà una difficoltà alta, mentre una creatura più mansueta è più semplice da addestrare. 

\subsection{Creature evocate}
Le creature evocate si comportano come animali addestrati dall'evocatore e potranno quindi seguire i suoi comandi direttamente.

\subsection{Impartire i propri comandi}

Dal momento che una creatura è stata \emph{addestrata}, per colui che è stato riconosciuto (dalla creatura stessa) come il padrone sarà possibile impartirle i propri comandi.\\
La riuscita è stabilita dal master: in alcuni casi la creatura potrebbe rifiutarsi di obbedire (a causa di un'azione troppo pericolosa o perché essa ha altri bisogni da soddisfare, come la fame).

\end{document}