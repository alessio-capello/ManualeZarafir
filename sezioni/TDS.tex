\documentclass[../manuale_main.tex]{subfiles}



\begin{document}

Il Tiro di Simulazione, abbreviato TDS, è il sistema che viene utilizato in questo manuale per stabilire la riuscita di ogni azione \textit{non banale}.
Per azione banale si intende ogni azione talmente semplice che si assume che il personaggio non possa sbagliare, se eseguita in condizioni normali. Alcuni esempi di azioni banali possono essere: \textit{passeggiare, raccogliere oggetti}.\\
Ovvimente possono essere presenti variabili che alterino le possibilità di riuscita delle azioni banali, in questi casi (stabiliti dal master), sarà necessario eseguire un TDS per conoscere l'esito.

\paragraph{Come funziona il TDS}\mbox{}\\
Il giocatore spiega al master cosa vorrebbe che il suo personaggio facesse in una determinata situazione. Il master stabilisce quindi una difficoltà associata all'azione in questione. Le difficoltà variano tra 0 e 10, con 10 la difficoltà massima. \\
In alcuni casi la difficoltà non viene stabilita dal master, ma è implicita nella situazione: colpire un personaggio con abilità \textit{evitare} pari a 4 e la specializzazione \textit{leggiadro} all'1, avrà difficoltà pari a 5 + i modificatori dati dall'arma dell'attaccante (e altri modificatori dati dalla situazione).\\
Noterete però che in questo manuale sono presenti azioni con difficoltà superiore al 10. Questo nasce dal fatto che si tratta di azioni decisamente fuori dal comune, come manipolare il tempo o rendere praticamente invisibili trappole o altri trabocchetti: azioni di questo genere superano la concezione di difficoltà massima pari a 10 a causa dell'unicità dell'azione in oggetto.\\
Un altro caso in cui il personaggio può dover eseguire azioni con difficoltà superiore al 10 è dato dalla presenza di modificatori alla difficoltà che gli rendono più complessa l'azione.\\
Il giocatore tira 3D6 e somma al risultato la difficoltà e i modificatori relativi: se il valore ottenuto è inferiore alla somma tra l'abilità utilizzata e la caratteristica a cui fa riferimento (ovvero la caratteristica associata all'abilità stessa), l'azione ha avuto successo, altrimenti l'azione è fallita.\\
In altre parole (se preferite), considerate la somma tra:\\ \textit{Valore statistica di riferimento + Valore abilità + modificatori - difficoltà}. \\Il valore ottenuto tirando 3D6 deve quindi essere inferiore al risultato ottenuto.

Vanno inoltre specificate due casistiche particolari:
\begin{itemize}
\item Un risultato con i 3D6 pari a \textbf{6 6 6} è considerato fallimento automatico, indipendentemente dalle caratteristiche e dalla difficoltà.
\item Un risultato con i 3D6 pari a \textbf{1 1 1} è considerato successo automatico, indipendentemente dalle caratteristiche e dalla difficoltà.
\end{itemize}

\subsection{Prova contrapposta}
Esiste inoltre un altro tipo di simulazione chiamato ``Prova Contrapposta”:
Si utilizza per tutte le azioni che servono per misurarsi con un'altra creatura in una specifica abilità (\textit{esempio: una sfida a braccio di ferro}).\\
Tutti i personaggi coinvolti nella “Prova Contrapposta” tireranno 3D6 e sommeranno al risultato ottenuto la caratteristica su cui si basa la prova e, se esiste, l'abilità piú adatta (tornando all'esempio del braccio di ferro: La caratteristica ``Forza'' e l’abilità ``Mani Nude'').\\
Chi otterrà il risultato più alto vincerà la prova contrapposta (nel caso di pareggio il master potrà decidere di fare ritirare i dadi o di considerare la prova risolta in parità, senza dunque un vero vincitore o un perdente).

\end{document}