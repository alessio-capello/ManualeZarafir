\documentclass[../manuale_main.tex]{subfiles}



\begin{document}

Per ostacolare i propri nemici o danneggiarli si possono utilizzare delle trappole.
Queste si dividono in base lo scopo che dovranno avere e al livello della trappola stessa.
sono presenti due categorie:
\begin{itemize}
\item Trappole offensive: fanno parte di questa categoria tutte le trappole e dispositivi mirati solamente a infliggere danno a chi li ha attivati. Il danno inflitto è pari a: \textit{Livello trappola + D6+ modificatori}.
\item Trappole bloccanti: fanno parte di questa categoria tutte le trappole e dispositivi mirati solamente a bloccare chi li abbia attivati. Per liberarsi da questo tipo di trappola è necessario superare un TDS su ``Forza'' a difficoltà pari a: \textit{Livello trappola + modificatori}.
\end{itemize}

\subsection{Costruire una trappola}
Per costruire e posizionare una trappola, il personaggio deve dichiarare il tipo di trappola che intende realizzare e il livello della trappola. Dopodiché costruirà e posizionerà la trappola (se possibile) superando con successo un TDS su ``Utilizzare trappole'' con difficoltà pari al livello della trappola.
Costruire una trappola richiede tempo e materiali, decisi dal master in base al tipo di trappola che il giocatore vuole realizzare.
Un personaggio può produrre trappole di un livello pari o inferiore al valore della propria abilità ``Utilizzare trappole”.

\subsection{Occultare una trappola}
Per rendere più difficile trovare una trappola già posizionata, il personaggio può fare un TDS su “Utilizzare trappole'' alla difficoltà desiderata. In caso di riuscita, la trappola verrà nascosta.\\
Il valore scelto per la difficoltà determinerà quanto sia difficile rivelarla.

\subsection{Rivelare una trappola}
Per poter trovare una trappola un personaggio dovrà superare un TDS su ``Utilizzare trappole'' con una difficoltà pari a quella che è stata utilizzata per nasconderla.\\
Se la trappola non è stata nascosta il valore della difficoltà base è pari a 0.\\
Quando una trappola viene individuata, è possibile disattivarla o aggirarla (se l'ambiente lo permette).

\subsection{Disattivare una trappola}
Per poter disattivare una trappola (già individuata) un personaggio dovrà superare un TDS su ``Utilizzare trappole'' con una difficoltà pari a: \textit{Livello trappola + modificatori}.



\end{document}